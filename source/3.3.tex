\chapter{昨夜的星辰(三)}

\section*{前言}
偏僻的晓饭馆里觥筹交错,只是一场非常普通的同事聚餐,但却是陆久和V经历的第一次正常的社交。

在轻松的气氛中和酒精的推动下,他们也像普通人一样交流着自己内心的真实感受。这是一种奇妙的感受,至少对陆久来说是这样。他第一次知道人的心情,竟然可以不只是沉重的浅灰色,而是如此色彩斑斓。

\lineseparator

\section*{}

谢振开车、雷蒙在副驾,而V则坐在谢振后面,留给陆久的只有副驾后边的位子。这是领导席吗,陆久心想。虽然不知道后面这两个位子是怎么分的,但陆久知道社交中座次是非常有讲究的,雷蒙和谢振一定是把上座让给了他。

第一次坐后面啊,陆久心想。在战区的时候他的位置在副驾,因为便于观察。而在上海……

陆久轻轻摇摇头,停下了自己正在回忆的事情。他现在不想想起曾经和他同乘的那些人。

谢振踩下油门,汽车朝西驶去。陆久有点意外,因为他还以为他们会朝东边的市区而去呢。

“不是去市区吗?”陆久问。

“市区里喝了酒可不敢开车了,嘿嘿。”谢振说着一笑,“被抓住了不仅驾照没收,搞不好还要蹲半年大牢。”

“啊。是吗。”

陆久可不知道酒后驾驶是如此严重的事情,他以为最多被口头教育和罚款呢。不过要是喝酒的,话让V把车开回来不就行……

但陆久还是把这句话憋了回去。

还好没说出来,陆久心想。要是在战区,他支派支派V倒是无可厚非的,不过这里已经是新的单位了。

刚上任就使唤女同事,那可是官僚作风;要是解释说他和V很熟悉,那就让人生疑了。

陆久忽然意识到他也学会组织话术了,什么该说什么不该说,不经意地就开始琢磨起来。这就是社交吗,陆久戏谑地心想。想不到自己上手得非常快呢。

汽车一路颠簸,似乎是走在山路上。V一声不响地望着车窗外面,而因为陆久看不到前方,渐渐感觉有点困。但他努力支着脑袋不让自己打瞌睡,因为他害怕自己万一要是靠在V身上睡着了,可要出洋相了。

开了大概有三四十分钟,汽车停了下来。陆久下车一看,他们已经到了山脚下的一座农家院式的饭店。

不过说是山脚下,其实距离真正的山脚还有至少二三十公里,因为这里的地势并不怎么崎岖。他觉得山很近,不过是因为远处的山脉十分巍峨罢了。

“那边是燕山?”陆久问谢振。

“是啊,陆主任来过?”谢振说。

“哪里,地图上看见的。”

“说起来我也算是在燕山脚下长大的,我家在张城……敢问陆主任是哪里人?”

“这个,呵。也算是北方人吧。”陆久皮笑肉不笑地敷衍了过去。

问他是哪里人……可惜他也不知道。

几个人走进了饭店,里边的装修颇有山户风格。房檐下挂着一串串干菜和谷物、院子里堆满了木柴,也不知道是不是真的用这些烧火做饭的。饭店的伙计一看到谢振,立即摆手招呼,显然谢振已经是这里的常客了。

“几位贵宾里边请。”小伙计把几个人让进了偏房,安排了一个小小的雅间。

雅间里有张小小的圆桌,谢振拉出对着门的椅子请陆久坐下,又让V做在陆久右边。然后他让雷蒙坐在陆久旁边,但雷蒙让了让,坐在了陆久对面。

小伙计拿来了菜单递给谢振,谢振又把菜点递给陆久。

“你来你来。”陆久摆摆手,“我不行。我怕这个,一看菜单我就眼晕。”

“姑娘有什么想吃的?”谢振又向着V询问。

“随便。”V看也没有看一眼菜单。

“那我就点些家常菜了。”谢振说着,对着菜单指点了一番。小伙计抄下了菜单,一溜小跑着向厨房送了过去。

“不知陆主任,能喝酒吗?”趁着菜没上来,谢振问道。

“啊……喝一点。”陆久说。

谢振拿过陆久面前的玻璃杯,把从车里带来的酒给陆久倒满。

“姑娘呢?”

“喝。”V淡然说道。

陆久微微扭头瞥了V一眼。她也喝酒?

人形能喝酒吗,陆久不知道。不过V既然说喝,那么也许喝一点也没什么?

陆久这才意识到,自己其实不了解V。虽然对她的性格有所体验,但是她吃什么、喝什么、喜欢什么,陆久完全一无所知。

“哎,我……”

“你什么你。”

谢振又给雷蒙倒酒,雷蒙想要推辞但没推掉,还是被谢振倒了满满一杯。然后谢振也把自己的酒杯倒满了。

“今天欢迎陆主任大驾光临,各位同事也凑齐了,正好熟悉熟悉。这里我年龄最大,所以我来组织一下,但我不是主客,还请各位努力发挥。”

几道凉菜上桌,谢振首先开口说了酒词。

“就咱们四个,就别什么主任不主任的了。”陆久说,“我这个人当不惯官,愿意和我坐在一起的都是朋友,没什么上下级的分别。”

“那怎么行。职位有高低,不能不分啊。”谢振说。

“是啊,您本来就是我们的上司,我们不能装不知道啊。”雷蒙也附和着说。

“可以,那就自便吧。”陆久脸色一沉,把酒杯朝前微微一推。

“这……”

陆久的这个动作谢振知道:推开酒杯表示不喝了,但刚入席就不喝了显然是不愉快的表现。谢振马上明白了,陆久是个不吃溜须拍马那一套的人,过多的恭维只能让他厌烦。

“好,陆老弟性情中人,果然名不虚传。”谢振把陆久的酒杯轻轻推了回去,“那我们就兄弟相称,这倒正合我意。”

“名不虚传”?陆久眉头微微一皱。他可没有在酒桌上传过什么大名,谢振怕是用错词语了吧。不过陆久没有计较,伸手把酒杯重新拿到自己跟前,算是接受了谢振的提议。

“那我们就开始吧?”谢振端起了酒杯说,“两位同志都是初次见面,幸会幸会。请!”

说着,他端起酒杯喝了一大口,雷蒙也抿了一小口。

陆久喝了一口,V则面无表情地一动不动,直到陆久碰了碰她示意,才跟着喝了一口。

“吃菜,吃菜。”谢振说着,几个人都动起了筷子。

按照北方的规矩,前三次要一起喝,在谢振的倡议下大家都喝了三次。觥筹间热菜很快就端了上来,摆了满满一桌。

陆久不知道这都是什么菜,因为他对烹饪没什么心得。他只知道如何安全卫生地把食物弄熟,但是什么菜和什么菜配在一起他就完全不懂了。在他看来这就是好几盘子蔬菜炒肉——只不过菜是山里的野菜、肉是山里的兽肉,不是市场上常见的东西罢了。

而V的注意力则一直放在着离她最近的菜上,完全不在意自己到底在吃什么。

“诸位,我老谢有几句话想说。”酒过三巡,谢振的杯子几乎空了,于是他又把杯子倒满端了起来。见谢振发言,几个人都看向了他。

“二位同志初来乍到,我老谢招待不周,没能趁早好好安排。”老谢说,“特别是姑娘到来的时候,我的表现太差了,简直丑态百出。我深刻地反省。这杯酒是我向你道歉的,请一定原谅我。”

说完,谢振一昂脖一口喝干了杯中的烈酒。这让陆久有些吃惊,倒不是因为谢振的酒量,而是因为谢振态度的转变。

前一天他还一幅和人形不共戴天的面孔,为何今天忽然如此恭敬了?

但V却完全没有被谢振打动,只是淡淡地说:“不要紧。”

听到V的话,谢振愕然地端着空空的酒杯呆住了。

陆久相信V是根本没有在意谢振的无礼,但她太过超然的姿态让陆久有点坐不住,因为V根本没动酒杯。

 “老谢干杯了,你也应该喝点表示回应。”陆久轻声提示V说。

无论她能不能喝酒,对喝酒的礼仪看来是完全不懂啊,陆久无奈地想着。

“嗯。”V说着拿起酒杯,也一口喝了下去。

“咳。”陆久忍不住清了清嗓子。虽不知V酒量几许,但她至少一点都没有谦虚。可是这不成拼酒了吗。

“好啊,二位真是海量,海量。”一旁的雷蒙急忙边说边给两个人填上了酒,打破了尴尬的气氛。

谢振用不敢相信的目光看了V一阵,然后笑了起来。

“好酒量,真是巾帼不让须眉。佩服。”谢振竖起大拇指对V说道,“敢问姑娘芳名?”

“陆薇。”V说道。

“姑娘也姓陆?”

“是的。”

谢振看了陆久一眼,陆久感觉如芒在背,于是默默伸手夹了一口菜。谢振又笑了笑。

“幸会、幸会。我叫谢振,你叫我老谢就行。”

“好。”

“那个,既然是同志们第一次聚餐,我也……敬陆哥一个。”雷蒙忽然端起了酒杯对陆久说道,“其实……我也想向您道歉。您那天问我我是不是不喜欢人形,我没对您说实话。请您原谅。” 

说完,雷蒙举杯一饮而尽。陆久一愣,随即也举杯致意、跟着干杯。

“没事。”陆久说,“谈不上什么原谅,每个人都有自己的想法,也不能全都……”

“不,您误会了。其实我呢,嗯,谢叔作证,我不是不喜欢人形……”放下杯子,雷蒙接着说道,“相反在我眼里,人形是和人类一样的。虽然有点不同,但我从来没有把人形当过物品对待。真的。”

说完,雷蒙在陆久和自己的杯子里又添上了酒。

既然如此,那还有什么可道歉的呢。陆久纳闷了,搞不懂雷蒙到底什么意思。

“哦,那就好。”陆久说,“其实人形和人类在我看来也是一样的。在我身边的都是朋友,我都当做一样的朋友对待……” 

“是吗。能和陆哥这么意气相投,真是太好了。来,再喝一个。”

说完雷蒙再次端起酒杯朝陆久示意,然后把刚倒上的酒干了

“……你喝慢点。”陆久说道。不是说雷蒙不能喝吗,可是一喝起来,这劲头丝毫不输谢振啊。

眼见无法推辞,陆久只好再次跟着干了。他看了一眼谢振,看到谢振正赞许地看着雷蒙,显然是对雷蒙的表现很满意。

他俩是想联合起来灌自己酒吗,陆久有点纳闷。没有理由。虽然自己喝这点还不至于醉了,但为什么呢,这里俩人葫芦里卖的什么药?

“陆哥,其实有件事我一直纳闷,想问您但又怕唐突。”放下酒杯,雷蒙说道。

“没事,你问吧。”陆久说。

“要不再喝一个……”

“再喝就醉了。”陆久忙说道,“有话尽管说,不用顾虑。”

“那我就说了。您和陆小姐都姓陆、而且很熟悉,请问是有……什么亲近关系吗?”

“……哎。”陆久不由得苦叫了一声。这个问题太尖锐,真是该来的躲不掉啊。该怎么解释呢,说旁边这家伙太任性,非要也姓陆?实在是不可理喻。

“没有。”陆久说,“我和陆薇……确实是认识,但不是你想的那样。只是朋友。”

“要是那样,我就放心了。”不知道是因为酒劲还是因为激动,雷蒙的脸通红通红的,“其实我想告诉您……我第一眼看见陆小姐,就……喜欢上了。所以我要是说我想追她……您,不会反对吧。”

此言一出,犹如平地落雷。不仅陆久呆住了、就连谢振都呆住了,餐桌上顿时陷入了沉默。

谢振看了看雷蒙、又看了看V,最后目光落在了陆久身上。

“我,这个——”陆久不知该如何作答,“不,这得问当事人的意见,你别问我啊。我哪能做得了主。”

“那,陆小姐……?”

“什么?”V淡淡地说。

“就是……那个,我刚说的那个……”

“我不太懂你的意思。”

“他说他想追求你!”谢振终于忍不住大声说了出来。

“什么是追求?”V不解地望向陆久。

她这是要我解释解释吗,陆久心想。

我为什么要解释这种事情,这与我何干?陆久有点想抓脑袋。不仅谢振,他感觉雷蒙、甚至V都是在故意针对他。为什么每个人要问问他呢的意见,这里面有他什么事??

简直就是鸿门宴。陆久感觉这三个人,是在故意联合起来难为他。

“意思就是……”陆久在肚里搜索着词汇说,“他喜欢你,对你有好感。”

“嗯。”

“所以希望和你的关系,能有进一步发展。”

“哦。”

“发展到以后,能一直在一起的地步……之类的。”

“不行。”

“咳、咳咳……”

正佯装吃菜、实则侧耳的谢振呛住了,一脸期待的雷蒙也呆住了。

当然这是难免的,这拒绝实在是太直白了。一般人就算不愿意,也会说考虑一下,然后私下再说不行。这当着几位朋友的面就直接说不行,肯定让人难以接受、绝对难以接受吧。

换了谁也难以接受。

但陆久反而想笑。他端起茶杯抿了一口,感到一种笑看纷争的快意。

让你们合伙拿我开涮,陆久幸灾乐祸地想着,你们这是自讨没趣。你们以为她是个好对付的角色吗?

“为什么?能告诉我原因吗?”雷蒙紧握的双手微微发抖,面带不甘地说道,“要是有什么不足的地方,我会努力改善的。你看我虽然不怎么帅,但是还有开发的空间,至少给个机会……”

“我喜欢陆久。”V毫不犹豫地说道,“我想和陆久在一起。”

噗。

陆久嘴里的茶喷了出来。

“别瞎说。”陆久借擦嘴之机,用纸巾掩口低声说道,“在座都是同事,乱说什么呢?”

V和陆久相处已经不是一天两天,“想和陆久在一起”这句话,从某种意义上来说该算是实话。可是就算是实话也要分场合。就算是那时的命令使然……但这样说出来,只会让别人对他们之间的关系产生误解。

再说了,她知道“喜欢”是什么意思吗?

“怎么了,不可以吗。”V不解地说道。

“别说了!”

“……嗯。”

“哈哈哈哈……”谢振再也忍不住笑了起来,“哎呀,这种事我可想不到。反转得太快了啊。哎呀。真他妈的神了,啊哈哈哈……”

“老谢,别跟着瞎起哄!”陆久尴尬地说道,脸上红一阵白一阵的。

“不行。”谢振捂着肚子说,“不行不行不行,陆老弟。姑娘在问你呢,你不能这么糊弄过去。问你可以不可以呢,你倒是说呀!”

“说什么说,唯恐天下不乱……”

“没事,陆哥,你说吧。”雷蒙幽幽地说道,“天涯何处无芳草,我懂的。所以不用顾忌我。你说了,我也就死心了。”

“你们是合伙的吗?”陆久叫道,“你们这是在车上商量好的吧?”

“没有。在车上我们什么都没说。”V依然是淡淡地说道。

“我没问你!”陆久气得快要七窍生烟了。

“啊哈,哈哈哈……”谢振笑得快趴下了,“管你问谁,这姑娘可不说假话!哈哈哈……就算不信我们,你也得信她啊,陆司令?”

“我……嗯?”

陆久忽然愣住了。他好像听到了一个已经有些遥远的称呼。

“就是你们吧。”谢振依然笑着,但脸上的调侃已经不见了,“‘老兵归营’。”

“什么?”陆久有些茫然。谢振好像说了点他有点熟悉的东西,但一时间想不起来是什么。

“看看这个。”

谢振拿起自己的手机拨弄了一番,然后递给陆久。陆久接过来一看,手机屏幕上是一幅照片。那是两年前的一张照片,当时曾经在军报之间广为流传——。

那是神色坚定的V,正扶着目光焦灼、浑身是血的陆久走下直升机。而在他们面前的,是一群举手敬礼的战术人形。

那张照片的名字,叫做《老兵归营》。

\section*{}

陆久脸上的笑容消失了,他没有说话,把手机还给了谢振。他这才明白为什么谢振会问V的名字,他该知道人形这种东西是没有名字的。

“陆久、陆薇。”谢振说,“难怪我看你们眼熟,原来是陆司令和薇副官。何其的失敬,我真是有眼不识泰山。”

“哦……哦!我想起来了,怪不得我觉得陆薇小姐那么……您就是那时候北部战区的……”听到谢振的话,雷蒙也明白了过来,睁大眼睛说道。

“都是过去的事情了。”陆久摇了摇头说,“已经没有什么陆司令了。我是陆久,只是陆久。”

“这该喝一杯了吧?”谢振举起酒杯说道,“以前……我也是个当兵的。这杯我要向老兵致敬!”

“对!”雷蒙也跟着说道,“我是个……想当兵,但是人家不要我。我也向您致敬。”

陆久看了他们一会儿,然后无奈地笑了笑。

“我不过也是以前当过兵罢了,老谢才是老兵。”他说,“喝吧。”

几轮烈酒下肚,几个人都有了些醉意,只有V除外。V虽然从不敬酒,却来者不拒跟着喝了不少,可她没有一点点饮过酒的表现,仿佛她喝下去的都是水一样。

“说起来这些来,关于我们身份的事情,还请各位保密。”陆久对雷蒙和谢振说道,“我们已经不是战区的人员了,也不想和以前的事情再有牵扯。”

“为什么呢。”雷蒙问道,“你们可是格里芬公司的名人。在我刚上班的时候,到处都能看见你们的那张照片……特别是陆哥,可以说是我那时候的偶像啊。”

“是有难言的内幕吧。”谢振说,“战场上的事情,很多都很很难讲清……有些事,也许连讲都不能讲。”

“是啊。”陆久微微叹了口气说,“一言难尽。”

“不过我还是羡慕你们,退役了但还能在一起工作。”雷蒙说,“虽然陆哥稍微沧桑了一些,但陆薇小姐还是和照片上一样好看。说实话,我那时第一眼看到她的照片时就在想,这就是我的梦中……嗯,算了。”

陆久闻言苦笑了一下,然后看了V一眼。人们都看到她的美,谁又曾了解她受的罪呢,就连陆久自己都不敢说了解。但V只是默默端坐在椅子上,仿佛完全没有听雷蒙他们说话。

“老谢。你也和战术人形一起……工作过吗。”陆久说。

“工作?呵呵。”谢振笑了一声,他知道陆久在说什么,“算是吧。你是想问,我为什么会不喜欢人形吧。”

“嗯。”陆久点了点头。

“好吧,既然你问起来了,我就告诉你好了。”谢振拿出两根烟,把其中一根递给陆久,“我呢,和你不一样。虽然也是战斗人员,但我不是个指挥官。我是个特殊勤务人员,你懂吧。”

“我知道。”陆久再次微微点了点头。公司有些专门执行特殊任务的人类武装人员,他们和广泛使用的战术人形是有区别的,被称作“特殊勤务人员”。之前在N21战区时前来收押“播音员”的那些人,就属于特勤人员。

“现在的特勤人员已经严格和战术人形分隔开了,除非经过会议决定,否则不会轻易和战术人形联合行动。但几年前不是这样。以前的特勤小队很喜欢让战术人形提供火力支援,因为她们对武器的控制非常精确。可是三年前,出了一次事故改变了这一切。你知道‘尤妮卡事件’吗。”

“是那次因为战术人形的失误,导致多名人类士兵伤亡的事件吗。”陆久说。

“没错,公司一直对外宣称如此。”谢振压低声音说道,“但实际上,那次不是人形的失误导致伤亡那么简单。为了避免产不良影响,公司隐瞒了许多事情,只是把责任都推到了那几个人形身上了事。”

“……你为什么会知道这些?”陆久问道。

“因为那次行动正是由我带队的。”谢振说,“受到那次事件的影响,我被调离了特勤小队……那次行动的代号就是‘尤妮卡’”。

说着,谢振抽了一大口烟。

“那次行动是在印度洋展开的,位置是马六甲海域。有一艘货船被怀疑藏有化学武器,公司受雇佣前去调查,一共出动了两个小队,而我的小队是其中之一。我们出动的时间是……午夜。

“那天的天气很不好,那片海域正处于热带风暴中,我们清楚地记得乘坐的运输机上下飘摇,就像狂风里的一片树叶。不过也正是借着天气的掩护,船上的守卫完全没有发现我们登船。潜入很顺利,我们没花多长时间就找到了装有神经毒剂的集装箱,然后准备把货物运回去。就在这时,出现了紧急情况:两枚巡航导弹从海里钻了出来,贴着海面朝着我们飞过来。我意识到这次行动一定牵涉到了某些军事大国,他们是想击沉这艘船掩盖证据。事不宜迟,我立即下令丢弃货物、只拿走纸质的文件,然后全速撤离。可是时间已经来不及了。只用了两分钟多点的时间,导弹就击中了货船,在船舷一侧撕开了一个巨大的洞,船体开始迅速下沉。不过幸运的是,我的小队队员全部安全登上了甲板,撤离的直升机就近在眼前。但另一个小队……却被困在了船舱中。”

“人形小队吗。”陆久问道。

“是的。我的小队一共八个人,全部都是人类士兵。但另一个小队是由四个战术人形组成的。在那种情况下,根据规定该怎么做,你一定知道吧。”

陆久点了点头。当人类和人形同时遇险时,人类的生命安全是第一位的,让人类士兵为了战术人形去涉险是绝对不允许的。

“然而你违反了规定?”陆久说。

“不,我的命令是丢弃那些人形,但是遭到了几名队员的反对。他们希望去营救那些被困人形,因为他们的强烈要求……我动摇了,同意了他们的要求。结果不仅那些人形没有营救回来,我们还损失了三个队员,我也因为违规决策而被撤职。在那之后,人类和人形之间的联合行动就变得非常谨慎了——不是因为彼此之间的不信任,而是因为这里面存在有巨大的道德层面的隐患。”

“但这……并不是那些人形的错。”陆久说。

“是啊。也许这是战争贩子们的错、是人形制造商的错。但归根结底是我的错。如果我坚持原则不允许他们去救援那些人形,事情就不会变成这样。但事已至此,再说什么也无法挽回了。从那之后……我就再也不想和人形打交道了。”

陆久看着谢振,想要安慰他几句,却不知道该说什么。谢振的心情他可以理解,因为和多数人的想法一样,对他来说战术人形只是一种军用设备,不值得让自己的兄弟为此送命。但他手下的士兵却并不全都这么认为,这件事本质上是人类之间观点的分歧。无论谢振对战术人形的看法如何,站在他那时的位置上,这都是一个艰难的抉择。

陆久原本认为自己还是很重视战术人形的,但细细想来他重视的不过是自己手下的人形。那时变种的伞病毒感染了21战区总部的时候,他也毫不犹豫地击毁了好几名21战区的战术人形,这种做法也许和谢振的想法没有什么区别。

“算了吧,都是过去的事了。”陆久说。

“是啊,反正也跟我没关系了。”谢振笑了笑,“没有冒犯的意思,我知道战术人形都是忠诚可靠的。如果陆老弟陷入危险,陆薇小姐一定会毫不犹豫地挺身而出吧。”

“她已经不是我的……”

“当然。陆久的生命安全对我来说,比任何事情都重要。”

陆久不太喜欢这个问题,但他刚要开口抗议就被V的话打断了。他看向V,只见V的脸上依然带着平静的表情,但说出这句话的时候,她的眼睛里透着坚定的光芒。

她为什么要说这些呢,是为了向自己表明立场吗,陆久心想。

不,那只是她对谢振的提问的回应。她没有必要向自己表明什么,因为她的意志,早就无数次地用她的行动去证明过了。

“我能理解那时你的队员的想法,不是每个士兵都把战术人形当做可以随时丢弃的设备的。虽然战术人形是为了降低人类伤亡而出现的‘代用品’,但对于很多人来说,她们也是和自己一同出生入死的同伴。”陆久说,“薇曾经和我共同经历了许多战斗,她不止一次救过我的小命。虽然她已经不是我手中的武装力量,但如果她落入险境,我也绝不会将她弃之不顾的。”

听到陆久的话,谢振没说什么,只是看着陆久点了点头。然后,他扭头对身边的雷蒙说:“我看你没机会了。”



\section*{}

那天几个人一直喝酒聊天到很晚,回去的时候已经过了午夜。谢振首先开车把陆久和V送回了公司,然后又拉着雷蒙不知所踪。临别的时候,谢振说他要和雷蒙去单身人士的俱乐部玩玩,就不带陆久这样有人为伴的人了。陆久很想说自己和V的关系不是他们所想象的那样,但无奈他已经喝得七荤八素,实在无力再去辩解。

在V的搀扶下陆久跌跌撞撞地回到了宿舍,进门差点被放在门口的被子绊个跟头。他本想把床和V分开,可惜已经腿脚发软实在做不到,只好一头栽倒在床上。

“那个。”陆久指着门口的被褥,磕磕绊绊地对V说,“你的。”

V领会了精神,把空着的床收拾了一下,然后铺上了被褥。

“这床铺盖好大。”V说。

“总比小了强。”陆久迷迷糊糊地说。

“你不要紧吧。”

“还好。头晕。”

“酒量不好,还喝这么多。”

“哈。哈哈……”

听到V的批评,陆久笑了起来。

“笑什么。”V说。

“笑你啊。真是个怪物。”陆久说,“我们四个人……喝了六瓶白酒,雷蒙……是半路跑了,没喝多少。剩下我们三个,喝了……差不多五瓶,就算我酒量再好也……可是你,为什么一点事都没有?”

“因为我可以……将酒精通过尿液排出。”V的脸微微一红,“难道人类,不行吗。”

“人类是,无法排出酒精的。只能通过肝脏分解。”陆久说,“难怪你那么能喝。我可不能和你比……”

“抱歉,我不知道。我给你倒点水喝吧。”说着,V去洗手间接了一杯水递给陆久。

“谢谢。”陆久说着接过杯子喝了一口,发现那杯水凉得像冰一样。

“这是……自来水?”

“是的。”

“好吧。”

因为实在是口渴得厉害,陆久没再说什么,端起杯子一饮而尽。冰冷的自来水灌进去,引得他胃里一阵痉挛、肚子里开始翻江倒海。

于是陆久挣扎着从床上坐了起来,然后连滚带爬地钻进了洗手间,抱着马桶剧烈地呕吐了起来。

“你怎么了?”V惊慌地问道,“没事吧?要不要去医院?”

“没事。”陆久说,“用水壶,把水加热一下。太凉了。”

“可是你吐得很厉害……”

“是胃的保护性反应。排出具有刺激性的食物、防止更多摄入有害物质导致中毒。去烧水吧。”

吐出了胃里的食物,陆久感觉好了许多。他冲洗了一下然后从洗手间走了出来,再次倒在床上。

“对不起。现在是冬季,水管里的水对人类来说温度太低了,是我考虑不周。”看着床上的陆久,V再次端来了热水,有些愧疚地说道。

“而且含有细菌,和对人体有害的……漂白剂。”陆久斜倚在床头,接过杯子喝了一口说,“人类的身体很脆弱。远不如人形那样坚韧……水必须要烧开,之后才能饮用。”

“我以后会注意的。”

“你呀。呵。”陆久笑了,“对人类的了解,其实很少啊。”

“所以才不行吗。”

“什么不行?”

“……喜欢你。”

陆久愣住了。他盯着V看了一阵,一直到确定自己不是出现了幻听,才微微转开了目光。

“胡说什么呢。”

“我没有胡说。”

“你知道什么是‘喜欢’吗。”

“我不确定。但雷蒙说他喜欢我,想和我一直在一起……但我想和你一直在一起。所以我觉得我喜欢你。”

……无懈可击的论证啊,陆久心想。从这个理论来推导,V的这种感情称之为“喜欢”,也许没有逻辑上的漏洞?他也想不清楚。

他感到天旋地转,思考对此时的他来说,是相当一件难事。

“雷蒙是个人类,他所谓的‘喜欢’……是个很宽泛的概念。”陆久闭着眼睛,揉着太阳穴说道,“和你说的这种,不是一回事。”

“因为我是人形,所以就不行吗。”

陆久被问住了。人形的感情,到底能不能和人类相提并论?他脑袋里嗡嗡作响,实在没法回答这个问题。

可惜自己不知道,究竟是谁给人形赋予了感情……陆久心想。不然,他一定要找到那个人去问一问……

嗯,一定得问一问。

“你也喝醉了吗。”陆久说。

“我没有。”

“那就别问……这种,莫名其妙的问题。”说着,陆久躺下盖上了被子。

“也就是说不行吗。”V有些失落地说。

“我没说……不行……”在陷入沉睡之前,陆久在迷迷糊糊地说道。