\specialsectioning{}
\chapter{补充部分:旅行}

“呼、呼……”

少女奋力地奔跑着,汗水正在沿着她的脸颊不断淌下,她也顾不上去擦拭。

快点……她对着自己说着。

已经看到前边那个模糊的身影了,只有两百米……也许还不到。

少女加快了脚步,又努力奔跑了一阵,但她实在跑不动了。

“站住……”她一边努力抬腿朝那个身影走着,一边用力喊道,但急促的呼吸让她没法喊出太大声,“等一等!”

那个身影明显听到了身后的呼声,但他并没有驻足回望。他只是稍稍停了一瞬,然后就继续快步朝前走去。

“等等……”少女又喊了一声,再次奔跑了起来。片刻后,她终于追上了那个人,那是个高大的男人。但那个男人依然没有停下。于是她伸手拉住了他的手臂。

“等,等一等……”少女上气不接下气地说着,男人终于停下了,但他未发一言,只是默默地站在原地。

“你……要去哪?”少女问道。

“这和你无关。”男人淡然说道。

“只要关乎你的安全,就和我有关。”终于缓了一口气的少女说着,“我从公司接受命令……”

“够了。”男人不耐烦地说着,“很显然,我在独自出逃。我正在离开公司,所以公司的行事准则已经不适用于我了,你不是不明白。”

“……”少女沉默了。

“公司的命令并没有对我的职责加以限制。”片刻后,她继续说道,“除非遭到你或者公司的解职,否则我的职责将一直有效。”

“那我就在此地宣布将你……”男人恼火地转过身说道,但话说了一半他停住了。

“不,”他的声音忽然变轻了,“你并非我拟聘的人员,我无权将你解职。如果你真的是在履行自己的职责,就请将此事汇报公司等公司决断吧。”

“……我不会离开的。”少女倔强地说道。没有必要互相打官腔了,这里没有别人,他们两个都知道彼此在想什么。

“你到底想干什么?”

“保护你的安全。”

“我很安全!”男人终于忍耐不住发怒了,“我已经离开战区了,还有什么会威胁我的安全!?”

少女没有说话,只是直直地看着男人的眼睛。他说的没错——除了子弹之外,没有什么能威胁到他的安全。但女孩想要保护的,不只是他的人身安全。

“如果有朝一日他迷失了,你一定要把他带回来啊。”女孩依然记得这样的约定,闺中密友之间的约定。

所以她不会就凭简单的几句话就被这个男人打发走。

“至少告诉我,你要去哪。”女孩轻声说道,“不然我会一直跟着你。”

“我是在逃亡。你知情不报就是共犯,你难道不知道吗。”男人的声音里带着疲惫。

“不,我没有这样的职责。我的义务就是确保你的安全,不分情况、不分场合。”

“呵,别自欺欺人了。”男人冷笑了一声,“你想跟着无所谓,但是有一件事我想你也明白:我是个人类,就算到了最后,他们最多也不过是把我重新扔进那个监牢。而你就不一样了,他们会把你——”

“我知道。”少女毫不在意地说道,“即使如此我也会尽责。我的结局和我的行为没有必然联系——我想你也明白。”

最后那句话,她是用几不可闻的声音说出来的。

但男人显然听得很清楚。所以他沉默地看了少女一阵。

“那就走吧。”男人叹了口气说道,“但是注意一点,不要让其他人觉得我们是在逃亡。”

说完,他开始继续前行,少女快步跟了上去。

“我不知道,你的这些行为当中,所谓的‘职责’到底占多大成分呢。”男人一边走一边低声嘟囔着。

“很大成分。”他身边的少女低声回应道。

\section*{}

“以前见过海吗。”

陆久一边说着,一边嗤啦地拉开了一罐啤酒。他身边的啤酒罐已经堆积得像小山一样了。

他身边的少女默默地摇了摇头。

“是吗。”陆久说道。他把目光投向那片月光下正在不断起伏着的水面的尽头,仿佛想要说些什么。但他终于还是什么都没说,只是把啤酒端到嘴边,然后仰头一饮而尽。

陆久和他“副官”离开N17战区已经4天了,他们先是搭过路车来到了条铁路的附近,然后扒着货车到达了秦市。今天应该是他们踏入秦市辖区的第一个晚上。

陆久说这个地方叫秦市——V可不知道这地方是什么地方,更不知道陆久为何会熟知这样的“旅行路线”。不过陆久对这行程似乎相当自信,看来他早有预谋。

两个人沿着铁路来到他们见到的第一个小镇之后,陆久做的第一件事就是买了一个旅行帐篷(V看到他兜里装着不少现金,这让她对陆久是早有预谋地逃亡更加深信不疑),其次就是搬了一大箱啤酒。

陆久很熟悉地跟小镇的居民攀谈着,从他们口里套到了不少关于这个地区的信息。陆久对他们说他和V是一对刚刚结婚的新婚夫妇,正在徒步旅行。人们纷纷对陆久能够娶到这样一位年轻貌美的妻子表示恭喜,而陆久似乎也陶醉在人们的祝福之中,但一转眼,陆久的脸上就又变成了那副淡漠的表情。

“人们总是会对男人保持警惕,但是一个已婚男人对他们来说却是安全的。”陆久这样对V解释说。

V表示理解地点了点头。

她倒无所谓,自己被介绍称随从还是什么她都不在乎。她知道陆久当然不能说“我是一位正在逃亡的指挥官,而这位是我的副官”,而说是兄妹的话,她带那副有着北欧血统的外表和陆久纯正的东亚脸孔,恐怕也会惹人生疑。

当然,两个外表迥异的异性,说是夫妇无疑最能让人理解。

没什么,反正也不是真的。V心想。再说了,其实她根本不明白 ,“夫妇”这个词……到底是什么意思?

两个人沿着不算宽阔但是却相当整洁的马路走着,似乎完全没有目的。当然,这只是对V而言,她知道陆久一定有个他要去的地方,只是他一直都没有说。

到了下午的时候,他们来到了一片宽广的区域旁——如果用V那为数不多的词汇去描述的话,那大概应该叫做“海边”,而那片不断发出 “哗啦哗啦”的声响并且看起来似乎无限宽广的水域,大概就是所谓的“海”。

这是一条沿海公路。

第一次看到这样的景观,V愣住了片刻。待她回过神,陆久已经走出很远了,于是她快步追上了陆久。

陆久显然对这样的东西已经不再陌生,他甚至没有去看一眼那片海。而V的心里则在反复想着一件事——

好大。

如此宽广、如此开阔,举目所及根本看不到尽头。而且还在不断起伏汹涌着。

是什么样的力量把这些水流推向沙滩又退回,不断重复永不休止?V深感迷惑。她长达十几年里积累的记忆中,只有关于战斗的情景。在这些情景当中,没有一个片段是关于“海”的。

她又不免地对眼前的东西产生了好奇。海里有什么?海有尽头吗?V心想。还有,海的尽头……又有些什么呢?

但陆久没有说任何关于海的事情,他根本就没有说一句话。他只是这样沉默地走着,沿着同样仿佛没有尽头的海岸线。

当太阳终于沉没在天空和海的相交之处的时候,陆久停了下来。

“今天就在这里扎营吧。”他说。

V也跟着停了下来,但她四顾了一番,这里什么都没有。

马路的一侧是灌木丛生的荒野,另一侧是一片长约百米的沙滩,再往那边就是无限的大海。

她隐约想起自己曾经在某个雨林中宿营,但那个地方至少有些高大的树木可以用来躲避夜行的动物。但这里什么都没。

如果到了晚上,海里出现了危险的生物……V有些担心地想着,但是她看了看陆久,知道他的决定不会改变了。

“好。你休息吧,我来放哨。”V点了点头说。

陆久有些不可思议地看了她一眼。他似乎是想笑,但却没有笑。

“不必了,”陆久一边说着一边把帐篷支在沙滩上,“这里很安全,没有值得警惕的东西。”

“……是吗。”听到陆久这样说,V稍微安心了一点,因为她知道陆久对这里比自己熟悉得多。

于是她从背包里拿出了在镇上买的压缩干粮。陆久曾让她物色一些口粮,但是V对食物没什么概念,所幸她在商店里找到了一些压缩饼干。这些饼干和她们在军营里领到的一款口粮很像,所以她就买了一大包。

陆久看到V拿出的东西,皱起了眉头。不过他没说什么。

V把两块饼干对起来磨碎放进饭盒,然后把附赠的生石灰倒在饭盒夹层,又在夹层和饭盒里倒进了水。

饭盒里立即发出了吱吱的声响,一会儿就微微冒出了热气。感觉温度差不多了,V拿起饭盒递到了陆久的面前。

陆久凝视着饭盒里的食物一阵,然后抬头看了看V。

“你这样,以后怎么能嫁得出去……”陆久嘴里嘟囔着。

“什么?”V不解地问道。

“没什么。”陆久接过他的晚餐,风卷残云般地吃了下去。看陆久开始用餐了,V也吃了一些饼干,但是她根本没有加热,只是就着水嚼了嚼就咽了下去。

“休息吧。”陆久吃完他的口粮,对着V说道,然后自己坐在了帐篷旁边,从背包里取出了他买的啤酒。

他嗤啦地拉开一罐,然后咕嘟咕嘟地几口就喝完了。

他是在,饮酒吗。在一旁默默看着的V心想。

“借助酒精的麻痹,忘记让自己感到难过的事情,暂时地逃离痛苦……”

好像有什么人曾经对自己说过饮酒这种行为。V想起那个人说过,陆久是会饮酒的,自己今天终于亲眼看到了。

那么陆久是在逃避什么、又是想要忘记什么呢。不言自明。

他一定是在逃避失去的痛苦、想要忘记那个曾经在她身边的人。但是那个人的影子,此刻一定就在陆久的眼前。

如果现在是在一片黑暗的丛林里,他大概很快就能安眠了吧。可惜今天的天空很晴朗,天上的月亮也很亮。月光照在海面,波涛泛出银色的光芒,周围的一切清晰如白昼。

嗤啦,陆久又开了一罐啤酒,然后一口喝了进去。

那一定很痛苦,V心想。他想要逃开的影子,一定正环绕他的身旁,可他却无计可施。

但自己又能做些什么呢。什么都做不了。自己承诺了要“带他回来”,可V现在才发现这种承诺是如此的不负责任——自己只不过是区区一个人形,能做什么呢。

陆久自己也说了,那是他心中“最优秀的战士、最美丽的女人”。自己永远都替代不了那个人。

自己除了在这里默默注视着那个无助的背影之外,根本什么都做不到。

陆久其实说得很对,V心想,那时候自己如果真的是为了履行职责的话,就该把情况上报公司。

但她却选择了和陆久一起逃亡。这算是什么呢。

这根本不是她的职责。这毫无疑问是帮助陆久犯错。这不违反背了公司给她的命令、也违背了她和那位“闺中密友”的约定。

不是出于职责、也不是出于约定,她只是凭着自己意愿去做着这样的事情。

她只是因为自己,一厢情愿地想要和他在一起罢了。

真是愚蠢,明知道这样下去不会有好的结果,还是自顾地这样做了。V在心里嘲笑着自己。

自己到底想要怎样?她自己都说不上来。

一切都由着他吧,V心想。无论他想去哪,自己跟着就是了。哪怕是天涯海角。

其他的,她可能真的没法再去做什么了。

一边这样想着,V轻轻地躺在了帐篷里,听着帐篷外的男人一罐又一罐地打开啤酒的声音,渐渐陷入了浅眠。

\section*{}

当V睁开眼睛的时候,天依然没有亮。事实上,这很可能正是午夜,因为月亮正在天顶。

她发觉自己身边躺着一个正在微微打鼾的男人,是陆久。

喝醉了、睡着了吗。她心想。这样也许他就会觉得好一点吧。至少在这一刻,他不必再被那些无法挽回的悔恨折磨了。

V轻轻起身,来到了帐篷外边,大海依然汹涌着,海风已经有些凉了。V看到海浪正在离他们很近的地方起落着,比下午太阳落山的时候要靠近得多。

想不到,海也是会移动的呢,她心想。这就是那种叫做“潮汐”的现象吧。但是她确信海浪终究不会触及他们的帐篷,因为帐篷就支在马路的路基下面,那里是不会被海量冲毁的。陆久支起帐篷的时候肯定也考虑到了这些。

V在帐篷前坐了下来,那里扔着一大堆啤酒罐,陆久已经把他买的所有饮料都喝完了。

……大海的尽头。V注视着银光闪烁的海面想着。如果可能的话,想去那里看一看,看看那里到底有些什么。

不过注定是不可能的吧。且不论自己的身份,就目前的情况而言,没有陆久的带领她甚至不知道自己该去往何方。

就在V看着海面出神的时候,她感到自己身后有动静。还没等她做出反应,她就感到自己被抱住了——那是一双粗壮有力的手臂,她闻到了浓重的酒气。不用说她也知道是谁。

她身后的人把头放在她的肩膀上,贪婪地嗅着她的头发气息。她知道这个人想要做什么,毕竟自己也是一具女性的躯体,也在散发着雌性的荷尔蒙。

如果是那样的话,事情就太容易了——无论这个人做什么,她都不会反抗,没有必要、也没有理由反抗。

但是正因如此,她心里也清楚地知道,身后的这个人,到最后一定什么都不会做。因为自己不是他心里想着的那一个。

果然,那个人在她的肩头伏了一会儿后,慢慢松开了他的手臂。他再次躺倒在帐篷里,然后发出了均匀的呼吸声。

当V再次醒来的时候,天依然没有亮,但是也快了。她身边没有人,但那个人离她并不远。

他就在帐篷外边,在他昨天呆过她也呆过的地方,呆呆地静坐着。

陆久安静地坐在那里,看着渐渐发红的天边,就好像他在那里坐了整整一夜一样。

如果昨天晚上V没有醒来,她一定会这样认为,因为昨天晚上的一切,都已经随着潮汐的退却而隐去了。那时她对大海一时的迷想和他心里忽然涌起的欲望,都已经烟消云散。

“我们走吧。”

当清晨的第一缕阳光照在陆久脸上的时候,他开口说道。

“好。”V轻声应道。

两个人依然沿着马路徒步走着,没用多长时间,他们来到了一个路口。

右边依然是沿着海岸线延伸的马路,而左边则是通向海拔渐高、树木也渐渐茂盛起来的地方。V向着那个方向看去,那边的远方高高耸立着庞大的土丘,大概该被称之为一座山。

该往哪个方向走呢,V心想。当然,她不知道。她连他们这是在去哪都不知道。

她也不想知道。她就想这样走下去,要是可能的话,最好是永远都不要走到那个目的地。

但是如果可以的话,她想去那座高耸的土丘那边看一看。那座山挡住了她的视线,她想知道山的后面有什么,就像想知道海的那边有什么一样。

她忽然发现自己想要知道很多无关紧要的东西。为什么呢,她思忖着。这些和自己有什么关系呢,就算知道又如何呢?这是一个人形应该去探求、去知道的东西吗。

她也说不清楚。她只是……只是想要知道而已。

“这边走。”

正当V有些出神的时候,她听到陆久开口说,然后向着左边的路走去。

波涛的声音渐渐远去了。V不知道跟着陆久走了多久,她感觉应该不是很长时间,因为太阳一直在从树叶的缝隙之间洒下,并没有西斜的迹象。

身边的光线正在渐渐变得柔和,海面和沙滩反射的刺目阳光已经完全看不见,略带咸味的海风也闻不到了。四处都是树叶泥土的清新气息。

秦市一定是个景观很好的地方,V心想,不仅傍海而且依山。这样的风景真是难得。

但是这一切对于陆久来说,似乎依然是稀松平常的,他没有兴奋地左顾右盼,而是依旧漠然地沿着道路不停地迈着步子。

他依然是朝着他的目的地前进的吧,V心想,所以他不会注意路上的风景。

他就像是一个走在回家路上的归乡人一样。

他是不是正在去往自己的家乡呢,V的心里忽然冒出了这样的想法。这倒是很有可能,因为V曾经听说过陆久曾经就是生于北部战区的人。

只不过这一点她从来都未能得到确认罢了。

随着渐渐深入山林,脚下的路变得越发狭窄了,水泥的马路地面也没有了,只剩下碎石铺成小路。

这是自然,在植被如此茂盛的地方修路肯定是非常困难的。

这条路究竟通向何方呢,V这样想着,下意识地回头望了望。

已经看不见了。

山里的路十分曲折,七弯八绕地走了一上午,来时的路早已隐没在树林之间没了踪影。

虽然有路,但是似乎已经失去了方向。只能任由这条路将他们带到什么地方了吧。

他们沿着小路走了一阵,V忽然又听到了水声。不是海边那种淘浪的声音,听起来倒像是流水的潺潺声音。

“……有水声。”V轻轻地对陆久说道。她确信陆久没有听到,因为这个声音在她的耳朵里也是很轻微的,而战术人形的听觉比人类更加敏锐。

但陆久似乎早就知道了。

“嗯,有水声。”他说。

没过多久,他们来到了一条河流旁边。那条河流有六七十米宽,水流十分平缓,几乎看不到流动的迹象,但是水流的声音却很清晰。

路到了这就就是尽头了,河流拦住了他们的去路,不能再往前走了。

“休息一下吧。”陆久说着,解下了身上的背包,放在了地上。

“好。”V说着,也解下了背包。差不多该到了用餐的时间了。

一边这样想着,V一边从背包里取出了压缩饼干,但是却被陆久制止了。

“这次我来吧。”他说。

V有些纳闷,她不知道陆久说的“他来”是什么意思。所有口粮都在V身上,陆久之前带着的一背包啤酒昨晚已经全部喝掉了。他要怎么来呢。

但V看到陆久抽出了腰后面的战术匕首,那是他从军区离开时所携带的唯一的武器。

这个动作让V心头一凛。她这才意识到自己竟然疏忽到了如此地步,才短短几天就忘记了陆久的身份。

仅仅几天没有战斗,她的眼中就连武器,都变得陌生了。V立刻绷紧了神经。

陆久显然也意识到了V的眼神变了,但是他没搭理她。

“去捡点干树枝。”陆久对V说道。

树林里干树枝要多少有多少,没过一会儿V就捡来一大堆。她把树枝装在陆久的背包里,稀里哗啦地倒在了地上。然后她注意到陆久似乎正在地上专心地寻找着什么。

她走过去,看见陆久的身边放着他们吃饭的饭盒,而饭盒里边……

有好几条棕红色的虫子,正在不断扭动着。

这些东西看得V起了一身鸡皮疙瘩。

“这是……”

虽然她已经习惯了顺从陆久,但还是忍不住开口问道。

“蚯蚓。”陆久头也不抬地说道,他似乎正在地上挖的就是这些虫子。

不会吧,V充满担忧地想着。她不会吃这种东西的,绝对不会。就算是命令,她也不会。

陆久转过身,看了一阵满脸焦虑的V,终于忍不住笑了起来。

“……你在想什么呢?”陆久说,“不,还是不要说出来了。不过就算你想吃,我也不会让你吃的。”

“不,我……”V的脸涨得通红,“我不想吃。”

陆久无奈地摇了摇头,似乎对V的见识深感遗憾。

“我是要钓鱼啊。”

“钓鱼?”V不解地问道,继“夫妇”之后,她又听到了一个新的词汇。

“看着。”陆久说着,V注意到他的身边放着一根长长的树枝。

陆久拧开战术匕首的手柄,从里边取出一卷尼龙线、和一根针。那是战场上用于缝合伤口的急救物资。

陆久从兜里掏出他的打火机,打着火,把针在上面烧了一下。然后他把那根烧红的针用衣角包住,轻轻一折。针被折弯了。

接着,陆久把尼龙线的一头缠在针上,另一头绑在了树枝的一端,又在靠近针的一边栓了一小块石子。

“就这样吧。没有竹竿,也只能用这个了。”陆久自言自语地说着,从饭盒里拿出一条虫子挂在针上。然后,他拿起树枝一甩,把勾着虫子的针甩进了河面。

“这是在,做什么?”V忍不住好奇地问道。

“继续看着。”陆久站在河边,头也不回地说道。

V完全不懂陆久是在干什么。但片刻之后,陆久猛然一抽树枝,一条鱼被甩上了岸。V注意到那条鱼是被陆久投下的针勾上来的。

“你怎么做到的?”V惊奇地问道。

“鱼钩藏在鱼饵里,鱼一咬自然就被勾住了。”陆久说道,“不过没想到这里的鱼这么容易上钩。”

V呆呆地盯着那条还在胡乱摆动打挺的鱼,惊讶得说不出话来。她只知道陆久是个战斗经验丰富的老兵,却不知道他还有这样的技能。

“别发呆了,杀了它。”陆久说道。

杀……V感到不知所措。怎么杀?她完全不知道。照着它的头来一枪?可惜她没有带枪。

她这才意识到,自己不会杀鱼。她……只会杀人。

“这也不会?我的大小姐啊。”陆久无奈地说道,拿起战术匕首,用刀背照着鱼头猛敲了一下。鱼立刻就不动了。

“鱼……死了?”V惊讶地问道。

“那还怎么着,难道还能是睡着了?”陆久好笑地答道。

大约一小时的时间,陆久就钓到了六条鱼。其实他钓到的鱼不只这些,不过有几条实在是太小了又被陆久扔回了河里。而一小时的时间里,V渐渐掌握了“处决鱼类”这一技术含量不高的技能。

陆久用战术匕首斩掉了鱼头、剖去内脏,把鱼用树枝穿了起来。然后他又在地上挖了个坑,把V收集的树枝扔在里边生了一堆火。接着,他把穿着鱼的树枝插在了火堆旁边。

“吃过烤蛇肉吗?”陆久忽然没来由地问了一句。

“……没有。”V答道。

“真没吃过吗。”陆久小声嘟囔着,“我本来想说烤鱼比烤蛇好吃呢。”

他说什么,烤蛇肉?V心里莫名其妙地想着。不知为何,V觉得这见都没见过的东西,好像并不陌生。

而且,烤鱼好像确实比烤蛇好吃。

\section*{}

野餐过后,两个人开始沿河岸向着河水流动的方向前进。

“秦市就在坐落海边。虽然有点绕远,但是沿着这条河应该能抵达入海口,我们从那里继续沿着海岸走,应该就能到。”陆久一边走一边说着。

“泅渡到河对岸,不是更快吗。”跟在他身后的V有些不解地说。

“我不喜欢下水,你又不是不知道。”陆久没好气地说道。

“……对不起。”V小心地说道。

听到V道歉的话,陆久站住了。

“对不起。”陆久轻声说,“该道歉的是我。你的确不知道。”

听到陆久的话,V明白了。

陆久不喜欢下水。这件事,自己……本该是知道的吗,V自嘲地想道。

自己是谁、自己算是什么呢。一无是处,就连个替代品都算不上。就连那个“自己”的替代品,都算不上。

“现在知道了。”V在陆久的身后,落寞地笑了笑说。

之后两个人默默地走着,谁都没有再说话。不知道是否是因为太过深入山林,他们走了很长一段时间,直到太阳开始渐渐下山,也没能看到河流的入海口。如果不是流水在指引着方向,V简直要怀疑他们是不是走错了。

“这可不好,难道今晚要在树林里过夜了?”陆久有些焦躁地说着,“早知如此就该走海边那条路。”

他是故意走这边的吗,V这才意识到。难道那时候他看出自己想往这边走了?

但这一切终究还是不得而知,陆久沉默地走着,虽然越走越快,但是当天完全黑下来的时候他们依然在山林之中。

“该死。”陆久骂了一句,“不能再走了,再继续前进不知道会遇到什么。山里的夜晚可不安全。”

“这里有危险的动物?”V警觉地问道。

“非常危险。”陆久阴沉地说道,“北部地区最凶猛的食肉动物就在这片山林里出没,要是遇到它可就完蛋了。”

“没关系,我们加快速度。真有危险的动物出现,我来拖住它,你趁机脱身。”V坚定地说道。

“别开玩笑了,那家伙可不是什么小猫小狗,它在这片森林没有任何天敌。”陆久摇了摇头说,“体重超过三百公斤、跑起来时速却能达到70公里,几秒钟就会把一个人撕成碎片。”

“什么动物这么凶猛?”V惊讶地问道。

“……虎。”

“那我们怎么办?”V的语气里明显开始担心了。

“这次可不比那次在雨林,这里的树又高又直,根本就爬不上去。”陆久低声说道,“最古老的办法,用火光驱散野兽。只能这样了。”

两个人已经不能再胡乱活动,因为对于这片山林里的掠食者来说,这夜晚和白昼是一样的。所幸山林里木材很充足,V和陆久急忙各自在附近收集了一些树枝,然后点起了篝火。为了防止背后遭遇突袭,他们把帐篷搭在了河岸和苟活之间。

“今天晚上睡不成了。我们必须轮流拾柴确保篝火不会熄灭。”陆久说道,“老虎的狩猎时间是黄昏到黎明,天亮它们就会回家睡觉,坚持到太阳出来我们就安全了。”

“好的。”V点了点头说道。

“趁着篝火还旺,赶紧收集一些柴火。随着附近的木柴被烧掉,我们将不得不去越来越远的地方捡树枝,到时候危险会成倍地增加。”

“知道了。”

“拿一根树枝当火把。遇到野兽不要跑——没用的,你跑不过它们。用火把驱赶它们,野兽都是怕人的。”

“好。”

虽然身处危险之中,听着陆久沉着地安排任务,V的心中莫名地涌起一种安心的感觉。在此刻V的眼里,陆久依然是那个可靠的指挥官。

V和陆久一起行动,迅速收集了附近能拿到的所有木柴,然后回到了篝火旁。所幸他们没有遇到任何野兽。

做好收集工作后,两个人围着篝火坐了下来,静静地等待着黎明的到来。不过这段时间还是太长了。虽然时间已经接近初夏,但是北方的夜晚至少还会持续七个小时,安安静静地坐着是很难熬的。

而且这几天一直在徒步行走,白天的体力消耗很大。V倒没什么,战术人形的精力要比人类充沛得多,但陆久就不一样了。前半夜刚刚过完,陆久的精神状态就明显有些恍然,他实在是又累又困很难坚持了。

V本想让陆久去休息自己放哨添火,但是她马上意识到这样不行。虽然她的精神还好,但是在这种环境里她没有任何经验去处理突发情况,就连森林里的野兽她都不能辨识,她需要陆久的指挥。

陆久虽然也知道这一点,不过看上去他已经无法拒绝自己正在拼命要求休息的大脑了。

必须做点什么来提提他的精神,V心想。

“陆……司令。”V开口说道。谈话自然是最好的选择,虽然她不知道该说些什么。

“嗯?”听到有人呼唤自己,陆久的精神稍稍集中了起来。

“那个……”V说。

根本没有话题可谈啊。V在心里叫苦。她本来就是不善谈吐的人,更糟糕的是陆久也一样。虽然曾经是他的副官,但是V和陆久一起共事时的一天里,说的话总共也就十几句,现在强行找一个话题去聊实在是太难为人了。

“我能,问一个问题吗。”V想了半天,终于说道。

“说吧。”听到V的话,陆久的精神稍微集中了一些。

“您刚才说道‘不比那次在雨林’。您曾经,在雨林里度过了几个晚上吗?”

“……这个,”陆久有些犹豫地说道,他没想到V会问这种问题,不知道她是什么意思,“啊,算是吧。”

“算是”?这种回答恐怕就连陆久自己都敷衍不过去。不过作为和陆久相处过一阵的人,V明白那就是陆久对“不想承认,但是又不得不承认”的事情的回答。

“那次……也是和我吗?”V说道。

话刚说完,V感觉自己的表达有点不对。于是她又纠正说:“我是说,和那个人……”

“……是和你。”陆久沉默了一阵,开口回答道。

“那不是我吧。”V的声音低了下去。陆久的这个经过思考但相当肯定能够的答复,让她觉得有点不快。她不想听陆久说“那个人”是她。

那个人不是她,V和陆久都知道。所以她不想听陆久这么说。

她不想听陆久为了安慰她,而故意这么说。

“那是你。”陆久加重了语气说道。

“……那不是我。”V焦躁地说道,稍稍将脸转向了一旁。

“那就是你。”陆久平静地说,“是和谁一起的,我比任何人都清楚。”

 “可是我完全不知道那些事情。”

“你只是忘了。”

“我不是忘了!”V稍稍提高了声调说道,“我根本没有经历过那些!”

“……你知道我是怎么判断一个人是谁的吗。”陆久轻声说道,“很多方面。说话的语气、做事的态度……一个人,在很多方面给我的感觉。那就是你——除了关于那段事情的回忆,那时的你,和现在的你没有任何区别。这一点我毫不怀疑。”

陆久曾经说过不想谈起那些已经过去的事情,但是他现在觉得有必要澄清一下。那个人为了他而牺牲了自己,她的身份陆久不容任何人否认、就算是她本人也不行。

“是吗……那么那时候,我们发生了什么?”被陆久坚定的语气所感染,V也不自觉地说出了“我们”这个词,仿佛那时和陆久在一起的真的就是她一样。这样的用词让陆久略感宽慰。

“我们一起穿越了敌人的封锁、趟过一片满是地雷的死亡地带,我们还在河流中央打捞撤离用的快艇。后来,我们遭遇了敌人的空袭,逃进了丛林。在丛林里我感染了疟疾,你四处奔走帮我收集草药,救活了我的小命。”

“后来呢。”

“后来……”陆久说着停了下来。

“后来?”

“后来,我们离开了那里。”

“我呢?”

“你?”

“是的,我后来怎样了?”V低声说道。

“……”

陆久沉默了。后来她……

“后来……你,失去了那些记忆。”片刻之后,陆久轻声说。

“……后来,我死了是吗。”V微微转过头,看着陆久说道。

当V说出这句话的时候,她看到陆久的身体明显地颤抖了一下。

“不,你没有死。你还活着,就在这里。”陆久说。

V再次转过了脸。她的眼睛里流出了泪水,但是陆久没有看到。她的眼泪流淌得很平静,陆久甚至没有察觉。

她知道自己,或者说那时候的“那个她”,死去了——她亲手结束了自己的生命。她一开始不知道到底为什么,但是想起陆久最后对她说的那段话,她现在明白了。

“……因为他们,生而为人。”

是的。虽然没有了那几天的完整记忆,但是最后的那段话,她一直都记得。当她听到陆久说她拥有“天赋的人格”的时,那一刹那的感动依然铭记在她心中。

所以无论她到底是不是当时的那个人,她都感到欣慰。她为了这个自己或者那个自己那时所做的一切,而感到欣慰。

也许那真就是自己,V心想。因为如果换作是她,她也会那么做。她将会无数次地、毫不犹豫地那么做。

那时那个V,在你心里是什么呢。眼前的这个V,在你心里又是什么呢。她忽然很想问这些。

但她终究没有去问,因为她知道,无论是什么,自己在陆久的心中,都不会是在她所希望的那个位置。V知道陆久昨天因为冲动而抱住她的时候,他心里想的,却是另一个他永远都无法再拥抱的人。所以她什么都没有去问。

“……是吗。”

她只是这样说道。

\section*{}

两个人就这样有一搭没一搭地聊着过去的事情,勉强维持着清醒。在黎明快要到来的时候,陆久终于因为疲劳而保持着坐姿睡着了。V虽然一直在默默地看着陆久,但终究没有忍心去唤醒他。

当陆久忽然惊醒的时候,阳光已经从树叶的缝隙之中洒下。陆久在附近发现了一些粪便,是这片森林的主人留下的,显然它昨天晚上曾经来过这里巡视领地。但是很幸运,它没有把它的客人当做晚餐。

两个人再次踏上了徒步旅行的旅途,午后刚过一点的时候,他们来到了河口。河口的水流十分平缓,而且很浅,陆久和V卷起裤管就蹚过了河。越过河口之后,两个人用压缩饼干填了填肚子,然后开始继续前进。

河口的另一边依然是沙滩。沙滩上的沙子很细,为了晾干被河水打湿的裤腿,两个人索性就赤着脚在海岸上走了起来。

“这是……什么?”

走着走着,V忽然停了下来。她在脚下的泥沙里发现了一些东西。

陆久转过身,看到V正好奇地低头看着脚下的一个灰白色的物件。那是一个小小的海螺,不过眼前这个是一个已经死去很久的空壳,海螺的表面都已经被海水腐蚀了。

“是海螺。”陆久说着,从V的脚下把那个贝壳刨了出来。

“海螺,是什么?”V不解地问道。

“海里的一种软体动物,这是它的壳。”

“它……已经死了吗?”

“死了很久了。”

“死了啊。”V的语气有些失望。陆久不明白V在失望什么,在他看来这些海里的又腥又咸的东西简直让他讨厌之极,但是V似乎对它们很感兴趣。

“据说把海螺放在耳边的话,能够听到海的声音。”为了安慰面色怅然的V,陆久说了一个成年人常说的故事。他把海螺甩了甩,然后在衣襟上擦干递给V。

“真的吗。”V将信将疑地接过海螺,放在耳边。

“真的呢,我听到呼呼的声音,就像海风一样……”V的脸上露出了吃惊的神色,还有一丝兴奋。

那不过是体内血液流动的声音罢了,陆久心想,但是他没有把这个故事说破。

一路上V都把那个海螺拿在手中,时不时地放在耳边听一听。虽然她依旧寡言少语,但是脸上那微微兴奋的表情是藏不住的,看得出她的心情很好。V对这个世界了解得太少了,从某些意义上讲她就像是一张白纸,甚至比小孩子还容易哄。

两个人徒步走了一天,一路都是和前些天相同的景色。但是当天色渐晚的时候,他们已经能够看到海岸线远处闪烁的光亮,那是城市灯火的光芒。

“明天这个时候,我们就该到达那个地方了吧。”看着海岸线的远方,陆久说道。V没有出声。

那又如何呢,V心想。那个地方、这个地方,对她来说没有区别。此刻的她只希望能在这个人身边就好。

V不知道自己这是怎么了,自从95牺牲后她就开始怀着这样强烈的念头,甚至就连公司的命令都被抛之脑后。她不知为何自己会开始如此地在意一个人。虽然她很想要看一看那些山和海的另一边的风景,但是如果能和这个人在一起,那么她就算哪都不去、就留在这片沙滩上也不要紧。

她的这种愿望已经远远不是“任务”范畴之内了,她想不明白自己到底是哪里出了问题。她不知道她只是得了一种病、一种这个年纪很容易患上的病,一种叫做“恋情”的病。

她没有出任何问题,只不过是喜欢上了一个男人。

只不过是一厢情愿地喜欢上了一个,大概永远都不可能和她在一起的男人。

当然,V也不懂什么是喜欢、什么又是爱,对她来说只要能看到陆久就足够了。关于未来,她是毫无概念的。一个没有情感经历的人形,心里泛起的,不过是懵懂的悸动。

所以V更不会知道,这些事对于陆久则是不同的。陆久经历过很多事情,也明白很多事情。他知道自己就是这枪林弹雨的命运、朝生暮死的人,在这乱世之下,他没有办法保证自己能安稳地看到明天,所以感情这种东西对他来说不是必须的。他也会为了得到和失去而伤感,但是在那之前,他有太多事情需要去面对。

他曾说他很后悔没能回应95的感情,但那更多的是追挽之词。如果他真的要回应的话,最后大概还是会婉转地拒绝吧。他知道自己无法对感情这东西负起责任。如果要爱一个人,他就要同样地爱惜自己,因为只有活着的人才能去好好地去爱别人。但是对于陆久这种终日徘徊在生死边缘的人来说,这一点是偏偏是他做不到的。

所以,他大概不会接受任何人,因为那种真切的责任,是他无法承担的。

春天的秦市天气通常很好,这一天的夜晚也一如既往地晴朗。当月亮升起到半空中的时候,陆久再次独自坐在帐篷旁边开始默默遥望远方,不过这一次,他没有酒可喝了。

V同样睡不着,于是也走出了帐篷,静静地站在陆久身边。

皎洁的月光之下,沉默的男人背后站着恬静的少女,他们的面前是正在微微起伏着的大海。这个夜里没有风。

这本该是一幅美丽而浪漫的画面,如果这两个人中间的任何一个,身上存在着关于“浪漫”的细胞的话。

“您是……在想她吗。”很长时间的沉默之后,V终于忍不住开口了。陆久显然知道V是在指谁。

“没有。”陆久简单地答道。

“那在想什么?”

“什么都不想。”

“怎么可能。”

“是真的。”

“可您看起来有些……情绪低落。”

“那是因为你看的是我的后背。如果你从前面看我的话,你会发现我和以前是一样的。”

这话V无法反驳,她的确是在从后面注视陆久。她说陆久情绪低落,只不过是她自己觉得陆久情绪低落罢了。陆久的样子,和他们在N17军区指挥部初次见面的时候,的确没有任何的改变。

是啊,V心想,自己凭什么说他失落呢。他一直都是这幅样子,不会愤怒也不会喜悦、但随时都能应付突然发生的一切。而且无论发生了什么,他都不会改变。

不过这种情况,也有一次例外。那是在那次战斗后,陆久回到N17战区的第一个夜晚。

那天晚上,本应该被“隔离”V离开了她的隔离室——她可以在一定范围内自由活动,而陆久也在那片被隔离的区域里。

那天晚上,当时间已过午夜的时候,V从窗户里看到陆久的单人病房里还亮着灯。于是她走了出去。

她走进了陆久的病房,看到赤裸着上身、身上缠着多处绷带和固定板的陆久正坐在病床上发呆。

“你不该来这里。”陆久瞥了一眼那位不速之客说道,“你正在被隔离,而我想自己呆着。”

“您的情绪看起来很不好。”V说,“如果自己呆着有可能给您带来危险的话,那我就有必要多加关注了。”

“请你离开。”陆久说,“我的情绪的确不好,不好到我现在就想毁掉这里的一切。所以请你离开吧,让我自己静一静。”

“那就毁掉这里的一切吧,如果那样你就能感觉好一些。”V不为所动地说,“我不会阻止你,但也不会离开。”

陆久沉默了。一阵子之后,他勉强地笑了笑。

“……算了吧,你说得对。”他说,“就算毁灭全世界又如何,已去之人不会再回来。我是这个战区的总指挥官,我得平静地面对这一切。明天战斗还会继续,就算激烈地宣泄自己内心的不满有又何帮助呢。不能改变任何事情。理智……是啊,我还没有失去理智。”

V默默地看了陆久一阵,然后走到了他的身边。

“请不要,再埋藏自己的心情了。”V说道。

“什么?”

“请不要,再假装若无其事了。”

“我没有假装。情况显而易见,我不能在这个时候……”

“请不要,再这样了。”V打断了陆久的话,轻轻将陆久的头揽在自己的胸前,“我知道您的心情。虽然无法说感同身受……但我知道您一定非常痛苦。”

“放手。”陆久的语气有些慌乱,“你在胡说些什么……”

“让我来为您分担吧,请不要再独自承受这一切了。就算心里的痛苦不能告诉任何人,至少可以让我来倾听。难过的话就请说出来,我可以安慰您。就算所有人都来伤害您,有我来保护您。就算所有人都在指责您,有我来原谅您。就算所有人都对您冷眼以待,至少我会一直陪在您的身旁,直到……您不再需要我的那一天。”

“你在说什么蠢话。我是战区的司令,怎么会需要你,区区一个人形的安慰?我是公司首屈一指的指挥官,我,我是……”

陆久的声音颤抖着,并且越来越小。终于,他把额头紧紧靠在了V的胸前。

V一手搂着陆久的脸庞,一手轻轻抚摸着陆久粗硬的头发茬。她感到陆久的肩膀瑟缩着,而且在不停地颤抖。

他是在哭泣吗,V心想。

他也许并没有他假装得那么坚强。他也许也需要什么人去抚慰他的伤痛。他也许,也需要一个人的陪伴。

但是他一定不想让任何人知道这些。

于是V轻轻地扬起头,让自己的目光离开了怀里的男人。

但是……这是什么。

V伸手轻轻抚摸了一下自己的脸。潮湿。某种液体正在从自己的眼眶中涌出。

她用手指沾了一点脸上的液体,涂在嘴唇上——咸的。这是……泪水吗。

V知道,人类在极端快乐或悲伤的时候会流泪。可为什么自己也会流泪呢。自己既不快乐、也不悲伤,而且自己根本就不是人类。人类的七情六欲,在自己身上本来是并不存在的。

忽然,V的胸口感到一丝异样的感觉,很疼。

很奇怪,这里没有受伤。可为什么,这里会感到疼呢?V也不明白。

她只能轻轻地搂着这个受伤的男人,一直到他彻底平静下来。

“谢谢。”良久,陆久终于低着头轻声说道,“我已经感觉好多了。”

于是V缓缓松开了她的臂膀,然后默默地转过身,走出了陆久的房间。

那可能是陆久第一次也是唯一的一次流露出自己的感情。此后,这件事他们谁都没有再提起过。

这样回想起来,V和陆久之间其实发生过很多事情。但不知为何,到最后却都会变得仿佛什么都没发生过。两个人似乎有时也有过亲密的关系,但只要一夜过去,他们的关系就会再次回到原点。

就像这些天一样、就像每一天一样,就像现在一样。

\section*{}

“你还不睡吗,已经是凌晨了。”陆久说道。

“你还不睡吗?”V反问。

“我是想睡了,但是‘赶紧睡吧’这样的话说出口来有点太……让人尴尬了,所以我在征求你的意见。”

“那就赶紧睡吧。”

陆久摊了摊手。他不是这个意思——他绝对不是在等女孩子先对他说出这样的话,虽然这话也没有任何其他的意义。但V根本就不食人间烟火,她不会考虑任何话里的弦外之音。

而且两个异性呆在同一个帐篷里,本身就是一件让人尴尬的事吧。

自己在纠结什么啊,她只是个人形而已。陆久有些无奈地想着,倒头钻进了帐篷。V也躺在了他的身旁。

“请问您那天……”忽然,V在陆久的背后轻声说道。

“什么?”

“前天晚上。”

“……怎么了?”

“您……喝了很多酒。之后,并没有睡着吧。”

“我睡着了。”

“您没有。您走出了帐篷,并且……“

“咳……”

陆久咳了一声试图打断V的发言,但是V并没有停下。

“您还拥抱了我。”

“是吗,”陆久难堪地说道,“我不记得了。”

“是的。”V肯定地说。

“对不起,那天我喝醉了。如有冒犯……”

“其实,您的确是在想她吧。”

“不,我……”

“然后把我当成她了吗。”

“……”

“我想您一定很难过。”

“……”

“需要我,为您做些什么吗?”

“……什么?”

“您知道的。女人的身体是对受伤男人的最好的安慰。”

“别说这种话。”听到这句话,陆久有些忍不住了。虽然有些愠怒,他还是压抑了自己的情绪,“你该爱惜自己的身体,那是维持你存在的根本。不要总是把自己当成一件任人使用的工具。”

他很反感V这样说,V早就该清楚。这话不仅毫不自重,而且简直就是对一个男人的挑衅。

他们见面的第一天,陆久就曾为这种事大为光火。

其实要说V一点也没有吸取教训,不如说她就是在故意找茬——一再地被忽视,总是让人开心不起来的。不过就连V自己也没有察觉到这一点罢了。

“合理地使用,我认为不算是不爱惜。”

“别再说了!”陆久低声喝道,语调里已经饱含怒气。

“难道我,真的就什么都做不了吗。”V轻声地说道,声音里充满了失落。

“不,这种事……”陆久一时语塞,不知道该如何回答,他心头的火气也瞬间消散了。

就算V说的是真的,对这种事她又能做什么呢。但是他又不忍对V说“没有什么你能做的”,V所做的一切都是为了他,这一点陆久是明白的。

“你所做的我都非常感激,无论是过去还是现在。”陆久低声说道,“但有些事情不是因为我们做了什么、或者没做什么就会改变,那不是任何人的过错。所以请不要再那样说了。”

V没有说话。陆久不知她是不是生气了,也不知她在想什么。

“好。”一阵之后,V终于轻声说道,“那么,能问个问题吗。”

又是一个问题,陆久有些凛然。V从来不问茶余饭后的闲事,这点陆久早就有所觉悟,不过他想了想觉得没什么大不了的。

他的确有很多不想说的事情,但是却没什么是不能说的。他说过不谈以前的事情,但是昨天不也对着V和盘托出了吗。这里只有他们两个,V想问什么陆久都可以告诉她。

V曾经不止一次救过他的命。对于陆久来说,他唯一不会怀疑的,就是V的忠诚。

“……可以,问吧。”陆久说。

“您,为什么要离开战区呢。”

陆久笑了笑。她终于还是问起这些了啊。

“累了、烦了。我也不知道,总之不想再干了。”

“您曾经说过,您亲自上阵是为了能把自己的士兵都带回家。”

“……只是,曾经。”

“已经不想保护她们了吗?”

“我保护不了她们。我做不到。”

“如果连您都做不到,那还有谁能做到呢?”

谁都做不到,陆久心想。但是这要怎么说呢。

只能实话实说了吧。

“……玩物、炮灰,战术人形在整个社会都是这样的地位,凭我一己之力是无法改变的。我想所有人都知道这一点,包括你在内。你们积极地响应我,只不过是逢场作戏,不想让我失望,其实我明白。对于你们的好意我很感谢,但我现在觉得没必要继续表演下去了。既然改变不了什么,继续下去也没有意义。”

“我认为有一点您说错了,因为无论他人如何,我没有这样想过。在我看来,您的所有决定都是发自内心的,而且您也用尽全力去践行了自己的诺言。您所说的一切……至少我,从来不曾怀疑。”

“现实就是这样,就算你相信,但还是于事无补。相信一件注定失败的事情,只能说你信错了。”

“我相信的并不是某件事,而是您做出的选择。所以就算现在您选择放弃之前的信念,我依然会相信并且追随您。”

“明知是错,你也依然选择去相信吗?为什么要这样做呢,这不符合逻辑吧。”

“因为我是个战术人形——您是陆久司令,而我是您的副官。我会无条件地相您,对我来说就是这样。这是我的准则,高于客观上的逻辑。”

“……是这样吗。”陆久微微叹了口气。

“是的。”

“你的准则我无意置评。不过离开了战区,我已经不是指挥官了,所以不要再叫我司令了。”

“那我呢。如果你不是陆司令,那我也就不是您的副官了。我又是什么呢。”

“你是我的……”陆久思考了一阵,“你是我的,伙伴。”

“‘伙伴’?那是什么?”V有些疑惑地问道。这个词和“夫妇”、“钓鱼”一样,同样是她没听过的。

“关系亲密的人和人之间都可以称之为‘伙伴’。”陆久说,“和‘指挥官’与‘副官’不同,伙伴之间是平等的,没有上下级的区别。”

“可是我是个人形。和一个人类平等,让我感到自己的身份越级了。”

“我们初次相识就是在战场。那时候,我还不是指挥官,只是一个普通士兵,而你也是。那时我们的关系是‘战友’。”陆久说道,“战友之间是平等的。如同两个战友都解除了兵役,那么称之为伙伴不足为过,不算是越级。况且你对我而言,一直都是如此。”

“……那么,我该如何称呼您呢。”

“直接称呼名字就好了。”陆久想了想说,“虽然那也许不是我的真名……但是,你不也这样叫过吗。叫我‘陆久’就好。”

\section*{}

“这里就是秦市了吗。”V稍微有些怀疑地问道。

这里没有她想象中的那样繁华,城镇规模和N17战区上的小镇相差无几。唯一不同的就是这座小镇建立在海岸边上,有着很多渔船和码头。

“不,这里不是秦市。这里是……”陆久思考了一阵,“我也说不上它的名字。这里是秦市南部的郊县,地图上这里应该是B-13海滩。”

“听起来像是什么军用地图上用的……代号。”这种描述让V有些纳闷。

“你大概没猜错。”陆久耸了耸肩。

两个人来到北镇是时候已经是下午,不过根据昨天晚上V目测的距离来推算,他们的速度算是比汽车还要快了。所以这里不是秦市——通过计算V也能得出这样的结论,因为他们的行进速度是和往常一样的。

但一直到目睹了城镇之后,V才说出了心中的疑惑。

“这里就是我们……”V说,“这里就是您的目的地吗。”

“是啊。”陆久答道,“这里就是我们的目的地。”

“这里是,您的故乡吗。”

听到这个问题,陆久愣住了。

“你为什么会这样想呢。”稍微沉默了一阵后,陆久说道。

“没什么,只是猜测。人都会想要去到他熟悉的地方吧。”

是吗,陆久心想。也不一定。世上纵使有许多无辜女人的眼泪,但依然无法阻止一些人逃离家园。

不过,此刻V的推断逻辑上正确。

“嗯……你说得没错。”陆久思考了片刻,“这里的确是我熟悉的地方,但是我想不是我的故乡。虽然我的记忆被干预过,想不起任何关于自己故乡的印象,但是我能确定不是这里。”

“那这里是什么地方呢。”

“我想是我曾经……战斗过的地方吧。”

战斗过的地方,V心想。想来陆久的记忆中,大概也不剩什么温馨的回忆了,他能记起的大多数只是战斗。

……和自己一样呢。难怪陆久会说他和自己很像,从某种意义上说他们的确有共同之处,那就是都是为了战斗而活着。

但是V现在觉得自己活着的意义,已经不只是战斗了。不知道陆久是怎么想的呢。

“那您来到这里,是要寻找,熟悉的人吗。”

听到这个问题,陆久笑了笑。

熟悉的人,自己哪里还有什么“熟悉的人”。自己被囚禁了多久,就连他自己都不知道——虽然在陆久印象中他的牢狱生活只有两三年,但是他知道自己的记忆不可靠。根据眼前的景象来看,他脱离这个世界已经远远不止两三年了。

“我也不知道。”陆久说,“就像你说的,只是觉得这里感觉很熟悉,所以就来到了这里。至于目标,很遗憾,完全没有。”

只是没有目的地前进吗,V心想,不过这样也不错。一路上都是没有见过的景色,虽然是出逃,却像是一场旅行一样。

唯一的不足之处,就是不知道哪天就会被公司发现和制裁吧。不过没关系。要是陆久不在乎,那V就更不在乎了。

两个人沿着街道一边慢慢走着,一边环顾着四周的景色。这里是一个很普通的小镇,有着整齐的街道和一排一排的住房。虽然房屋很多,但是街上人却不多,让人感到有些冷清。

“我记忆中,这里曾经是个避暑度假的圣地,在盛夏到来前街上都有些萧条。看来如今依然是这样啊。”陆久说。

V似懂非懂地点了点头。“度假”,她心想,又是一个新词。是指在没有工作的时候的放松和休息吗?她不知道。人形是没有假期的,除了养护时间就是待命。

“那我们,是要去哪里呢。”V轻声问道,陆久摇了摇头。他只是跟着感觉来到了这里,但是具体要去的地方他真的没有。

两个人漫无目的地游荡着,经过了小镇的市场。这里是镇上生活必需品的集散地,不过集市上除了海产、蔬菜和日常用品外,也没有什么稀奇的特产。陆久和V漫步其中,他们引起了人们好奇的目光——不过人们主要是在看V。陆久和这里的人们外貌看上去完全是一样的,而V这种外国血统的人形则要受人关注多了。

不过在这个地域偏远的小镇上,人形的使用并不普及,没有人意识到V和其他人的不同,只是把她当成了一个远道而来的异乡客。

“人们都在看你呢。”陆久对V低声说道。

“啊,我有什么不自然的地方吗。”V紧张地问道。

不自然倒没有,只是V的外表太引人注目了:高挑的身材比例几近完美、肤色白皙如雪,米色的头发稍稍长长了,有几缕已经垂到了肩膀。这样一位美丽的女孩,想不被人注意都难。

陆久微微皱了皱眉。毕竟他们正在出逃,引人注目是他要极力避免的。

“过来。”他低声说道。V不明所以地朝陆久靠近了一点。

陆久却伸手揽住了V的腰,一把把她拉到了自己跟前。

“陆,你……”V大惊失色地说道,“你这是……”

“嘘,安静。”陆久轻轻说,“别紧张,继续慢慢往前走。”

“……好。”

V稍稍安定了下来,依偎在陆久身边慢慢走着,在旁人看来他们就像一对关系融洽的情侣一样。看见V已经有了伴侣,人们渐渐失去了对V的兴趣,目光也纷纷转移到了其他的地方。

“他们不看我了吗。”V小声问道。

“基本不看了。”陆久说。

“为什么?”

和人类的求偶本能有关……这种事要怎么才能和你说明白。陆久心想。

“没什么。”他说。

两个人慢慢走着,缓缓穿过了市场,来到了环绕小镇的沿海公路上。正当陆久心里如释重负的时候,他们忽然被一个人拦住了去路。那是一位面容慈祥的老妇人,正在举着木质的货架沿路售卖贝壳制作的工艺品。

“您好,先生。要买一串珍珠或贝壳送给您美丽的妻子吗?”那个老妇人带着微笑说道。

她把自己当做游客了吧,陆久心想。不,自己此刻扮演的确就是游客。

看来自己的演技过关了。这些沿街售卖的人眼光通常相当锐利,是不是本地人、兜里有没有钱一眼就能看出来,能骗过这样的街头小贩并不容易。

不过,这样的话……要说不买也不行了吧。真是高明的推销,陆久暗想。这个时候男人通常已经无法拒绝了,而女人对这些饰品是没有抵抗力的。一句话出口,生意就做成了一多半,不愧是位老手。

“怎么样,喜欢哪一件?”陆久在脸上挂出微笑着对V说道。

“嗯,我……看一看。”V支支吾吾地说着。虽然她不擅逢场作戏,但也明白了陆久的意图,于是她装作感兴趣地翻看着那些吊坠和项链。

那些造型各异的贝壳项链,确实让V很有些好奇。陆久给她的那个风化的贝壳她还留着,不过这些精心制作的饰品比那个贝壳精美多了。V从那些饰品里发现了一个海螺制作的吊坠,那是把一个海螺从中间切割开、再打磨抛光制成的,海螺玲珑而复杂的内部花纹一览无余,形状非常精致。

V把那件吊饰拿在手里,仔细地看着,明显已经爱不释手。

“就要这个吧。”陆久一边说着一边把一张钞票递给那位老妇。想不到她会喜欢形状复杂的饰品啊,陆久心想。

老妇人笑嘻嘻地找出了零钱,然后从货架上取下了那个吊饰,但却没有交给V,而是递到了陆久手里。陆久楞了一下才明白她的意思。

“给你。”陆久微笑着走到V的跟前,把那个吊坠挂在了她的脖子上,并为她整理了一下被风吹得有些凌乱的头发。

“谢谢。”V红着脸小声说道。虽然她知道陆久的行为只是在即兴表演,但还是因为这样亲昵的动作羞得微微低下了头。

“真是位可爱的女孩。”看到V的样子,就连卖饰品的老妇都忍俊不禁了,“祝你们永结同心。”

“谢谢您。”陆久笑着说道,挥挥手告别了那个妇人。

两个人沿着马路继续走着,陆久忽然感到气氛有些奇怪,因为他们似乎陷入了微妙的沉默之中——虽然已经走出很远了,V依然保持着微微低着头的姿态。看到V略带羞涩的表情,陆久感到有些好笑。

毕竟是个不更事的孩子,陆久心想。也不知她昨晚说什么“合理使用”时的气势到哪去了。

“那个人已经走了。”陆久对V说道。V这才恍然回过了神。

她停下了脚步,然后轻轻从脖子上摘下了那个挂件,递到了陆久面前。

“怎么,不喜欢这个?”陆久奇怪地问道。

V没有说话,只是轻轻地摇了摇头。

“不……我想还是不必了。”

陆久心里有些困惑。V明显很喜欢这件挂饰,但为何又不肯收下呢。而且就这样把它丢掉也未免有些可惜了。

陆久看了V一阵之后,轻轻把她伸过来的手推了回去。

“虽然不是值钱的东西,至少挺漂亮,不是吗。”他说,“既然是你挑的,就拿着吧。”

V犹豫了一下,然后收下了陆久的礼物。不过这次她没有把那个饰品挂在脖子上,而是塞进了口袋里。

“……陆久。”V轻声说道。

“嗯?”

“我也有件东西……”

V说着,从腿上的包里取出一件黑色的东西。

V腿上的小包,陆久不是没有留意过,那似乎是V在公司领到的制服的配件。因为V的那套洋装除了连衣裙的侧面有两个口袋之外,身上没有什么能装东西的地方,所以她经常在腿上挂着一个小包来盛放杂物。陆久知道那个包里装着些物件,但他也没想过里边是什么。但今天V从里边取出东西来,陆久这才注意到。

他接过V递上的东西看了一眼,发现是一幅手套,一幅皮革缝制的战术手套。

“这是?”

“很久以前……”V说,“在战区的时候,从镇上买的。不过一直也没有机会给你……”

陆久和V几乎每天都会见面,“没有机会”这样的说法显然是一种托辞,大概是指“懒得给你”或者“不想给你”之类的。不过陆久注意的不是这些细节,而是V提起的另外的事情。

在战区的时候,从镇上……

陆久仔细想了想,他对这件事有点印象。那次自己去参加公司的会议,V私自外出很晚才回来,自己还曾为此发火。但他没想到V居然还给自己带了礼物。

陆久把手套戴在手上,活动了一下手指。这幅手套不仅大小正合适、完全贴合了他的手型,而且触感十分细腻,戴在手上几乎和裸手无异。陆久又把手套脱了下来,放在面前仔细端详着。

不仅材质很好而且做工优良,掌心和手背关节易磨损的地方都缝有双层的皮革加固,整个手套还布满了透气的小孔。这绝非流水线上的产品,而是手工制作的高级皮具。

不过,她怎么会想起送自己手套呢?陆久感到有些奇怪。他知道V几乎一点都不懂世故,出门还为别人带件礼物这种事情,恐怕她是想不到的。大概是听从了什么人的建议吧。

陆久忽然想到,那天V并不是独自外出的,和她同行的人还有一个。

“谢谢。”陆久说,“非常精致,我很喜欢。”

说着,他将手套轻轻对折了一下,塞进了自己胸前的口袋。然后,他对着V表示感谢地笑了笑。

但是V却没有说话。因为陆久虽然在笑着,但他眼里的那一丝闪躲却没能逃过V的眼睛。

V知道,他一定是又想起了某个人。

那时候,V本想等陆久回到军营就把手套交给他的,但是因为陆久提前回营后发现她开小差了于是大发雷霆,所以那时就没有拿出来。之后这件东西就一直放在V的包里。

她并不是忘记了,而是不知道该如何开口,也不知道陆久会作何反应。他会接受吗、会喜欢吗,这些问题让她反复踟蹰,最后终于不了了之。

今天,她在接受陆久的礼物时忽然想起了这幅手套,于是就拿了出来,但是想不到还是选错了时机。

或许,自己永远都不该拿出这件东西来。这样就不会让他想起那些伤怀的往事了,V心想。她想要为自己的冒失道歉,但是又想起自己昨晚已经答应过陆久不再说那些事情了,于是只好有点不知所措地沉默着。

但是陆久并没有沉浸在回忆中,他再次开始前进,于是V也跟了上去。

陆久走得很快,仿佛已经明确了要去哪里。他沿着公路走了一段,然后停了下来——水泥铺就的马路到这里就没有了。

这里大概就是小镇的边缘,再往前没有道路也没有房屋,只有一望无际的大海。马路尽头的一侧有一片长长的房屋,风格古旧而且看起来有些破落。

这些房屋每一座都有三层,而且都带着院子。相比小镇里多数只有一层或两层的民房,它们显得更加豪华和高大。但是这些房屋多数似乎已经人去楼空很久,窗户上的玻璃很多都残缺不全,有些甚至房顶都塌了。海风吹过这些破败的建筑时,发出呜呜的响声,仿佛有许多人在同时呜咽和悲鸣。但也有几座建筑虽然破旧,但是门窗依然完整,依然有人在居住的迹象。

陆久走下马路,在海滩上踱着步子。他低着走慢慢走着,仿佛在寻找着什么。片刻之后,他停住了。

V走到了陆久跟前,她看见陆久正在注视着他的脚下——那里有一截从地面之下刺出的铁竿。

那大概是一个警示牌或者标竿残存的一段,但是上面的部分早就不见了,仅存的下部也锈蚀不堪,不仔细看根本就不会注意到。

陆久蹲下身,用手在那段铁竿下面刨起了沙子。在他往下挖了三四十厘米之后,终于挖到了铁竿的基座,一块方形的水泥块。

陆久拂去水泥块上的沙子,只见水泥上隐约用阴文刻着几个字:

“B13”

V俯下身仔细看了一眼那几个字符,感到有些熟悉。B13?

她想起不久前陆久曾经说过这里是B-13海滩,这个残存的遗迹应该就是海滩上的标识了。这么说,陆久果然曾经来过这里。

他说他没有目的,看来不完全是真的,V心想。他一直都在朝着这个方向前进,但也许只是刚刚才下定了要来这个地方的决心。

陆久从铁竿出发沿着海滩走了几步,然后又折回去向另一个放行走了几步。最后,他抬起头,看向了那一片残破的旧房。

V知道陆久是在定位。他本应该用绳子绕着他确定的点来画出圆形的交点的,但是因为他要找的目标比较大,所以不需要特别精确。

果然,陆久在估算了一阵之后,举步朝着那片旧房走去。他来到了其中一座房子的门前,这座房子前院的铁门锈迹斑斑,已经关都关不严了。门没有锁,陆久推开了一条缝走了进去。

笃、笃笃。陆久轻轻敲了敲老房子的木门。那扇门同样也已经年久失修,而且很久都没有漆过了,腐朽的木头已经发白。

“谁呀?”门里传来一个老迈的声音,听起来像是一个女人。

“我是一个外乡的旅行者,想向您……打听一个地方。”陆久说。

“你沿着马路走,那边有个市场,去那里打听吧。”里边的人缓缓说道,“这里已经是镇子的尽头了,没有地方可去啦。”

“我就是从那边过来的,但那里的人都不知道。他们说这里住着小镇最早的居民,让我来这里问问。”

“这里除了几座旧房子,什么都没有。”里边的人说,“你要找的,是什么地方呀?”

“‘雷区’。”

“……”

里边的人没有说话。陆久等了好一阵,房间的门被打开了,一个上年纪的女人走了出来。

那个女人大约已经年过花甲,也许是因为长久缺乏运动的原因,她的身体看起来佝偻而虚弱,显得十分苍老。

“这里没有什么‘雷区’,你找错了吧。”她颤颤巍巍地说着,“我一辈子都住在这里,从来没听过你说的这个地方。”

陆久没有说话,只是默默地看了那个年迈的妇人。过了一会儿,他微微点了点头。

“也许吧。”陆久说,“我也很久都没有来过这里了,可能是我记错了。”

说着,陆久转身朝着院子外面走去。但当他走到庭院的门前的时候,他听到背后的老妇开口说话了。

“这里的确曾经有一片雷区。但是你怎么会 ‘记得’的呢?”她说。

陆久停下了脚步。

“难道说,真的是你吗……士官长先生?”

\section*{}

“你能够认出我来吗。”陆久问。

“当然啦,你的样子,和那时几乎没有变化嘛。”老妇慢悠悠地说着,“但是我也因此不敢认啊,因为都已经过去四十年啦。我连你的名字都早就忘记啦,只记得他们一直叫你‘士官长’。”

四十年。陆久心里一颤。

这是他第一次知道关于自己过去的线索。他勉强能记得自己曾经在这里执行任务,以及任务的一些模糊的细节,但是他没有想到已经是那么久之前的事情了。

他本以为,自己受到干预的记忆只是发生在十年以内的事情。

“没关系,因为我也记不起你名字了。”为了掩饰自己的震惊,陆久努力笑了笑。

老妇人在摇椅上慢慢地摇着,和陆久随意地聊着年代久远的往事,而V则静静地坐在陆久身边默默倾听。当这位老人对陆久说这里没有什么雷区的时候,陆久真的以为自己的记忆已经完全错乱了。但过了一阵之后,他还是被这位老人请进了屋里。

“但是我不可能忘记你,毕竟是你手下的孩子们,从雷区里救了我的女儿。”老人缓缓说道,“那时还有个孩子牺牲了吧,真可惜。后来他们都去哪啦?”

“他们……”陆久说道,“我也不知道。应该都回去了吧,我们的任务结束后,国家把他们都释放了。”

陆久撒了谎,因为国家并没有释放他们。释放他们的是陆久,这就是他“抗命”罪行中的一条。但是陆久觉得没有必要说出这些。

毕竟,就连他自己都想不起来的事情,都已经实在太多了。

“是你把他们放了吧?”老人显然知道陆久说的不是事实。

“……你怎么知道的?”

“你的事情,我从你的那位军官朋友那里听说了一点。”老人说,“我一开始还在一直打听你的消息,后来就打听不到了,他们说你被抓起来啦。”

“军官朋友”吗,陆久心想。他的这位军官朋友,到底是谁呢,他毫无印象。陆久记得自己在曾经的战斗生涯中有一位长官,陆久一直把他称作“中士”,难道就是他吗?但陆久完全想不起来那个人到底是谁。

“是的,我……那时的我触犯了军法,但是好像所犯的罪行并不止这一条。”陆久点了点头,“之后我被清洗了记忆,所以你说的那些事情我虽有印象,但是已经记不起细节了。但是这个地方我倒记得很清楚。”

“这么说,你是被强制休眠、又被再社会化改造了呀。”老人说,“难怪你依然这么年轻啊。”

“……基本就是这样了。”

“不过你终于被放出来了,是洗脱罪名了吧?”老人问道,“我就知道会有这么一天的,你肯定不是那种为非作歹的人啊。虽然已经物是人非,但是沉冤昭雪终究是好事呀。”

“不……”陆久笑了笑,“不是沉冤昭雪,应该说是戴罪立功了吧。”

“……是吗。你真的犯了罪呀。”老人说,“不过,既然已经还清罪债,那以前的事情也就不提了吧。你来这里是干嘛来啦?”

“没什么。我在徒步旅行,路过这里就想来看看。”

“要在这里呆一阵子吗?”

“也许吧。也许会在这里挣点盘缠,然后再接着走。”

“哈哈,你还真是潇洒啊。”老人笑了起来,“不过也好。人嘛,活着就要做自己想做的事情。这位女孩是谁呀?”

“是我的……”陆久摸了摸下巴,“我的朋友。”

“朋友啊。”老人点了点头,“真好。这么年轻、这么漂亮。她是个外国人吧?”

“啊,是啊。”

“你们结婚了吗?”

“呃,不……”

陆久再次摸了摸下巴。他不知道这位老人是怎么理解“朋友”这个词的,怎么就要“结婚”了?

不过陆久忽然想起,他和V之前一直是假扮配偶的身份一路逃亡的,现在如果否认的话,就和前边所说的不符了。

考虑到也许暂时要在这个小镇呆一阵子,陆久需要继续使用这个身份,所以他决定适当地搪塞一下。

“还没。”陆久敷衍地说着,“不过也许用不了多久……”

“要珍惜眼前人啊。”听到陆久的话,老人谆谆劝导着,“时下局势这么乱,遇到一个两心相悦的人可不容易。”

“是是。”陆久赶紧扯开了话题,“对了,你一个人住在这里吗?你的女儿呢。”

“她呀,”老人的声音变低了,“不在啦。”

“不在……”

“死啦。”

“……是吗。”听到这个消息,就连陆久都感到了一丝震惊,“对不起,我没想到会这样。”

“没事儿,很久以前的事啦。”老人轻轻叹了口气,“她不顾我的反对嫁了个当兵的,结果当兵的去打仗,战死了。她受不了打击终日郁郁寡欢精神恍惚,身体越来越差,没两年也随着去了。”

“没有留下后人吗?”

“倒是有个孩子,我还帮她带过几年。孩子五岁的时候她死了,孩子就被军队接走了。据说现在也当了军官,不过我后来都没见过他啦。”

“没有联系过?”

“联系过,但是联系不了。据说是国家重点培养的人,生活严格保密,不能见面。但是他倒没忘记我,经常给我寄钱。”

“……是这样吗。”

老人的话让陆久听得有些烦乱。他隐约回想起一些当时在这里执行任务的事情,那时他是在这片海滩带队排雷,当时就寄宿在这个老人家里——当然那时她还是个年轻的少妇,带着一个五六岁的小女孩。老人的丈夫是本地人,在战乱中死去了,没想到老人的孩子也受了战争的牵连。

陆久微微转头瞥了一眼身边的V,发现她正听得入神,虽然端正地坐着,但是显然没有注意到陆久在看她。

这些事对她来说,就像是在听故事一样吧,陆久心想。不过,这些事情也确实该叫做“故事”了。

“没什么,我都这把年纪了,早就看开了。”老人笑了笑说,“人的一辈子啊,就是这样。不知道什么时候就完了。所以我才说你潇洒,做自己想做的事,很好。”

“哪有什么潇洒。”陆久也微微笑了笑,“无非是除了打仗什么都不会,现在打仗不想打了,就不知道该干什么了。”

“那就走走停停吧。”老人说,“不过呀,虽然打仗不是什么好事,但总得有人去打。你去了,说不定就有人得救了;你不去,也许好多人就会死。你比别人能打,你不去谁去呢?”

听到这话陆久笑了起来。

这段话倒是挑不出毛病,不过这也太慷他人之慨了。打仗可是掉脑袋的事,能打的人就该死吗?陆久可不这么认为。

“我知道你笑什么,你肯定是笑我站着说话不腰疼。”老人也笑了,“不过呀,这话可不是我说的。这是我那短命的女婿说的。他总是满口‘职责’啊‘大义’的,让我非常讨厌。不过他死了这么多年之后,想起他说的话,好像也有点道理。”

陆久依然咧着嘴,但是他心里已经不笑了。那这倒算一番豪言壮语,他心想,听起来是个军人说的话。

不过这些已经和陆久没有关系了。谁会得救、谁会死去,他再也不想管,也不会去管了。

“这里,有住的地方吗?”陆久忽然问道。

\section*{}

老人一个人住在三层的楼房里,房间自然多得是。但是陆久没有住在她家,而是在市场的另一端找了一个潮湿而阴暗的房子租了下来。

因为他知道自己在做什么,他已经不想再和牵连到任何人。

那天午夜的时候,V靠在沙发上静静地睡着了。而陆久则一直和老人聊到天色发白。

“我走了。”太阳即将升起的时候,陆久站在老人庭院的门前说着,“不用出来送了。”

“我不出去。我这老骨头也吹不得海风啦。”老人站在窗前说。

“我可能呆不了几天就走,到时候就不来和你告别了。”

“没事。我都没想到还能再见着你。已经挺好啦。”

陆久摆了摆手,就这样头也不回地离开了老人的家。

四十年,他心想。这个老人说不定是这世上他曾经认识的人里面,最后一个还活着的了。而这次告别也一定是永诀了。

但陆久也不觉得有什么可牵挂。毕竟除了在这个老人(那时还是个年轻少妇)家里当过几个月房客之外,他们本来也有没更深一层的关系了。

救了老人的女儿的命的不是陆久,而是陆久手下的兵。而那个兵也已经死了。

属于陆久的过去,早就已经烟消云散。这位老人不过是那片残迹般的烟雾中的最后一缕。

陆久的找到的临时居所非常靠近海边。那是一套只有一间居室、一间洗手间和一间厨房的半地下室,而且所有门窗都朝向东方,过了上午阳光几乎就照不进来了,因此房租相当低廉。不过这所房子有一样优点,那就是坐在窗前就能看到大海。

陆久决定到镇上找一份工作。虽然他出来的时候身上带着不少现金,但是他不打算再用这些钱了。他知道如果过多使用的话,那么顺着这些钞票的编号,公司迟早会找到他。

但镇上没什么陆久能干的活儿:手工艺他不会、叫卖的差使他也做不来,杀人的活计又没有市场。修理枪械他倒是很拿手,可惜这里的五金店里不卖枪。转来转去,他终于在码头上找到了适合他的工作——当装着鱼或者货物的船靠岸的时候,他可以帮忙搬运这些东西。简单说,就是靠出卖体力的装卸工。

而V则在屋里深居简出地待命,当满身疲惫的陆久回来时,她就帮着清洗一下他那满是鱼腥味的外衣。

生活就这样持续着,日复一日。陆久的收入虽然不多,但是对于没有什么开销的他们来说已经足够。虽然一开始V还不能完全适应这种几乎是无所事事的“居家生活”,但是一段时间之后她也渐渐习惯了,并学会了做家务和烹饪一些简单的饭菜。作为战术人形她是没有搭载家政模块的,但是她拥有很好的学习的能力。虽然她下厨的手艺始终让人不敢恭维,但是陆久是个很好打发的人,他从不对现成的饭菜提意见。

两个人依然使用着夫妇的身份作为伪装,悄无声息地蛰伏着。他们的生活平静而规律:陆久每天早上日出时出门、日落时回家,两个人一起默默地过晚饭吃饭,然后V收拾好桌子就去休息。而陆久则会坐在窗户前,独自饮酒到深夜,然后在沙发上合衣而眠。

这样的日子会持续到何时呢,V毫无概念。她所做的只是没有目的地陪伴在陆久身旁。如果说她对这样的生活有什么希冀的话,那就是希望时间能抚平陆久心里的伤痕,希望他总有一天能走出那些灰暗的过往。

V在心里并不排斥平淡的生活——如果这平淡能够长久地维持下去的话。她从不期待陆久能再去够建功立业,甚至没有想过他能重回战场。就算是变成一个普通人,只要他能够安心平静地度过每一天,V就感到满意了。

如果真的能这样的话,那也算完成了她和那位闺中密友的约定了吧。V这样想着。

然而事情却并没有向她所希望的方向发展。陆久虽然每天都是带着淡然的面容离开寓所的,但是每天他回来的时候,他都会变得比前一天变得更加沉默。而陆久对着窗外饮酒的时间也变得越来越长、休息得越来越晚。

V知道他依然被往日的回忆折磨着,而且因为无法排遣胸中的苦闷,他正在陷得越来越深。V的心里充满担忧,却又感到一筹莫展。她看到陆久正在不停下坠,她虽想把他拉出泥潭,但却从来都没有抓住过他的手。

终于,也许是因为他心里的压抑已经无法承担、也许是因为他索性自暴自弃了,陆久越过了自己从不曾触碰过的底线。一个下雨的夜晚,当陆久再次独自坐在窗户前喝酒到深夜的时候,V凭着直觉感到陆久已经到了极限,他需要一个出口来发泄心中的积郁。所以当她被那个男人压在身下的时候,她没有丝毫的反抗。

如果自己能成为他发泄的出口的话,那这样就很好,V在心里想着。如果这样能够派上用场的话,那她也就不再那么一无是处了吧。

然而虽然有所准备,但是V总归还是毫无经验,只能静静地任由陆久摆布。当陆久粗暴地进入她的身体时,她只感到撕裂般的疼痛,这样的交融毫无美感可言。

那一晚他们做了一整夜,一直到天光微亮才相拥而眠。当第二天V醒来的时候,甚至就连走路都感到疼痛,子弹贯穿身体的时候都没有这样痛苦。

不仅是身体,V还感到了一阵心里的莫名刺痛,那是她从来都没有过的感觉。她也不知这到底是为了什么——是因为陆久,还是因为她自己。她明知在陆久心里自己不过是另一个人的替代品,但她依然心甘情愿地扮演着这样的角色,这让她第一次为自己感到可悲。

但她又觉得,陆久的痛苦也许更深。

她知道,只有曾经拥有过,失去的时候才会感到痛彻心扉。她生来一无所有,因此永远不必为失去了什么而感到悲伤;而陆久,曾经拥有的一切都失去了——他的名字、他的过去,还有他爱过的女人。V不知道陆久到底失去了多少,也无法想象那会是怎样的感受。

但是她的心里又有着一丝感动,特别是看着陆久在她的身边安睡的时候。只有那个时候,这头牢笼困兽般的男人才变得像食草动物一样安静。V默默注视着陆久胡子拉碴的睡脸,发现女人不仅有被男人保护的需要,而且也有去保护受伤的男人的需要。

这具残破的身躯能派上用场,真是太好了,V心想。虽然很疼,但是也很高兴。因为这微薄的安慰,也许就是她能给陆久的一切。

但不论是温柔乡还是英雄冢,女人的身体的确是男人的愈心良药。因为就连陆久,在那之后也似乎渐渐发生了改变。每天早上陆久在V的身边醒来的时候,他的面色都是安详的。白天的陆久内心似乎总是平静而柔和,仿如秋天远空的云朵、如春日午后的阳光。就连V甚至也因此产生了一种错觉,让她觉得陆久真的在自己身上得到了一丝的治愈。可是每到夜晚,陆久又会变回那头心中的饥渴仿佛永远无法填满的野兽,凶猛地撕扯着V的身体,似乎要将她的血肉和灵魂全部吞噬。

渐渐地,V对这种疼痛变得麻木了。她已经不在乎自己身体的感受,在她的心中,只要这个男人喜欢就可以了。

——是的,对于V来说,只要陆久喜欢就够了。就算世界在此刻终结,她也没有什么值得遗憾的事情,因为她对曾经拥有的一切已经感到满足。

她本该像那些人形一样在无尽的战斗中不断轮回,直到彻底毁灭,但她却拥有了意料之外的这么多。她有过这段穿越高山和河流的旅程、有过这些缀满璀璨繁星的夜晚,还有这片不知尽头在何方的海。她有过冬雪,有过春风,有过秋雨和夏日的骄阳,还有过一个凶恶如猛虎、又脆弱如伤兽的男人。难道这还不够吗?

虽然她根本不知爱为何物,却依然凭着自己的本能,去付出着她所有不多的一切。

虽然她的身体无法孕育生命,但那个人在这具残缺的身躯里注入的生命,让她觉得自己的生命也变得鲜活。

虽然她不懂人生的意义,但她也开始确信自己是在真切地活着:因为那微感疼痛的身体,就是她活着的证明。

\section*{}

这样的生活持续了四个月。

到盛夏来临的时候,V的头发已经披到了肩膀,为了保持清爽她在后脑勺上扎了一个简单的马尾。

陆久已经不怎么在寓所里饮酒了,他的习惯变成了下班后去码头上的酒吧里喝几杯再回去。而V也渐渐外出得多了起来,不再是只在采购日常用品的时候出门一趟。

她偶尔也会感觉着时间去酒吧找陆久,每次她走进酒吧都会引来一片赞叹的口哨,当她坐在陆久面前的时候,又会引来一片发牢骚的嘘声。不过好在V从来不会在酒吧长久逗留,她和陆久的对话总是只有两句:

“回去吧。”V说。

“好的。”陆久答道。

陆久本就是个沉默寡言的人,他每次来到酒吧都会独自坐在靠窗的角落里,默默望着窗外的海面喝到月亮升起才离开。陆久的酒量在酒吧的常客里是数一数二的,不过这一点只有酒保才知道,因为陆久从来不和其他人打交道。

但陆久后来也成了酒吧里的名人,在他离开北镇很久以后,人们依然会谈论那个外乡人的故事。他出名的第二个原因就是因为他有一位美得让其他男人两眼发直的妻子,而第一个原因,则是他曾一个人阻止了一场酒吧里的大混战。

那一天,陆久如同往常一样坐在了靠窗户的角落里,他忽然意识到身边的气氛有些不对。

抛开面色如土的酒吧老板和几乎已经有点神经质的酒保不说,酒吧里居然同时出现了两伙不该同时出现的人——以“船长”为首的渔民们和以治安官为首的海上治安巡查队。

年老干瘦的船长和年轻彪悍的治安官正坐在一张桌子上沉默相对,而其他人则分坐在两边,正在一边摆弄着手里的武器一边相互瞪视,显然都是有备而来。至于为何而来,自不必说,肯定不是要一起开个充满喜庆气氛的联欢派对。

这个小镇上巡查队和渔民之间的关系向来紧张。不过虽然私下冲突不断,但是在他们老大的努力协调或者说相互制约下,一直都还维持着微妙的平衡,至少在台面上是如此。但是今天的局势似乎分外严峻。

陆久这一介草民是从来不关心这些“江湖”上的事情的,作为外乡人的他也鲜有受到帮派拉拢的时候,在镇民的眼中他只是个不为人知的小人物,为利益而斗来斗去的事情不会和他有关。不过那天陆久还是被卷入了一场声势浩大的帮派斗争中,只因为他看似无心的一句话。

陆久向来是有自己的立场的,只要他的酒还在杯子里,就算外面已经打得血流成河,他也绝对不闻不问。毕竟他已经不想再去管别人的事情了。

如果他能把这一立场贯彻到底,那么明哲保身并不是问题。但遗憾的是那天陆久有些喝醉了。

“要是实在谈不拢,为什么不派个代表出来解决呢。”陆久站在大战一触即发的两伙人跟前说道,“简单快捷,还免得两败俱伤。

这话似乎深得手持武器的治安队员和渔民们的赞同,但是也深得两伙人的首领的不满。

今天的事情,靠谈判显然解决不了。那么派个代表,又该派谁呢?恐怕每个人都会觉得是两个帮派的首领。不过这早就不是武将单挑的古代战争了,首领们之所以纠集这么多人火拼,一来是因为对自己的帮派战斗力很自信、二来就算战斗不利,自己也能保全身。首领亲自去械斗的话,养这么多部下还有何用呢?

“你是个什么东西,敢在这里大放厥词?!”年轻的治安官首先叫了起来。

“是啊。一个外乡人,凭什么对我们的事情指手画脚?”年迈的船长也冷声说道。

“没什么。”看到自己的意见不受欢迎,陆久摆了摆手说,“只是个偶然经过的路人。请诸位继续吧,我自便了。”

说着陆久朝着酒吧门外走去,但是他已经走不了了。他的话让两位首领都没有了台阶可下,怎么可能让他如此拍屁股走人。

“留步啊,老兄。”治安官冷笑着拦住了陆久,“搅浑了水就想一走了之,没那么简单吧?”

陆久看了一眼船长,发现他也正愠怒地看着自己,显然他也不会同意自己就这么离开。

“那你想怎样?”陆久微微一笑。

他并不是对目前的处境非常乐观。纵然陆久身手很好,但是赤手空拳面对一屋子面色不善、手持凶器的暴徒,他没有什么正面对抗的把握。

如果是V的话也许还能招架一阵,至少可以全身而退,但陆久只是一介血肉之躯的普通人。

不过,他只是对此事不甚在意罢了。

“是你说的要派个代表的吧。那么要是现在派你当代表,你会代表那一边呢?”

这个问题问得十分狡猾,显然无论陆久说代表哪一边,都会惹怒另一边。不过,考虑到没有哪一边想被陆久这样的无名小卒代表,说不定他无论怎么回答都会同时惹怒两边。

“我代表我自己。”陆久想都没想就说到。

“代表……你自己?”治安官因为吃惊而瞪大了眼睛,“你是说你是想代表自己,作为第三股势力参与进来?”

“我不想参与。但是想要走出这个门外,看来没有其他办法了吧。”陆久淡然说道。

“哈哈,哈哈哈……”治安官大笑起来,“很自信啊,老兄!这个回答倒是出人意料!”

周围的人听到陆久这句话,也跟着纷纷哄笑了起来,但是陆久没有说话。

“你太狂妄了,年轻人。”一直沉默的船长厉声说道,“这里没有你的事情,赶紧离开吧。”

“放屁!”治安官喝道,“老头子,你放他一马我不管,但得先过了我这一关再说!”

“他和我们的事情没有关系。”船长说,“我们事我们自己解决。”

“无关?呵呵,少废话!今天的事情,见者有份!”

“既然如此,那这样吧。”陆久打断了两个人的争执,“我先和警长先生这边切磋切磋,等我输了你们再解决自己的事情也不迟,如何?”

“……想死的话我不管,可别指望有人救你。”船长冷冷地说道。

“我知道。”陆久说着转向了治安官,“我就一个人。您是要自己上,还是所有人一起来?”

治安官盯着陆久看了一阵。虽然对他来说所有人一拥而上把陆久制服更为轻松,但是那么做的话,难免会让对面的渔民笑话,在自己这边可能也难以服众。

所以他决定单独会会陆久,毕竟陆久虽然身材高大,但看上去也不像什么武林高手。而治安官则几乎每天都在斗殴,对自己的身手已经相当自信了。

“好啊……有种!就让我会会你。”治安官脱下了自己的外套扔给手下人,“都给我让开!”

巡警们纷纷向后退去,渔民们见状也后退让出了一块地方,两个人被围在人群中间。

当多年后治安官先生回想起当时的情景的时候,他总结出自己犯下的一个严重的错误,那就是选择了和陆久单挑。

他很轻视陆久、完全没有把陆久放在眼里,这都不算错误。因为就算他很重视陆久,把陆久全都放在了眼里,他最后还是赢不了,顶多是多周旋几分钟罢了。毕竟实力悬殊。

那次两个人都只出了一招,就在这一招之间就决定了胜负。

治安官一记凌厉的直拳直攻陆久正脸,但陆久躲都没躲。他微微曲腿向前倾身,然后用额头直接顶向那个带风的拳头,同时挥出了自己的右拳。两个人的拳头同时击中了对方。

陆久的眉弓被打破了,血流了他一脸。但治安官中拳的地方是左耳后,这一拳打歪了他的下巴、强烈震撼了他的颈椎,几乎将他打昏过去。

“妈的……”不甘认输的治安官从腰里掏出了他的手枪,人们在惊呼声种纷纷趴了下去。治安官挣扎着想要击毙陆久,但是还没等他子弹上膛陆久就来到了他跟前。然后他感到手里的枪一震,听到咔嚓一声。

那把枪依然握在自己手里,只不过枪的套筒和枪机一起不见了。陆久只用了一瞬间就拆掉了他的枪——是的,是拆掉,而非卸下了那把手枪。

“1911式手枪不能上膛待击,过了这么多年依然没有改进吗。”陆久淡然说道,语气里似乎有些失望。

治安官输了,这已经不需要他的承认。这个不起眼的外乡人在战斗方面远远超过了他,一对一的对抗,治安官毫无胜算。

“那么,我就告辞了。”

陆久说着,转身朝着酒吧门外走去,但是他再次被拦了下来。

“慢着。”一直站在一旁的船长开口了,“在下也想和你比试比试。”

说着,船长递上了一块抹布。

“哦?”陆久接过抹布擦了一下脸上的血,有些奇怪地回应道。

他没想到这个时候船长会拦住他,刚才这些渔民对自己的敌意并不太深。

不过陆久思考了一下明白了过来:此时自己就这样离去的话,酒吧里恐怕依然难免一场血战,因为士气受挫的巡查队肯定不会再派什么代表了。他们只会把自己的怒气都发泄到船长和他的人身上。

“好吧。”陆久微微点了点头,“怎么比?”

“‘拳怕少壮’,比拳头我这个老头子是比不过年轻人的。”船长笑了笑说,“比枪吧。看阁下刚才拆枪的那一手,一定也是个好抢手。”

“行。”

“去海边。”

两个人走出酒吧,朝着海滩走去,后边跟了一大群人。这些人因为好奇,也顾不上彼此之间的矛盾,结伴而行完全没有了刚才的杀气。

“看见那个浮标了吗?”来到海边,船长指着远处海面上说道。陆久眯眼看向那个方向,那里的确有一个发光的浮筒,正随着海浪上下起伏着。

差不多有四五百米,陆久心想。

“拿两把枪。”船长对着手下的人说道,然后转向陆久,“选一把吧。”

手下人送来了两把步枪,陆久随手接过一把枪看了看。

……温彻斯特杠杆式步枪。陆久忍着心里的惊讶没开口问这两把枪是从哪来的,就算是在自己以前的那个时代,这种东西也算是古董了。

不要说光学瞄具,枪上就连卡尺都没有,只有一个照门一个准星。新手拿在手里,恐怕连测距都没法测。

“轮流射击那个浮标,谁先打中谁就算赢。怎么样?”船长完全没有注意到陆久的神情。

“不比。”陆久断然说道。

“……怎么?”对陆久毫不犹豫的拒绝,船长微微有些吃惊。

“‘枪怕老郎’,论枪法我这种毛头小子怎么比得过老猎人。”陆久用船长之前的话回敬道。

“那你说怎么比?”

“比不靠枪法的。”

“打枪不比枪法比什么?”船长奇怪地问道。

“手枪决斗。”

船长眯着眼看了陆久一阵,然后冷冷一笑。“手枪决斗”,这种古老的对决游戏,几乎和温彻斯特杠杆步枪的历史一样悠久。

在不远的距离上,一声令下,两人同时拔枪,谁开枪快谁就赢。不过人和人拔枪的速度,一般差不了太多,所以这种对决靠的不全是技巧……特别是老手之间,胜负之分往往有很大的运气成分。

因此陆久的提议让船长一时有些犹豫。

“我同意这个人的话!”见船长犹豫不决,方才败落的治安官煽动地说道,“除了你自己,你那把老枪鬼才会用。手枪决斗谁也别说自己不会,很公平!”

陆久也点了点头,他就知道自己的提议会有人附和——毕竟有个人刚才挨了狠狠的一拳,不见血的比赛,他一定不想看到。

“……好!”听到这番话,不想在人前失了面子的船长说道,“就按你说的。换枪!”

说着船长把手里的步枪递给了手下人,然后撩起了衣襟。陆久看到船长的腰里别着一把左轮手枪,看来他真的是对古朴武器情有独钟。

“用我的枪!”治安官走了过来,向陆久递上了他那刚刚装好的1911,然后颇有深意地看了陆久一眼。

陆久接过枪,发现枪很沉——里边的弹夹是满载的。

“小崽子,你想耍什么花招?”船长警惕地说道。

“放屁,我是怕你拿把哑枪坑人!枪在他手里,我能耍什么花招,我难道会给他一把假枪帮你获胜?!”治安官立即回敬道。

啪!一声清脆的枪响结束了两个人的争论,陆久朝着天上开了一枪。

“是把好枪。”陆久说着把枪别在了腰间,“这是一场公平的对决,没有人耍花招。准备开始吧。”

“我来发信号。”治安官说道。他对这次对决充满期待——无论是谁赢他都高兴,因为这两个人他都看不过眼。

船长狠狠瞪了治安官一眼,然后看向了陆久。

“这是你选择的方式。枪弹无眼,要是你死了,也不要怪我。”他说。

“持枪之人,自有饮弹之心。不用多说。”陆久答道。

说完,两个人站定,各自拉开了架势。

“就用这个吧。酒瓶落地,同时开枪。”治安官不知从哪掏出一瓶啤酒猛灌了几口,“预备——走你!”

啤酒瓶被高高抛上了天空,没喝完的啤酒洒得到处都是。陆久和船长专注地看着对方,用余光留意着空中旋转的酒瓶。几秒钟之后,酒瓶落在地上,发出一声碎裂的声音。陆久和船长同时拔出了枪。

啪啪!

响起两个枪响,陆久倒下了,船长也跪在了地上。

但陆久马上又站了起来。

在酒瓶落地的一瞬,陆久没有立即开枪,而是在拔枪的同时倒了下去。子弹贴着陆久的肩膀飞了过去,而船长被打中了大腿。

“……你!”半跪在地上的船长再次举起枪对准了陆久,“你……使诈!”

“胜者生、败者亡,使诈又如何。另外我已经手下留情了,刚才我完全可以打爆你的脑袋。”陆久同样用枪指着船长说道,“战争之中,兵不厌诈。你输了。”

“你塔码的使诈!不公平!”看到陆久用计取胜,激愤的渔民纷纷举起了武器,气氛再次紧张了起来。

“放下武器!”船长厉声喝道,“他的枪里还有子弹,你们不是他的对手。”

陆久举着枪静静地站着,一动不动。终于,船长首先垂下了枪口,渔民们也跟着放下了武器。

“说得好,兵不厌诈。”船长低下头说道,“我输了。我心服口服。”

陆久也收起了手中的枪。

“既然二位都输了,那么这次就算是平手了吧。”陆久说着朝治安官走去,把那把手枪递了过去,“作为胜利者,我希望以后,最好别再打了。”

说完,陆久朝着回去的方向走去。

“……不简单啊,这老兄。”治安官接过那把枪看了一阵,最后还是退出子弹把枪塞进了枪套,轻声说道。

他就不怕被人暗算吗?他身后的这些人个个手持武器,他却没有一丝防备,治安官心想。不过也许这才是胜利者的姿态吧,如果不能让人心服,那就不能算大获全胜。

“等一等!”被部下扶着的船长对着陆久的背影喊道,“你说战争之中兵不厌诈,莫非你……是个兵吗?”

“……”

陆久稍微停了停,没有说话。片刻后,他再次沉默着迈开了脚步。

\section*{}

陆久回到居所之后,已经是深夜。黑暗之中,陆久看到V正坐在床头等他。

陆久走进洗手间,打开灯,看到镜子里的自己。他额头的血迹已经干涸,眉弓上被撕裂的伤口已经结痂了。他清洗了伤口,然后用胶布固定了一下。

“你受伤了。”陆久听到V在他的背后说道。

“和人斗殴,挨了一拳。”陆久头也不回地说。

“和什么人斗殴?”

“几个镇民。”

V没有再说话,她知道陆久经历的不可能是“斗殴”这么简单的事情,这个镇子上没人能够轻易打破陆久的脑门。她也想不出陆久会为了什么理由和人斗殴。

但她最终还是什么都没问。

“睡吧。”她说。

陆久走向沙发,坐了下来,然后睡着了。他那个晚上没有去V的床上,从那以后,也再没有去过。

陆久的离开是在几天之后。那天早上陆久如往常一样去码头“工作”,但是V已经敏锐地意识到陆久可能不会回来了。因为他在离去之前,比往常多看了V几秒。

“我走了。”站在门口,陆久说道。

“……早点回来。”V回答。虽然心中已经有了猜测,但她还是一如既往地回应着。就在最后的诀别之际,两个人依然没有告别。

傍晚,有人推开了陆久居所的门,但走进门来的却不是陆久。

V警惕地作出了战斗反应,但是马上又放下了架势。走进来的是酒吧的老板,虽然见面不多,但V认识他,他应该不是带有敌意的人物。

虽然不知道发生了什么,但是V明白,自己的猜测马上将会被证实。

果然,酒吧老板用平静地语调对V说道:

“民用人形Vector,你的所有人陆久已经将你的使用权交付予我。你的所有人在我处借用了大量现金,并承诺由你在我的酒吧工作来代偿债务,因此你将在我的酒吧充当服务人员,直到你的劳动抵偿他所借的债务、或者其他人将债务偿清。现在请跟我走。”

说完,他递上了一份由陆久签署的文件的副本。V接过去看了一眼,看到下面果然有陆久的签名,于是将文件还给了酒吧老板。

上面写的文字和条款,她一句都没看。

“知道了。”她轻声说道,“我们走吧。”