\chapter{昨夜的星辰(九)}

\section*{前言}

大雪中只有两个人的旅行,本应是非常温馨甚至浪漫的,但却又隐隐埋藏着一丝不安。两个人心中怀着的是不同的感情:V希望能够回到他们上次分别的地方,让两个人断裂的联系重新开始;而陆久则想在那里说出当时本该说出、却没有说出的话,给彼此的关系做一个最终的清算。

\lineseparator

一些与主题有关的废话:

第三部分也接近尾声了……我希望能利用这个假期把它整理完。

烧钱已经是个热度严重下滑的IP了,Vector更是过气中的过气,这个故事的点击量通常每篇只有不到百次,关注的读者更是少得可怜,只有那么三四位。不过不要紧,这个故事本来也是老瓶老酒重新整理出来的,依然能有人支持我很高兴。

第三部分的主要内容是一些琐碎的日常生活,陆久和V一直在努力试着像是两个“普通人”那样,在人群之中过着平凡的生活。这样的生活让他们收获很多,甚至产生了“我也可以像普通人一样生活”的错觉。但他们知道,至少是陆久知道,这一切也不过是相互迁就的表演。

关于V,陆久在心中确信的有两点:

一、无论V怎样表现得像是一个人类,但她终究不是人类;

二、无论V怎样装作像一个人类那样去行动和思考,但她终究还是没有把自己当做人类。

陆久也不知道到底什么才是“人类”,但对于自然出生的他,物种同类间的认同感是底线。陆久希望V能够认为她自己是个人类,以此来掩盖陆久内心的不安,但V并不这样想。V在心底到底是怎样认识她自己的,陆久感到扑朔迷离。

陆久还不知道对自己来说,什么东西是重要的。他不知道自己如果选择像人一样活下去,需要一个怎样的伴侣。他不知道,让他无法释怀的,只是一些没有什么意义的执念——也许是还没能想开、也许是还没有认识到。

他总会认识到的,在死掉之前。但不知道当他明白这些的时候,那些他本该珍惜的事情,会不会还在。

为了逃离阖家欢聚的春节,陆久带着V,开上了他们的二手陆巡,踏上了通往过去的旅行。他们去了陆久年轻时曾经驻扎过和服役过的地方——秦岛市,其中的北镇正是他们曾经分别的地方。大雪中只有两个人的旅行,本应是非常温馨甚至浪漫的,但却又隐隐透露着一丝不安,两个人心中怀着的是不同的感情:V希望能够回到他们上次分别的地方,让两个人断裂的联系重新开始;而陆久则想在那里说出当时本该说出、却没有说出的话,给彼此的关系做一个最终的清算。

而在这之后,陆久和V到底是终于能够放下对彼此的执念、了无牵挂一身轻松地去往各自该去的地方;还是依然被剪不断理还乱的羁绊捆绑在一起,在痛苦中无法挣脱?这超长的两节,希望能把这些梳理清楚。

正文从下一页开始,喜欢的朋友请收藏关注。

\lineseparator
\section*{}

当陆久睁开眼睛的时候已过黄昏,视野里已经有些发暗了。也许是因为难得地彻底停止了患得患失,他不知道什么时候睡着了。

他首先看了一下手腕上的计时器,确认了时间是晚上的六点,然后他才发觉自己正靠在什么人的身上,而不是靠着沙发的靠背。陆久赶紧坐直了身子,看到自己正在靠着的果然是V的肩膀,因为他确定自己还在公寓的客厅里。而V则一直坐在沙发上,位置没有丝毫改变,也许是不想惊醒陆久。

“啊,不好意思。”陆久说,“我不小心睡着了。”

“没有,只是一小会儿而已。”V说,“你放下茶杯就没动静了,在你倒在我身上之前,我都没有发现你睡着了。”

“看来我有点放松过头了。”陆久感觉有点尴尬,因为他似乎听到V说话的语气仿佛带着笑意,一定是因为他的睡相不怎么样。

“没什么,这样也挺好。难得能休息一下。”

“我没有……打呼噜什么的吧。”

“没有。你从不打呼噜。”

“……是吗。”

这个回答让陆久不知道是该欣慰,还是更加尴尬。没有因为睡着而失态自然很好,不过V的说法好像她很了解某些内情一样,这让陆久稍微有些不安。

算了吧,陆久心想,被她了解到内情也是没办法的,因为反正……也不是第一次了。但在V的面前他越来越容易失去防备这一点,还是需要自省的。

“说真的,忽然无事可做了也让人很觉得不习惯。”陆久说,“所以我觉得我们该讨论一下假期的安排了。”

“你不是说有些想法吗。”

“我说了吗?”

“昨天说了吧。”

陆久想了想,好像确实说过,不过他也想听听V的意见。

“你有什么意见?”

“我……没什么意见。”

“那么就去走走吧。”

“出去走走?”

“是啊。别人的春节是在家和家人一起度过的,而我们这种没有家人的人,留在屋里干什么呢。难得有这样无所事事的时间,不如出去走走。有什么想去的地方吗?”

“我——”

“不要急着说没有。想一想。”

“我……”V低头考虑了一会儿,“我想去海边。”

……海。

想要去海边吗。陆久知道距离最近的海岸在哪,那个地方很久,比N17战区还要近。

“现在可是冬季,北方的海边不仅很冷,说不定海面已经冻结了。”

“是吗。我只是随便说说,不行就算了。”

“我没说不行,但是我们得准备一下。”陆久说着拿起了自己的手机,“告诉我怎么预约宾馆?”

V有些意外地看了陆久一眼,然后帮他打开了搜索引擎。

陆久低头专注地拨弄着手机,没有再说话。过了一阵子,他抬起头看向V,发现V也正在看着他。

“为什么忽然想要去海边呢。”陆久说。

“没什么。只是……想去看看。”V移开了目光。

陆久闻言有点困惑。只是想去看看?莫非那时候还没看够吗。

不过说起来,虽然那时候离海那么近,好像确实没有认真地去看过一次。

回想起来,在他们共同经历过的时光里,不乏留下回忆的地方。广袤的森林、宽阔的河流、沙化的无人区、冰雪覆盖的荒原、城市的废墟、旧世界的遗迹、和不知尽头在何处的苍茫大海,那都是些令人惊叹的奇迹。可那时的他们却一直忙着辗转奔波,忽略了就在身边的景观。

“那就去吧。”陆久放下了手机说道,“什么时候动身,现在就出发吗?”

“现在?”陆久的话显然让V有点吃惊,但她马上就点了点头,“……好的,现在就出发吧。”

看来是认真的呢,陆久戏谑地想着,本来只是开玩耍,想不到V竟然毫不犹豫就同意了。还好她没说要去南美洲的雨林。

陆久和V都是行动派,两个人收拾了一些简单的行李,然后像那时候那样再次踏上了旅途。不过这次他们有了自己的交通工具,所以不必再去扒火车,而且也不再是不辞而别——虽然空荡荡的房间里根本没有需要道别的人,但至少在临走前陆久把春联贴在了门口。

在加油站里加满油箱,陆久驾驶着汽车驶入了向北的高速公路。要去的地方和他们所在的城市相邻,汽车行驶了四个小时,然后在凌晨时分离开了高速公路。

虽然陆久是特意在城市的繁华地段预定了宾馆,但抵达时他发现宾馆里几乎无人入住,停车场里空荡荡的十分冷清。这也难怪:一因为这是一座以避暑旅游为主要产业的城市,在冬季总会显得有些萧条;二来在这万家团圆的节日里,漂泊在外的人更是少之又少。

把汽车停好,陆久和V拎着行李箱走进了宾馆的前台。前台负责接待的是一个娇小的民用人形,虽然是凌晨时分,但她一丝不苟地坐在柜台里面,脸上没有一丝倦容。看见陆久走进来,那个人形立即站了起来。

“您好,欢迎光临。请问有预定吗?”

“有。”陆久把手机上的预约号给接待员看了看。

“好的,请给我您的证件。”

陆久又把自己那张来历不明的身份证递了过去。

“和您确认一下,下榻的只有您一位对吧。”

“不,我们是两个人。”

接待员楞了一下,然后看了看陆久身边的V。

“随行的人形不需要登记。”她笑了笑, 拿起了对讲机,“客房部,201018房间有贵宾入住,请接待。”

说完,接待员然后把证件还给陆久,顺便递上了房卡。陆久耸了耸肩,接过房卡向电梯走去。

这座宾馆的招待规格不低,虽然是标准的双人间,但宾馆的房间很宽敞,而且里面地毯、台灯、沙发、电脑之类的设施一应俱全,相比之下陆久在分公司的宿舍就显得有些简陋了。陆久放下行李,拉开了房间落地窗的窗帘朝外望了望,城市的夜景尽收眼底。

只不过,这夜景只有街道黯淡的路灯和楼宇里透出的星星点点的光。这座城市的夜晚远不如上海那样灯火辉煌,毕竟不是什么大都会。

“屋里是不是有点冷?”陆久忽然说道。厚厚的户外服装还在他身上没有脱下,但陆久在屋里并没有感到有多暖和,手脚依然有些发凉。

“现在的室内温度是14度,对人类来说确实不高。”V看了看梳妆台前的温度计说。

室外的温度已经零下10度,这座大楼上应该是有供暖的。不过鉴于入住的人数不多,供暖似乎并没有全部开启。于是陆久拿起了床头的电话,拨通了前台的号码。

“您好,这里是前台。请问有什么能为您服务的吗。”接起电话的听起来就是刚才的接待员人形。

“房间里温度很低,供暖是不是有问题?”

“是这样的,我们已经为您开启电热供暖系统,但需要一点时间来预热,然后很快就会暖和起来。”接待员说道,“在达到预设温度前,建议您可以先让随行的人形为您取暖。”

听到这样的建议,陆久看了V一眼。

“好的。知道了。”陆久放下了电话。

“怎么了吗。”V说。她注意到陆久刚才看她了。

“没什么。”

陆久倒不是不能忍耐一会儿,因为这房间里的温度比室外高多了。只是接待员后面说的那一句让他有些不快,因为她显然是把V当做一件物品了。

但陆久没有说什么,因为他知道那个接待员人形的建议是善意的。她的逻辑并没有错,错的是没有把V当做物品的陆久。

“不能提高温度?”V说。

“能。不过是电热功能,所以要预热一会儿。”

“你觉得冷吗。”

“不冷。”

“要是冷的话……”

“不用了。”

陆久打断了V的话,然后脱下外套躺在了床上。不知为何,他心里感觉莫名地有些烦乱。

“就算是人类,也可以挤在一起取暖吧。”V也脱下外套走到了陆久旁边,“我可以也躺在这里吗。”

陆久没说话,不知为何,V竟然看穿了他的想法。他抬眼看了看屋里的另一张床,但最终还是往旁边挪了挪,给V腾了点地方。

V躺在陆久的旁边,然后拉起被子盖在两个人身上。

“你不喜欢我这样,是吗。”V轻声说。

“不。我只是……”

“在你宿舍里的时候,你也是这样的。当我说我们可以睡一张床的时候,你的表情并不高兴。”

“我没有。”

“在北镇的时候也是,宁可一个人独自窝在座椅上睡觉。”

“……别再说了。”

“小芮说得对,人形的地位确实不能和人类相提并论。但我真的不觉得有什么,我并不在意其他人的看法。”

不需要在意其他人的看法,陆久心想,一个人并不是因为自己的出身、别人的态度而成为了人,而是因为——

但陆久没有把这些话说出口,因为在经历了这些年之后,他已经知道事情不是他想象得那么理想主义。就算是他,在心里其实也很在意其他人的目光。毕竟人类都是社会性的动物——改变一个人很难、改变这个世界则难如登天,这就是他一直想远离人群的原因。

“说什么莫名其妙的话。”陆久把头转向一边说道。他感觉有些疲倦,不想再去想小芮说的那些。

知道陆久不想谈论这个话题,V也没有再说下去。

“抱我一下……可以吗。”

陆久把脸转向V,看到她正面对着自己侧卧着,身体微微蜷缩。于是他伸出胳膊拥住了V。

“暖和一点了没有。”枕在陆久的臂弯,V轻声说。

“暖和多了。谢谢。”

“那我去另外一张床上了,不然这样你也休息不好。开了一夜车,睡一会儿吧。”

“……”

陆久没有说好,也没有说不好。但V还是轻轻起身,然后脱下衣服躺在另一张床上盖上了被子。

过了一阵,陆久感觉屋子里的温度升高了不少,而且V似乎已经睡着了。于是他也躺平身体闭上了眼睛。

\section*{}

陆久再次醒来的时候已经快到中午,一夜的舟车劳顿确实让他有些疲劳。他睁开眼就看到V正穿戴整齐地端坐在写字台前,正默默地看着自己。

这让陆久心里稍稍有些懊恼,因为他昨天还暗暗决定要警醒一点,但这次依然是完全不知道V什么时候醒的。

陆久坐起身,穿好衣服去洗手间洗了洗脸,然后走到窗前拉开窗帘朝外面望了一眼。

天空十分阴霾、铅灰色的云层很厚,似乎是要下雪的样子。而街道上则冷冷清清的,多数店铺已经关张,只有两三家快餐和便利店还开着门。

“看来没有海鲜吃了。”陆久说。

V明白陆久的意思,因为她也算在海边生活过半年多,知道渔民通常只在夏秋才出海打渔。不过说起海鲜,她好像只有那次和船老大、警长几个人一起出海的时候在船上吃过一点烤鱼,其他的时候在宁老板的酒吧里吃的都是些家常便饭。

“没关系,我对那些也不感兴趣。”V说。

“我也是。说真的我一般会尽量避免吃那些东西,因为……你知道吧,水里的东西都让我退避三舍。”

陆久说着笑了起来,V也跟着微微笑了笑。她知道陆久很抵触看不到底的深水,还有水里生活的动物也让他没有什么好感。

“随便吃点什么好了。”V说。

“啊,然后也该出发了。”

“出发去哪?”

“你不是说要去海边吗。”

“这就去海边吗。”

“不,并不是去海边。”陆久说,“这附近……我也有点想去的地方。”

“……啊。”

V似乎是想问陆久想要去哪,但最后还是没问,也许是觉得问了也不知道陆久说出来的是什么地方。

“在西部的郊区。”陆久自己说出了答案,“这个城市西郊的山林之中,曾有一片重要人物避暑度假的行宫。但那些别墅在战争中遭到了毁坏……据我所知,现在已经改建成度假村了。昨天我搜索了一下,有一处度假村在春节期间依然正常营业,那里就是我们要去的地方。”

为了节省时间,陆久没有在店里用餐,他在西式的快餐店里买了两份包含热饮料的套餐,然后设定好GPS导航,一路向西北方向而去。随着他们渐渐远离了市区,天空也变得越发阴沉了。等到公路两边的树木渐渐茂盛、路面也开始有了高低起伏的时候,天空中终于开始飘起了雪花。

“下雪了呢。”V轻声说。

“唔。”

陆久没有说什么,两个人望着车窗外各有所思。关于上一次下雪的记忆,两个人应该是相同的,因此他们都知道那是一段让人不愿意触碰的回忆。

中午出发,两人在车上解决了午餐,抵达度假村的时候已是午后。因为天气原因,本该光照充足的午后反而有些昏暗,雪也渐渐大了起来。

度假村的区域十分广阔,里面不仅有坐落在山林间的大片别墅,还有诸多人工已和然景观。但是冬季大概多数工作人员都已经回家了,所以园子里路上的积雪无人清理。陆久驱车穿过了大理石堆砌的气派大门,然后朝着别墅里区面驶去,在背后留下一行深色的车辙。

片刻后,汽车停在了一幢别墅的门前。陆久下车,V也跟着下了车,然后他们直接走进了那所高大的复式建筑的大门。

“不需要入住登记、领取钥匙的吗?”陆久脱下大衣挂在了客厅的衣架上,V在旁边问道。V有些惊讶,因为陆久一推门就走进了屋子,屋子不仅没有上锁,而且客厅的壁炉里还燃烧着新劈的木柴。

“在预定房间的时候就已经登记了,钥匙就放在门后面。”陆久活动了一下四肢,坐在壁炉前宽大的沙发上说,“这里和宾馆不一样,我们租下的是整座房子,所以不需要去前台办理手续。”

V在客厅里走了一圈,发现这座豪宅的客厅不仅宽阔得超乎想象,而且布置着许多做工精致的木质家具,就像常年有人居住的人家一样。她开始有些好奇,陆久怎么会知道这种地方。

“你以前来过这里吗。”V说。

“是的。”陆久点了点头,“不过是很久以前了。”

“……什么时候?”V不知道陆久还有这样的经历。在她的记忆中,战区的工作通常都很忙,而陆久也不是有闲情逸致外出游玩的人。

“大概四十多年前吧。”

“四十多……”

V一时间没有反应过来。片刻后,她才意识到陆久说的是在他被投入监狱之前的事情。

“还记得吗,我曾经在这个城市郊区的某片海滩上,执行过清理爆炸物的任务。”陆久说,“那时候,这里就是我们战区的指挥部所在的地方。”

“所以你才想来这里的?”

“呵,算是吧。不过除了山林之外,似乎已经看不到熟悉的东西了。”

“毕竟已经过去那么多年了。”

“不过也有些东西,不是几十年的时间就能抹去的。”陆久说,“怎么样,一起去看一看雪中的山景吧。”

陆久的话让V楞了一下,不过随即她就点了点头:“好。”

陆久起身从门后的设备筒里拿了两根伸缩的合金手杖,把其中一根递给了V。然后,两个人再次穿戴好户外服装、扎紧衣服的束带,走出了房间。

外面的雪纷纷扬扬地下个不停,好像比来的时候更大了,地面上的路几乎已经被完全覆盖。道路两旁的树上有些枝头的叶子还没掉完,落上积雪好像是开放着的白色花朵。这个午后一点风都没有,树林间安静极了,能够听到的只有两个人踩在雪上发出的咯吱咯吱的微微声响。

沿着不算陡峭的山坡走了一阵,他们停了下来,靠在防护锁链上稍事休息。

“真个是好地方。”陆久说,“到处都这么安静。”

“你很对安静的地方情有独钟呢。”

“唔,热闹的地方容易让人感到烦躁,所以我不喜欢的就是熙熙攘攘的人群。”陆久说,“不过郝丽安女士之前倒是交代,要我学着融入人群、治一治自己的社会性人格障碍。呵,其实我也是在消极怠工呢。”

“我也不喜欢人群,不过我觉得这没什么。为什么一定要和不喜欢的人在一起呢。”

“话虽如此,但人终究是社会性的动物,不能和其他人相处的人格是不完整的。”

“我知道。”V说,“我的性格一直都让很多人不喜欢,也许这就是天生的缺陷。”

“我不是在说你……”陆久说,“其实我也是这样。算了吧,我们两个也是半斤八两。”

“你交流能力比我强多了。”V摇了摇头,“如果没有你,我根本不知道该如何跟雷蒙、谢振还有小芮的姐姐们交流。”

“那只是因为我和人类打过的交道比你多一些。说实话,我也并不是特别喜欢,和那些人们呆在一起。”

“不过……我喜欢和你在一起。”V说,“和你在一起的时候,我从来没有感到过排斥。”

陆久看着身边的女孩,没有说话。他应该是第二次听到这句话了吧,也许是第三次?说出这句话的时候V的眼睛里没有暧昧的神色、也没有任何掩饰。

她似乎只是说出自己心中所想,陆久心想。但这所谓的“喜欢”,只是因为朝夕相处而变得彼此信任和习惯,还是有另外的深意呢。在V的心中到底是怎样理解人类的感情的、她所说的喜欢和其他人所说的喜欢是否是同一个概念?陆久不得而知。他就连对自己心中感情的认识,都是模糊不清的。

但有一点陆久可以肯定,那就是V在身边的时候他也没有感到过排斥、而且对环境的戒备也开始变得越来越放松。于是陆久笑了笑说:

“谢谢。我也喜欢和你在一起。”

\section*{}

休息了片刻之后,两个人继续沿着山路向前走去。他们转过几个弯、穿过了一片茂密的树林,下榻的别墅已经完全隐没在森林中看不到了。因为徒步登山在雪地上行走十分耗费体力,陆久的步伐放慢了一些,两个人像散步一样一前一后地在落雪的山路上慢慢走着。

“我们这是要去哪?”一边走着,V忽然开口问道。

“山顶。”

“那里有什么?”

“瞭望台。”

“瞭望台?”

“是的。别墅区坐落的地方,曾经是古代镇守此处的军营所在,所以这里有一些古代建筑的遗迹。”

“……我不知道那是什么。”

“没什么出奇的,只是一堆砖石的废墟。”陆久说,“不过视野应该不错。”

他们又走了一段,拐过了一个陡峭的山崖,来到了这段山脊的顶峰。山顶上有一小块空旷的平地,上边有一座堡垒样的建筑,大约有五六米高。那座堡垒是用石头堆砌的,看起来非常坚固,但在经年累月的岁月中已经被侵蚀得有些残破了。

陆久踏上那些破败垮塌的台阶,登上了堡垒,V紧随其后。堡垒的顶部中间已经有些塌陷,但四缘依然平整而坚固。两个人站在台顶上举目远眺,所见之处山峦叠嶂、飞雪连天,苍茫间气势说不出的磅礴。

“这里的风景还不错吧。”陆久说道。他看见身边的女孩怔怔地望着远方,已经有点呆住了。

“好高的山、好大的雪,”V喃喃地说着,“好壮观……”

“是啊。”陆久说着用手扫了扫垛墙上的雪,坐在了垛口上,“瞭望台都修在制高点的位置,通常都是观景的好地方。”

“请不要坐在那里,那个位置很危险。”V立即提醒陆久说道。垛墙之下是陡峭的山崖,如果不小心掉下去可就要命丧黄泉了,而且那些年久失修的垛墙显然已经很不牢靠——虽然眼前的景观让人流连,但V并没有忽略陆久的举动。

陆久笑了笑,起身离开了垛口。“很危险”的话,他已经有很长一段时间没有听过了。

“那么,那些又是什么?”见陆久去了安全的地方,V又看着远处说道。

“……是墙。”

陆久知道V在问什么,她说的是那些连绵如长龙一样蜿蜒、将各个山巅的瞭望台连接起来的残垣。那是古人用于防范外敌入侵的屏障——历经几个世纪的天地变幻,依然矗立不倒的古长城。

“这墙是做什么用的?”

“是一种防御工事。”

“防御工事?”

“是的。这是古人修建的用于抵御外敌的壁垒,人们把它叫做‘长城’。”

“在这样的山林里修筑这么大规模的工事可不容易。”V明显被陆久的话震惊了,“这些瞭望塔和墙彼此相连,长得看不到尽头呢。”

“的确。而且那时候也没有现代的大型机械,全都是靠人力修建,工程难度可想而知。”

陆久说着,从怀里掏出一个保温杯。他拧开杯子,往盖子里倒了一些冒着热气的水,递给了V。

“喝吗。”

V接过杯子,喝了一口。

“味道怪怪的。这是什么?”

“掺了酒的热水。”陆久也端起杯子喝了一口,“故人戍边,冬天经常靠饮酒来活血驱寒。不过大多数时间他们只能喝冷酒,我们比他们幸运,我们有保温杯。”

V喝掉杯盖里的水,然后把盖子呈到了陆久面前:“再来一杯。”

陆久又给V倒了一杯水,V一饮而尽,然后呼出一口白气。

“谢谢,我感觉暖和多了。”

“从这里向东几十公里的海边,就是长城的起点。这些城墙的修建,耗费了数以百万计的人力以及上百年的时间,说是见证了许多王朝的兴衰,也不足为过。汉语里有个叫做‘边塞’的词,指的就是这种地方——这里就是几百年前的边境线,城墙之外,就是战火连绵、动荡不安的塞外地区。我们面前的,可以说就是古代的战场前线。”

“那么我们刚才所在的别墅,就是古代的指挥官坐镇的军营了吧。”

“呵,就是这么回事吧。”

“可是现在无论是边境还是军营,都已经是一片废墟了。”

“这里的山石树木历经了千百年岁月里战火的涤荡,但如今人们只有在怀古的时候,才会来看它们一眼。在漫长的时间之中,没有什么永垂不朽的东西。”

“但战争并没有停止。战场和前线,只不过是换到了别的地方。”

V说着向着西北方望去,陆久知道在她的目光所达不到的地方,正是北部战区的方向。

的确如此,陆久心想,因为人类的历史就是一部战争史。他们自从从树上下来就在彼此争斗不息,几万年间,只有自相残杀这件事从来都没有停止过。

只不过如今在一些地方的战场上,奔行于前线的已经不是满面倦容的士兵,而是一些娇弱可人的少女。

“那些事情和我们已经没有关系了。”陆久说着,走下了瞭望台。

“陆久。”

听到呼唤,陆久停住脚步,转身朝后面望去。他看到V没有跟着他下来,而是依然站在城墙上,正低头俯视着自己。

“怎么?”

“我们……也会是这样的吗。”V说,“就像这些城池和壁垒一样,总有一天,也会被人们所遗忘的吧?”

会吗?当然会。事实上陆久不觉得有人会记得他们这样的人。如果是在战场上倒下,他们最多也只会变成纪念碑下刻着的名字,很快就被人淡忘掉。而现在他们远离了战场,在后方过着平静的生活的时候,恐怕更没什么人会在忽然之间想起他们吧。

但引起陆久思考的不是这个。他不明白V到底为什么要这么问,是什么让她忽然提起这些,她以前从不在意的事情呢。

“我们也许明天就会被遗忘、也许今天就已经被遗忘了。但那又如何。”陆久说,“我们已经不再是战争的机器。我们有自己的人生。我们自己选择要成为怎样的人。”

“那你会如何选择呢。”

“选择什么?”

“你说过,对异性的追求也是人类的本能,那你会如何选择呢?”V说,“你一直都对民用人形十分关心,你说我们不是因为得到了别人的认可才成为了‘人’,而是因为我们自己认为自己是人、因为我们希望能够被当做人来对待、因为我们……生而为人。那么你会选择一个人形做自己的伴侣吗?”

“……你在说什么?”

“雷蒙说他喜欢我,我说我喜欢你。可你告诉我,雷蒙所谓的喜欢和我说的不是一回事。我不明白这两个‘喜欢’的概念要达到怎样的标准才能被一视同仁,但在我的心中,喜欢一个人的心情,我只知道一种。我想知道我对你的喜欢……你会接受吗?”

陆久没有说话。大雪依然在下无声地个不停,覆盖了目所能及的一切,让陆久产生了一种错觉,仿佛这苍茫而寂静的天地之间,只剩下了他们两个人。他昂着头,默默地注视着城墙上的少女,而那个少女也正在注视着他、等待着他的回答。

一瞬间,陆久忽然明白,他那套“V不懂人类之间的感情”的说辞,已经无法再欺骗任何人了。

是言听计从也好、是逆来顺受也好,V从来没有质问过陆久对她到底是怎样的感情,就算陆久一直都在假装对她所做的一切视而不见。但就算喜欢一个人的感受对每个人来说都不同,那种心情一定都是一样的。这个少女正在向他袒露心声,那也是她对自己人权的宣告——她有权喜欢一个人,无论那个人接不接受,他都不能用“她不懂”这种践踏别人人格的理由,将她的告白置若罔闻。

“……下来。”沉默了一阵后,陆久说。

V手扶垛墙纵身一跃,直接跳下了五六米高的瞭望台。她的姿势犹如猫科动物一样轻捷矫健,落地带起的气流吹开地上一大片积雪,也将她肩膀和头发上的落雪全都荡开了。然后,她用轻快而坚定的步伐走到了陆久的面前。

“为什么要忽然说这些?”陆久问。

“这些遗迹有好几百年的历史了吧。它们一定曾是非常重要的战略据点,让许多人为之浴血厮杀,但如今已经在沉寂中慢慢凋零、慢慢被遗忘了。我不想和它们一样。”V说,“我没有千百年的岁月可等待,人形的寿命很短、比人类还要短,我应该比人类更加珍惜这来之不易的生命。所以,我想知道你的回答。”

V的话让陆久惊讶。他不知道V是从什么时候开始,已经成长成了如此优秀的一个……人。

她是如此的坦诚和勇敢,毫不犹豫地就说出了陆久深藏在内心却不敢去面对的事情。他们的寿命都是非常短暂的,每当面对犹如雨点般倾泻在头上的枪弹的时候,他们的生命也许已经到了最后的几秒。但明知如此,陆久却一直都没有面对过自己的真心,因为他没有那样的勇气。

他总是在瞻前顾后地患得患失,从来都不曾抓紧过那些贵如珍宝却转瞬即逝的美好的东西。他失去了这么多,但直到现在也没有真心反省过一次。他也许马上就将要有一无所有,却还在麻木不仁地虚度着仅有的——

“我知道了。”陆久看着V的眼睛,点了点头说道,“你的提问,我需要考虑一下。在我们结束旅行之前给你答复,可以吗。”

“……可以。”

“你那句话说得很好,我们都该珍惜自己的生命,因为它也许并不像我们想象的那么长。你给我上了重要的一课,谢谢。”

“不,我不是想……”

听到陆久的称赞,V的脸上微微发红,下意识地移开了目光。一瞬间,她身上那股当仁不让的气势消失了,V再次恢复成了那个有些不知所措的羞怯少女。但她的话,陆久已经铭记在了心头。

“好了,我们回去吧。山里的天黑得很快,到时候路就没那么好走了。”

“嗯。”

两个人沿着来时的路向回去的方向走去,但这一次V没有再跟在陆久的身后,而是并排走在了他的身边。V伸出手,轻轻挽住了陆久的胳膊,陆久也没有挣开。

\section*{}

当两个人回到别墅的时候,天已经完全黑了。陆久很庆幸自己及时作出了回去的决定,因为就算没有游荡的野兽,在没有照明、又滑又险的山脊上走夜路,也不是闹着玩的。在门廊上脱下大衣、抖掉上边的落雪,陆久看到屋子里亮着灯——那不是他们离开时壁炉里的火焰,而是有人打开了照明的灯光。

陆久和V一起走进房间,看到屋子里正端正地站着一个少女,那显然是为他们提供服务的人形,她的脖子上也戴着一个蓝色的颈环。陆久还闻到一股食物的香味。 

“欢迎回来,陆先生。”那个人形少女见陆久进来,恭敬地鞠了个躬,“晚餐已经准备好了,因为不知道您去了哪里,我就擅自先为您盛了上来。如果您觉得有些凉的话,我可以为您加热一下。”

陆久看见宽大的餐桌上,晚餐已经布置好了,而且正在散发着腾腾的热气,显然是刚刚摆上来。

“不用了,谢谢。”陆久说,“我出门的时候没有留下联系方式,真是麻烦你们了。”

“不用客气,这也是我们服务的一部分。”那个人形微微点头,“那就等您用餐之后我再来打扫。对了,院子里的温泉已经可以使用,但热水会在早上六点停止循环。”

“知道了。”

“有什么需要的,可以随时用这个和召唤我。请慢用。”

说完,人形少女将一个手环一样的呼叫器放在了桌子上,然后再次鞠躬离开了房间。

“就座吧。”人形少女走后,陆久对V说。

“这里的服务可真周全。”V有些感慨地说道。

“的确,毕竟是曾经接待重要人士的地方。”陆久耸了耸肩,坐在了餐桌前。

晚餐是玉米面熬的粥、清炒的蔬菜还有酱汁调制的冷脍,应该是标准的招待餐。虽然算不上豪华,但对于白天只吃了个干巴巴的汉堡、下午又在雪地里走了一下午的陆久来说,这些饭菜已经足够勾起他的食欲了。

“我就不客气了。”

陆久首先拿起了筷子开始扫荡面前的食物,V看了看陆久,也拿起了餐具。两个人只用了几分钟时间,就把餐桌上的食物消灭了将近一半。

“喝吗。”晚餐吃下一半,陆久拿起了餐桌上的酒瓶对V说道。

“喝一点吧。”

陆久拿过两个酒杯分别倒满酒,然后把其中的一个放在了V的面前。

“你平时,不喝酒吧。人形会喝酒吗?”

“因人而异,有些会喝。但我从来不喝。”

“那怎么今天……?”

“只有你一个人喝的话,一定没有兴致吧。”

“哈。这话倒是不假。”

陆久笑了笑。他自己是从不独自喝酒的,不过V不该知道这些酒桌上的潜规则才对。

“谁告诉你这些的?”

“谢振。他对我说,一个人喝酒会很寂寞,如果你喝酒的话,我一定要陪着你。”

“我就知道是这个老酒鬼。那么,嗯,那个……”

陆久耸耸肩端起了酒杯,却不知道该说什么。因为按照规矩,这时候该说些酒词的,但他想不出来该为什么祝贺。

“新年快乐。”V也端起杯子,轻声说道。

“对。新年快乐……”陆久长长地出了口气说,“今天是旧历的最后一天,是除夕了啊。真正是个喝酒的好日子呢。嗯,新年快乐!”

说完,两个人轻轻碰了碰酒杯。陆久稍稍抿了一口,而V则高高地端起酒杯——

然后被陆久伸手捏住了杯底。

“别干杯。”陆久笑着说道,“我可喝不过你。”

V楞了一下,然后脸上微微一红,也稍稍抿了一口然后放下了酒杯。

“不是说喝多一点,是表示尊敬的吗。”

“又是老谢告诉你的?你不要学酒场上那一套。”

“……是雷蒙告诉我的。”

这两个人,都教了他的副官些什么啊,陆久皱起了眉头。回去了一定要好好教训他们一番。

不过,V已经不是自己的副官了吧。陆久笑着摇了摇头。

虽然V在战区呆了大概只有半年多,但他偶尔还是会下意识地就把她当成了自己的人。偶尔还是会想起那些……

本不该再去想起的事情。

两个人在沉默中一边吃菜一边对饮,不知不觉就把整整一瓶酒喝完了。V不知道自己喝的是什么酒,酒水这种东西对她来说是毫无概念的,但陆久知道那可是瓶陈年好酒,在这个地方估计更得卖上天价了。

不过,管他呢,陆久心想。钟鼓馔玉何足贵,但愿长醉不复醒。

酒足饭饱,陆久有些微醺。他站起身走到了落地窗前,看到院子里的雪已经很厚了,而天上依旧在飘着雪花。北方的雪,有时候会一连下上好几天。

新年的雪是祥瑞之兆,不仅增添了节日气息,更寄托了人们对美好生活的期待。只不过对于陆久来说,也没什么美好的事情值得期待了。

“院子里好像有温泉。”陆久背对着V说,“要去泡一泡吗。”

“……好。”

V提任何问题就同意了,但她估计也不知道温泉是什么。邀请女孩子共赴浴场,可不算什么体面之行,但此时的陆久已经有了五分醉意,心里不再想那么多礼义廉耻的事了。

陆久拉开落地的推拉门,立刻一股刺骨的寒风吹了进来,但陆久却一点也不觉得冷。他走入庭院,看到院子里亮着柔和的灯光。

庭院里种着几颗造型别致的松树,显然是经过了静心的修剪,一丛丛的枝头落漫白雪更显苍翠。一条石板铺成的小路从门前直通院子的角落,那个角落笼罩在白茫茫的氤氲之中,隐约像是一个池塘,应该就是服务生所说的温泉吧。

而那条石板路上则湿漉漉地全是雪水,却没有一片雪花留在上面。陆久猜想,这条路下面大概就是循环热水的管道,这巧妙的设计正是为了清除路上的积雪。

陆久沿着石板路向温泉池走去,在一片雾气里看到了更衣的厢房。厢房里挂着好几件各式男女浴袍,陆久取下一件换好,然后把自己的衣服挂在了衣架上。然后,他又提起厢房里的茶壶倒了一壶热水,拿起两个茶杯走了出去。

当他把茶壶茶杯放在温泉池边的台子上的时候,忽然感觉到似乎少了点什么东西——好像少了个人?

他这才想起来,自己自顾地走进厢房的时候,V好像没有来。他走出那团蒸腾的雾气,看到V正站在门口望着院子里的雪景,显然是不知道陆久去了哪里。

“在这边。”

陆久朝V摆了摆手,沿石板路朝V的方向走去。因为喝了酒的原因,而且正赤着脚,陆久的脚下有些踉跄。他走了两步,忽然感觉脚下被石板绊了一下,然后身子一斜向前跌倒在了雪地上。

“陆久!”V看到陆久跌倒,一个箭步跑到了他的面前,“怎么了,没事吧?”

“呵。”陆久笑了一声。

院子的地面是泥地,不仅种着草坪而且还下了厚厚的一层雪,绊倒摔一跤是不会有什么大碍的。但陆久却没有立即起身,因为他感觉很舒服。

在酒意上涌、全身仿佛燃烧一样燥热的时候,这冰冷的雪贴在皮肤上有种说不出的凉爽。

“醉卧沙场君莫笑,古来征战几人回……”陆久一边喃喃说着一边慢慢爬了起来,“啊。要是在古代,这光景也算不错呢……”

V困惑地看着陆久,不知道发生了什么,因为陆久的样子很奇怪,而且还在说着一些她不太懂的话。

她还没有意识到,陆久已经有点醉了。

“我没事。”陆久说着指了指厢房,然后朝着温泉池走去,“浴袍在那边。”

V去更衣,陆久率先走进了温泉水池。他坐在齐腰深的热水中,感觉意识有些恍惚。他知道热水会让酒精更快地进入血液,饮酒后泡温泉是件危险的事情,但他并不在乎。他的一生经历了太多危险的事情,这一点点的险情不算什么。

“你喝醉了。”陆久闭着眼睛,忽然听到身边有人说话。他睁开眼,看到穿着洁白浴袍的V正坐在他的身边。

“是的。”陆久再次闭上眼睛说道,“饮酒后泡在热水里会醉得更快,也许会失去知觉。所以这次要劳你多留心,不然我大概就要在这里溺水了。”

“既然会醉,为什么还要喝酒呢。”V说。

“那不正是喝酒的意义所在吗。”陆久说,“喝醉了、不省人事,也就感觉不到这世间的痛苦,也就暂时地解脱了。”

“原来酒精,真的是一个精神上的避世之所啊。”

“正是这样。”

“怪不得你那时候……总是喝酒。”

“什么时候?”

“忘了吗。”

陆久想了想,大概想起了V说的是什么时候。她是说在北镇的时候吧。

“能忘了更好……”陆久说。

“那就忘了吧。喝水吗。”

V端起身后的茶具,倒了杯茶递给陆久。

陆久伸手拿过摆在一旁的茶盘丢在水面上,茶盘漂浮了起来,然后陆久把两个茶杯放在了茶盘上。

“喝吧。”陆久举起茶杯说。

“呵。你这气势,还像是在喝酒一样呢。”V发出了一声轻笑。

“就当是喝酒吧,干杯。”陆久说,“这次……我们该为什么干杯呢。”

“为了我们两个人。”

陆久举着杯子的手停住了,他转头看了V一眼。

“说得对。”陆久望着正在飘雪的天空说,“经历了那么多之后我们不仅还活着,而且,此时此刻还能在一起……嗯,为我们自己干杯。”

茶杯碰撞,两个人喝下了各自的茶,V又在茶杯里添上了水。

“17战区,其实是个不错的地方。”陆久说,“虽然有些荒凉,但是和这里一样,大多数时间都非常安静。我其实挺喜欢那里。”

“是吗。”

“嗯。那时我刚到战区的时候,条件非常艰苦,就连指挥部都是随意搭建的帐篷、手下只有六七个战术人形。补给之类的,根本没有专门的后勤队伍运送,只能等旁边的21战区那边的补给车辆给顺便带过来。很多的时候补给青黄不接,我们都要自己外出寻觅食物,简直就像是山里的猎人一样,呵。”

“你有过狩猎的经验吗。”

“我记得自己曾经在山林里经历过一场很长时间的游击战。记忆中的山林就和这里差不多,17战区的山林也是这样的,营地外面不到两公里的地方就是森林,我们的营地其实就坐落在森林的边缘。冬天我们在雪地上清理出一片空地,然后制作一个简易的陷阱,或者干脆挖个坑然后在里边放个铁桶,然后把一些口粮当做诱饵放在上面,很容易就能捕捉到动物……那片森林里人迹罕至,动物们,特别是兔子之类的小动物,几乎没有任何警惕性。有时候一个坑里会掉进去好几只兔子,我们一次吃不完,就用树枝编了笼子把它们圈养起来。但到最后往往姑娘们都舍不得杀它们,最后只好放生……结果兔子们一个个养得膘肥体壮,我们反而倒贴了好多口粮,哈……”

“哈,那可真是得不偿失。”V听了也跟着笑了起来。

“嗯,所以我就学聪明了,捉到的动物直接杀掉剥皮做成肉干。营地里没有小动物可玩了,姑娘们心情低落了好一阵子,不过在那种严酷的环境下,实在是没有余裕去考虑个人心情。你大概不知道,别看我们的姑娘们作战都非常骁勇,但她们可不敢杀小动物。所以营地里剔骨腌肉的活计都是我一个人干的。哎,堂堂战区指挥官竟然还要干屠夫和厨子的活儿,克鲁格说的建功立业到底在哪呢?就这样的事情,我干了一年多,一直到基地扩建才算终于解放了。我一想起来这件事,就觉得又好气又好笑。”

“是吗。真是……辛苦你了。”

虽然没有发出笑声,但陆久看到V说这话的时候全身都在发抖,显然是已经乐不可支。

“我去拿支烟抽。”陆久说着从水里站了起来,但忽然眼前一晕又坐下了。

“你怎么了?”V立刻关切地问道。

“没事,猛一下从水里站起来,有点头晕。”

“你呆在这里,我去给你拿。”

“好。烟在大衣的口袋里,顺便把打火机也拿过来。”

“知道了。”

V说着起身向房间走去,但走了两步也俯身蹲在了地上。

“怎么了?”陆久问。

“没事。我也有点……头晕,感觉天旋地转的,难以保持平衡。”V说。

“你不会也喝醉了吧?”陆久笑着说,“我还以为你千杯不醉呢。”

“一定是因为这热水。”V有点难为情地说道,赶紧站起来快步朝屋里走去。

\section*{}

只消片刻,V就回来了,手里拿着陆久的烟和火机。因为害怕烟叶受潮,陆久没有把烟放在水面的茶盘里,而是把烟放在了池边的台子上。

“17战区的荒野,其实景色也很不错。”陆久抽了一口烟,感觉清醒了一些,“春天四野里万物复苏,从冰雪融化开始,你能够看着荒原如何一天天地变成绿色。夏天的原野里是最富有生机的,各种动物鸟类在林间穿梭,甚至可以说有点吵闹。秋天云淡风轻、天气总是非常晴朗,天空中经常能看到南飞的候鸟……只有秋末的时候,才会有一阵阴郁连绵的天气。而冬天就像现在,很早就会下雪,而且一下就是整整一冬天,让人什么都不想干、只想冬眠。”

“听你说的,简直就像是一片田园的景观呢。”V轻声说。

“要是没有那些敌人,也差不多就是一片田园景观吧。”

“可你说的那些景色,我却一次都没有看到过。”

“那是因为……”

陆久想起当V到来的生活,17号区域已经成了N17战区,不仅营地扩大了,而且附近为了防止有敌人埋伏,军营外方圆十公里全部被清理成了平地,还在五公里的位置设立了岗哨和战壕。而且V在N17战区只呆了短短的五个月,从秋末开始没有等到春暖花开,就因为陪着自己“逃亡”而离开了战区。

“你只是没有赶上而已。”陆久说,“其实这样的风景哪里都会有,17战区不过是这个世界的一个小小的角落,这个世界本来就有很多美丽的风景……只是我们没有看到。”

“我们以后,会看到的吧。”

“嗯,一定会的。”

“陆久。”

“唔?”

“我能……”

“问吧。”陆久看着V笑着说,“以后有什么问题直接问就好,省去这种毫无意义的开场白吧。”

“嗯。不过,感觉你今天有点不一样呢。”

“怎么说?”

“你比以前健谈多了。很多事情……17战区的那些事情,你以前从来都不会说的。”

“那是因为我喝醉了。”陆久说,“也许我明天就会忘记今天晚上的谈话,谁知道呢。酒精会刺激人的大脑,影响人的记忆力啊。所以我今天无论说了什么,只要明天不记得了,就和没说一样,不是吗。”

“那我就问了。”

“问吧。”

“你对95……”

“嗯?”

但V没有再说下去,只是默默地抬起头,望着漆黑的天空。

“……算了,还是不问了。”过了一阵,V轻声说。

“95是我最亲密的战友,一直到最后都是。我们无数次地并肩作战、出生入死,手把手一起建立了N17战区。”陆久说 ,“但你要问我对她的感情,我对她只有同志之情。我知道她对我的想法,但我却没有告过诉她我自己的想法……我感到很遗憾。无论事情是怎样,我都该告诉她的。但我没有那样的勇气,也没有……那样的心情。那时候的我总觉得来日方长,心里想的只有战斗。可惜……一直到最后,我也……”

“别说了。”V低声打断了陆久的话,“我明白了。已经回答得足够了。”

陆久点点头,没再说下去。他长长地出了口气,然后靠在水池边,又点了一支烟。

“对不起。”V轻声说。

“怎么了。”

“我不该问那些让你难过的事情。”

“没什么。事情都过去那么久了不是吗。”

“可你心里还在介怀吧。”

“人的心里总有些还不清的债。每个人都是这样的,不用在意。”

“我能为你做些什么吗。”

“什么?”

“能够减轻你痛苦的事情。什么都可以。”

陆久笑了。

“有啊。”

“告诉我。”

“好好活着。做个好人。”

V不可思议地盯着陆久看了一阵,然后也笑了。

“你可是提了些不得了的要求呢。”

“做不到吗?”

“我不知道。说实话,感觉有点强人所难。”

“那就换一件能做到的吧。”陆久说,“给我倒杯水。”

V用手指把水面上的茶盘勾了过来,给陆久倒了一杯水。

“谢谢。”陆久喝了一口说道,“我觉得你又有点以前的样子了。”

“什么时候?”

“南美洲那时候。”

“我……以前是什么样的?”

“处处挑人的毛病、很难搞好关系,但倒是很坦诚,所以总是当面找我的麻烦。”

“后来我不那样了吗。”

“嗯。当我再见到你的时候,你就变了。变得阴郁消沉,似乎对这个世界没有了一丝好感。那两年,你去哪了?”

“我……不知道。我只知道我在南美洲的时候……失去了记忆。再次苏醒的时候,已经是五个月之后的事情了,我走出培养槽,放在面前的是熟悉的服装和武器。后来就一直在总部做一些无足轻重的琐碎工作,有时候没有工作的时候,就什么都不做地备勤。直到一年多之后的有一天,总部忽然给我一封调令,让我去某个偏远地区工作……然后,就遇到了你。”

“你还记得南美洲的事情吗。”

“记得。”

“一直记得?”

“嗯。”

“那你见到我的时候为什么装得像是第一次见面一样?”

“因为我……不知道该怎样做。我从来没有经历过和一个人分别那么久后的再见,我不知道这么长的时间你又有怎样的变化,我甚至不能确定那个人……到底还是不是你。”

陆久注视着V,沉默着没有说话。他看到V的脸上满是不知所措,显然是对自己的谴责感到十分惶恐,就像一个做错了事情却不知该如何挽回的孩子一样。

“不是我,还能是谁。你这个傻瓜。”

“我是个傻瓜。”V低下头说,“对不起。”

“我不是在责怪你。”陆久抱怨地说道,“别动不动就道歉,你这样让我也很紧张。不必那么小心翼翼的,我们已经很熟悉了吧,和以前那样随意点不行吗?”

“可你不是说以前的我很难相处——”

V的话还没说完,脸色突然一沉。陆久还没来得及反应过来发生了什么,就被V从水里提了起来仰面朝天地扔在了水池外的雪地上。然后,V飞身扑向了陆久,把陆久死死压在了地上。

“你搞什么——”

“别动!”V低声喝道,“听到那个声音了吗,从西北方向传来的,是敌袭!”

远处似乎的确隐隐传来了的接连不断爆炸声。陆久立刻安静了,职业军人的素养让他对“敌袭”两个字格外敏感。但当他侧耳倾听了一阵后,他紧绷的身体放松了下来。

“时间过了午夜了吧。”陆久说。

“是刚刚过。但是这——”

“是贺岁的爆竹声啊。我们已经步入新的一年了。”

“爆竹是……?”

“不先从我身上下去吗?”

听到陆久淡定的语调,V这才将信将疑地从陆久的身上爬了下来。

“爆竹是为了发出声响而制作的小型爆炸物,燃放爆竹是庆祝新年的重要传统。现在我们该拜年了。”陆久说着坐了起来,“该说点什么新年贺词呢。啊,真麻烦,还是新年好吧。”

“……什么?”V显然对陆久的话完全不解。

“说‘新年好’就行了。”

“新年好。”

“……嗯。好。”

V显然还是不怎么明白新年的喜庆,而陆久也没太大感触。新的一年对他们来说,又意味着什么呢,似乎一切如故。

说到底,他们到这个地方来,不就是为了逃离合家欢庆的春节的吗。

陆久全身湿漉漉地坐在雪地上,忽然感觉有些冷,醉意似乎已经下去大半了。他记得刚才的对话原本很融洽,但刹那间他忽然又在心中感到一阵失落的空虚。是因为酒力已过的原因吗。

“我感觉有点冷。我们回屋里吧。”陆久说。

“好。”

\section*{}

两个人在厢房里换了一身干燥的浴袍,然后回到了屋子里。屋里餐桌上已经被收拾得干干净净,而且壁炉里也填了新的木柴,看来是那个服务人形已经来过了。陆久半躺在宽大的沙发上,V则在从外面带进来的茶壶里添满了水,然后又倒了两杯茶放在茶几上。

“不去休息吗?”V坐在陆久的身边说。

“守岁啊。新年的凌晨,是要一直清醒着到天明的。不过我也不怎么困。”

“你说的‘年’,到底是什么?”

“那个啊。你知道神话传说吗。” 

“神话传说……?”

“是一些人们口口相传的故事,描述关于……那些混沌的年代,所发生的传奇。”

“我知道这个概念,但没听过多少实例,所以不太理解。” 

“现代的很多习俗都来自于古代的传说,因为那时候人们对世界的认识还不像现在这样客观。他们认为很多自然现象都是更高一级生命的意志,而自己所做的仪式性行为,会对这些高级生命产生影响。春节就是一种这样仪式的集合体。”

“原来是这样。”

“传说中‘年’是一种凶猛的怪物,每到冬天就会出来袭击人们的村落。人们发现这种怪物害怕红色和巨大的响声,于是在灯上装上红色的灯罩、穿上红色的衣服,并且点燃竹子来驱赶这只怪物。这个习俗流传下来,演变成了挂灯笼和放爆竹。”

“你那时贴在公寓门前‘对联’,也是这样的一习俗?”

“是啊。古时候人们相信天上的神明会在新年期间来到凡间巡查,所以就把自己的祈愿写在门前,希望路过的神明能够看到。”

“天上真的有神明吗。”

“这个……”

当然没有,陆久本打算这么说的,因为就算是小孩子也知道传说和现实的区别。但他转念一想,那只不过是在这个国家里不推崇宗教而已。举国信教的地方有很多,在那些国家里人们对常识的认识,则又是另一种状态了吧。

“我也不知道。”陆久说,“如果不信奉神明,那就是没有;但对于信奉神明的人来说,也许神明就在他们的心中。”

“你是不信神明的吧。”

“是的。对我来说……这个世界上是不存在神明的。你呢,你怎么想?”

“我想,对于人形来说,人类就是神明。”

陆久沉默了。他只是随口而出的无心之言,因为他知道人形和人类不同,信仰这种东西她们恐怕根本理解不了。但他没想到V竟然会说出这种话。

的确,那些自己不可违抗的创造者、掌握着自己生杀予夺之权的存在,不是神明又是什么呢。但这样的回答,让陆久莫名地感到胸口一阵压抑。

人类到底算是什么,陆久心想。自私、贪婪又怯懦,不可一世地自认为是这个世界的主人,只有在欺凌比自己更加弱小的生物时才能感到一丝可笑的尊严。一边肆无忌惮地挥霍着资源、不断侵占着其他生物的生存空间,一边不停地相互勾心斗角、不断地对彼此发动战争,时时站在自我毁灭的边缘。这种文明简直就是这个星球上的公害。

“人类不是神明,只不过是一群渺小又可怜的生物罢了。”陆久说,“因为无助,才创造了神明作为祈拜的偶像;因为害怕孤独,才给创造了人形作为相依的伙伴。事实上,人类之所以总是把其他生命当做精神寄托,是因为他们深知最不可信的,就是他们自己。”

“……你对人类的评价很悲观呢。”听到陆久这番话,V显然有些吃惊。

“也许因为我是个不正常的人类吧。”陆久自嘲地笑了笑,然后端起茶杯喝了一口,“难得良辰美景,不要说这些让人烦闷的话题了。对了,那时候我离开之后……你又在那里呆了一段时间吧。” 

虽然刻意回避了地点的名字,但V显然知道陆久说的是什么地方。

“是的。酒吧的宁先生招募我到酒吧做招待员,我在他那里居留了大概有四个月的样子。”

“他的生意一定不错。”

“嗯,每天前来喝酒用餐的人很多,我经常都很忙。我之前不知道酒吧竟然是如此忙碌的地方。”

“他那间酒吧,大概也不是什么忙碌的地方……不过你感觉忙碌的话,那是因为你的到来,帮他招揽了很多客人吧。”

“客人的增加,和我有关?”陆久的话让V稍稍有些吃惊。

“怎么,你一点都没有意识到吗。”陆久说,“宁老板也没有说过吗?”

“没有,我不知道……为什么,会和我有关?”

“当然是因为,你是个美人啊。”

“我……”

听到陆久的恭维,V没有害羞或者否认,而是露出了有些茫然的表情。她抬起胳膊,看着自己从宽松的浴袍袖子里伸出来的洁白如玉的前臂和双手,仿佛在看着别人的躯体。

“原来,我是美人啊。”

“只要长着眼睛的人,都会这样认为的。”

“……是吗。”

“是啊。酒馆的生活怎么样?我想那是你第一次和那么多人类接触吧。”

“的确,虽然是民用人形,但我从来没有做过服务行业,第一天工作就手忙脚乱的。不过还好,客人都很宽容、宁先生也很照顾我。对了,船长和警长两个人也是酒吧里的常客,他们给了我不少帮助。”

“那两个人啊。后来他们没再发生冲突吧。”

“没有。不过据宁先生说,他们以前的关系有些紧张,还是你从中协调后才缓和下来的?”

“咳,我只是和他们打了一架。因为没有分出胜负,所以我就建议他们以后别再相处拆台了。”

“是你把他们都打败了吧?警长都告诉我了,而且说了不少你的好话。”

“……别提那个了。后来呢?”

“后来基本就是日复一日的劳作了。船长曾经邀请我出海,我们去了挺远的海上捕鱼,其他的就没什么了……对了,有件事我一直没有告诉你。”

“什么事?”

“在我在酒吧工作的时候,警长曾经找过我一次,请求我协助他们的一次……作战行动。他们在当地发现了贩售违禁药物的团伙,但是因为战斗力不足,所以希望我能充当战斗人员。”

“你同意了,是吧?”

“……是的。警长说,如果你在的话一定会出手相助,所以我就……同意了他的请求。这件事一直没有向你报告,对不起。”

陆久忽然想起来,那时候他返回北镇去“偷走”V留下的武器的时候,曾经注意到留下的弹药很少。现在想一想,V那时候竟然真的使用过那把枪。

“造成人员伤亡了吗。”

“我击伤了两个敌对目标,但是没有杀死他们。”

“你果然拥有特别的火力许可。”

“是的。因为我是一个……非法的人形。”

“你在一个非法的人类面前多愁善感什么呢。继续说之后的事情吧。”

“那次行动船帮的人也参与了,船长还在行动中受了伤,有好一阵没有去酒吧。至于行动之后的事情我不太了解,不过看警长的样子应该是解决了。在那之后就没什么值得一提的事情了,我在酒吧里又工作了大概一个多月,郝丽安女士出现了,将我召回了总部。”

“是吗。”

“嗯。我那段时间里经历的事情,大概就是这样。”

“我那时候不辞而别……难为你了。”

“没什么……对于我来说,这也是和人类相处的宝贵经验。不过你欠宁先生的钱,后来还清了吗。”

“那家伙啊。我已经还得绰绰有余了。”

“那就好。”

“薇。”

“嗯?”

“比起海边,你更想去的是北镇吧。”

“嗯……有一点。”

“呵,你也学会掩饰自己的想法了呢。其实你一开始想去的,就是那里吧?”

“嗯。但是我怕你……不想去。”

“我是……也没有不想去。”

陆久确实是不怎么想去,因为那里实在没有留下什么美好的回忆。不过他现在已经想开了。

既然有V陪在身边,他还有什么可瞻前顾后的呢。难道他还怕睹物伤情吗。

“要是你不想去的话,就不去了。”察觉到陆久那一瞬的犹豫,V说道。

“去吧。明天一早就出发。”陆久说,“我们下一个目的地就去北镇。”

“可是雪还没有停。这么大的雪中走山路,太危险了。”V摇了摇头。

“那要等到什么时候?”

“至少等到太阳出来。”

“北方的雪可是会下一整个冬天的,要是太阳一直不出来怎么办,总不能等到春天到来吧?”

“那就一直等下去好了。”

“准备持久作战了吗。”陆久笑了,“到时候公司里该乱套了,那不就和上次的逃亡一样了吗。”

“那不是正好吗,反正你也不喜欢人多的地方。”V不以为然地说,“这样就可以名正言顺地逃亡了。”

“说得好。那就一直呆在这里,然后祈祷大雪就像是冰川时期一样,一万年都不会停下吧。”

“要是那样就好了。”

“简直就像是在说梦话。”

“没有可能吗。”

“怎么可能。”

虽然像是在说梦话,但这种事情要是能实现的话,陆久有那么一瞬间,还真的稍微有些当真地想过。

大雪围困了这座山林,而冰雪消融的时候,永远都不会到来。这样他们就可以“名正言顺”地,或者“迫不得已”地,将这一起共度的时间无限地延长。

前提是……如果真能实现的话。