\chapter{昨夜的星辰(七)}

\section*{前言}

活着、死去,这些事情对于以前的陆久来说,并没有什么特别的意义。他总是在按部就班地活着,从不曾考虑过生命的意义,因为他认为自己这样的人,就算是死去也不会有人为之哀悼。

这样的人,是从什么时候开始明白生命的价值的呢。大概就是从看到那些活着的人,为了死去的人而悲痛的时候。那时候,他才开始了思考:自己失去珍惜的人会是怎样的心情,自己死去是否也会有人承受同样的悲伤。

因此,他也就感到了死亡是沉重的、剥夺人的生命是罪恶的,感到了活着是一件值得庆幸和珍惜的事情。

\lineseparator
\section*{}

早上,陆久从房间里出来的时候V已经在客厅等候了。她的表情很平静,似乎听从了陆久的建议,没有再去想关于以前的事情。两个人如往常一样在八点三十分准时抵达了办公室。

这一天的工作和昨天一样,早早地就有一大堆信件堆在面前,几个人一直忙碌着到了中午才稍稍喘了口气。到了下午下班的时间,谢振首先起身离开了办公室,但雷蒙却没有走。当陆久正想要闭眼休息一会儿干涩的眼睛时,雷蒙走到了他的座位旁边。

“陆主任,我有点事情想向您……汇报一下。”

“唔,怎么了?”

陆久揉了揉眼睛,坐直了身子。但雷蒙却没有继续说下去,而是朝着V望了一眼。

“我能和您单独谈谈吗。”他说。

陆久疑惑地看着雷蒙,心里有点纳闷。难道有什么需要保密的事情要说吗。

“知道了。薇,请你去先把车热一下吧。”

“好的。”V看了陆久一眼,表示明白地点了点头,起身走出了办公室并关好了门。

“不好意思,我不是不信任陆薇小姐,但这件事知道的人少一点会比较好。”V离开后,雷蒙从抽屉里取出一件东西递给陆久,“陆主任,我们今天收到了……这个。”

陆久接过来一看,是一个很普通的蓝色信封,寄件人是公司总部、收件人是一个他不认识的名字,地址则是这个城市下辖的某个区县。

看起来这只是一封从总部发出,需要代为递送信件。陆久知道由总部发出的信件是不需要审查的,直接按照流程投递即可,却不知雷蒙为何要把它交给自己。

“这封信,有什么问题吗。”陆久说。

“蓝色的信封……是装阵亡通知书专用的封套。”雷蒙轻声说。

“……是这样吗。”

陆久不由得感到了惊讶。再有十几天的时间就是春节了,在这样阖家团圆的节日前收到这种东西,对一个家庭来说会是怎样的噩耗,就连曾见惯了死亡的陆久也不忍去想。

“我们该怎样处理这封信件呢。”陆久说。

“依照惯例,信检中心在收到‘蓝信封’的时候会向综合办公室汇报,然后由综合办去安排公司专门的人员将信件亲自送往收件人的家中、并进行安抚工作。这件事……通常是由分公司的一把手负责出面。”

“那么,我是要将这封信亲自交付综合办公室吗。”

“流程上本该如此,不过……”雷蒙似乎有些犹豫,“有件事我不知道该不该说。”

“说吧。”

雷蒙看了看陆久,又看了看门口。然后,他仿佛下定决心一般,深深吸了一口气。

“这位阵亡的士兵我认识,他曾经是谢振手下的人。在谢振刚刚来到信检中心的时候,这个人曾经来找过他,那时我们还一起吃过一顿饭。所以我在想的是,这件事要不要和老谢说呢。如果报到综合办的话,迟早是会被老谢知道的,而我们要是没有提前知会他……到时候他恐怕不会高兴吧。”

陆久沉默了一阵,然后伸手摸了摸下巴。战斗总会有牺牲,如果是以前的陆久,他是不会为这种事情动容的。但现在不同了。

他已经不是运筹帷幄的指挥官,甚至就连战斗人员都不算了。他必须站在别人的角度设身处地地去想一想,必须像个普通人那样去同情、去理解别人的心情。

所以他意识到,这件事远比他想象的要复杂得多。

既然是谢振的往日战友,那么于情倒是该让他知道,但陆久觉得没有理由这样做。谢振已经不在特勤中队了,这件事也就和他没有关系了……而且这样悲伤的事情,谢振知道了一定会影响他的情绪。但如果不告诉他,就像雷蒙所说的,以后老谢知道了恐怕也会不高兴。这让陆久一时犯了难。

“我觉得,老谢知道了这件事只会难过,还是不要让他知道的好。”沉思了一阵后,陆久开口说道,“如果我们去把这封信送到家属手中,符合公司的规定吗。”

“公司只是说通知书须内部人员亲自送达家属,但没有明文规定必须由谁送……由分公司的最高负责人出面,只是为了表示公司的重视而形成的惯例。您如果您愿意去的话,综合办公室应该不会反对,毕竟,如此沉重的事情……您也能理解吧,谁都不会愿意去做的。如果有人主动请缨,我想他们反而要松一口气。”

“是这样。这封信,先放在我这里吧。”陆久点了点头说,“我考虑一下怎么办最为妥当,然后我来和综合办汇报。”

“好的。”雷蒙点了点头没再说什么。他把那个蓝色的信封放在陆久桌子上,然后离开了办公室。

\section*{}

陆久看了看那个信封,上面的名字他并不认识。这是当然的,他从来没有和特勤人员打过交道。他把信放在了自己的抽屉里,然后想了想,又把信取出来塞进了自己的兜里。

陆久走出离开办公室,V正在门口等着他。他一言不发地上了车,和V一起回到了公寓。两个人依然是在昨天吃饭的地方解决了晚饭,但这次陆久相当沉默,吃饭的时候一句话都没有说。

“发生什么事了吗。你看上去精神不太好。”

回到公寓,陆久依然坐在客厅里发呆,这时候V终于忍不住问道。

“我们收到了一封……阵亡通知书。”陆久长长地出了口气,从兜里取出那封信放在桌子上,“我在思考该如何处理这件事。”

“按照标准流程,该怎样处理的呢。”

“交给综合办公室,让他们安排人员去递送。”

“你不打算这么做吗。”

“是的。这个阵亡的士兵……是谢振曾经的战友。我不想让他知道这件事。”

“为什么呢。”

陆久没有说话,抬头看了看V。

“你觉得我们应该让谢振知道这件事吗。”陆久说。

“我觉得……应该吧。”V想了想说,“毕竟是曾经和他有关的人,他有权知道。”

“正因为如此,我才和你的想法相反。谢振是非常珍重战友的人,他知道了这件事也不会让死者复生,但一定会非常悲痛。我认为这只会给他徒增烦恼。”

“可他总有一天会知道的。”

“如果他知道的时候这件事已经过去了很久,至少他就不会那么难受。”

“我觉得我们不该向谢振隐瞒这件事。那样做……不诚实。”

陆久看了看V,然后笑了笑。是啊,陆久心想,“诚实”。那也许正是V身上最美好的品质。但对于人类来说,过于诚实有时候也是一种鲁莽。

“你总是很诚实,薇。我毫不否认诚实是一种美德。但有时候我们讷口不言,才是真正的人文关怀。”

“我不明白。”

“你一定认为说出真相并没有错。当然,如实陈述,任何人都会觉得是合乎情理的。但在人类的世界里却并不总是如此。我们了解到的真相,不去透露给那些不需要知道的人,这样那些人就能活得更心安理得——既然他们知道了也不会改变任何事情、知道了反而会徒增苦恼,那我们为何非要让他们知道不可呢。所以说,有时候不把自己知道的事情说出来,也是一种慈悲。有句话叫做‘沉默是金’,就是说一些时候,保持沉默才是最恰当的。”

“ ‘知道了也不会有所改变,所以还是不知道为好’,以前你曾说过这样的话,但你后来也承认这是一种逃避。所以这样的思考方式我不太理解。我觉得无论如何那些事实就摆在那里,如果早一些知道的话,也许就……”

V似乎对陆久的逻辑早有微词,但当她无意间扫了一眼桌子上的信封的时候,她的眼睛忽然微微睁大了。

“怎么了?”

陆久有些奇怪地说道。他注意到V那总是很平静的表情,出现了一些细微的变化。

“不,没有。那个……”

V的样子忽然慌乱了起来,就连说话都有些语无伦次。

“你想起什么事了吗。”陆久更加不解了。

“没什么。”V说道。

“你的样子可不是‘没什么’的样子,我看显然有什么吧。”陆久又疑惑又好笑地说,“你知道自己根本不会撒谎的,更不会隐藏自己的想法。”

“……但是,我不想说。”

陆久闻言一愣。V拒绝回答他的问询,这还是第一次。明明昨天晚上陆久的提问让V回忆得非常痛苦,她也不曾推脱。

“那能告诉我不说的理由吗。”

“因为……‘沉默是金’。”

陆久皱起了眉头。这家伙在说什么啊。词语学得倒是很快,可这个显然用错地方了吧。

“我已经知道了你有话要说,再沉默就没用了。”陆久说,“你不想说我也不会勉强你,但要是和我们眼前的这件事有关的话,我还是希望你能告诉我。因为我正在为此而伤脑筋,也许你知道的事情能对我有所帮助。”

“……不会有帮助的。”V低声说,“只会起反作用。”

听了V的话,陆久也沉默了。他从桌子上拿起那封信,凝视了一阵,然后又放在了桌子上。

“就算你不说,我也能大概地推断出来。”陆久说,“这封信是从总部寄来的,你不可能知道里边的内容。那么只有一个可能:你认识信封上的这个名字,对吧。”

“……”

V沉默着没有出声。

“你说会起反作用,莫非我也认识他?”陆久接着说道,“不,不会的。我一个特勤人员都不认识。不过我倒是见过几个,但也非常有限。但这倒是个有价值的消息:你不是热衷交际的人,认识的特勤人员会有谁呢。这里面和我见过面的就更少了。我想我的搜索范围已经缩小了很多了。”

“……铁杉树行动。”仿佛放弃了一样,V终于叹了口气,开口说道。

“铁杉树”?陆久觉得这个词有些耳熟,但一时想不起在哪听过。

“在南宁。”V又接着说道。

南宁……?

陆久一时间还是没有把这两个词联系起来。但当他意识到V所说的事情的时候,他感到如遭雷击、脑海一片空白。

毫无疑问V说的是那一次,因为陆久只去过一次南宁。

“难道他是……”

陆久难以置信地说道,他感到嗓子有些发干。

“是的,”V低声说,“南宁那次行动的四名特勤人员之一。我曾经在那次行动的简报上看到过他的名字。”

\section*{}

陆久感觉有点头晕,于是轻轻靠在了沙发的靠背上。这么说,眼前这名牺牲的士兵、谢振的战友,是在那次行动中阵亡的……

若是这样,那么他可以说是被陆久杀死的。

“……没事吧。”看到陆久失魂落魄的样子,V屈身凑到了他跟前,关切地问道。

“没什么。”陆久摇了摇头。

“你的样子可不是‘没什么’的样子。”V学着陆久说道。

“我……只是感觉有点,出乎意料。我实在没有想到会是那些人……”

“你的推测完全正确,我还以为……你都知道了。”

“……我其实是骗你的,我一个特勤人员都不认识,又怎么会知道你认识的是哪个呢。”陆久勉强笑了笑,“我甚至觉得我们都见过的特勤人员根本不存在。”

“你……”

听了陆久的话,V才意识到自己被骗了,一时间有点生气。但看到陆久的样子,她又生不起气来了。

“好了,我知道怎么回事了。”陆久说,“我还真得感谢你,提供了一条非常重要的信息。”

“然后你打算怎么办?”

“按原计划办。我会亲自把这封信交给这位士兵的家属的,所以这件事也请你保密。”

“还是交给综合办公室吧,不要勉强自己。”

陆久看了看桌子上的信、又抬头看了看V。他看到V虽然表情平静,但眼里满是关切。于是陆久长长地出了一口气。

“这件事确实有点麻烦。不过我想还是我亲自出面比较好,我觉得我还能应付。”

“我和你一起去。”

“没有必要。”陆久说,“这种事我一个人去就够了。”

V没有说什么,只是看了陆久一阵。

“我要去。”她轻声说。

陆久抬起头看着V的眼睛,但V毫不退缩。陆久在她的眼中看到了熟悉的眼神,让他想起了在战区的时候,他决定向临近战区发起支援的那个晚上。

“我又不是去战场。”陆久说。

“可你的脸色,比去战场时还要沉重。”

“……好吧。”过了一阵,陆久终于无可奈何地说道,“那就一起去吧。”

听到陆久的这句话,V才移开了盯着他的目光。

“你的表情很难看,所以一个人去的话我不放心。”V小声说,“如果觉得我碍事,我可以在车里等你。”

“不,我只是不想让你也去承受这种……算了。”陆久说,“说实话,我一个人去心里也没把握,你如果愿意和我一起,我倒很高兴。”

“好的。”听了陆久的话,V的表情明快了许多,“那我们什么时候出发呢。”

“很快,不过在那之前我们有些东西要准备一下。我记得在战区的时候给你发过常服,那套衣服还有吗。”

“有。一直放在我的行李中,但从来没穿过。”

“很好,明天上午把它拿到洗衣店整理一下。我的军装……被没收了,我也要去找套像样的衣服来。”

“要去商店里买吗?”

“不,商店里没有那种服饰。我得找个人帮帮忙。”

\section*{}

说完,陆久拿起电话拨了一个号码。没过多久,电话接通了,里面传来一阵嘈杂的吵闹声。

“晚上好,准将先生。”陆久说。

“什么事?快点说,我现在很忙。”

“我听出来了。那就长话短说吧,我需要我的军装常服。”

“这种事可别找我。想要的话,自己去找克老爷子请示。”

“那就换个说法,我要一套和我以前那套尺码完全一样的常服。”陆久说,“这种公司人手一套的衣服,犯不着去找大老板吧,分公司难道没有后勤处吗?”

“一套衣服倒不在话下,但老板要是看到你穿着军装,那么给你这套衣服的人就不会有好果子吃了,知道吧。你以为你的军装为什么被扒了,难道你不知道克老板有多烦你?给你军装,那不是打他老人家的脸吗?”

“……我不会在公司里穿,而且我保证不会让任何公司里的人看到。”

“你要去干什么?”

“与你无关。”

“好的,我也没空听你瞎撤淡。再见。”

“别挂!我真的急需这东西。”

“少废话。要么告诉我理由,要么就别浪费我的时间了。”

“我要去……给一个士兵的家属,递送阵亡通知书。我不能穿着便装去。”

“……”

电话那边沉默了。陆久听到里边的嘈杂声渐渐小了,想必是手持电话的人去了一个相对安静的地方。

“行啊,有你的。”过了片刻准将先生说道,“我正琢磨该怎样让你死心呢,没想到你竟然找了个我无法拒绝的理由。”

“是你自己要问的,我没必要编这种故事。”陆久稍微有些恼火地说道。

“我到希望你是编的,这样我就会对你的创作才能刮目相看了。”准将先生嘲弄地说道,“我会跟那边的人打个招呼,不过你办完这件事,要马上脱下那身衣服好好藏起来才行,不然我麻烦就大了。还有别的事吗?”

“没有了。这次算我欠你一个情。”

“哼,你欠我的情只有一个吗。”

说完一句挖苦的话,对方挂断了电话。陆久也没说什么,把手机随手扔到了一边。

“明天你先去准备自己的衣服,我去公司拿了东西就出发。”

“知道了。”V说,“我们……该怎么做呢。”

“你觉得呢。”陆久反问道。他也没有这样的经验,所以想先听听V会发表怎样的看法。

“我……觉得只能把事情如实相告,并争取家属的宽恕吧。”

不可能得到宽恕的,陆久心想,这可不是鞠躬道歉就能算了的事情。而且他们是代表公司去安抚遗属的情绪,而不是去负荆请罪。

“那要是家属不肯宽恕呢。”

“他们如果要责难的话……就由我来承担吧。”

陆久摇了摇头。几乎毫无和人相处的经验的V思考方式太过简单了,她的意见陆久实在无法采纳。

“不能那样。”陆久说,“我们此去的目的是为了抚慰家属的情绪,而不是去刺激他们。我们不能让任何人知道这个士兵的死和我们有关,包括谢振和雷蒙。”

“那不是成了……”

“薇。那些遗属,失去了他们的家庭成员,那是他们最重要的人。”陆久打断了V的话,“他们要是知道自己的亲人只是毫无价值地——” 

如果是我死了,杀死我的人向你请求宽恕,你会怎么想?陆久想问V这个问题,但他没有说出口。

“别忘了我说过的话,‘沉默是金’。”

“……我知道了。”

“我们得把这通知书交给他们,但之后的事情只能随机应变了。到时候事情就让我来处理吧,你不必说话。虽然我也不是什么擅长交际的人,但至少在人情世故这方面,我比你经验稍微多一些。”

“好的。”

\section*{}

早晨陆久很早就去了公司,当他经过公司大门的时候被警卫处拦了下来。

“陆主任,昨天晚上后勤处委托我们将这个交给您。”执勤的人形少女恭敬地向陆久递上了一个包裹。陆久接过包裹摸了一下,里面的东西显然是衣服。

效率真高啊,陆久心想。大人物说话就是不一样。

“谢谢。”

陆久把衣服放在车里,来到了办公室。谢振和雷蒙都还没有来,办公室里一个人都没有,于是陆久便留了一张字条说民给自己和V要出去一天。

回到公寓,V不在屋里,应该是带着衣服去洗衣店熨烫了,于是陆久决定先试一试刚取来的衣服。他脱下身上的便装,穿好军装走进洗手间,看着半身镜里的自己。

那套常服非常合身,就像量身裁剪的一样,他的样子和在战区时没有多大区别。不过不同的是,镜子里的那张脸却有所不同——也许是想到今天要去做的事情,虽然他努力地做出平静的样子,但眉宇间还是难掩凝重的神色。

自己就是带着这幅表情四处走来走去的吗,陆久心想,怪不得会被V担心。他做了几次深呼吸、用力揉了揉脸抹去了脸上的阴郁,然后开始对着镜子里的自己思考关于那封慰问信的事情。

无论他怎样做,对于那个士兵的家属来说这都是个难以接受的噩耗,陆久很清楚这一点。但他希望自己能尽量妥善地完成自己的任务。

敲门之后,该怎样自我介绍、又该怎样打招呼呢。该站在离门口多远的距离比较合适?是该先敬礼,还是先握手?陆久一遍遍地揣摩着自己的动作,在心中模拟着悲痛的家属可能出现的反应已经自己该如何应对。当陆久听到门口传来有人进来的动静的时候,他低头看了看手上的计时器,发现时间已经过去了一个多小时。

陆久走出洗手间,正碰上刚刚进门,手里还拿着熨烫过的衣服的V。看到衣帽整齐的陆久,V楞了一下,她的眼睛微微睁大了。

“怎么了。”陆久有点不解地问道,他注意到V的表情有点奇怪。

“没什么。你的衣服……很合身。”V小声说了一句。

陆久更加不解了,他整理了一下领带,又低头看了看自己的衣服。这身衣服和他当时在战区的时候穿的应该是完全一样的,合不合身V总不该现在才发现。

“和在战区的时候有什么不同吗。”

“和那时候完全一样。”V说,“所以才觉得有点……熟悉的感觉。”

“也就是说看着还算顺眼?”

“是的。”

陆久这才恍然大悟,V大概是想称赞他的装扮,却没有合适的词汇可以形容。形象得到了V的肯定,让陆久稍稍多了一点自信。

“那就好。虽然不能光明正大地穿,呵……但我希望这套制服能符合我所要扮演的形象。”陆久说,“你也试试自己的衣服吧,如果没有问题我们就准备出发。”

“好。”V说着把手里的衣服放在沙发上,然后开始在客厅里换衣服。这种毫不避讳的做法陆久并不感到意外,不过在V脱下衣服的时候陆久还是转过身回避了一下。

“好了。”

听到V的话,陆久转过身。当他看到V的新装时候,不由得也和V刚走进门时那样楞了一下。

她那身衣服非常……合身。陆久忽然明白为什么V要这么说了,因为他一时间也找不到恰当的词语来形容。

白衬衣、黑长裤,深灰色的小西服和深蓝色的领带,女装的常服要比男士黑白两色的常服的色彩丰富一些,但依然保持了军装庄正素雅的风格。以前在战区的时候V总是穿着作为民用人形时穿着的那身洋装,虽然优雅但是谈不上什么气质,看久了就觉得不过是日常服饰而已。但眼前的这身制服穿在身上,让人感到庄严而不失大方,秘书官端庄秀丽的气质立即油然而生。陆久有些后悔,早知这身制服的观赏效果如此出众,他那时就下令所有人在营地都要穿常服了。

“好,很适合庄严肃穆的场合。”陆久点了点头赞许地说道,“希望我们能够……做好这件事。”

“听起来你好像没什么信心。”V说。

“确实。”陆久叹了口气,“虽然参加过很多战斗,但我从来没有和士兵的家属们会过面,更何况是遗属。我也不知道,该怎样……”

“我们只要把这个消息传达给他们就行了。”

“我明白。虽然他们会很悲痛,但我们是什么都改变不了的。不过这件事——”

“别去想那些了。”V打断了陆久的话,“你不是说了吗,我们是去安抚他们,而不是去请罪的。顺其自然吧。”

“你说的对,想太多也没用。”意识到自己正在从V那里寻求安慰,陆久有些不好意思地笑了笑,“时间有限,我们出发吧。”

简单地在楼下用餐后,两个人走向了他们的越野车。V本想坐在司机位上,却被陆久阻止了:

“开车是男人的工作。” 陆久说。

\section*{}

虽然不太明白这样分工的理由,但V没有多问,只是按照陆久的意愿坐在了副驾上。陆久已经在地图上查好了目的地的位置,那是这座燕山脚下的城市周边的郊县,距离大概有六十多公里,还不算太远。

这是个风和日丽的冬日午后,虽然气温很低但阳光很好,照在身上让人感觉很温暖。两个人坐在飞速行驶的汽车里随着地势的起伏在马路上时上时下,村庄和田野在身边交替掠过,他们工作和生活的城市就这样在身后渐渐远去了。

“今天的天气真好。”汽车行驶了一阵,一直沉默着的V忽然开口说道。

“是啊。”陆久说。

天气的确很好,但陆久的情绪并没有跟着变好,因为他们不是去出游。想到那天在南宁发生的事情,陆久的心里只感觉到沉重。

“还在想那天的事情?”

“啊……没有。只是在思考该怎么和那些家属开口去说。”

“……你以前从来没有这样茫然失措过。”

陆久知道V说得没错。自从他得知这个阵亡士兵的身份之后,那天晚上的战斗场景就一直在他脑海里反复播放,他不断地猜测这位士兵是那天阵亡的四个人中的哪一个。他知道这样的猜测是没有结果的,但他越是这样想就越是无法自已。

“也许吧。”陆久说,“我也不知道自己这是怎么了。我知道去想那些事情毫无意义,以前我从不会过多考虑这样的事情,但现在却感觉有点控制不住自己。”

“那是因为我们都变了。”

陆久觉得这句话有点耳熟,微微扭头看了V一眼。他看到身边的女孩只是默默地注视着前方,脸上依然是不变的平静表情。

“那么是变好了,还是变坏了呢。”陆久说。

“变好了。你渐渐开始关注身边的事情,我觉得是好的改变。”

V的语气很肯定,这让陆久也开始思考起她的话来。曾经就连生死都不放在心上的人,忽然变得多愁善感,真的是件好事吗。

“你也变了。”过了一会儿陆久笑了笑说,“变得学会安慰人了。”

地图上六十公里的距离放到平面上就不止六十公里了,而且山区的路也高高低低地起伏不定,汽车行驶了大约两个小时才到达了目的地的郊区。在进入郊县之后,陆久先在路边停下了车,然后在地图上仔细地再次确认了那位士兵的家的位置。

那是一片小小的村庄,村子里全都是一排排整齐排列的房子,高的有三层、低的有两层,一看就是居民们自己盖起来的。这样的建筑风格陆久非常熟悉,他能够感到自己大概也曾经在这样的村庄里生活过,但具体的细节却又若有似无地无法捕捉。

“第三排第四户,就在那边。”隔着田地,陆久凝望着不远处的一片房屋说道。

“这里的房子有点像北镇呢。”V说。

“嗯,北方的村庄都是这样。”

“我们要过去吗。”

“稍等一会儿。”

陆久站在车后面,点燃了一根烟。他抽了一口,然后吐出烟气、整理了一下自己的情绪。

“我感觉你有些紧张。”

V说的没错,陆久确实心里在惴惴不安。第一次上战场的士兵也许就是这样的感受吧,虽然陆久已经记不起来自己第一次去打仗时的情景了。

“是啊,说实话我现在有点后悔揽下这事情了。”

“要是感觉不好,我们就回去吧。”

陆久笑了笑,他有时候不知道V到底是真的不谙世事,还是在故意讽刺他。就这么临阵脱逃,且不说分公司或者谢振知道了会怎么想,就连皮尔斯和雷蒙那边都交代不了吧。

“还不至于。”陆久说,“至少我们要面对的人没有武装。走吧。”

说完,陆久扔掉烟头、拉紧了领带,然后坐到了副驾上。

驾驶员换成了V,只用了几分钟,他们就来到了那座房子前。房子没有直接坐落在街道上,而是和街道之间隔着一片宽阔的庭院,而院门则是敞开的,说明家里有人。陆久下车后整理了一下衣服,把蓝色的信封拿在手里,深吸了一口气,然后向着庭院内走去。V则迅速跟在了他身后。

穿过修葺整齐的院子,陆久来到了房子的大门前,然后伸手在门上轻轻扣了三下。

片刻后,门开了一条缝,陆久看到一个瘦弱的年轻女人走了出来。

\section*{}

“请问这里是檀春野先生家吗。”陆久轻声说。

“你们……”

那个女人打量了陆久和V一番,显然没有想到家里会有这样的客人来访。看着这两个身穿军装的人,她一开始甚至没有反应过来他们是干什么的。

陆久稍稍迟疑了一下,没有立即自我介绍。因为他意识到很多雇佣兵都没有对自己的家人公开过自己的“工作”,也许这个士兵的家属还不知道“格里芬公司”这个组织到底是干什么的。

不过过了片刻,陆久知道自己不需要为这些事操心,因为这个女人显然知道檀春野是做什么工作的。陆久看到面前的女人的脸色变了,在短短几秒钟的时间里,他见证了一个人的表情从迷惑到惊讶、再到悲伤的变化全过程。

“你们走错门了,”那个女人用发抖的声音说着,“这里不是檀春野家。你们找错了!”

说完,她咣当一声关上了门。

“我们找错门了吗。”V在陆久背后轻轻说道。

“我想没有。”陆久沉声说。“稍等一会儿吧。”

果然,过了一阵,门再次打开了。不过这次走出来的不是刚才的年轻女人,而是一个头发有些花白的老人。那个老人大概有六十岁上下,身体有些佝偻手里还拿着拐杖,似乎腿上有残疾。

“请问这里是檀春野先生家吗。”陆久再次轻声说道。

“是的,你们是谁?” 老人打量了陆久一番,开口问道。

“我们是格里芬公司的办事人员。”陆久说,“您是檀春野先生的家属吧。”

“我是他父亲。”

陆久闻言,点了点头,然后轻轻吸了一口气:

“您好,檀先生。非常遗憾,我们为您带来了悲痛的消息。”

听到陆久的话,老人并没有什么反应,只是直直地看着陆久的眼睛。陆久没有回避,而是用尽量肃穆的目光回望着那个人。过了一阵,那个老人终于点了点头。

“说吧。”他说。

“您的儿子在几个月前的战斗行动中牺牲了。我代表格里芬公司将这一让不幸的消息转达给您,并向您致以最沉痛的哀悼,和最深切的慰问。”

说完,陆久递上了那个蓝色的信封。

老人接过信封看了一眼,没有说什么。但陆久听到屋里传来一声尖利的哀嚎。

“对不起,是我的儿媳在哭。她实在承受不了这样的打击。”陆久面前的老人用沙哑的声音说道,“你们的消息我收到了,让你们听到这样不堪的声音实在是不好意思。”

“这样悲痛的消息任谁都会感到难以接受。”陆久说,“说实话倒是您的镇定让我感到有些……吃惊。”

“你是春野的上司吗。”老人说。

“不,我们不是同一个部门。不过我曾经也是作战人员,所以也可以说是他的战友。”

“那就容易解释了。”老人说着提起了左边的裤腿,“我也曾经是战斗人员,这是最后一次战斗时留下的纪念。”

陆久看到他的左腿上装的是金属的义肢。于是陆久立正向老人敬了个军礼:

“作为军营的晚辈,我向您致敬。”

“不必,我已经不是军人了。”老人微微摇了摇头,“我负伤退役的时候春野才4岁,不过那时候他倒是很喜欢这条铁腿。后来他长大后想去当兵,我不同意,于是他去上了大学。大学毕业后据说去了个什么保安公司,呵呵,我就知道他不是普通的保安。”

“他是个英勇的战士。”

“他是怎么死的?”

“他在一次战斗中孤身掩护几位战友撤离,结果陷入了困境。很遗憾,当救援部队抵达的时候已经太晚了。”

“他救了很多人吗。”

“是的。得益于他的大无畏牺牲,有多位士兵得以安全撤离危险区域。”

“那就好。”老人点了点头说,“那就好……我也是靠别人的牺牲,才只丢了一条腿就活了下来。那时候我如果也留下断后,我们的队伍也许就不会……算了。知道他没有白白死去我很高兴,他比我要更加勇敢。”

“他值得您为之骄傲。”

听到陆久的话,老人笑了笑。

“年轻人,我不是不懂政治部那一套。”老人说,“战斗中很多士兵都是默默无闻地牺牲的,也许是被流弹击中、也许是被炮火波及,没有几个真正死得像个英雄。很多美丽的故事都是用来慰藉那些家属的,因为人死终归不能复生。但就算这是个故事,我也很感谢你给我带来的安慰。”

“这不是故事。我以军人的荣誉保证,我所说的句句属实。”

老人没说什么,只是再次对陆久笑了笑。陆久不能确定他到底有没有相信自己的话,但他可以看到那双眼睛里的悲伤已经难以抑制。

“抚恤事宜公司会另派人员前来联系的,届时遗资遗物也会一并送达。”陆久说,“檀春野的英勇事迹将永远激励我们,也请您节哀。”

说完,陆久再次敬礼,然后转身离开了正在强忍着丧子之痛的老人。

\section*{}

短短不足半小时的谈话,让陆久感到心力交瘁,犹如经历了几天不眠不休的战斗一样疲倦。他走到汽车跟前的时候,直接拉开车门坐在了后排。

V默默地启动汽车,朝着归途而去。开了半小时之后,她把汽车在路边的田野里停了下来。

“你还好吧。”V扭头对独自坐在后面的陆久说道。

“还好。”陆久说。V看了陆久一阵,然后走下汽车,从另一侧登上了后排坐在了陆久的旁边。

“你的脸色很差。”V说。

“没什么,稍微有点恶心。哈。”陆久笑了,“真是的,我果然不擅长做这种事。让你看了这么一场拙劣的表演,实在是太不好意思了。”

“我觉得你成功地说服了那个老人,他应该是相信了你的话。”

“那又如何,事实是怎样的你我都很清楚,那个士兵根本没什么英勇事迹,因为他是被自己反叛的同僚杀死的。一个活生生的人,死得像一只蝼蚁一般。但我们不能吐露事实,因为这就是人类的交际,面不改色地说谎就是其中的主要内容。军人的荣誉?哈,价值不过一句廉价的谎言。”

“至少有人从中得到了宽慰。我有点理解你所说的‘沉默是金’了。”

“呵呵。去他妈的。”陆久冷笑了一声,然后骂了一句脏话。

听到陆久的咒骂,V沉默了片刻。然后,她朝着陆久靠了过去,抓住陆久的胳膊将他拉到了跟前,然后从背后环抱住了他的肩膀。

起初陆久对V的动作有点意外,但当他明白V在做什么之后,他没有抗拒,而是顺从地把头靠在了V的胸前。

“在我们离开的时候,我看到那个老人的眼睛里除了悲痛还有感激。”V轻声说着,“我觉得你做了正确的事情,这让我学习到了很多东西。所以不要那么自责。”

“啊。我只是觉得有点累。”

“那是因为你尽力了。稍事休息一下吧。”

“好的。”陆久听从了V的劝告,在V的怀里闭上了眼睛。

当陆久醒来的时候天已经黑了。他睁开眼睛感到的第一件事就是手脚有些冰凉,因为日落后外面气温很低,而汽车又没有开空调,所以他感觉有些冷。不过至少他的身上没有感到冷,因为他不仅披着两件大衣,而且V依然在紧紧地抱着他用体温为他取暖。

“你醒了。”察觉到陆久动了,V轻声说道。

“啊。”陆久急忙坐直身子离开了V的肩头,“抱歉,我睡了很久吗。”

陆久其实睡的时间并不长,但因为北方的冬季白天很短,所以太阳已经落山。

“只是一小会儿。”V说,“感觉好些了吗。”

“好多了。谢谢。”陆久有点不好意思地说,“我们继续上路吧。”

“好,我来开车吧。”

两个人再次坐在了汽车的前面,开始朝着回去的方向走去。大约一个多小时之后,他们回到了自己的住所。

走进房间,陆久首先按照和皮尔斯的约定,换下了制服仔细叠好,然后把衣服放在了衣柜里。而V则端坐在沙发上,不知道陆久为何回来就急匆匆地先去换一身衣服。

“陆久。”V忽然开口说道。

“嗯?”

“那时候第一次出来开门的那位女士,是那个士兵的妻子吗。”

“应该是这样吧。”

“据我所知,特勤人员也经常外出作战,很少回家。所以她平时和那个士兵见面的机会应该不多吧。”

“是的,所以我们才有那么多的信件需要处理。虽然打电话或者发电子邮件会更方便快捷,但手写的书信更有实感、更能寄托人们的思念吧。为什么突然问起这些了?”

“没什么。想到总是要等着一个不知能否归来的人,忽然觉得她这样的人类也很可怜。”

当然。胡麻好种无人种,正是归时底不归,苦等征人总是令人心焦……不过年纪轻轻就成了寡妇,不更是悲哀的事情吗。而且这种话从V的口中说出,让陆久感到很有些讽刺,人类竟然被人形所同情,不知道这算不算另一种悲哀。

陆久意识到V也开始思考关于人类社会的问题了,但她的思路恐怕还是和普通人有些差异。

“所以我才一直觉得才要和身边的人少点瓜葛。自己要是出了什么事,也少点人为之伤怀。”陆久有些自嘲地说着,朝自己的卧室走去。一直到他睡着,客厅的灯都没有熄灭,V似乎还在思考今天发生的事情。