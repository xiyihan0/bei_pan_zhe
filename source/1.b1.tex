\chapter{附录一:罪犯H的日志}
\paragraph*{文件0105:}

DM行动纪要:关于军官H屠杀、抗命、危险行为以及推定背叛罪行的定罪依据(副本)

\lineseparator

\paragraph*{说明:}

以下是军官H在执行“DM”行动时的纪要日志,由军官军官H记录,军官F提供。

注:此文档为0105文件副本,因保密需要文件中人名以及时间、地点均略去或作简写。其真实信息参见文件正本及附件。

\lineseparator

\begin{enumerate}
    \item **月**日(0)
    
今日抵达Q市。落地后由阔别一年多的军官K接待,老友久别重逢,心内十分高兴。

从K处获悉,此次行动主旨为清除Q市滩头敌军所遗留之爆炸物,称之为DM行动,第一阶段为期约3-4个月。

总部配备的排爆人员为没有或者缺乏任务经验的新人,称之为“学员”。我负责训练以及监督其行动之任务。

抵达指挥站后,认识此次行动负责人之一,军官F。在F的带领下,检阅了我手下的第一期学员,共计16人。

学员均为战败国之战俘,以少年居多,其中有军士2人。
    \item **月**日(0)
    
多数学员的排爆经验较为匮乏,正在由F组织进行简单训练。训练完毕后,我将带领学员负责Q-B13海滩排爆工作。

B13海滩为沙质海滩,海洋环境温和,曾为Q市度假地。海滩的爆炸物有较为详细的图纸和数字资料,适合作为第一阶段任务的适应性训练。

根据F所述,当第一阶段任务结束时,存活学员将被允许作为投降人员释放回国。但我的任务未作知会,似乎还将继续。所幸战争已经结束,纵然尚需在此地执行任务,但归营只是时间问题。

今日在排爆训练中发生意外事故,有学员一人死亡。
    \item **月**日(1)
    
今日风和日丽,海滩天气良好。15名学员均已到位,营地位于海滨度假村的一所别墅内,为征用之民房。这里曾有别墅群存在,但在战争中已悉数被毁,现仅存一座楼房,有六间房间、一座仓库。

别墅的主人A先生在卫国战斗中牺牲,现有A太太及其女儿B小姐(6岁)管理。

我住在楼房一楼,A太太及女儿住在二楼,学员被安置于仓库中。

略加操练后,排爆工作终于起步。如进展顺利,约4个月内可结束第一阶段,届时应是初秋。

因学员多有恐惧心理,第一日的行动进度缓慢。相信几日后速度将会有所提高。
    \item **月**日(3)
    
今日是行动展开第三日,指挥站送来补给若干、以及一条排爆犬用于协助搜索爆炸物以及看管学员。排爆犬性格活泼、惹人喜爱,我将它命名为M犬。

但指挥站送来的补给中,仅有本人的口粮,没有学员的口粮。考虑到学员已经三日未进食,我将所有口粮置于大锅中煮成汤水让学员分饮。但因为食物有限,学员饮下后体力恢复不多。

于是今日休息时间较前二日早一小时。
    \item **月**日(7)
    
今日是行动展开第七日,指挥站依然没有送来学员的口粮,询其原因,答复曰战俘之补给不是首要事项。

这几日都在向A太太借粮用以勉强维持学员行动,如这样的情况持续,恐怕三日内学员将无法行动。另外因为我用粮食饲育敌人士兵,A太太对我态度多有抗拒。

我已就此事向军官K求助,希望一、二日内得到解决。

学员之间已经出现身体不适的症状,推断是过度饥饿所致。

行动开展一周后学员的经验都已经比较熟练,但是因为身体状况,效率未能显著提高。
    \item **月**日(9)
    
学员内爆发群体疾病。

学员A向我坦白,他们因为饥饿难耐偷吃了A女士给禽类准备的饲料,因而生病。

饲料主要材料为麦糠,纵然导致腹泻,却不该引起发热症状,因此我到禽舍实地调查。发现大量鼠粪,推断学员因吃下含微生物之鼠粪而致病。

我命学员饮下海水后扣喉咙呕吐洗胃,用清洁的水冲洗全身、暴晒衣物。然后今日休息一天。

指挥站的补给仍未送达,因此我驱车前往,取回食物和药品若干。

回到营地后将食物及药品让学员服下,学员的精神明显好转。

我指派学员A为本批学员的负责人,负责组织和监督学员的日常生活。
    \item **月**日(11)
    
今日多数学员已基本正常,年轻的躯体恢复速度很快。但仍有三人十分虚弱。

其中一人学员B向我请假一小时,未予准许。另有学员A似乎为请假者的朋友,表示愿承担前者的工作,请求我准假一小时,未予准许。

下午,之前的请假者学员B因为神志不清误触地雷引信,爆炸导致双臂损毁,伤势严重。因营地设施简陋无法救治,只能简单包扎后送往指挥站的伤兵营。

我本对受伤学员存活的可能性不报希望,但诸多学员一致央求,我不得已而施救。

对我而言,这些学员本身就是用以消耗的耗材,对指挥站而言恐怕更是如此。
    \item **月**日(12)
    
今日是行动展开第十二日,指挥站终于将学员的口粮送达。这也是托K施加压力的结果。

虽然只有粗粮且分量不多,但是至少能够保证学员有体力去完成排爆工作。

目前我们面临的主要是反步兵地雷,排爆方式为在沙滩上使用铁钎匍匐探索,找到地雷后徒手挖出并拆除引信。这一方式虽不及金属探测器高效,但在资源有限的情况下,只能用此等简易排雷法。

所幸学员学习很快,此刻排爆已经接近计划中速度,且没有更多人员伤亡出现。

目前沙滩上已经清理出一片安全区域,相信第一阶段最终结束时间,不会比预计中推迟很多。
    \item **月**日(20)
    
今日是行动展开第二十日,总体工作约完成五分之一。沙滩上的安全区域面积已经颇为可观,因此拟划出一片区域供学员操练运动。

我去往指挥站汇报行动进度,会见了K及军官F,军官F对我的行动进度表示赞赏,K为我多加美言。

在指挥站我欲顺便探视之前受伤的学员B,获悉学员B已于几日前死于伤口感染。此事并不出我所料,但为了不影响学员们的士气,我谎称学员B正在康复。

学员A似乎对我带来的消息并不信任。于是我在四下无人时,对学员A通报了学员B的死讯。他对此事表示理解,但事后在我不注意的地方哭泣了很久。
    \item **月**日(29)
    
今日天气晴朗,海滩温度适宜外出放风。因为一个月来行动进展速度让人满意,我决定给学员放假一日,以休息放松身心。

学员多为少年,在沙滩上组织足球运动玩耍、以及跑步竞赛,心情十分欢愉。我认定适当的休息和娱乐对于提高行动效率是有益的。

下午指挥站又送来两人,以补充学员数量。这些学员之间相处得十分融洽,此时我感到管理这支队伍的难度并不太高。这让我积累了宝贵的经验。

现在战争业已结束,或许我们不该一直将这些少年当做敌人对待。
    \item **月**日(30)
    
今日是黑色的一日。

我带领M犬于滩头巡视时,在安全区域触发了地雷。因为躲避及时,我未受重伤,但M犬却在爆炸中丧生。

我询问学员的时候,他们说已经仔细检查过所有爆炸物,和指挥站上提供的材料中数量一致。这说明指挥站所提供的材料数字也许并不准确——又或者,这些学员存在混淆视听、甚至消极怠工的行为。无论原因是哪一个,我都知道,此刻的海滩安全区已经并不安全。

我想我对于他们的管理过于松懈了,在朝夕相处的过程中我渐渐淡忘了他们是敌人这一事实、忘记了这些地雷正是他们和他们的战友亲手埋下的。负责这一区域的学员被我亲手处决。

我要求他们并排手挽手走过他们所清理过的所有区域,以保证这片区域的彻底安全。一名军士表示抗议,我亦当场将他处决。其他人不敢再有怨言。

这些学员走过他们清理过的区域后,没有再次出现爆炸。我想他们今后会对自己的任务更加认真。
    \item **月**日(40)
    
学员A向我汇报,他发明了使用铁框分割沙滩的方法让工作更加严密、保证万无一失。我对这个方法并不感兴趣,但还是同意推广。

我更中意的是用他们的双脚去探查他们所排查过的区域,这样就算为了他们自己负责,也不敢对自己的工作有所懈怠。

这些少年只是些孩子,也许他们对于自己在战争中犯下的罪过不该全权负责。我知道他们和我一样,只是遵守命令的军人。但我无法确定自己到底该不该相信他们,以及能信任他们到何种程度。

毕竟他们和那些我在战场上杀死的、以及杀死我的战友的人们,是同一群人。我不会再心怀友善地去对待他们。
    \item **月**日(50)

这些天的行动进展平稳,距离预期目标只剩下一半了。我向指挥部汇报了工作情况,指挥部对我的工作表示赞赏。

K表示十分期待工作结束后能再和我并肩作战,但我觉得可能不会有这样的机会了。一个原因是这里的战争已经结束,我们的战功足够我们晋升到中层军官,不必再亲临战场;另一原因是因为这些天的工作让我感到十分疲惫,甚至比我所经历的一线作战更糟。我甚至考虑过能够在适当的时候解甲归田。

但是我知道K是那种向往战场的人,他的心中充满了战斗的荣耀,肯定不会想象退役之类的事情。所以我并没有说出我的想法。
    \item **月**日(56)

今天发生了意外的事情。

B小姐,也就是A女士的女儿,独自外出玩耍的时候误入雷区。当A女士发现自己的女儿坐在隔离线之外和洋娃娃玩耍的时候,她吓得魂飞魄散——虽然那个小女孩离隔离线只有二十米距离。

而此时的我正在指挥站听取关于下一阶段行动的简报。

当我回到营地的时候发现学员宿舍里只有一个人,躺在床上面色抑郁地望着仓库的顶棚,他就是学员A。

我问他其他学员哪里去了,是不是逃走了。他不说话。

过了好一阵子,他才说,在滩头。

我赶到滩头的时候,看见那些学员们都在那里——正在用他们每天都在练习的技能,试图在隔离线和那个小女孩之间清理出一条安全通道。他们已经排出了两颗地雷。

虽然他们都知道只要耐心等待,最终每个人都会是安全的,但那个孩子似乎已经失去耐心了。她不断站起来又坐下,只是因为她母亲的苦苦哀求。我认为她已经不会在原地呆太久了。

这时候学员A站了出来。他不顾危险地走到了小女孩身边,开始和她说话,安抚她的情绪。

十五分钟后,安全通道建成了。一个学员过去把那个孩子抱了回来,但学员A却没有过来。他直直地走向了海边——那边是一大片没有经过排查的雷区。

他无视我们的呼喊漠然地走着,终于在走出二十多米后,被一颗地雷炸成了碎片。
    \item **月**日(57)

所有的学员都被学员A的死震撼了,低落的情绪在学员之间蔓延。也许每个人都像学员A一样,丧失了活下去的希望,只是没有表现出来。学员A的自杀一幕让他们积累在心头的绝望开始爆发了。

我只好让他们放假一天,并对他们进行动员和鼓励。

我终于意识到,对他们来说,战争还没有结束——他们依然都在每天面对着死亡。可是在这场还在继续的战争中,我又扮演着怎样的角色呢。

我曾经以为他们只是一群用于消耗的耗材,但是我发现我错了。也许对于我来说的确如此,但对于他们来说,我是他们初入战场时带队的中士。

我想起了自己还是新兵的那些日子。如果中士已经不顾手下新兵的死活,那这些士兵该是如何的绝望呢。我对自己的行为感到可耻。

无论如何,我们的工作已经进行大半,这是唯一给我安慰的事情。军官F曾经承诺工作结束后这些人就可以回家,我认为我可以完成这项工作。

我想起那时我的带队中士,也就是如今的军官K说过的话:如果你不能带领你的士兵走向胜利,那么至少带领他们回家。不该有更多人在这里死去了。
    \item **月**日(70)

这两周里又有一名学员因为事故而死去,但我一直在鼓励他们、他们也在鼓励着自己。

这不是任何人的过错,既然战争还在继续,那么出现伤亡总是难免的。

这是一群优秀的军人,虽然他们是如此的年轻,虽然他们曾经是我的敌人。这群孩子纵然在夜里因为想家和害怕而哭泣,但是在面对白天的战场的时候,也从不退却。

第一阶段的行动已经接近尾声,回家的愿望正在驱使他们努力前进。我希望他们每一个人都能顺利踏上归途。
    \item **月**日(82)

今天事故再次发生了,有三个学员在装卸已经拆除引信的地雷时牺牲。

那堆本来应该安全的东西因为不明的原因发生了爆炸,三百多颗地雷把方圆百米范围内的一切都炸成了灰,而我们永远无法知道到底是他妈的为什么。

他们已经离回家只有一步之遥了,我昨天还在听他们谈论回家后要做些什么。

真他妈的混蛋。
    \item **月**日(85)

所有的爆炸物都已经排查干净了。今天是我这三个月以来最为轻松的一天。

我请求A女士为我的学员们做一些可口的饭菜,她欣然接受了。我用自己的军饷支付给她报酬,她不肯收,但我还是把钱塞到了她手里。

我在饭桌上为他们祝贺,那些孩子们喜极而泣,让我也不由得心里为之触动。

明天我就去向指挥站汇报,学员们已经开始准备自己的行李了。
    \item **月**日(86)

今天我到指挥站汇报工作,顺便了解了一下第二阶段的行动。

下一阶段的工作重点是N39地区,依然是滩头。不过这次可没有B13海滩那么简单了。那里部署的都是些构造复杂的地雷,而且海滩十分泥泞,也没有爆炸物的准确数字。

另外,我的学员们将不被允许回家——他们作为有经验的排爆人员,将继续他们的任务,一直到Q市海滩上的爆炸物全部排除,或者全部阵亡。

真是讽刺。我其实一开始就该想到这一点的,活下来的都是经验丰富的士兵,战斗还未结束怎么可能让他们退役。

指挥站的人们,他们从一开始就根本没有打算让这些孩子活着离开。他们不仅欺骗了那些战俘,还欺骗了自己人,因为最真挚的表演就是要自己也深信不疑——这种事我早就了然于心,却不肯相信会发生在自己身上。
    \item **月**日(87)

我虽然反对指挥部的做法,但是和F的交涉没有任何结果。他说这一切都是上边的意思,无论是我、是他还是K,都没有选择的余地。我知道他说的是真的。

K也劝我顺其自然,不要过多考虑那些于己无关的人,但我不能不考虑。纵然他们曾经是我的敌人,但现在已经是我的战友。我从不抛弃战友。

违抗上级命令显然是错误的,可是如果这命令本身也是错误的,又该如何抉择呢?
\end{enumerate}
\lineseparator
\paragraph*{附注:}

以上为军官H所做的行动纪要全文。

在最后一次纪要的一天后,H本该奉命将剩余战俘送往N39地区,但是他没有照做,而是私自释放了那几个战俘。

前去接收战俘的官员,在B13滩头误触地雷身亡。根据**检举,此事军官H嫌疑重大,军官H因此事被送往军事法庭。

\lineseparator
\paragraph*{附:}

检方起诉事项、被告定罪以及其他人员情况:
\begin{itemize}
    \item 放任战俘逃亡是既定事实,因此军官H的抗命罪名成立。
    \item 军官H曾枪杀两名战俘,屠杀罪名成立。
    \item 因对于交接友军的死亡原因不作供述,军官H的背叛罪名推定成立,但因证据不足暂不定罪。
    \item 军官H在与军官F的争论中行为过激,被认为有危险行为罪。
    \item 逃亡的战俘下落不明,推测已经离境,不再进一步追查。
    \item 军官H的好友军官K,确认对此事并不知情,和H的行为没有关系。
    \item 军官F存在失职,但事发后积极检举,作降职处理以期后效。
\end{itemize}

\lineseparator
文件记载结束。