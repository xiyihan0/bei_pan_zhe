\chapter{背叛者(八)}

\section*{前言}


陆久把散兵游勇组成的队伍交给了45,然后自己去单刀赴会。他知道回去不会有好事,因为这样的事情克鲁格已经向他预告过了,并且暗示了他躲远一点。但他还是要回去,因为他想知道克鲁格那边到底发生了什么。

在离开前,陆久遣散了NT-77,并交给她一项任务;而45显然也不会闲着,每个人似乎都很忙。悬而未决的一切即将尘埃落定,虽然陆久尚不知道会发生什么,但他已经隐约预感到,决定何去何从的时刻就要到了。

\lineseparator


故事又回到了最初的起点,在写到这一章的时候,作者也感觉恍若隔世。因为实在是太久了,久到作者本人都沉浸在冗长的剧情中,几乎经忘记了陆久和克鲁格最后会拔枪相向。

但该来的终究会来,只要故事在继续,这一章的剧情就是躲不过去的。

话不多说,一起来看一看本文标题中的大叛徒,究竟是怎样表演的吧。

\section*{}

“您好,陆司令。虽然您没有直接向我们提供AR小队的情报,但我们毕竟还是搭了您的便车,因此我决定向您提供一些有用的参考消息。考虑到您可能没有时间去关注AR小队的战况,而且就算关注了也不会改变您的态度,所以我以视频形式向您发布了这条消息。当您看到这条消息的时候,已经发生了这样的事情:AR小队落入了铁血的埋伏圈,并且被困住了。但如我所料的是,军方出手为她们解了围,当然是出于军方自己的目的。现在AR小队遭受了伤亡,并且主要人员落在了军方手里.另外,同时伤亡的还有SOG派去侦查和协助AR小队的两个队员,但所幸队本人长无恙。我相信,AR小队一定十分迫切地想要撤离并摆脱军方的控制,因此我在她们可能会途经的路上做了接应的准备。同时,为了争取更高的回报,我擅自将手中的兵力分出一股,用于搜救其他失散人形。虽然不知道您回到总部后会发生什么事情,但我有着强烈的预感,我们还会有后续合作。所以,祝我们双方的行动都能顺利。”

陆久是在和45分道扬镳90分钟后收到的视频。正如45所言,他已经决定了自己要做的事情,因此他没有马上去查看到底发生了什么。虽然45提供的情报足以引人关注,但陆久相信局势仍然尚在45把握之下——既然不是即时通讯,那么就还没到生死攸关的时刻。

陆久去往总部用了两个多小时的时间,抵达时已经快要接近中午。临时指挥部里面空荡荡的,多数屋子都没有人,但楼房的前面还有一个哨兵,这说明至少这个机关的首脑还在。

陆久走进楼房,在楼道里听到了说话的声音,于是他在楼道口驻足倾听了一阵。

“如果我去请求父亲尽量斡旋的话,说不定军方那边,也不是非要把您……”

“我不是请求你帮助我,而是在请求你帮助格里芬公司。”

元帅的办公室里,一老一少两个声音在交谈。

格里芬公司的秘密指挥部已经人去楼空,元帅的办公室除了元帅秘书和门口的两个警卫员之外,其他人员几乎都已经离开了。办公室本身十分宽敞,在撤走了人员和重要的文件之后,更显空旷,两个人大声说话时甚至有些回音。

“小皮尔斯,要说交情,我和你父亲的交情在某些方面比你还要深,我想你也知道。”说话的人是克鲁格,“如果斡旋能有帮助,我会亲自去和他说的,不必劳你为难。但我们都知道眼下的情况,不是靠卖个人情就能度过的难关。我需要的不是你的父亲,而是你的帮助。”

“我也要听我父亲……也要服从总部的安排。”回答的是皮尔斯,声音似乎有些为难。

“有句话叫做‘将在外,君命有所不受’,我想你肯定知道是什么意思。”克鲁格说,“你父亲并不能完全掌握这里的情况。我也许没有多长时间在这里发号施令了,但我们的部队依然在行动,他们依然会需要空中支援。我向你保证我们依然会按照合约里的约定请求服务和支付费用,就算格里芬的账户被冻结,我也会请第三方代为支付的。能够决定这里的事情的,只有你,皮尔斯准将。”

“对不起,元帅阁下,我还是希望这件事能够与总部沟通后,再做答复。”

“你年轻有为、风流倜傥,每个人都觉得你以后绝对是一号叱咤风云的人物,我也一直这么认为。”克鲁格说,“你难道就甘愿一辈子都只做老爸的乖儿子吗。你也该长大了。”

“我风光的外表之下是什么,您应该比‘每个人’先生更了解。”皮尔斯说,“而且,我得做谁的儿子,也不是能由我来选择的事情。”

“有时候,我觉得你还不如那个陆久。”克鲁格说,“虽然那家伙自作主张惹了不少麻烦,但总好过凡事都要请示上级的将军。”

“我承认不如陆司令。”皮尔斯笑了笑,“我的命运看似风光,实则早已被安排得明明白白了;而陆司令虽在随波逐流,但他的未来却有无限的可能。如您所见,我就连飞机都不能开了,而陆司令能做出什么样的事情、扮演什么样的角色,谁都不敢说。”

“……那个老家伙,还是不让你开飞机吗。”

“啊,一如既往嘛。”

“我觉得你们父子之间,是该好好沟通沟通了。”

“我也曾这样考虑过,但后来想明白了。我是可以问他到底怎样才能让我开飞机,但我的未来,还有无数件事情要做决定,难道每一件都要去请示吗?在他要替我作出决定的所有事情里,开飞机也许未必是最重要的那一件。限制我的并不是禁飞令,而是‘没有决定权’这件事本身。”

“想要决定所有自己的事情,不要说是你,这世上谁也做不到。能争取一件是一件吧,你说呢。”

“您说的也有道理。不过我还是认为服从那个人的意志是第一位的,不然我的囹圄只会越来越小。”

“我明白你的难处。这样吧,我来亲自写一封信向你的父亲说明。如果他因为这件事责怪你,你就把信给他,告诉他,是我向你请求的。看在多年前的战友的情面上,他不会把账记在你头上的。怎么样?”

“……那好吧。”皮尔斯考虑了一阵说,“如果是您的请求的话,就算他不高兴,大概也——”

咚、咚、咚。

办公室的门响了三声,门外有什么人在敲门。所有人的目光都投向了门口。

克鲁格已经把所有事情都安排下去,这个时候不该有任何人来会见了。再说就算有,也该由门口的两位哨兵来通报,怎么会有人来敲门呢。

“什么人?”克鲁格沉声问道。

“陆久参上。”门外传来一个声音。

克鲁格看了皮尔斯一眼,皮尔斯摊开了手,表示不知道怎么回事;接着他又看向郝丽安,郝丽安也微微摇了摇头。

“进来。”克鲁格说。

门开了,陆久走了进来。

陆久首先看向郝丽安,见郝丽安也在看着他,而且微微皱着眉。然后陆久又看向了皮尔斯,但没有和皮尔斯搭话。

“我并没有召见你吧。”克鲁格对陆久说,“按照我的命令,你此刻应该在前线执行任务才对。为何擅离职守?”

“是因为前线出现了紧急情况,总部又失去了联系,所以我才来亲自汇报。”陆久回答。

“总部失去联系时对部队进行指挥,不正是你们这些指挥官的职责吗?”克鲁格冷冷地说道,“难道你是来这里让我来代替你指挥战斗的?”

陆久感到,自己的到来让克鲁格很不满。他意识到克鲁格切断了总部和外面的联系,正是因为不愿意让人知道总部里正在发生的事情。

“……至少我希望您能了解一下发生了什么。”陆久说,“昨天下午,SOG小组侦查到AR小队正在向铁血的埋伏圈行进。而刚才我得到了更紧急的消息,军方……”

“用不着你告诉我那些。”克鲁格打断了陆久的话,打开了全息地图,“比起AR小队,我认为军方先抓到的一定会是我,因为我坐在这里就是为了等他们。”

全息地图亮了起来,陆久惊讶地发现上面的内容和他上次见到的不一样,各部队的位置,包括南部军团的主力、AR小队、SOG小组和军方、铁血的情况都一览无余。陆久这才明白皮尔斯一直为格里芬提供的“服务”详情。

而陆久也清晰地看到,AR小队和大量军方武装已经零距离接触。他推测应该AR小队应该还没有受到军方的攻击,不然,军方只需一瞬间就能把她们消灭。

“AR小队已经被军方控制了吗。”克鲁格喃喃地说道。

“正是如此。”陆久说。

“军方不会消灭她们的,至少暂时不会。因为她们对军方还有很大的用。但如果AR小队被军方控制,那对我们来说可就麻烦了。必须制造机会让她们撤离。”

“但我们无法和AR小队建立通讯。”

“你是这么想的吗。我大概知道你为什么要找我请示了。”克鲁格严厉地看了陆久一眼,拿起了通讯器按下了一个按钮。

“喂,是我。”克鲁格对着话筒说道,“是的,我已经知道了。告诉你的AR小队,准备和军方脱离。我会派人进攻军方,枪声就是信号。嗯,就这样。”

克鲁格没有说出对方的名字,但陆久听得出和克鲁格说话的,毫无疑问是帕斯卡。

“等等,我集结到的部队离AR小队还有些距离。如果需要她们出击……”陆久说。

“有多少人?”克鲁格再次打断了陆久的话。

“大约50多人。”陆久回答。

克鲁格没再理会陆久,而是再次拿起通讯器,按下了另一个按钮。

“SOG小组吗。我是克鲁格。AR小队现在需要撤离,你们负责向军方发起进攻,吸引军方的注意力。我随后会派人去支援你们,在支援抵达前,不允许后退。明白的话就马上行动。对,现在。”

让SOG小组做先头部队吗,陆久心里想着,手心捏了一把汗。军方能够轻易扫灭铁血,战斗力决不能小觑,SOG小队主动出击势必万分危险,只希望支援能最快速度抵达。

“这里是总部。对,我是克鲁格。队伍集结得怎样了?”克鲁格再次拿起通讯器,“马上即将有一股约50人的散兵队伍向你靠拢,接应她们后马上向出口位置移动,搜救行动结束了。”

这是在向谁下达命令,陆久不太确定。难道是南部军团?“出口位置”又是什么……

但没等陆久想清楚,克鲁格再次拿起了通讯器。

“格里芬的士兵,请听着,这里是格里芬最高负责人克鲁格在向你们亲自下达命令。”克鲁格说,“我现在将南部军团主力部队集结点的坐标发给你们每一个人。收到后立即开始向坐标方向移动,并与主力部队汇合,然后听从南部军团的指挥。在到达集结点之前,你们不得接受任何人的命令,这是最优先的指令。”

听到克鲁格的最后一个命令,陆久完全明白了。克鲁格首先下令SOG去骚扰军方为AR小队创造撤离机会,然后命令南部军团准备接应格里芬的散兵,等到散兵和大部队汇合就开始撤离。

但这里面有个问题,那就是克鲁格许诺会为SOG小队派出支援,但他并没有向任何部队下达支援命令。

“SOG小队只剩三名成员,她们无论如何都无法对抗军方的武力的。”陆久说。

“我没要求她们消灭军方。我只是让她们为AR小队争取撤退的时间。”克鲁格漠然答道。

“那SOG小组怎么办,谁来掩护她们撤退?”陆久说。

“难道你认为我们有必要为了三个人形,再派出一支部队?”克鲁格反问道。

陆久沉默了。

从数学上看,毫无疑问克鲁格的策略是损失最低的,这笔账是个人都会算。舍车保帅的做法在兵法里并不少见,但此刻这对陆久来说,是无法接受的。

“克鲁格元帅。即使会付出较大的代价,我们通常也会去营救身陷危机的士兵,因为这是一种同仇敌忾的精神。”过了片刻,陆久终于开口说,“如果知道自己落单就会被抛弃,那么士兵就会怯于作战,因为谁都有可能有落单的时候。就算是以多换少,也要营救落单的战友,这事关军队的士气,不能用简单的伤亡数字计算。”

“的确如此,但那是针对人类而言,而SOG小组是战术人形。”克鲁格冷冷地说道,“我们讨论过这个话题,不是吗——人类会反抗、而机器则不会。我之所以推广这些军用设备来替代人类士兵,就是为了减少人类的伤亡,并且在关键时刻可以毫不犹豫地舍弃也不会影响到士气。”

陆久无言以对。战术人形存在的意义,就连她们自己都知道——Vector说过,她们不是什么兵种,因为她们根本不是人类,只是G\&K公司的财产;95说过,她们被制造出来就是未了代替人类去战斗,代替人类去受伤、代替人类去死。这是每个人都知道的现实,陆久当然也知道,一开始就知道。

他明白自己一厢情愿的想法和这个世界相矛盾,怀着这种想法迟早是会付出代价的,而这一天终于到了。

呵,陆久暗笑自己的迂腐。克鲁格说得当然是正确的,如果自己早一些想明白,能免去多少无畏的辗转反侧。

这次无论如何也不可能挽回局面了,陆久心想。克鲁格已经剥夺了他的部队,就算没有,那些部队也不可能听他的指挥去反抗克鲁格。

他想起45的“忠告”,那个精明的姑娘已经提醒过他会发生什么样的事情了,但他没有听——也许听了也没用。他只是抱着一丝“事情不会到那种地步”的侥幸心理来到总部的,却没想到局势已经远超了他预估的底线。

陆久有些失落地笑了笑。他曾经嘲笑皮尔斯因为不能开飞机而终日失魂落魄,像个小姑娘一样郁郁寡欢,今天他终于品尝到这种滋味了。

算了吧,陆久对自己说,他已经尽力去改变局势了,只不过这个结局是在预料之中。这次他真正是无计可施了,因为他要对抗的不仅是敌人的部队,还有整个人类社会的认知。

他陆久是人类,而人形是异类,不管人形有没有感情,终归是人类的仿制品。和异类之间的感情是不可能有结果的,无论今夕何夕,只要人类还在统治这个世界,这一条规则就绝不会变。

曾经的那些海誓山盟,只是为了在内心不知所往的时候寻求一些安慰。而现在,要去的方向已经很明白,是时候和过去错误的一切划清界限了。

只不过,有一个小问题,那就是陆久已经决定了要站在界限的另一边,哪怕是“错”的——

因为和自己心爱的姑娘比起来,这个该死的世界算它妈个屁啊。

“……克鲁格元帅,我强烈要求派兵支援SOG小队。”陆久深吸了一口气,然后立正、认真地说道。

“不行。我们没有那么多的兵力,去突破军方和铁血势力交错的战场。”见陆久提出正式请求,克鲁格也用官方态度进行了回复。

“那么,请派出运输机去接应她们撤离。”

“那也做不到,我们和皮尔斯将军协定的支援条款里,明文写了不得为了低价值目标而让飞行员陷入险境。这一点你可以向你那边的皮尔斯准将确认。”

陆久没有去看皮尔斯,因为他知道克鲁格说的是真的,而且那些事情皮尔斯也左右不了。

“那么,我请求让我亲自去救援SOG小队。”沉默了一阵后,陆久沉声说。

“那更不可能。不要讨价还价了,陆久。”克鲁格有些疲倦地说道,“大部队的损失、空中支援的损失或者是指挥人员的损失,都是不可接受的。别忘了,你也是公司财产的一部分。没人愿意看到牺牲,但这就是战争的代价、是不可避免的,你该明白。”

“……”

陆久当然明白,比谁都明白。他知道自己已经没有其他选择了。

沉默了一阵,陆久一言不发地转身朝着指挥部之外走去。

“陆久,站住!”

克鲁格低声喝道,闻声而至的警卫从门外走了进来,试图拦住陆久。但没跑几步他们便停下了,因为他们看到了,或者应该说每个人都看到了,陆久的手正按在他胸前的手枪上。

“让开。妨碍我的人后果自负,我不会重复第二次。”陆久说。他的声音很低,但却字字入耳,犹如回荡在山谷里的枪声。

“元帅的命令你不仅不执行、还用武力对抗,你难道是想反叛吗?!”郝丽安厉声喝道,“警卫,抓住他!”

“都别动!”克鲁格举起手,阻止了剑拔弩张的警卫和陆久,“你们几个后退。不要去拿武器,那家伙会开枪的、而且会比你们更快。”

几个警卫闻言向后退到了门口,但并没有出门,眼睛也没有离开陆久。

“陆久,仔细想想。”克鲁格缓缓说道,“我知道你一向体恤下属,你的心情我明白。但从更长远的角度考虑一下、从自己的身份考虑一下。你是北部军团的总指挥官,你的一举一动,都对整个公司的指挥官们有着表率作用。战争总有伤亡,如果一个SOG小队陷入绝境就需要一个战区司令去营救,那么别的指挥官该怎么办?他们的部下也可能会面对伤亡,他们会因为你的行为而产生压力、因为担心牺牲而不敢用兵。你明白吗?”

陆久稍稍垂下了目光,他知道克鲁格说的是对的,作为身居高位的人,必须比其他人更顾全大局。但他的手并没有从手枪上离开。

“你说得对,克鲁格元帅。我知道我做了个坏榜样……也许我一直都在做坏榜样。是我辜负了你的信任,我一开始就错了。但我必须要去,不管其他人怎么想、不管其他人是否理解。”

克鲁格眯着眼睛看着陆久,他知道心意已决的陆久,自己是阻止不了的。

“你让我非常失望,陆久。”

“我很抱歉。”

说完,陆久转身飞奔出了总指挥部的大厅。

“皮尔斯。”陆久走出门外后,克鲁格说道。

“是,元帅先生。”皮尔斯说道。

“去阻止陆久。你知道他的行为意味着什么——那不仅是条不归路、而且会有无数人被他牵连。作为朋友,你有阻止他的义务,我相信你会有办法的。”克鲁格说着顿了顿,“虽然看来我们已经不需要空中支援了,但我也不会让你白干活。无论任何手段,只要拦下他,我就去向你父亲求情解除你的禁飞令。这件事我自信还能做到。去吧,记住:如果有必要的话,做你该做的。”

“我知道了。”皮尔斯摸了摸腰上的左轮手枪,沉声说道。

“还有,给你个忠告:当心背后。”

“是。”

\section*{}

最后还是做了最坏的选择啊。

一边用力拧着全地形摩托车的油门,陆久一边无奈地想着。克鲁格的理论无懈可击,但陆久还是不能接受,只因为一点——SOG小队里的那个人,不仅仅是陆久的下属。

那是只属于陆久的、这世上唯一属于陆久的东西。对陆久来说,那就是他的一切、那就是这个世界存在的意义。如果那个人不存在了,那么这个世界也没必要存在了。

所以陆久一定要去救她,不管她是人形还是人类、不管这个世界会怎样。

虽然也想过要做一个好的军官、好的雇员,但看来都失败了,陆久暗叹。看着那些高远的理想时,他并没有去追求,而是终于选择了目光短浅的东西。不过这一次,他一点也没有因为自己的选择而后悔。

陆久回到45的营地时,营地里只有少数的几个人形,404小队的人除了45一个都不在。而45则在那简陋的营地里拿着通讯器,表情平静地说着什么,似乎是在指挥部队。

“您回来了,陆司令。”看见陆久,45立即转向了他,“比我预想的还要快呢。”

“你总能什么都预想到,真不简单。”陆久不带感情地说道。

“哪里,因为我听说SOG小队奉命去吸引军方的火力,所以我感觉您应该马上就该回来了。”

“这么说你已经有对策了吗。”陆久说道。

“您又想怎么做?”45反问。

“我要去救她们。”陆久说。

“那这里面的利害关系,不用我多说了吧。”45笑了笑,陆久的回答并没有出她意料,“要救SOG小队,付出的代价不仅仅是几十个人形,还有影响更为深远的东西。”

“我知道。”

“即便如此您也不在乎吗?”

“不在乎。”

45盯着陆久看了片刻。

“我就跟您摊牌吧。”45说,“这个世界越乱,我们这样的人越是有事可做;而秩序之下,我们的活动空间反而越来越小。所以您如果想那么做,我倒是喜闻乐见的。但真正让我担忧的,是您到底能做到哪一步、能不能达到您预期的目标。”

“是能不能达成我的目标,还是能不能达成你的目标?”

“我想我们的目标是一致的,就结果而言。”

“这么说,你是在指望我了吗。”

“就是这样。”45说,“您这样的人并不多见,而眼下的情况也并不常有。从我的角度来看,现在正是天时、地利、人和三位一体的时刻,可谓机不可失、时不再来。”

“我真的值得你这样看好吗。”陆久笑了笑,“虽然不知道你到底在图谋什么,但我感觉不是功名利禄这类的小事,对吧。”

“我谋求的对您来说绝对不值一哂,但此刻我们能成就的事情,就不一样了。”45也笑了,“而且就算是失败了,于我而言也几乎没有损失,不是吗。”

“呵,这倒是。你几乎没有下注,却把把都在坐庄。”陆久冷笑了一声,“我承认你很精明,但这样的玩法我不太喜欢。”

“您不喜欢我是理所当然的,您是靠实力生存,而我们只能靠耍小把戏混口饭吃。”45丝毫不以为忤,“只不过我能保证,这一把下的注,决不会让您吃亏。”

“别废话了,我没兴趣。”陆久厌烦地说道,“既然你什么都知道,也该知道我没时间和你胡扯。我要走了。”

“416和G11正在协助NT77占领铁血的雷达站用以发出广播,而9也带人去支援SOG小组了。”45说,“我留在这里,就是等着和您谈谈您的下一步计划,希望您能花几分钟听听我的意见。”

听到45的话,陆久停了下来。

“你知道我想要什么吗。”陆久问。

“无非是找到那个姑娘,然后离开是非之地。”

“那你知道我要怎么做?”

“我猜是要搞个大新闻?”

陆久眯起眼睛看了45一阵,因为45的话揭穿了他内心最深处的秘密,那个秘密甚至就连陆久都在假装不知道。

“你是怎么知道的?”陆久说。

“NT77是个简单的家伙。我假装递给她武器,顺便看到了您在她手里写的那个号码,而她完全没有意识到我在刺探情报。”45说,“然后顺着那个号码我找到了一个人……一个叫‘科宁斯’的老记者,他主持的那台节目以敢于不计后果地揭露事实真相而闻名,大概能算是全世界收视率最高的新闻节目了吧。说实话,您竟然还认识如此有影响力的人,让我很吃惊。您交给NT77的东西,我我猜是您和帕斯卡‘合作’时得到的资料,可能还不只于此。二代民用人形的制造和应用细节,是这个行业最大的秘密,这些资料交到那个记者手里会发生什么会发生什么,嘻嘻,我都不敢想象。那就是您的底牌了吧,我说得对吗?”

“……也许吧。”陆久说。

陆久就是打算借科宁斯之手将民用人形的秘密公之于众,45已经看穿了自己的整个计划,他没有必要再隐瞒了。陆久本来是为了不让45听到才写下那个号码的,却忘了交代NT77要严防泄密。不过事已至此,再说什么也晚了,他没有能力再去重新筹谋,只能希望45信守承诺,不会妨碍他。

“那你有什么想说的?”陆久说。

“您的计划很好,但还不够。”45说,“您的证据很客观,却不够生动,稍加歪曲就会变成居心叵测的谎言,因为您没有足够的说服力。我认为您还需要人证。您需要另一个人来亲口说出您想要说出的观点:人形就是人类的同类,而人类正在把自己的同类当做动物去压迫和奴役……让一个人亲口说出这些,人们才会相信。那得是一个声名显赫的人,让人们就算不愿意相信,也不得不相信。”

“可是有谁会公开证实那些事情呢。”陆久说,“我不认识什么社会名流,不可能让他们附和我的观点,更别说是公开表述了。”

“您的观点是不争的事实,虽然有一些人为了利益而极力回避,但更多的人是在沉默中观望。”45说着摘下自己的发卡,别在了陆久领子后面,“据我所知,您也认识一位有话语权的大人,而今天的这个局面,可以说他就是始作俑者。我相信您有办法让他说出我们想听的话,对吗。”

陆久思考了一阵,然后笑了笑。他刚才还觉得不知该如何行动,但45的几句话却如拨云见日,让陆久心中豁然开朗。

“克鲁格对我的恩情你们应该很清楚。不去报答帮助过我的人、却要反过来利用他,这就是你给我的建议?”

“事到如今,您还在乎这个?他马上就要落入大牢,现在也只有这点价值了。”45微微一笑,“再说,他帮助您,不也是为了利用您吗。说不定他的灵魂,也会因此得救呢?”

“……对一些人来说,这个世界就像一间铁屋,没有窗户而且难以摧毁。他们被囚禁在里面、昏昏欲睡,最终全部都会被闷死。但这些人是在睡梦中死去的,不会感觉到将死的悲哀;而如果叫醒这些人,也许只能让他们徒受临终之苦。你觉得唤醒他们没问题吗。”

“只要有人醒来,就不能说这屋子没有摧毁的可能,而且许多人本来就是在装睡。”45说,“至于对不对,那不是我们需要操心的事情。”

“唉,”陆久叹了口气,“我不久前还说,不相信世界的命运会落在几个人身上这种事,但现在相反的事情似乎就要应验到自己身上了。”

“您不相信的是这个世界会被几个人拯救。”45说,“但几个人要毁灭这个世界,可是小菜一碟。您忘了三战是如何爆发的了吗?”

“我可不希望这个世界因为我而毁灭。我还想好好过日子呢。”

“那我们是再次达成一致了,嘻嘻。这是我们过上好日子的唯一途径呢。”45说道,“在野外垂钓时经常有些水域的水非常清澈,鱼能够看到岸上的人,因此很难上钩。您知道这时候渔夫们会怎么做吗?”

“让我想想。”陆久摸了摸下巴说,“莫非是,把水搅浑?”

“……没想到您也是一位钓鱼老手呢。”45微微吃惊地说道。

“老手谈不上,是另一位渔夫教我的。”陆久说,“但搅混水,我倒自认很在行。”

\section*{}

接受克鲁格的委托后,皮尔斯并没有立即去追捕陆久,而是驾车回到了自己的空军基地。

自然,动用空中火力要比在地面拦截陆久容易得多也安全得多,但皮尔斯也没有回他的指挥部,而是直接进了五号机库。那里停放着一架老式的A-12e攻击机。

那架飞机曾经是皮尔斯的座驾,陪着皮尔斯经历了许多次战斗,包括让他声名鹊起、也让他结束了自己飞行生涯的斯普利特之战。这架飞机在失去动力后奇迹般地平安着陆,让皮尔斯捡回了一条命,皮尔斯事后设法收购了这架飞机,并将它悄悄转移到了自己的名下。在那之后,又因为机缘巧合被陆久“借”了去。

毫不夸张地说,这架飞机不仅是皮尔斯的爱机,也是他和陆久之间“友谊”的见证。而此刻,它正静静地停在这间即将被撤掉的机库中。

“准将先生,你在这儿干嘛呢?”

正当皮尔斯默默凝望着那架飞机的时候,他的身后传来一个声音。说话的是英菲尔德。

“没什么,找点酒喝。”皮尔斯说。

“在机库里面找酒喝?”英菲尔德揶揄地说道,“我倒是听说在赫鲁晓夫时代,俄国飞行员因为缺酒,时常会偷喝飞机的防冻液。莫非你也喜欢这种口味?”

“不,我只是字面意思上的找酒喝。我上次和陆久在这里喝酒,还剩了半瓶威士忌。”皮尔斯摇了摇手里的瓶子说,“另外,只有米格战斗机才用乙醇做防冻液的溶剂,A-12的防冻液溶剂是甲醇,喝了眼会瞎的。”

“看来你还没喝多。”

“当然,我还没开始喝呢。”

“出去了一趟,克鲁格老板给你交代什么任务了?”

“一开始还是在谈空中支援的事情。不过发生了点情况,老爷子的想法变了,他委托我去办另外一件事。”

“什么事?”

“干掉一个人。”

“谁?”

“陆久。”

“哈哈,那可真是……和《不可能任务》里的剧情一样了。想必会非常惊险刺激,说不定还有美女为伴?”

“是吧,我也觉得是这样。以后说不定会有以我为原型而拍摄的电影呢。”

说完,皮尔斯拧开瓶盖灌了一口。

“你是说真的?”英菲尔德正色说。

“是啊,真的。克鲁格甚至许诺只要我办成这件事,就向我老爹求情取消我的禁飞令。”

“这……难以置信。为什么?”

“那当然是因为,陆久把那位大人惹恼了呗。还能为什么?”

“我不相信。克鲁格也算是个军事家,平时也是理智而冷静的,这时候怎么可能因为个人喜怒而下这种命令?”

“也许吧,不过他很不高兴可不是假的。SOG小组陷入了军方的包围圈,但克鲁格拒绝去救援,于是陆久就自己去了。克鲁格对陆久抗命一事十分不满,下令将陆久军法处置、以肃军纪。”

“SOG怎么会陷入军方的包围?根据我们侦查到的情报,她们一直都和军方以及AR小队保持了安全距离。真正陷入军方包围是该是AR小队才对啊?”

“没错,所以克鲁格向SOG小组下令吸引军方的注意,让AR小队有机会逃脱。”

“他这是要把SOG小组当弃子了?”

“显然就是这样。”

“难怪陆司令……不,还是不对。SOG小队也是格里芬的士兵,陆司令要去救她们也不违情理,克鲁格何至于如此大动肝火?”

“哼。”

皮尔斯没有回答,又灌了一口酒,这让英菲尔德有些不解。

“到底为什么?”英菲尔德追问。

“问来问去的,烦不烦?你是我的上司吗?”

皮尔斯恼怒地说道,这突如其来的呵斥让英菲尔德一惊。

“……抱歉,我僭越了。”英菲尔德说。

“你认为陆久是何许人也,格里芬的指挥官、克鲁格的战友?那都不是最重要的。”皮尔斯说着转过了身看着英菲尔德,“最重要的是,陆久是个人类。克鲁格不会允许陆久冒生命危险去救一个人形,因为这打破了克鲁格最重要的信条:人形是为了代替人类流血而存在的。这也是他成立格里芬公司的初衷。陆久不顾一切地去救Vector,这种行为不仅仅是本末倒置,还会动摇人类绝对在人形之上的地位、动摇了这个世界上最重要的一条秩序,明白吗。”

“明白了。”英菲尔德笑了笑,“是我考虑不周,看问题看得太短浅了。”

“当然。你也不过是个战术人形,对人类的理解还差得远。”

“既然如此,为什么你不去……处理克鲁格的委托呢。”

“因为我不想去。”

“为什……哦,不。没什么。”

“李,你觉得对我来说,飞行很重要吗。”

“当然很重要。”

“是最重要的事吗。”

“我认为……是的。”

皮尔斯又灌了一口酒。

“当然。”皮尔斯笑了笑,“我做梦都想着能再开这架飞机呢。”

“那你……?”

“但我同时也很好奇,如果陆久真的按他想的去做,到底会发生什么。”

“你和陆久的利益相去甚远,我觉得无论发生什么,对你来说都不会是好事。”

“谁知道呢,说不定是好事呢?”

“我不明白。”英菲尔德耸了耸肩说,“毕竟我只是个战术人形。”

“当然,你以为你是什么?”

“不过,你要是因为过去的情谊而不忍,不放让我代劳好了。我只要远远地在直升机上开一枪,保证陆司令不会受一点苦。这件事,我这个战术人形还是可以做的。”

听了英菲尔德的话,皮尔斯笑了笑。接着,他举起酒瓶灌了一大口,然后拧上了瓶盖。

“你以为我是因为心中踟蹰,才在这里喝酒的?”

“我暂时想不到其他原因。”

“你错了,李。”皮尔斯说,“我也是个军人,你知道军人就是为了去做那些没有原则和理由的事情而存在的。我不会因为考虑个人感情而对该做的事情畏首畏尾。我在这里的唯一理由,就是想看看陆久到底能搞出什么名堂。”

“你不是说他会动摇这个世界的秩序吗。这些事情难道不重要?”

“对于安居乐业的人们来说,当然重要。他们在幸福之中、他们不想改变现有的生活。”皮尔斯走到了英菲尔德跟前,看着英菲尔德的眼睛说道,“但对我来说呢。对我来说,这个世界是好的吗。如果它就这样一成不变,我们的未来,会是幸福的吗?”

英菲尔德视线下垂,没有去看皮尔斯直视自己的眼睛。她注意到皮尔斯说了“我们的未来”这句话,而他们的未来,已经差不多到了可以以小时来计算的时候。

“我不知道。”英菲尔德轻声说,“我不知道这个世界的未来是否还属于我,但它一定属于你。”

“哼,但我不那么认为。就让我们来赌一把,看看这个世界的时运到底如何吧。”皮尔斯掏出自己的左轮手枪在手里转了一圈,“我就在这里呆着,不管陆久想干什么,我都不会去干预。但如果他自己跑来我的面前,那就只能说我们就是这种——唔,那个词叫什么来着……‘缘分’?”

\section*{}

“我猜你不是因为回心转意才回来的。”克鲁格说道。

这个问题其实不需要回答,因为眼下的事情,看不到任何谈判的余地——这间屋子里持有武器的人都已经瞄准了目标,就等扣下扳机了。

陆久的背后是四个卫兵,他们的枪口全部对准着陆久,随时准备着开火,而陆久手里的枪也正指着克鲁格。谁的枪会更快,士兵们并没有十足把握,所以他们才没有立即把陆久打成筛子。

“也可以说我回心转意了。”陆久说,“只不过,朝向的不是您所希望的那个方向。”

“呵,无所谓。你只要遵从自己的内心就好,毕竟抢拿在你手里不是吗。”克鲁格笑了一声说道,“不过在那之前,我希望能让郝丽安离开。这间屋子里的事情,和她没关系。”

“可以。”陆久说。

“我不会离开的!”郝丽安愤怒地说道。她虽然强装镇定,但颤抖的声音中却难掩紧张和恐惧。虽然是元帅的秘书,但其实她并没有经历过真枪实弹的战斗。

“我得和陆先生谈谈,而你也有自己的任务。南部军团和许多指挥官都需要你的协助。”克鲁格说,“你要肩负起格里芬的未来,这些事情,我们不是已经说好了吗。”

“可是……”

“那些陈谷子烂芝麻的,都发生在你出生之前,没有理由把你牵扯进来。”克鲁格笑了笑说,“去吧,孩子。这是男人之间的事情。”

郝丽安沉默了。她看了看克鲁格,又狠狠地看了陆久一眼,然后转身走出了元帅的办公室。

“呵。”克鲁格笑了一声,“姑娘们总是喜欢感情用事。你还记得那次行动吗,在东非?最后说要撤离的时候,黛雅也是这么倔强地不肯走,就连你也劝不动她。”

“不记得了。”陆久漠然答道。

“不记得就算了。”克鲁格叹了口气说道,“说吧,你为何而来?”

“为了做个了断。”陆久说。

“了断?你想了断什么?”

“过去的一切。”

“……知道吗,我可不是在江河日下的时候才找到你的。”克鲁格靠在豪华的办座椅里,嘶哑的声音里带着一丝疲惫,“我将你捞出大牢的时候,整个公司的事业正如日中天。我本来希望有了你的协助,我更能如虎添翼……这本该是顺水推舟的事情,其实就算没有你,我一样能够实现自己的理想。比起从你那里得到什么,我更在意的是能够有你来见证,见证我们曾经一同为之奋斗的一切——真的,我甚至曾经幻想过那一刻的情景……可你却让我如此失望。”

陆久没有说话,他并没有打算反驳。他来到此处不是为了替自己辩解,而是为了听到克鲁格的发言。他毫不怀疑,45正在通过自己身上的发卡收集资料……搞不好是现场直播。

“只有大势已去却不肯接受现实的人才喜欢问为什么,但是想不到,这竟然也成了此刻我想问的问题。”见陆久没有说话,克鲁格继续说道,“是我把身负重伤的你背出战场、是我把被判处终身监禁的你救出死牢,也是我给了你功名利禄。就算你不知恩图报,至少我不该遭到如此背叛……不,无论是曾经还是现在,我都相信你不是这样的人。告诉我,这到底是为了什么?”

“事已至此,无论答案为何,都不能改变任何事情。”陆久说,“我们今天这样相对,并非因为个人感情,而是因为只有你的死,才能改变这错误的一切。所以我只能请你舍生取义。”

“‘舍生取义’?呵……我难道不是一直都在这样做吗。金钱、地位、荣誉,这些东西只要假以时日我都能拥有,但我追求的难道是这些吗?几十年的经营和奋斗,是为了实现一个伟大的梦想——我曾经以为,那也应该是你的梦想……而你却将我所建立的一切都毁于一旦,只是为了一堆军用设备?是我看错你了吗。”

“军用设备”。听到这个词,陆久笑了笑。他寻找的第一个关键词已经出现了。

“曾经我们一同出入战场,你救过我很多次,而我在这个新世界里的一切,也都是你给的。你说得没错,于情于理,我都该对你感恩戴德……但就算背负叛徒的罪名,我依然要这么做。如果你一定想要知道为什么的话,我告诉你:我已经无法忍受那些战术少女们被当做连牲畜都不如的‘设备’来驱赶和奴役了,不仅仅是SOG小组,还有所有的民用人形。她们有自己的情感和意志、有自己的愿望和诉求,她们理应得到和人类一样的权利和尊重,她们的命运不该由你这样的人来决定。她们就如同我们曾经发誓要守护的那些人一样,并不是因为受到我们的认可才被称之为‘人’,而是因为她们认为自己是人、是因为她们希望像人一样被对待,是因为她们……生而为人。”

“呵呵呵呵……”克鲁格嘶哑地笑了,“好一个‘生而为人’。这么说,你是要把那些工厂里的造物都叫做‘人’,并且决意要为它们争取一席之地了?这就是你背叛我的理由吗,真让人痛心。我的意志你并不曾去了解过,就像你不肯去了解自己一样。那些人形原本不过是供人类消遣的玩偶,它们永远不配和人类平起平坐。让它们替人类去牺牲,是将它们卑微的存在赋予更崇高、更有意义的价值!我是为真正的‘人类’福祉而奋斗,我为此倾注了毕生心血,你根本不知道我付出了什么、付出了多少。现在,只是因为自己狭隘的一己之见,就要武断地把这一切彻底推翻?你一定觉得自己就是正义,对吗?”

“克鲁格。坦白说,没有人比我更不想这么做。如果一开始我就提出我的想法,也许今天就不会是这样,这件事也有我的责任。可惜,我们已经走得太远,远到我们都已经错过了拨乱反正的机会。”陆久沉声说道。已经足够了,他心想,现在只差最后一步——

“呵,是啊。是走得太远了……”克鲁格冷笑一声,“如果一开始我也能好好考虑的话,那么也许很多事都不会发生。但就算一切都已经无法纠正,我们依然有自己的解决问题的办法,你和我这样的人,不就是为了这个目的才被训练出来的吗?”

说着,克鲁格的脚尖轻轻踩下了一个按钮。砰然一声,办公桌的前面冒出一圈防弹玻璃将克鲁格包围了起来,他所坐着的位置地板向两侧滑开,克鲁格随着座椅向着地板之下沉了下去。

“永别了,曾经的战友。你还不如在那时就战死沙场,也好过苟活到今天,然后这样死掉——真是可耻。”随着一阵嘲讽的话语,克鲁格彻底消失了。陆久见势不妙,立即侧身向着办公桌旁边的窗户扑过去,但为时已晚,他身后的士兵已经扣下了扳机。

砰砰砰、砰砰……

嘭!!

正当陆久在地上一边翻滚、一边祈祷着自己能躲开飞来的子弹时,他眼前忽然闪过一道刺目白光、然后是一声巨响。那响声产生的强烈震撼让他感觉天旋地转、挣扎着却怎么都爬不起来,而枪声也暂时地停止了。

然后,他感觉自己的身体忽然离了地——

接着,是一阵下坠的感觉。

