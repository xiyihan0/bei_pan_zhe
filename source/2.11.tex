\chapter{战争之人(尾声)}
\section*{前言}
文韬武略能征善战的指挥官,内心却是一片迷茫。技术拔尖工于心计的科学家,暗暗期待着能有人去依靠。而不知为何而生的战术少女,内心却坚定地明白一件事,那就是不顾一切地追随自己所爱的人。

每个人都有自己的角色,每个人都有自己的位置,每个人都有自己不得不去的方向。无论身居何位,在命运的大潮中,每个人都身不由己。

自己是谁、从何而来、要去向何方,正是每个人一生都在不断寻解的问题。

\lineseparator


\section*{}

“又可笑又愚蠢吧,这剧情。”

女飞行员听到她面前的男人用疲惫的声音说道。

飞行器上的帆布座椅不够宽也不够长,躺在上边更是硌得骨头都疼,所以这个男人索性蜷缩着侧躺在了机舱的中间。

她启动了自动驾驶模式,距离到达目的地的公司需要约两小时四十分钟。期间她起身查看了一下后边的乘客,竟然意外地发现他没有睡着。

对于那位乘客的问题,她不知道该怎样回答。事实上,虽然她旁听了这个男人和另一个女人之间的谈话,但依然不太明白发生了什么。她只知道这件事好像和自己有点关系。

“不。”V说,“我不太明白,你们之间谈论的事情。”

“不明白也好。”陆久翻过身仰面朝上躺着,目光和正在低头向下看着他的V对在了一起,“这就是人类之间的感情,虚伪而难以捉摸、充满了各怀鬼胎的相互揣度。不明白才是最好的。”

“……我可理解为,您刚刚结束了一段失败的恋情吗。”

听到这句话,陆久差点笑出来。总结得真是精辟,完全不像是这个什么都不懂的家伙该说出来的话。她这是从哪学来的?

不过他和帕斯卡之间的,真的是恋情吗。

陆久一时也无法给出答案。如果那看起来是一些类似恋情的东西,也许是因为他们在努力地把它装扮成那样。但陆久和帕斯卡都知道,从自利而发的感情,最终是不可能开出爱情的花朵的。

“不。”陆久说道,“我和帕斯卡女士之间的不是什么恋情,只是一场在彼此身上寻找自己所需的东西的实验。”

“但实验的结果,似乎不尽人意。”

“是啊。不过也不能怪她,想要从别人身上找到自己的归宿,本身就是不现实的。”

“那么您所寻求的,是什么呢。”

“我也不知道。”陆久轻声说,“我不知道……也许,什么都没有。”

陆久并非故意含糊其辞,现在的他真的不知道自己在做些什么。回到公司会面临怎样的处罚呢,也许会再次被关进监狱或者强制冬眠吧,说不定这次他不会再醒过来了。

那他所经历的这么多,到底是为了什么呢。

“我想,问您一个问题。”V说。

又有问题吗,陆久心想。她上次提问是什么时候了,是在北镇的海边吧。再上次是在N17战区的指挥部。

不知为何这些事情陆久记得特别清楚,可能是因为V的每次提出的问题都很尖锐,总是在不断诘问他的内心、让他不得不去审视自己。

而自我审视,是陆久所厌恶的。此刻他只想得过且过,就像以前一样。

不过,他不会拒绝V的提问。毕竟到了这个时候,还有什么不能说的呢。

“问吧。”陆久说。

“如果没有我,您和帕斯卡女士之间就不会是这样的结局吧。”

“也许吧。”陆久说。无法否认,V的出现改变了很多事情。

“让我活下来会产生怎样的影响,我想您不是没有考虑过。您为什么不放任我自生自灭呢。”

“你会自生吗?”陆久反问道,“你只会自灭吧。”

“有什么关系吗。”

陆久默默地看了V一阵,没有说话。他不知道V是不是诚心要这么说,但这样的对话实属老生常谈。以往的这个时候他早该大发雷霆了吧。

但今天他没有。

就算V的态度再怎样让他失望,他也不会再发火了。他没有那样的时间可浪费。

因为,这是他们能够共度的最后时光了。

“坦白说,因为我不想让你死。”陆久说,“我希望你能活着,明白吗。”

“我……”

陆久的话让V不知所措,因为她没想到陆久竟然回答得如此直白。在V的印象中,陆久虽然不是一个诸多面孔的男人,但也他很少谈论自己的想法、从来不把自己的内心示人。但不知为何,这一刻陆久的态度却是如此的明确。

虽然不知道这些时间里陆久身上发生了什么,但V能够感到此刻陆久变化。这个人已经和以前有所不同,他似乎已经……不再掩饰自己的内心了。

“我们认识已经好几年了,但是在一起的时间并不算多,而且经历的许多事情都很沉重。”陆久笑了笑说,“也许在你眼里这是一段蹉跎时光,但对我而言那都是美好的回忆。就算已经时日无多,但我很感谢你那时候的陪伴。”

已经“时日无多”了吗。那一瞬间,V明白了个男人为何如此坦诚的原因——对于未来,他已经彻底失去希望了。而她却和以前一样,只能束手无策地站在一旁。

飞行器的机舱里灯光很暗,但她依然能够看到陆久在对她微笑,他的笑容很温暖。上一次看到他这样的笑容,V记得是很久以前他们在N17战区指挥部里“冰释前嫌”的时候。那时的陆久是个偏远地区的小指挥官,而她,是陆久的副官。

回想起来,那些日子是多么灿烂而耀眼啊,V心想。陆司令、95队长、静默小组,还有侦查员05……他们曾经就在自己的身边,近得触手可及。但如今这些人都远去了,就像一片随风飘去的落叶,举目遥望却早已经不见影踪。

“……要是不想回公司,就去别处吧。现在改变航线还来得及。”V说着坐了下来,坐在了陆久身边的软凳上。

陆久没有说话,V的提议让他感到有些吃惊。如果说她之前饶过陆久一条小命尚且情有可原,这次的想法可是在公然抗命了。

“不。我说过由我向克鲁格汇报并承担所有后果,我不会溜之大吉的。”陆久否定地说道。

“可是回到公司,您一定会受到严厉的责罚。说不定……”

“克鲁格可是命令你看着我,别让我到处乱跑。”陆久打断了V的话,“之前你的任务受阻是因为我的原因,只要我去说明这一切就可以免除你的责任;但你刚才的提议不同,那是毋庸置疑的公然抗命。作为一个人形,无法完成任务也许可以说是因为能力不足,但故意不执行命令会是怎样的后果,你该知道吧。”

“那又如何。”V毫不在意地说,“我已经被认定是一个失败的人形,那种事对我来说,已经无所谓了。”

“但对我来说很重要。所以我不允许你那么做。”陆久平静地说道,“说真的,在酒店发现那个刺客竟然是你的时候,我其实很高兴,就算你是来杀我的。我想这大概是南美洲的疟疾留下的后遗症。的确,那时只要抹除你就能掩盖住帕斯卡的那些越界之举,但我不会那么做、就算再让我选择一次或者一万次,我也不会那么做。不是因为你是无辜的,而是因为对我来说……你很重要。”

V没有说话。陆久在南美洲罹患疟疾的事情她倒没忘,不过疟疾真的会留下后遗症吗。她对人类的生理机制不是很了解,只是觉得这后遗症的症状可真有点奇怪。

另外,陆久所说的“重要”的涵义,她也不能理解。不过这件事倒没什么影响——既然度陆久来说很重要,那么对她来说应该也很重要就是了。

“……水蛭素过敏、疟疾后遗症,您真是位体质特别的人呢。”V说。

“是啊,真是不好意思。”陆久听了V的话,只得尴尬地一笑。

\section*{}

“上次你从这间屋子离开前,我说过希望下次还是能在这里见到你。你果然兑现了自己的诺言,只不过经过上和我想得有点不一样。”

时间已经是凌晨,格里芬公司的一号办公室里只有两个人。克鲁格坐在宽大的办公桌后面,站在他面前的则是刚刚回到公司的外派人员陆久。

克鲁格的话显然是在讥讽,但他的脸上却没有蔑视的神色。除了不动如山的威严,他的声音里只有隐约的一丝倦意。

“所有的罪责都在我,我不会为自己辩解。”陆久说,“公司的任何处罚我都没有异议。”

“你太自负了。你不需要辩解、也不需要有异议,因为你根本没有发言的权力,知道吗。”

“是。”

“唉。”克鲁格叹了口气,“抽烟吗?”

“……抽。”

一盒烟被扔到了陆久的面前。陆久弯腰捡了起来,掏出一根烟点上。

“你不必解释发生了什么事情,因为我比你更清楚。”克鲁格也点燃了一根香烟,“关于帕斯卡、还有他的老板哈维尔,我很久以前就认识他们了。这些人我要比你了解的多得多。”

“是。”

“我知道帕斯卡早就想离开IOP公司了。事实上,她能在这个屋檐下隐忍那么久,我都感到吃惊。不过,她这次不单单是为了自己的跳槽做试探,她还另有目的,那就是把我和格里芬公司也拉下水。而且她成功了。从现在开始,我不得不为她所做的事情做掩护,我和哈维尔的关系也将因此受到影响。这是一次非常严重的事故,对公司造成的损失可以说是难以估量的。”

“全都是我的过失。”

“不,是我的过失。我深知帕斯卡的性格还把你派到了她那里,这件事从一开始就是个失策。我和帕斯卡是多年的相识,本该了解她的为人……哈维尔委托我帮他除掉几个竞争对手的时候,我就猜到这件事可能会和帕斯卡有关,但没想到她竟然大胆到亲自和哈维尔的眼中钉接触。不过即便如此,我也没有兴趣关心这些事,因为那毕竟还算是IOP内部的事情。但我觉得帕斯卡一定不会单干的,她会把能拖下水的人都拖下水,结果果不其然。虽然我从来都没有对她的情义和气节抱过什么幻想,但这次察觉得实在太晚了。通过这件事,你也该了解帕斯卡这个人是怎样的人了吧。”

“……是。”

“是吗,我看不是。我想你到现在还觉得帕斯卡也是个任人摆布、弱小无助的受害者,那可是她最擅长扮演的角色。你是出于同情她的境遇才替她办事说话的,我没说错吧?”

“帕斯卡她……告诉我关于她和IOP、以及之前的科研组织的事情了。她说IOP公司是她最大的阻碍,她想要摆脱IOP的控制,也说了您和哈维尔的关系。这些她并没有银芒,只是并请求我为她保密。但据我所见,她也没有想到这次前来干涉的人,会是来自格里芬公司。”

“她提到她之前所属的科研组织了吗。她都说了些什么?”

“她说了她和几位科学家因为‘遗迹’而遭受迫害的事情,还有如何进入了IOP公司、以及如何和您结下的友情。”

“……就连遗迹的事情都说了?她还真是信任你,呵呵。”克鲁格冷笑了一声,“帕斯卡这次真是做足了功课,处心积虑地想把你拉拢到她那一边。眼光一如既往地敏锐啊。那你呢,知道自己的立场吗?”

“我……”

“据我所知,关于你的过去,你一直都没有去了解过,是吗。”

“是的。”

“因为不感兴趣?”

“不,因为没有意义。”

“呵,你这么想也没错,毕竟对你来说这已经是个新世界了。不过,历史永远是有意义的,特别是关于自己的。”

说着,克鲁格从抽屉里取出了一张纸放在桌子上。

“看看这个。”

陆久伸手拿起那张纸,那是一张照片的复印件。照片已经很有些年头,不仅残破而且斑驳不堪、许多地方都模糊了。照片上是一群人的合影,可以看出那些人都是士兵,因为他们都全副武装。

那些人中有许多面部经过技术处理,被遮蔽了容貌。陆久的目光扫过那几个有限的可以看到脸的人,发现其中有一个人和克鲁格很像。

“这是您年轻时的照片吗。”陆久说。

“不只是我。再仔细看看。”克鲁格回答。

陆久继续仔细查看着那张照片,他的目光落在了克鲁格旁边的一个人身上。

那个人的容貌,也很熟悉。陆久看着那个人,眼睛渐渐睁大了。

不,不止是熟悉。那个人的样子,和他自己一模一样。

“这……是?”

“这也是你‘年轻时’的照片。”克鲁格说,“还有印象吗。”

“不,我……想不起……”陆久感到一阵眩晕,随即感到头痛欲裂。那张照片他确实有印象,但是何时何地的照片,他却无论如何都想不起来。

“有意思。再社会化改造,真是精妙的洗脑术呢。”克鲁格嘲讽地说道,“能让一个凶狠的罪犯变成良民、也能让一个优秀的战士变成傻瓜……也许还不如傻瓜。”

说完,克鲁格收回了那张图片。

“要把你再扔回监狱里易如反掌,但我不会那么做的,因为那样我就得不偿失了。不过你也别想再给我胡作非为了。”克鲁格说,“明天一早郝丽安会带你去你的新岗位,我们有一个安全又清闲的差使,到了那里你会有很多闲暇时间去弄清楚自己到底是何许人也。你该好好思考思考了。现在,你可以解散了。”

“是。”陆久说着,转身朝着克鲁格办公室的门外走去。

“等等。”克鲁格说道。

陆久闻言停了下来,他听到克鲁格起身走到了他的身后,然后感到后背一凉。

“去新岗位之前,先把不必要的东西留下。”克鲁格从陆久的后腰上取走了他藏在背后的手枪,“以后,你不用再碰枪了。”

\section*{}

就这么走了呢,帕斯卡怅然地心想。手上点燃的香烟还没熄灭,片刻前和她说话的人却已经不在了。

虽说自己早就料到他最终也不会答应,可竟然就连头也不回就走了。不喜欢的人,无论怎样讨好他都不会喜欢,真是郎心如铁。

算了吧,帕斯卡自我安慰地想着,不过是个男人。她从来不缺男人这种东西。

帕斯卡抽了一口手上的烟,被烟气呛得猛烈地咳嗽了起来——

在那个人面前自称会抽烟,不过是在虚张声势,因为香烟是扮演坏女人的必备工具。但做不来的事情终究还是做不来,不是靠装装样子就能行的。

帕斯卡用身体诱惑了多少个男人,就连她自己都说不清。因为那种事情对于她来说,已经不是需要刻意去记住的事情。就像是用货币换取自己需要的商品一样,在她的价值观里,这不过是资源交换的一种形式。

而对于这些男人中的大多数来说也是一样的,自己的身体不过是满足他们欲望的工具:“那个帕斯卡也不过是我的胯下玩物”——不仅仅是生理上繁殖欲望的满足、更是为了这样心理上的征服欲望的满足,男人们才会为她肆意挥霍着手中的财富或权力。

她对于自己所取悦过的多少男人多数没有什么好感,这些人的欲望通常不仅下作而且粗暴。但在这些男人中间,至少有一个自己并不讨厌,他就是那个陆久。

也许该说,她甚至有一点喜欢他那种愚直的温柔。

他为什么就不再回头看自己一眼呢,帕斯卡有点哀怨地想。总算相识一场,难道他对自己真的一点感情都没有吗。如果他回头,说不定自己就会动摇、说不定自己就会——

就会用真心挽留他,然后再重新开始?

她会为了陆久放弃自己的野心吗,帕斯卡自问。会像一个普通的女人那样,和那个男人一起过着平凡的生活?

不可能。就连她自己都觉得不可能。

那她为什么还要说出结婚的话呢。为他弹过的琴、唱过的歌、喝过的酒,到底算是什么呢。

真的只是为了所谓的计划、只是为了把他拉拢到自己身边吗。

那时在酒吧里说结婚的时候,其实她是很期待陆久的答复的。但是因为太紧张了,所以才假装醉酒,没想到后来真的睡着了。别看她在床上阅人无数,但那可是她第一次向别人求婚。

她很想告诉陆久,就算不答应也没什么。那时候陆久说会认真考虑时,她已经非常感激。真的。被一个人接纳的感觉,很幸福。

帕斯卡想了一阵,然后失落地笑了笑。可惜这一切都已经是镜花水月。没能说出来的话,她再也不会有机会去说了。

她知道自己应该反省,她每次都抓不住自己喜欢的人,她错失幸福的原因实在太多。她总是那么自私贪婪、急功近利,她的心里只有她自己、她从来都没有想过要不顾一切地去爱什么人。所以每个人离开她的时候,都是这样决然不肯反顾。

但那并非全部的原因。

另外的原因,是她在这条不归之路上走得太远了,这就是她无法回头的命运。

她只是对那个“如果当初没有这样”的假设还有所留恋。她只是明知道自己的最终结局就是万劫不复,但还是希望能够被什么人所救赎。

不过,要是他真的答应了的话……

帕斯卡揉了揉自己的额头。不,他当然不会答应,他怎么可能答应呢。

但如果,如果他肯答应的话?

向他妥协,也许,也不是不可能……

……

思考这些永远不会发生的假如,让帕斯卡感到很累,于是她靠着墙坐了下来。接着,她抱着自己的肩膀、把头枕在了膝盖上。

然后,在寂静的办公室里,再次恢复了单身的总工程师女士,终于无声地哭了起来。

\section{第二章的后记}
终于整理完了第二章,好鸡脖累啊。

帕斯卡机关算尽太聪明,但最后还是不抵Vector的一句“可以”。不能说帕斯卡不爱陆久,只能说她爱自己的野心更多,所以她总是抓不住自己的爱情。她爱过许多人,也有许多人爱过她,但当相爱的两个人没有把对方放在最重要的位置上时,这样的爱情必然是有隐患的。其实这正是爱情真正的样子:因为没有人能够一人处世,所以两个人之间的感情总是会掺杂它物、总是会瑕疵,不存在牢不可破,这就是真实的世俗之爱。

但在这个故事里,却有一个例外,那就是Vector。V只有陆久,她为了陆久做什么都可以。牺牲、背德,都无所谓,即使遭到陆久的侵犯也不会对他有一丝怨恨,她的爱是纯洁无暇的——纯洁、盲目、不求回报的理想之爱。

帕斯卡在官方的故事中着墨极少,特别是在性格和做事方面,她可以说是我原创的一个角色,也是我最喜欢的角色。她在这个故事中代表的,就是世俗之爱。相较于Vector这种只会出现在故事中的理想爱人,帕斯卡是每个人都会遇到的真实的女人。但世俗之爱终究难敌理想之爱,可能是因为我这个人还是有点理想主义情愫的吧,笑

故事写到帕斯卡孤独地在办公室无声地哭出来时,我也已经泪流满面。希望每个人最终都能和自己喜欢的人在一起,或者,最终能够真心喜欢和自己在一起的人。