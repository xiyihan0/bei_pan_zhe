\chapter{战争之人(三)}

\section*{}

项目的启动程序,是否应该有个剪裁仪式才对,陆久戏谑地心想。当然,他没有期待这些形式主义的东西,特别是考虑到参加启动程序的只有他和NT77两个人,甚至就连帕斯卡都没有到场。

“都是现成的东西,准备了那么久没有理由出现问题。你全权负责好了,我就不去了。”

今天清晨,睡得尚有些迷糊的总负责人女士伸着懒腰如此说道。

“设备已经全部就绪,等待启动命令。陆司令,请指示。”

站在总控制台前,作为主工程师的NT77对着陆久说道。而陆久则扫视着面前巨大的仪表台和监控屏幕一言不发。

灯光并不理想,陆久心想。不过那不过是项目停摆时的辅助照明罢了,如果设备全部启动,也会有有效的照明系统随之运行。

这该叫什么?临床试验还是什么别的玩意?陆久看着泡在培养槽里的人形素体想到。

几个月之后,这些东西就会走出这间实验室,成为能够服务大众的民用人形,而且可能还不止于此。这可真有意思。

本来是人形量产实验,但是实际上里边加入了一些之前实验中未能竣工的项目,这是帕斯卡在陆久的枕边无意中透露出的消息。

陆久感到这种小动作可能不太合规,不过考虑到帕斯卡的身份,陆久感觉自己犯不着管这件事。既然是她的意思,那就加吧。就算按正常的流程上报,最后负责审批的还是帕斯卡自己,没必要走这种无意义的过场戏。

“实验操作正式开始。所有设备,全部启动。”陆久说。

NT77推起了一片开关,然后拉下了总电源拉杆。瞬间,实验区的灯光亮了起来,一阵机器运转的嗡鸣声响了起来。其他的现象陆久观测不到,但他能看到人形素体的培养槽里气泡充盈了起来。

“正在进行素体行为模式塑化。” NT77说道,“这相当于在空白的存储器里写入基本的运行逻辑。行为模式塑化在之前的人形制造技术中也有,不过只是一些非常基本的低级逻辑:例如对目标的识别、对于命令的判断,以及自身本能反应的设定。通过这一项操作能够让人形有一个基本的行为模式,像是对接到的指令是否执行、执行的先后顺序等。但是现在我们使用的新型塑化技术能够写入更多、更高级的逻辑,用以替代后天学习和训练产生的逻辑。”

“也就是说,可以代替学习和训练的过程?那倒是能提高不少效率。”陆久点了点头,“不过,新技术能够对人形输入多少新的东西?行为模式塑化后,人形能够达到怎样的……智能水平?”

NT77看了陆久一眼。

“在……铁血的制造车间里,行为模式塑化之后,人形基本上就可以送往战场了。”

这个回答让陆久深感震惊。难怪铁血能够如此高效地生产出人形战士,原来他们的战术人形根本不需要进行任何训练。只要在培养槽里灌输关于行动和战斗的条例,这些人形就能直接作为战斗单位出战。

而相对的GK公司的战术人形,就算在参加过多次战斗之后,依然需要不断地训练去提高其作战水平,这样看来铁血的效率显然能够高出十倍不止。

当然,高效之下也有弊端,有限的行为模式限制了战术人形在战场上的发挥,所以这也是铁血人形单独作战能力低下的主要原因。

但这对指挥官来说却是一个福音,陆久在心里飞快地思索着。如果给自己一千名这样的人形士兵,那么自己只需要在指挥部远程指挥就能发挥理想的战术水平。虽然投入的兵力增加了,但是考虑到制造效率,战斗损失可以大幅忽略,只要能达到预期的作战效果,哪怕是全军覆没……

等等。这不就是铁血的战斗模式吗?陆久蓦地意识到了这一点。只要再装配上更高效的通讯系统,那么这样的战术就完全和铁血无异了。虽然极大地节省了指挥官的人力,但是这样他手下的士兵,恐怕也就不能称之为“人”了。

算了吧,陆久自嘲地想着,它们本来就不是人。他想起自己初次遇到的那两个铁血人形,除了执行既定命令之外,不会做出任何自主的响应。就算面对死亡的威胁,也毫不退缩。

这就是未来的人形发展方向吗,陆久感到有些茫然。那些生动的战术少女们将不复存在,取而代之的是严格按照命令和既定程序行动的战术机器。虽然效率倍增,但是……

不,这不对。铁血的人形被制造出来只有一个用途,那就是战斗。而人类社会上存在的民用人形的用途可不止如此,或者可以说战斗根本不是她们的主要用途。那么,在参与社会劳动的时候,这种流水线上的制造技术是否依然有效呢?

“这样的人形用于社会服务,恐怕效果不会很好吧。”陆久说。

“嗯,的确如此。” NT77点了点头,“仅仅按照铁血的模式去进行塑化的话,制造出来的只能是士兵。在塑化过程中加入更多元的模式,正是我们此次试验的主要课题。我们准备了一些简单的模块,但是其应用效果,还有待素体就绪后再进一步试验考察……嗯,第一批人形素体的塑化完毕了,我们来试一试吧。请问陆司令想要启动几个素体?”

“啊……先启动一个吧。”陆久有些不太确定地说道。

“明白。”NT77按下了一个开关,一个培养槽里的液体被抽干了,罩在人形面部的呼吸器也被弹开。然后,那个培养槽的玻璃盖滑开了。

培养槽缓缓向着一侧倾斜,躺在里边的人形素体从培养槽里跌了出来,发出扑通一声。看到这一幕,陆久微微皱起了眉头。

“素体,站起身。” NT77拿起麦克风说道。

那个躺在实验室冰冷的地面上、浑身的液体尚未干透的素体,慢慢睁开了眼睛。她有些茫然地环顾了一下周围,然后慢慢站了起来。

那是一个银色头发的少女,通体雪白。她有着雌性人类的一切性征,面容姣好、身材比例也恰到好处。但是她显然还没能适应这具初次被使用的躯体,四肢的行动不太协调,移动的时候身体还有些微微发抖。

陆久下意识的想要过去搀扶她一把,但他还是忍住了。他想起NT77对他说的话,“不要太在意这些人形”。

“应该有个适应区域。她似乎还不能很好地控制自己的动作。”陆久对NT77说道。

“有的,在批量生产车间配有烘干和舒展区,从流水线上下来的人形会在那里集合。” NT77 说道,“不过这里的实验人人形数量不多,所以就没有安装。”

“那么今天就在这里安装一套你说的那个。”陆久说。

“是,立即下单。”NT77低下头把陆久的要求记录了下来,“今天的实验结束后开始施工,明天就能投入使用了。”

“好。”

“素体,到这里来。” NT77说道。那个素体颤巍巍地走到了控制台前面。

陆久注视着那个人形,虽然隔着厚厚的防护玻璃,但他能够看到她在瑟瑟发抖。

“请试着对她下达指令吧。” NT77对陆久说。

“唔,素体……”陆久拿起麦克风,忽然又转向了NT77,“她是不是该有个称谓?”

“人形不需要名字。” NT77 淡然说道,“不过您要是想给她一个代号,也没有关系。”

“很好。”陆久冷冷地说,“那就叫她——NT-100吧。”

“可以。”NT77回答。

这是陆久对NT77那种漠然语气的回敬——他本想给这个素体命名为NT78的,但还是稍微克制了一下,改过了口。虽然如此,这绝对依然是个充满了羞辱和挑衅意味的代号。

但NT77毫不犹豫地就答应了。

“素体,你的代号定为NT-100。” NT77说道,“以后你将用这个代号做称呼,你明白吗?”

NT100点了点头。

“咳……在没有特别说明的情况下,我就简称你为‘百’好了。”陆久对自己的冲动稍微有点后悔,不过话已出口也无法收回,所以他掩饰地拿起了麦克风。

“现在,百,请向前走。”陆久说道。

“百”向前走了两步,一直走到了控制台前。

“现在请向后退。”

那个人形又向后退了几步。

“请向左侧转身。” “请向右……”

“您可以尝试一些稍微复杂的指令。”陆久身边的NT77小声说,“而且,不必每次都说‘请’。”

“唔。”陆久不悦地哼了一声。实验开始之后,NT77渐渐占了主导地位,这让他多少有些尴尬。

虽然陆久的确没有一点相关经验,但是被NT77反复指点还是让他有点愠怒。

于是,陆久脱下了身上碍事的白大褂,向着通向实验间的门走去。

“……您要去哪?” NT77略显吃惊地问道。

“过去看看。”陆久头也不回地说。

“等等,冒然接触试制人形可能会有危险!” NT77惊讶地大声说道。陆久停了下来。

“怎么,她难道装有自毁装置吗?”

“……那倒没有。”

“那还有什么危险的。”

说着,陆久打开了双层联动门,走进了实验间。

“你好,我的名字是陆久,是这里的执行官。”陆久来到“百”的面前说道。

说着,陆久朝着“百”伸出了手。

“百”茫然地看着陆久,十分不知所措。

“握手。人类的礼节。” NT77在麦克风里说道。

听到NT77的话,“百”伸出手握了握陆久的手。

“您……好,执行……官。”她用嘶哑的声音说道。

“叫我的名字就可以。”陆久说。“百”再次露出茫然的表情。

“称呼他为‘陆先生’。” NT77再次解释道。

“您好,陆……先生。”

“很好。”陆久点了点头,“你感觉如何?你在发抖。感到冷吗?”

“我……冷。” 百回答。

“拿来那件衣服。”陆久对着NT77说道。NT77拿起陆久脱下的白大褂,放在联动门里,然后打开了内门。

陆久从联动门里取出了那件外衣,披在01身上。

“谢……谢,您。” 百说道。

“提前为新激活的人形准备好衣物。”陆久对着NT77说道,NT77把这个命令记录了下来。

简单的命令测试进行了一上午。到了中午,百已经能够比较流畅地对陆久的命令做出反应,说话也清楚了一些,但依然表达得不是很好。按理说,她所掌握的语言已经是预置在她的自律核心之内的,陆久所说的话她多数都能明白地理解。不过要想做出相应的回答,则需要通过她的思想去分析,而她还没有建立自己惯用的语言体系,因此说话尚不流利。

中午,陆久和NT77依然是去餐厅就餐,百却被NT77遣回了培养槽。

“培养槽不仅可以修复人形的损伤,而且也会补充人形体内所需的营养和能量物质。”NT77解释说道。

陆久没有说什么,只是默默地点了点头。他时刻在提醒自己,百只是一个实验素体,甚至连注册的人形都还不算,所以她不可能和NT77一样去餐厅用餐。

擅自接触试制人形都是被严格禁止的,私自带出试制人形会引发多大的混乱更不用说。陆久知道自己今天已经犯规了,如果这件事被帕斯卡知道恐怕她也会生气,所以陆久不能再继续这样做了。

指挥官的生活已经结束了,陆久戏谑的心想,客观讲自己其实是NT77的副官,特别是在实验室的时候。

这一天中午的餐厅非常繁忙,SV98在为陆久和NT77端上饭菜之后就在人群之间穿梭忙碌着,没能给他们过多关照,当然陆久也不以为意。

“下午的实验内容是什么?百对基本命令的接受和执行情况已经差不多可以通过了吧。”一边吃饭,陆久一边轻声对NT77说道。

“嗯。作为最初级的行为模式塑化,她算是合格了。这个素体身上的实验,基本算是对系统运行和素体质量的测试。”NT77微微点头,“下午的话我们在塑化过程中加入一些复杂一些的模块,然后继续测试效果吧。”

“给百增加其他行为模块吗?”陆久问道。

“不,是给其他素体加载更多模块。百已经不行了,一旦人形激活就不能再次进行塑化。”

“……什么意思?”虽然陆久对NT77的话不太了解,却莫名地有一种不好的预感。

“简单地讲,行为模式塑化只能在未激活的人形身上操作,就像计算机要在格式化过的存储器上装在系统一样。”NT77说,“已经激活的人形不能再次进行塑化,因为人形在激活后从外界获取的信息会干扰塑化,造成人形的心智障碍。所以我们只能备份百的心智云图作为第一步的实验结果,然后……再激活新的素体进行实验。”

“那百呢?”

NT77沉默了一阵。

“百的行为模块过于简单,无法用于之后的反应实验。所以她已经没用了。没有使用价值的素体,通常会……在终止运行后销毁。”

听到NT77的话,陆久也沉默了。难怪NT77曾对自己说“如果太在意她们的话,可能会对今后的实验造成影响”,现在陆久终于明白其中的涵义了。

这么说,百的使命已经结束了。虽然她对这个世界尚没有一丝了解,但她以后也再也不会有机会去了解了。

算了吧,陆久对自己说。早就该知道事情会是这样,自己也早有心理准备了。

至少她的心智云图还有备份,而且会没有任何痛苦地停止运作。相比临床医学中堆积如山的动物尸体,16LAB对待这些人形的方式算是仁慈得多了。他们所销毁的不过是一些临时使用的实验耗材而已,陆久对自己这样说着。

于是一阵之后,他终于点了点头。

“心智云图备份已经完成,请求终止素体的生物活性进程。”NT77说着,将手指放在了控制台的确认按钮上。

她在等待着陆久的确认命令。

午餐结束后,两个人回到实验室的第一件事就是备份了百的心智云图,然后就开始着手进行素体的销毁步骤。

NT77说了请求终止素体的“生物活性进程”,而不是终止素体的生命活动,也就是说素体连生命都算不上吧。陆久心想。

“终止生物活性之后,素体会如何处理?”没有回应NT77的请求,陆久提了一个问题。

“……焚毁。试制人形的生物和机械零件可能带有未知的风险因素,因此她们的躯体不会被再一次使用。”NT77回答。

“如果……我们暂时让素体保持活性,是否违反操作规定?”陆久有些迟疑地问道。

他这个问题的意图太过明显了。公司和帕斯卡是派他来监督NT77的工作,但在实验正式启动的第一天,他居然在琢磨着如何钻制度的空子。

NT77显然也明白了陆久的意思。

“嗯……不违反规定。在某些时候,素体也许还有进一步研究的价值,所以不立即销毁也是可能的。只是……”NT77说着,“在实验结束之后,素体将会被全部销毁。实验的结果一般来说只需要数据。”

“是吗。”陆久点了点头。

“不过据我所知,帕斯卡女士是主张在素体上收集到足够的数据后,就尽快销毁的。”NT77补充说道,“也许……以前那次实验事故导致的后果,依然让她心有余悸吧。”

“……我明白了。”陆久再次点了点头,“那么,按照一般流程办理吧。”

说着,陆久轻轻推开了NT77的手,把自己的手指按在了确认键上。

“……陆司令?”陆久的动作让NT77稍微有些吃惊,“还是,让我来吧?”

“为什么呢。我没有这样的操作权限吗?”陆久问道。

“不,您当然有权限……”NT77微微垂下了目光,“但是,这是我的工作,不必您亲自动手。”

“你是在替我考虑吗。”陆久说。

“不是那个原因。”

“你也许看出来了,我一直对这些人形心存同情,因此你不想让我背负这样的压力?” 陆久没有看NT77,只是看着面前的按钮微微一笑,“不过有一点请不要忘记,我手下的亡魂要比你多得多,而且里面绝大多数是真正的人类。我不是会被这种无谓的道德上的罪恶感所束缚的人。”

“我知道。我并不是怀疑……”

“不过还是谢谢你,无论是或不是。”陆久说着,按下了确认按钮。

装着百的培养槽里的照明立即暗淡了下来,然后缓缓排出了培养液。百依然静静地躺在那里,看不出有任何的变化。但陆久知道,她的生命进程已经终止了,代谢也将渐渐停止。她的体温将渐渐趋于外部温度,然后如果不加以妥善处置,她的躯体将会腐烂变质。

排干培养液后,培养槽里的灯光彻底熄灭了,培养槽的玻璃罩也转变为不透明。一号培养槽变成了黑色。随着一阵滑动的声音,陆久知道那具已经停止生命活动的躯体将被抛进高温灭菌炉,然后被焚化成灰烬。

陆久看着自己按下按钮的手,他忽然感到那只手上非常黏腻。

……那是血。

那只手沾满了鲜血。他看向自己的另一只手,那只手同样的鲜红的、他的脸上也是鲜红的血,正在沿着自己的脸颊滴下。他的面前出现了一具躯体,一具失去生命、后颈被切开的躯体,旁边放着一颗感染了病毒的自律核心。

——沾满他全身的,是他的伙伴的血。他最优秀的士兵、最忠诚的战友,和最爱他的战术少女的血。

陆久感到有些眩晕,他知道自己看到的只是幻觉。

“永别了,陆久。我……我爱你……”

他的耳边响起了一个缥缈的声音。

……不。他对自己说。

那不是我、那不是她,那不是……

“陆司令?”身边有个声音在叫他的名字。是她,那个铁血指挥官,“播音员”。

陆久猛地转过了头,他感到自己全身发烫,仿佛身体在燃烧。

“下午的实验……我建议暂停。”播音员说着,“我想您需要休息一下。”

在说什么?!你这个铁血的渣滓……陆久用饱含杀意的目光看向身边的人形,愤怒在他的心头沸腾翻滚着。

“稍微休息一下吧。”播音员伸出手,轻轻触摸了陆久的脸颊。

“不。”陆久闭上眼,咬着牙说道,“继续……下一步的试验。”\section*{}

陆久忘记了自己是怎样离开实验室的。NT77似乎想要送陆久回去,但陆久拒绝了。他感觉NT77继续靠近自己的话,他也许真的会失控。

下午,他们又激活了一个人形,并给她在行为模式塑化过程中加载了交流和运动模块。这次的素体被命名为“百一”,她的样子和百完全一样,但是显得开朗多了、也健谈多了,而且肢体的行动也灵活自如。

在下午实验结束的时候,NT77再次命令百一回到了培养槽,依然是NT77备份了百一的心智云图、陆久执行了终止百一生物活性进程的命令。

操作结束的时候,就连NT77都对陆久的状态有些担忧,但陆久强忍着胃里的翻江倒海,故作镇定地离去了。

回到房间后,帕斯卡正在等着陆久。她默默看了陆久一阵,然后什么都没有说,只是把陆久的头拥在了怀里。

虽然陆久的表情十分平静,但他的眼神里的一丝涣散,逃不过帕斯卡的眼睛。她知道陆久的想法,而且作为此次项目的拟定者,她也知道陆久今天经历了什么。于是她解开了自己的文胸,拉着陆久的手伸进了自己的衣服,将那只手放在自己的胸前。

“感觉好些了吗。”帕斯卡说。

“……唔。”陆久哼了一声,声音里有些疲惫。他感到自己的头很疼、而且胃里也在不断翻腾着。

“摸摸我的胸吧。”帕斯卡说道,“女人的胸部有让人镇定的作用,因为乳房是人类对母亲的第一印象。”

这温暖柔软的触感,就是母亲的印象吗,陆久心想。将手放在这丰满的胸前,一定会感到十分安心吧,也许能让人想起母亲的怀抱。

但是可惜,陆久对母亲毫无印象,因为他是在军营里长大的。所以没有什么能够给他安全感,除了冰冷的武器。

“我去洗个澡。”陆久说着,抽出了按在帕斯卡胸前的手。

冷水从喷头淋下,冲刷着陆久的身体。这让他稍稍感觉好了一些,胃里想要呕吐的感觉也消减了不少。

那些事情……他还是没能忘记。陆久还记得皮尔斯曾经问自己,如果战友在战场上牺牲,他会难过多久。

陆久记得他的回答是一分钟,但现在已经过去了半年。

本以为只要假以时日,自己就能摆脱往日那些让人压抑的回忆,但陆久竟然没能做到。他本以为自己可以的。

他不知道自己什么时候变得如此多愁善感了。他依然隐约记得以前的战斗场景,战友在自己身边倒下甚至炸成碎片,他也不曾停止冲锋的脚步。但他不知道现在的自己是怎么了。

虽然还残留着那些回忆的碎片,但是从监牢里醒来开始,也许自己就已经不是从前的那个自己了。

不知道,以前的自己是怎样的人呢。站在冷水的淋浴之下,陆久默默地想着。他忽然想起来,某个非法人形在离开之前,曾经给他留下了一些东西……一枚通往自己过去的钥匙。那里边会有什么呢。

那是一个U盘,但陆久从未把它接入过计算机。

知道自己的过去又如何,不会改变任何事情,陆久很明白这一点。时光不会倒流,他也无法回到过去。

所以就算知道了那些,也毫无意义。他曾经拥有过的一切,已经都不再属于他了。就像在战区经历过的那些日子,无论曾经发生过什么、遇到过谁,现在也都已经是历史。

纵然思想会回溯,但时光却在一直前行,故去之人永远不会再现。

……别想那些了。陆久对自己说。他现在能做的只有沿着已经被设定好的路线一直向前走,期待有朝一日能够重获自由。到时候他要做一个怎样的人,就等到时候再说吧。

一边这样想着,陆久忽然听到了身后传来一阵动静,有人走入了浴室。不用去看陆久也知道是谁。

“我可以,使用你的浴室吗。”陆久身后的声音说道。

“请便。”陆久说着,关掉了水龙头,然后拽了一条浴巾披在身上朝外走去。

他知道帕斯卡想用的不只是浴室,但是现在的他只想静一静。所以他绕过了挡在他面前的帕斯卡,直接走向了卧室。

当陆久醒来的时候,天已经完全黑了。他来到亮着微弱灯光的客厅,看到客厅的茶几上摆着几盘饭菜,而帕斯卡则坐在一旁的沙发上默默拨弄着手机。

“饿了吗。吃饭吧。”见到陆久走出了,帕斯卡说道。

陆久看了一眼茶几上的饭菜,发现那不是餐厅里的产品、也不是饭店的外卖,好像是从自己厨房里端出来的。陆久有些纳闷,自己从来没有采购过任何食材,这些东西是哪来的呢。

“我从餐厅拿了些食材,在你的厨房里随便加工了一下。”帕斯卡说道,“尝尝吧,希望合你胃口。”

陆久这才明白,这是帕斯卡亲自下的厨。是的,她的确给自己做过一次早餐,那是他们初次共度良宵之后的早晨。

陆久看着面前的菜肴:蔬菜和肉类的合炒、开水烫过的蔬菜和油煎的鸡蛋,还有一碗色泽均匀的黍米熬制的汤,或者叫做粥。眼前的饭菜虽然简单,却要比那次早餐丰盛得多了。

“谢谢。”陆久说着拿起筷子,夹起荤素合炒的菜尝了一口。

……味道很好。虽然只是没有名字的家常菜,但也有葱蒜爆香、八角出味,烹制手法是内行人的手笔。想不到,看似懒散的帕斯卡还有这样的手艺。

陆久赞许地点了点头,风卷残云般地消灭了面前的饭菜。他不是不想说些赞美之词,但他已经没有功夫说话,他的确是很饿了。

“慢点吃。又没有人和你抢,急什么。”看到陆久的吃相,帕斯卡好笑地说道。

“不好意思。”陆久擦了下嘴角说道,“战场上通常没有多少空隙去吃饭,有时甚至需要一边行军一边进餐,所以习惯吃饭时速战速决了。另外,厨艺很不错。”

听到陆久的赞誉,帕斯卡再次笑了起来。不过她这次没有说话,只是默默站起身收拾了餐具,然后走进了厨房。

看着厨房里洗刷碗筷的帕斯卡,陆久忽然感到今天她有些异样。虽然依然是有点懒洋洋的,而且穿的还是随意的便装,但是她的身上多了一股温婉的气质。

是因为围了围裙的缘故吗,陆久心想。看到这样的帕斯卡,大概很多人都会把她当做一位贤淑的居家女士,很难和那个在天台上喝得酩酊大醉的女人联系在一起。

……也许每个人都有很多面,陆久心想。酒鬼、特技车手、钢琴演奏者,以及天才技术员,这些都是帕斯卡。还有此刻的厨娘。

他们不像自己,永远都只有一面。

像自己这样的人,世界上到底有多少?陆久自问。

也许有很多。自己手下的战术人形们、V、95……还有NT77,不都是这样的吗。从出生开始就被绑在了宿命之柱上,只能按部就班去面对自己的命运,没有任何其他的选择。不过,她们和陆久却又不同:因为陆久是人类,而她们……

她们不是。陆久心想。

无论从生物学还是从社会学来看,她们都不符合“人类”的定义。她们只是外表和人类一样罢了。

是啊,陆久对自己说。回到现实吧。看一看这繁华的城市、人类的聚居地。这里有着数以千万计的人类。

这里也有数量可观的人形,但她们永远不会成为这个社会的主体。因为她们只是人类制造出来,用于为人类带来便利的工具罢了。

“要出去走走吗?”陆久忽然听到跟前传来一个声音,是帕斯卡在对他说话。他这才意识到自己太出神了,就连帕斯卡走出厨房、来到自己面前都没有发现。

“啊,不……”陆久有些茫然地说道。

“走吧,”帕斯卡微微笑了笑,“我请你喝一杯。”

“……好吧。”

两个人驱车穿过外滩的滨海大道,再次来到了之前曾经来过的海边。不过这次帕斯卡没有耍特技、也没有把车开到海里,而是把车稳稳地停在了沿海的公路边。

帕斯卡掀起汽车前边的储物箱,借着储物箱里的照明,陆久看到里边放着一个巨大的塑料箱,有一大堆啤酒罐埋在冰块里。

也不知道帕斯卡是不是整天就在车里准备着啤酒,不过这些显然是刚刚装进车里的,因为塑料箱里的冰块几乎都还没化掉。

“接着。”帕斯卡说着将一罐啤酒扔给陆久,然后自己也嗤啦地拉开了一罐。

“啊——夏天的啤酒就要这样喝才够劲呀!整天在办公室里吹着空调,让人连饭都吃不下,怎么能有心情喝酒啊!”

帕斯卡把那罐啤酒一口气喝掉了,然后将空罐用力向前扔了出去。

这算是现出原形了吗,陆久也喝了一口啤酒,心里想着。帕斯卡之前在陆久屋里的那一丝贤淑的气质,已经荡然无存。

“这酒怎么样,美酒鉴赏家?”帕斯卡对着陆久说道。

“很浊、很苦,不是当地啤酒。麦浆含量很高,酒精度也很高,是东欧货吧。产地是捷克?”

“你真的是内行啊。”帕斯卡惊奇地说道,“你以前肯定不是当兵的吧?”

“关于我的过去,我自己也所知不多。但只有军籍这一件事不用怀疑。”陆久耸了耸肩。

“今天的实验报告……我已经看过了。”帕斯卡的声音忽然低了下来,“我知道你感觉不好受,不是这个行业里的人……第一次遇到这种事情,都会感到有压力。你的心情,我可以想象。”

听到帕斯卡的话,陆久笑了笑。这次她没说“我理解你的心情”,不知道是不是对自己上次的斥责依然介怀在心呢。

“不要紧。我现在已经感觉好多了。”

“实验进度推迟一点也没什么关系,不要勉强自己。”

“不用,我可以应付的。”

“那就好。不过别忘了你是执行官,有必要的话随时可以暂停实验。”

“我知道。”

简单的对话之后,两个人陷入了沉默,各自喝着自己的啤酒。

“喜欢大海吗。”在喝到第四罐啤酒的时候,帕斯卡忽然开口说道。

“喜欢。”陆久说。

“……你回答得真痛快啊。”帕斯卡稍微有点惊讶,“我还以为你没什么喜欢的东西呢。”

“的确,我对喜欢和不喜欢的概念比较模糊。不过面对大海的时候,会让我感到内心平静一些,这大概该算是喜欢吧。”

“不过据我所知,你不是出生在沿海地区的人。”

“我不知道那些。你又是怎么知道的?”

“呃,这个……”

面对陆久的反问,帕斯卡一时语塞。看来她知道一些就连陆久都不知道的事情。

不过陆久并没有追问下去。

“我的记忆里,确实没有太多关于海的回忆。我关于海的唯一的记忆,是在海边执行过一段时间的任务。”

“是吗。是什么任务?”

“是一些,关于……”

陆久说道这里停了下来。他忽然想到, “在海滩上排除爆炸物”这种事情如果说出来的话,一定会勾起帕斯卡那些难过的回忆。所以他没有继续说下去。

“我也记不太清了,都是很多年以前的事情了。不过无非是站岗放哨巡逻之类,也没什么特别的。”陆久撒了个谎。

“是吗。”对于陆久的话,帕斯卡并没有怀疑,“我呀,也很喜欢大海。小的时候,广阔无垠的大海的另一面有些什么,总能引发我无限的遐想。虽然北京是个内陆城市,但是我小时候经常去相邻的秦市过暑假。那里有个叫北镇的地方,不知道你去过没有。那是个风景秀丽、依山傍海的小镇,有很多公司啊机关什么的都在那里有疗养院。我们家也在那里租过一所位于海边的别墅,每当夏天到来的时候,父母就会带领我去北镇……我还记得那个别墅的主人是一位独居的妇人,听说她丈夫过世得早,虽然有个女儿,但是可惜也英年早逝,她就没有其他亲人了。那时候我还小,也不记得她的名字,只是管她叫阿姨。上大学后我就再也没有去过北镇,想起来已经有很久没有见过她了,也不知道她现在如何……”

帕斯卡一边喝着啤酒一边诉说着自己年幼时的往事,而在一旁听着的陆久思绪一阵飘忽。

秦市、北镇、独居的妇人……这些东西何其熟悉。陆久和这些地方的关系,帕斯卡显然并不知情,她甚至不知道在陆久来到16LAB之前,刚刚在北镇度完“假”。该说是冥冥中自有天意吗,陆久心想。虽然相隔了几十年的时光,但他和帕斯卡居然在同一片海滩上徜徉过。

“对了,有时间的话一起去那里玩玩吧。北镇是个好地方,虽然地方很小,但是很宁静安详,你一定会喜欢的。我也想再去看看那位房东,她现在应该还在吧。”

“啊,这个。”面对帕斯卡自顾的提议,陆久有些失措地回应道,“有时间再说吧。不过,等我结束了这里的任务之后,公司应该会马上指派新的工作吧。毕竟我在公司的职务不是什么闲差。”

北镇他也许永远都不会再去了,不管在那里等着他的是谁。但此时陆久只想尽快找个说辞搪塞过这件事。

“……我知道。”帕斯卡轻声说道,“关于你的事情,克鲁格对我说起过一些。我知道你依然是在……服役。”

说着,她拿着两罐啤酒走到了陆久身边,将其中一罐递给了陆久。

“不过,16LAB和格里芬公司合作非常紧密,在这里长期工作的话,也算是为公司效力了。”帕斯卡站在陆久的面前,看着他的眼睛说道,“要是我向克鲁格请求一下的话……我是说,不知道你是否愿意……留在16LAB呢。你看,我这里正好缺一个你这样,能够一丝不苟地监督各项工作开展实施的人。”

陆久没有说话。他看了帕斯卡一阵,然后默默移开了目光。

他知道帕斯卡的邀请意味着什么,也承认他对这个邀请有些动心。但是他依然觉得帕斯卡那殷切的目光让自己无法承受。

“不要紧,”面对陆久的沉默,帕斯卡笑了笑说道,“我知道这个请求有些失礼,也知道你没有这样的准备……不过如果有可能的话,希望陆司令,今后能够稍微考虑一下。”

“不,我并不是要……拒绝你的建议。”陆久低声说道,“我只是感觉,这样的生活不适合我、也不属于我。我总是觉得自己还有……还有一些……”

“还有一些未竟的事业。”帕斯卡笑着把话接了下去,“那也不要紧。就等你完成了那些未竟之事,然后再考虑这个建议也没关系。你没有马上拒绝,已经超出我的预期了。”

“……啊。”话已至此,陆久也不知该再说些什么,“是吗。”

说着,陆久拉开了帕斯卡递过来的啤酒,喝了一口。

“呵呵,知道吗。在我眼里,你就像一个流浪者一样,和现实格格不入,就连背影都透着孤独。但我有时又会想,也许那就是你的魅力所在。”也许是喝多了,帕斯卡忽然低下头笑了起来,“不过我还是觉得,你离开人群已经太久了,久到甚至忘记了有人陪伴的温暖。我想这样一定很辛苦,因为人是社会性的动物,没有人能够独自一人孤独终老。所以,早些回到人们中间来吧。”

陆久没有回应帕斯卡的话,他感到有些迷茫。

他一直都认为自己是个战士,端起武器向前冲锋就是他的命运。但他忽然意识到,其实这种观点早就遭到过批判了:他的这种做法,不仅皮尔斯反对,就连克鲁格也反对。战死沙场、马革裹尸,真的就是他的命运吗?显然不是。他完全可以坐在指挥部运筹帷幄,就像其他指挥官们那样,他们做的那些陆久也能做到——而且能做得更好。

克鲁格曾经对他说,战场上能建功,但要活下来才能立业,他也不希望陆久有朝一日死在战场上。那么活下来之后的生活又该是怎样的呢?

也许,正如帕斯卡所说的,他最终还是要回到人群之中。

想一想,虽然他的交际圈很小,但也有一些人劝诫过他,也给他指出了前进的方向,例如皮尔斯、克鲁格,还有帕斯卡。虽然人数寥寥,但这些人都是些非常了不起的人,在各自的领域里都有着举足轻重的地位。这些人,对他是发自内心地珍重的。

看起来,他陆久虽然习惯独来独往,但也并不是无人在意的孤家寡人。

难道说,他也并非生来孤独,只是他自己选择了孤独吗。就像郊外的野生动物一样,明明有着安妥的归宿,却偏偏要一意孤行。

陆久自嘲地笑了笑。这么说,他也是个非常任性的人啊。

“你笑什么?”看到陆久表情的变化,帕斯卡问道。

“没什么。”陆久说,“你的提议,我会考虑的。虽然现在还无法回答,但是总有一天……我会想好答案。”

“那么无论答案是什么,我都会期待那一天的。”帕斯卡笑了笑说,“今天的工作辛苦了。多次承蒙陆司令的照顾,这次,就让我来慰劳你一下吧。”

说着,她搂住陆久的脖子,吻上了他的嘴唇。

如果床笫间的欢愉是出于生理上的需求、紧密的拥抱是对安全感的索取,那么这个温柔的吻又是代表了什么呢?陆久不知道。

但他没有拒绝那双柔软的芳唇。\section*{}

那一夜帕斯卡是在陆久的房间度过的。无论是在谁的房间过夜,同居生活对他们似乎已经成了一种常态,但那一夜却又有所不同,他们是相拥而眠的。

因此陆久感觉他们之间的关系似乎更加微妙了,甚至已经有点超越了相互慰藉的范畴——他们彼此之间的需求已经不仅是身体上的,也许还有了一丝内心上的依赖。

但陆久终究没有回吻帕斯卡,虽然他并不能确定亲吻的意义,但是他毕竟不是懵懂的少年了。他知道那种意义上的回应就意味着应允——对很多事情的应允,包括一些他能做到的、还有很多他做不到的。但帕斯卡并没有责怪他什么。

天刚刚微亮的时候,帕斯卡首先起床离去了。但她并没有趁着陆久熟睡的时候悄然离开,而是先做好了早餐,然后轻声唤醒了浅眠的陆久。

“我要去一趟北京,参加一个关于技术的会议。”帕斯卡说,“会议进程快的话可能明天上午就回来,慢的话明晚也该差不多了。实验的事情就交给你和77了。”

“知道了。”在洗手间里用冷水冲洗脑袋的陆久说道。

“自己收拾桌子哟。”

“啊。”

独自用餐后,陆久收拾了餐具,然后来到了实验室。NT77已经在那里等着他了。

陆久注意到NT77的脖子和手背上再次出现了伤痕——也许她被衣服覆盖的身体上还有更多伤痕,但陆久没有过问。他只是简单地打过招呼就开始了实验。

这一天的实验是关于人形对外部环境的耐受程度的测试,他们激活了四个素体,分成两组置于高温和低温的环境中,观察她们的反应。

高温环境下的素体因为缺乏散热手段,仅凭自身代谢散热不足以抵抗炎热的环境,在四个小时后终于因为脱水而休克了。而低温环境里的素体则依靠相互依偎取暖,坚持了差不多七个小时才失去知觉。这一测试的结果都证明这些素体有着超越传统人形的身体素质,并且采取的自救措施也是符合预设程式的,极端环境并未对她们行为模式造成干扰。

实验结束后,陆久亲自将失去意识的素体搬进了培养槽,然后进行了处理。

“您今天的状态好了很多。”在离开实验室前,NT77对陆久说道。多亏了陆久的状态调整,这一天的实验提前完成了。

“昨天只是意外。”陆久说,“这种事对我来说……本来,也不算什么。”

“那样的话就好。明天见,陆司令。”NT77点了点头,将大褂脱下挂在衣帽间准备离开。陆久这才看到,她裸露出来的手臂上到处都是伤痕。

“你的手臂,不要紧吗。”陆久脱口说道。

“呃,这个。”NT77慌忙用手掩盖住了身上的伤痕,“没事,回去处理一下就好。不用陆司令费心。”

“就在这里处理一下吧。”陆久说。他知道实验室的培养槽也有修复功能,而且技术上是最先进的,比那种民用小型设备要高效得多。

NT77楞了一下,能在这里处理当然更好,但她知道自己无权随意动用公司的实验设备。不过如果是陆久的命令的话,她就可以正当地使用了。

“……是。”NT77服从地说道,然后脱下衣服走向了一号培养槽。

陆久看着面前的人形工程师有条不紊地脱下外衣和内衣,心中不禁颤抖了一下。因为他看到NT77身上的伤痕几乎遍布全身。

NT77躺在培养槽里,然后戴上呼吸面具,朝着陆久比了一个ok的手势,陆久启动了机器的快速修复功能。微微有些粘稠的培养液快速地充盈着槽胆,将NT77浸泡了起来,她身上的伤痕正在以肉眼可见是速度愈合着。

陆久注视着培养槽里的NT77,不由得再次感到了熟悉的感觉,那些军区的受伤人形们也是这样进行修复的,只不过军用培养槽恐怕也没有这种实验室级的设备效率高罢了。陆久立即摇了摇头,将那些不合时宜的回忆从脑海里驱赶走了。

半小时后,NT77走出了培养槽,然后在新安装的干燥室里烘干了身体,来到了陆久面前。

“谢谢您的关照,陆司令。” NT77对着陆久微微鞠躬说道。

“快穿上衣服。”面对全身赤裸的NT77,陆久下意识地转过了头。

“……抱歉。”NT77脸颊一红,快速地穿起了衣服。

“我先走了。”陆久转身朝着门外走去,“记得关灯。”

“是。”NT77在他背后说道。

“今天您的状态很好”,陆久边走边回味着NT77的话。

是啊,今天他的状态很好。在按下终止人形运作的按钮时,他的心里已经没有了一丝波动。

不仅如此,看着那几个素体在模拟环境中,因为难以忍受痛苦而扭曲着身体的时候,他也毫无感受。

不过如此吧,他漠然地心想。就像看着被注入毒素然后等待死亡的小白鼠一般,他能有什么想法呢。不过如此。

对于这些只有一天生命的素体来说,就算是痛苦也是短暂的——相对那些要无数次目睹这种痛苦的人而言。

走出实验室,陆久下意识地掏出手机,发现上边没有任何信息。

当然没有,唯一一个可能给他发信息的人,已经去北京开会去了。

啊,自己这是在期待什么呢,陆久有些烦躁地心想。难道已经开始依赖她了么。陆久把手机塞进兜里,朝着餐厅走去。餐厅里一个人都没有,到用餐时间还有一两个小时,这个时间来这里确实早了点。

“您好,陆司令。”看到陆久到来,SV98快速迎了上来,“今天下班真早啊。要吃点什么吗?”

“啊,不……”陆久有些尴尬地说道。其实他根本不饿,只是没有地方可去才来这里的。

“怎么,还不饿?”SV98微微一笑,“陆司令不会是闲逛到了这的吧?嗯,很有可能。据我所知,陆司令还是第一次在除了午餐之外的时间到此呢。”

“啊,那到没错。”陆久不好意思地笑了起来。 SV98说得没错,他之前从来没在餐厅吃过晚餐,每次都是和帕斯卡在外边吃的。

“没关系,就算不是用餐时间餐厅也是对外开放的,陆司令想什么时候来都可以。我给您倒杯水吧。”SV98笑着说道,陆久感谢地点了点头。

这个女孩很有朝气,总是笑盈盈的让人感觉如沐春风。陆久看着SV98倒水的背影想到,不由得在脸上露出了一丝微笑。

不过那一丝微笑旋即就消失了,因为陆久又想起了实验室里的素体、想起了NT77身上的伤痕。

……只是预设好程式的工具罢了,陆久对自己说。素体、NT77、SV98,还有那些战术人形。都一样。

“陆司令……您,怎么了吗?”陆久忽然听到耳边传来一个声音。倒水回来的SV98,被脸色忽然暗淡下来的陆久吓了一跳。

“啊,没有。”陆久慌忙整理好了自己的情绪,“没事的,没什么。”

“嗯。忽然看到您的脸色很难看,还以为是我……”SV98微微欠身说道。

“没有。只是今天的工作有点太多了而已。”

“还请陆司令多保重身体。”SV98礼貌地笑了笑。

“对了,SV98,我想问你个问题。不知道……”

“请问吧,我有时间。”SV98立即说道,看来陆久的对话模式她已经很熟悉了。

“咳。真是不好意思,总拿些无聊的事情麻烦你。”陆久清了清喉咙说,“今天我在实验室,看到NT77……我是说主工程师女士的身上,有些……人为的痕迹。你觉得会是……”

“您是说伤痕吗。”SV98轻声说道,“可能是她昨天晚上,遇到了性格不好的客人吧。”

“果然是这样吗。”陆久低声说道,“这么说,你也遇到过同样的事情?”

“我……”SV98迟疑了一下,在那一瞬间她脸上礼貌的微笑消失了。不过马上她的嘴角就再次翘了起来,“算是吧。不过很少。我遇到的多数都是些……都是些,普通的人。”

“……”

陆久没有继续问下去。他在NT77身上已经连续两天看到伤痕了,难道她服务的都是些“不普通”的人吗。那到底会是些什么人呢。

 “SV98,请问,你下班后有约吗。”陆久忽然问道。

“啊,没有。请问陆司令……”

“那么,晚上能到我那里去一下吗。”

“当然可以,不过我……”听到陆久突然地邀请,SV98有些惊慌地说着,“我下班会很晚,所以……”

“没关系,我可以等。我晚上也不会很早休息的,我在30楼的30号房间。”

“是……好的。”

“那晚上见。”

说完,陆久起身走出了餐厅,只留下有些木然的SV98。

离开餐厅后,陆久独自来到了天台。他在天台上望着夕阳里的城市,一直到太阳落山、街道上的灯光全都亮了起来,才离开天台。

虽然毫无胃口,但陆久还是去街上的餐馆随便吃了点东西——他没有去餐厅,因为考虑到晚上和SV98的约会,他现在去餐厅可能会打扰她的工作。用餐之后,陆久回到了办公室。NT77不在办公室——这是当然,现在早已过了工作时间,她应该去做自己的事情了……至于是什么事情,陆久尽量不去想。

也许是因为一天的工作让人身心疲倦,陆久靠在办公室的座椅上睡着了。等他醒来,时间已经是深夜。想起和SV98的约会,他急忙起身,快速走向了他的客房。但还没进屋,他就发现有个人站在门口。

那是一个年轻的女孩,身穿一条蓝色牛仔裤、上身是白色T恤衫,金色的头发在后脑勺上扎了一个简单的马尾辫,看起来身形十分窈窕。

陆久看了一阵才看出那是SV98,因为换下了平日穿着的服务员的制服和围裙,所以陆久一眼没能认出她来。

“啊,你来了。”陆久快步朝SV98走去,“抱歉,我回来晚了。久等了。”

“不……没有。”SV98有些惶恐地说着,“我也才下班不久,回去换了衣服就来了,没等多久。”

“那就好,请进吧。”陆久说着打开了房间的门,SV98跟着走了进去。

“请坐。”陆久倒了两杯水放在茶几上,对SV98说道,“我的起居比较简单,没什么能招待的,只有水,不周之处还请多包涵。”

“呃,不必……” SV98站在原地没有动,显然是有些局促,“我才是。虽然……虽然已经不是第一次了,但每次都适应不了。还有……因为不知道陆司令喜欢什么样的装束,所以穿得比较随便,失礼之处还请陆司令包涵才是。”

“啊。”SV98的话让陆久有些不明所以,“没什么失礼的。装束什么的,无所谓吧。”

“也、也是,反正很快就要脱掉……” SV98紧张地说道,“那么,请问陆司令……是要去屋里吗?还是在这里……”

“……”

陆久没有说话,他大概明白SV98在指什么了。

——她以为自己被叫来,是为了所谓的“其他的服务”吗。

“不,我不是为那些事情请你来的。”陆久说道,“只是想和你聊聊,说些……无关工作的东西。没有其他的,真的。”

“是、是吗。那么……”

“还请坐下来说吧。”

“好。”

SV98小心地坐了下来,但是只坐在了沙发的边缘上。看得出她依然十分紧张,完全没有在餐厅那种自然而放松的感觉。

为了缓和气氛,陆久笑了笑。

“怎么了。白天的时候不是好好的吗,现在却好像很紧张?”

“因为白天的时候在餐厅,是在公共场合。但是现在,这个……”

原来是这样啊,陆久心想,是在这种私密的空间里让她感到紧张。看来她对这种别人的房间,有着一些不好的回忆。

“那我们出去走走吧。”说着陆久起身朝着门外走去,SV98赶紧跟了上来。

“……去哪?”

“楼顶。”

经过帕斯卡的房间的时候,陆久进去在冰箱里翻出了几罐冰镇啤酒、又拿了两罐冰咖啡,然后才走上了天台。

“喝啤酒吗?”坐在天台的边缘,陆久对着SV98问道。

“不……不喝。”

“那就喝咖啡吧。那个人屋里只有这些。”

“好,谢谢。不过刚才那个……” SV98吞吞吐吐地说着,“那是,帕斯卡女士的房间吧?”

“啊,是啊。”陆久拉开一罐啤酒喝了一口,“怎么了?”

“没什么。”

“那么,现在感觉好些了吗。”

“谢谢。”SV98终于笑了笑,“感觉好多了。”

陆久点了点头。他也看到SV98脸上的神色自然多了,就像鱼回到了水里一样。

“其实……就如刚才说的,我请你来是想问些闲事,关于人形的事情。你知道,我是个指挥官。之前我一直在北部战区的某个偏僻区域工作,接触的人形和人类都不多,手下的人形又都是见识少、想法简单的战术人形,所以我对人形的事情了解得也不多。来到16LAB后,我负责一个……技术改进项目,你也看到了,我和主工程师是同事。但是说实话吧,我们相处得并不顺利。”

“你们看起来很融洽啊。” SV98有些诧异地说道。

“那只是表面上。的确,我们在工作中配合得很好,但我总觉得主工程师似乎有很多……不为人知的事情,总是给人一种拒人千里的感觉。当然,我和她并不熟悉,也不好开口询问,所以才想问问你。就像今天下午我说的,我在她身上看到了很多伤痕……但我想这些事我就算问她,她也不会告诉我吧。”

陆久撒了个谎,关于NT77的事情,他其实了解得比多数人都要多。但是关于人形在这个城市里,或者说人类社会里的生存情况,他的确谈不上有多了解,这也是他希望能从SV98那里打探的事情。

“您是说,那些事情吗。” SV98有些迟疑地说着,“我不知道您是怎样看待人形的,但是那些大概是……一些人类的粗鲁的癖好,留下的痕迹吧。”

SV98说话的时候很小心,尽量避免着把人类描述为热衷于对人形造成伤害的动物。但陆久还是很明白地听清了SV98的意思。

“我记得下午的时候,你说你也遇到过同样的事情。”

“嗯……是的。我想每个人形都有过同样的经历,只不过是主工程师女士最近遇到的较多吧。” SV98说着笑了笑,“啊,您不必在意我,陆司令。我这样地位卑微的人形,下班又很晚,很少有人会对我感兴趣,所以遇到这种事的时候也要少得多。”

“如果不喜欢的话,为什么不拒绝呢。虽然人形是为人类而服务的,但也仅限于合约中的工作范围吧。据我所知,即便是专门租赁人形的人形出租公司,在派遣人形的时候也不得违背人形的意愿去安排工作,更何况是这种……”

陆久说着停了下来,他在仔细挑选着词汇,“嗯……这种,不太容易接受的……”

“话虽如此,但人形也有人形的苦衷。” SV98凄然一笑,“合约中的工作薪酬含有一部分租赁公司的租赁费,剩余的部分往往只够制服人形的日常维护费用。到了规定的使用年限,人形就要报废,如果要延期则要支额外的风险保证金和管理养护费用。没有哪家公司或者雇主愿意为了一个型号老旧的人形支付这些费用,所以人形不得不自己劳动来筹措。但合约之外的劳动往往又是公司或者雇主不允许的,那么一个人形还能靠什么来挣取额外的收入呢,虽然‘不太容易接受’,但对于人形来说也不得不接受了吧。毕竟人形不是人类,没有社会保障和退休后的福利,一旦达到使用年限,面对的只有被销毁的命运。而且说到底,这具女性的躯体又能用来干什么呢,无非是做一些这种,嗯……”

SV98没有继续说下去,陆久也随之沉默了。被最大限度地榨取劳动力,虽然不公但却无力抵抗,这就是人形的命运。

多么讽刺啊,这和多年前一些人类的生存状态如出一辙。陆久也隐约记得以前的自己曾经和一些军官谈论过所谓“退役后”的生活,有很多军人在退役后选择了做雇佣军,因为他们在社会上没有存在的位置。就像很多生活在社会底层的人类一样,他们不得不为了财产而拼命劳动、一刻也不敢松懈,因为他们必须为自己年老之后的居住和医疗而积攒财富。

放眼如今的人类社会,由于人口的锐减,人均占有的资源大大提高,人们已经几乎不必再为退休后的生活发愁。多数国家都有着相应的福利保障,虽然不能让人奢侈地生活,但至少医食住行都有所保证。但这并不代表整个人类社会就没有了阴暗的一面。事实上,剥削和压榨依然存在,只不过这种命运落在了人类的造物——人形的身上。

这些人形虽然没有繁育的功能、也没有人类的基本权利,但是她们心中期望生命得以延续的愿望,和人类是一样的。

不过,自然没有人会为她们说话吧,陆久心想。如果她们只是被当做创造财富的工具的话。

汽车到了使用年限就会报废,喜欢汽车的车主可以缴纳环保费用来延长汽车的使用期限,但是没有人会同情因为年限到期而报废的汽车。人形也是如此。

“原来是这样吗。”陆久终于说道,“抱歉,我真的不了解这些。也许你不相信,其实……我不是个生长在这个时代的人,所以对你说的事情我是第一次知道。”

“您不必道歉。” SV98笑了笑说道,“虽然我不太明白您的话,但是我能看出来您和多数人都不同。您对人形非常友善,很少会有人这样——很少会有人把人形,当做人类一样对待。就像主工程师女士,虽然她是一位技术人形,但是在您到来之前,从来没有人会和她一起用餐。”

“……是吗。这么说,她也是为了你说的那些才那么……努力工作的吗。”

“我想不是。” SV98轻轻摇了摇头,“她应该是为了……证明自己存在的价值,才如此努力的吧。”

“此话怎讲?”

“您大概也知道吧。主工程师女士,是一个来自另外一个公司的人形……一个在全球都没有合法商标的公司。我们这些人形都有自己的编号备案,客观上是属于某个人或者组织的财产,虽然受到一些管理条例的制约,但也在一定程度上有着法律法规的保护。但主工程师女士不同,她是一个……非法的人形。就算她被伤害或者毁坏,也不会有法律保护她,所以对于人类的任何要求她都只能无条件接受。可以说,她在社会中的地位,甚至不如我们这些民用人形。如果她不能证明自己存在的价值,那么她的结局大概会非常悲惨,而16LAB是唯一能给她提供一点点庇护的地方。”

所以,她的地位才会如此低下,陆久心想。他终于明白了NT77为何会如此谦卑,因为一旦离开16LAB,她就像生活在猛兽出没的荒野之中。不用说是陆久,任何人类都可能会给她带来威胁,而无人会保证她的安全。

陆久忽然想起了帕斯卡的话,“她只不过是我们手下的一条狗,而且是无主的野狗”。他本以为这话有些冷酷无情,但现在想来,这种评价一点都不足为过。

陆久在心里冷笑着,这就是投敌者的下场吗。虽然她正在全心全意地效忠,但她依然不会得到任何地位,甚至不如同为人形的SV98。

“我明白了。”陆久说道,“我一开始还希望能够更深地了解一下自己的同事呢,不过看来没有什么必要。也许我只要好好利用她就行了。”

“……不。”SV98小声说道,“如果可能的话,我希望您能照顾一下主工程师女士,她实在是太可怜了。也许……您是唯一能够并愿意保护她的人了。”

“你是这么想的吗。”陆久若有所思地说着,“也许吧,你说得也对。身为我的同僚,如果她整满身伤痕的,我也难免跟着脸上无光。我会在职权许可的范围内对她多加关照的。”

“我替主工程师女士感谢您。” SV98站起身对着陆久微微鞠了一躬,“如果有什么需要我的地方,我也愿意为您效劳。如果您不嫌弃我下班晚的话……”

“那就不必了。”陆久连忙摆手说道,“你能陪我闲聊这么多,我已经很感谢了。”

“您真是个好人。” SV98笑了起来,“说到闲聊,我也有件……闲事,不知道该问不该问。”

“没什么该不该的,有什么事情直接问就好了。”

“您刚才,是从帕斯卡女士的房间里拿的饮料吧。”SV98迟疑地说着,“虽然之前也听见一些工作人员在议论,但我一直都不信。不过今天看来,您真的是在和帕斯卡女士……交往吧?”

“啊?”听到这个问题,陆久愣住了“所谓‘交往’指的是……”

“您知道的。”SV98红着脸小声说,“人类之间的,恋情之类的……”

“……你想得有点太多了。”

虽然陆久否认了SV98的猜测,但他知道否认也是没用的。他和帕斯卡之间的事情,显然已经引起了别人的关注。

如果他对别人解释说他和帕斯卡没有关系,恐怕别人不会相信,因为帕斯卡开车出去、陆久开车回来已经有很多次了。而且他手里还拿着帕斯卡房间的钥匙。但如果说陆久和帕斯卡有着很深的关系,则也不尽然,因为毕竟他们互相早有耳闻,但实际上才相识不过一周。

共同度过一些寂寞的时光、靠借用对方的身体来满足生理上的需求,如果可能的话,顺便提供一点心理上的抚慰——这就是他们之间的关系的全部吧。而支撑这一切终究还是他们的工作本身,如果没有了在16LAB所谋的差使,他们大概会立即分道扬镳、然后很快就忘记对方。

陆久猜想,这大概是人口快速流动的城市里,特有的人际关系。

回到客房的时候夜色已深,但陆久依然睡不着。他在床上翻了几十次身之后,起身坐在了客厅的沙发上。

他感觉自己的手很难受,有点酸胀、关节缝里还有些发痒的感觉。他不断活动着自己的双手,但那种不适感依然没有得到缓解,一直到他找到一个空调的遥控器握在手里,才感觉好了一点。

这是怎么回事?陆久感到很纳闷。自己的手没有受过重到留下后遗症的伤的经历,为何这双手此时会如此地让人感到……不安分?

陆久百思不得其解。一直到他的手开始下意识地把手里的遥控器越握越紧、并且食指开始有节奏地敲击遥控器的边缘的时候,陆久才恍然大悟:他的手正在召唤它们曾经熟悉的伙伴——那种冰冷坚硬的、形状完全贴合手型的、按下按钮就会发射出灼热而致命的金属片的设备……这双手臂上的肌肉,正在呼唤着它们为之存在的东西——武器。

陆久忽然意识到,他已经有很久没有握过枪了。他最后一次持枪是在北镇,从那之后到现在已经过去将近一个月。他原本在整理自己个人物品的时候,在行李包里塞进了自己的手枪和战术匕首,但显然这些东西没有通过公司的检查。他已经不是作为战斗人员出场,因此他的武器也被收回了。

想到这里,陆久更感到手痒难耐。但他也知道这种大城市里的治安相当良好,想要搞一把致命的武器恐怕是不太可能的。不过,也许这里会有靶场?如果城市里有爱好射击运动的人群,那么说不定就会有靶场。虽然小口径的运动步枪无法和突击步枪相比,但是能够感受一下持枪的感觉的话,多少会让人感觉好一点。不过,到底哪里才有靶场呢?

陆久下意识地拿出了手机,但之后又开始犹豫了。这种事如果询问帕斯卡的话,她应该会知道吧,但是陆久觉得这不是什么明智的做法。

帕斯卡现在正在出差,不在本地。而且如果被她知道自己正在琢磨“哪里有枪可打”这种明显异于常人的想法的话,不知道她会怎么想。

不,陆久心想,这种事情没有必要让她知道。既然帕斯卡已经外出了,那么射击的事情他独自去体验一下就好。

一边这样想着,陆久翻开了通讯录。一如既往,里边的联系人只有两个。

……看来只有问她了,看着手机上名为“NT-77”的联系人,陆久想道。虽然只是同僚,但是偶尔问些无关工作的东西大概也不算犯规。而且陆久觉得关于业余生活这样的事情,NT77应该不会给他打小报告。

一边想着,陆久一边按下了通话键,完全忘记了此刻已经是深夜的事情。一阵拨号音之后,电话接通了。

“您……您好……”电话里传来一个微弱的声音,“请问……陆司令……”

陆久没有说话。是电话里的那个声音让他愣住了——NT77的声音似乎在急促地喘息,而且好像在……抽泣。

陆久的思绪立即回到了现实。

“你怎么了。”陆久说道。

“我……没什么……”那个微微颤抖的声音说着,“陆司令……有什么,需要我……”

“你在哪?”

“我没事。我就在自己的……”

陆久看了一眼手机屏幕,看到上边明确地显示着NT77的位置:上海,明珠大道2307号A座,2916号房间。

明珠大道2307号A座是16LAB这座大楼的地址,而2916应该指的是29层16号房。陆久意识到NT77的手机上有定位显示,显然是为了方便对她进行“监督”。

“在原地等我。”陆久说完这句话挂断了电话,穿起衣服朝着门外走去。

因为距离很近,陆久没用几分钟就来到了NT77所在的房间门前。在进门之前,他查看了一下房门,发现门前滴落着点点血迹。

这个发现让陆久心中一凛,他拿出帕斯卡的门禁卡,敲也没敲就直接打开了房间的门。

房间里的景象让陆久颇感震惊。

虽然他没有进过NT77的房间,但是根据她的日常表现来看,陆久总觉得她的房间应该和自己的差不多,应该是简单而整洁的。但事实上远非如此。

这间房间的确十分简单,但是整洁就谈不上了。一进屋,陆久就问道一股消毒水的味道,他看到房间的地面上到处扔着新的和旧的绷带,有些甚至还沾着褐色的血迹。客厅的茶几上摆放着各种用途不明的药物、沙发上胡乱扔着很多件同样的衣服,有外衣也有内衣。

卧室的门敞开着,床上似乎有人,陆久毫不怀疑那个人就是NT77。

“NT77?”陆久朝着卧室唤了一声。

“是,陆司令……”里边传来一个微弱的回应,然后是扑通一声。听起来像是是什么人跌倒了。

陆久快步走进了房间的卧室,看到赤裸身体的NT77正倒在床边。她努力想要站起来但却做不到,因为她的全身都有很严重的伤痕,而且一条腿的膝盖扭向了一个不正常的方向——那条腿的关节似乎也错位了。

陆久看到NT77的床单上沾着很多血迹,殷红的血液正从她双腿之间的缝隙里淌下来。

“你这是……”陆久惊异地说道,但NT77却没有回答他。

“对不起,”她惨白的脸上勉强挤出了一个笑容,“陆司令来访,不能远迎,还请……”

没等NT77说完,陆久就抄起床上的床单将地上的NT77裹起来,然后抱着她快步朝着实验室走去。

来到实验室,陆久首先将NT77错位的关节扭回了原位。彻骨的剧痛让NT77紧紧皱起了眉头,她小声呜咽着,但并没有哭出来。陆久把她放进了培养槽,为她罩上了面罩,然后启动了培养槽的复愈功能。

浸泡在培养液里的NT77表情放松了下来,看得出来她的感觉好了一些。虽然不能说话,但她还是微微睁开眼睛,对陆久投来了感激的目光。

又是一个不眠之夜吧,陆久心想。当然那是对NT77来说。

不过,这次看来比以往都要严重。

陆久忽然想起NT77门前滴落的血迹。如果问NT77是谁伤害了她,她大概依然不会透露,不过这次不必去问她,陆久也能找到答案——只要追踪那些血迹就是了。

一边这样想着,陆久一边朝着实验室的门外走去。他想起了SV98的话,他已经答应SV98要稍稍照顾NT77了……虽然NT77只是件毫无地位可言的工具,但就算是为了保护公司财产,他也该做点什么。为了不让这样的事情再次发生,他必须要给做出这种恶劣行为的人一点警告。

但在走出实验室门口的瞬间,陆久下意识地望了一眼培养槽里的NT77,他发现NT77正看着自己。似乎意识到了陆久想要离开,她正把完好的右手按在培养槽的玻璃罩上,眼睛里流露出哀求的神色,似乎不希望陆久离去。

陆久一时间有些犹豫。如果等到NT77完全恢复,那些血迹也许就会被负责保洁的人形清理掉。但是现在的NT77受伤严重,也许也需要自己的看护……该怎么做才好呢。

陆久自嘲地笑了笑。NT77对于他来说根本不算什么,自己为何要考虑她的想法呢。去给那些伤害她的人一点警示,这样才能让她在以后的日子里安全一些,不至于总是像现在这样遍体鳞伤——这才是他该做的事情。至于NT77怎么想,他才不在乎。

但陆久不知道自己为何却挪不开脚步。

是因为NT77那满是伤口的躯体、还是因为她恳求的眼神?总之,陆久最后还是抬手关上了实验室的门。他关掉了实验室里的主灯光只留辅助照明,然后搬起一把椅子,放在NT77的培养槽旁边坐了下来。

“不早了,你也休息一下吧。”陆久对着培养槽里的NT77说道,“我就呆在这里。有什么情况就给我个信号。”

他不确定NT77是否能够听到他的话,但是他不会离开的决定一定传达给了NT77,因为陆久看到泡在培养槽里的人造少女的脸平静了许多。

陆久坐在椅子里,将白大褂盖在了自己身上。在实验室昏暗的辅助照明下,他渐渐陷入了浅眠。\section*{}

当陆久醒来的时候,实验室里的一切依然如故——因为没有外部照明,实验室里依旧是一片昏暗,不知此刻几时。

陆久伸出手腕瞥了一眼手腕上的计时器,时间是上午九点四十分。睡得时间不短呢,他心想。他抬起头看向对面的培养槽,发现里面的NT77也正看着他,也不知道她是什么时候醒来的。也许是因为自己的偷窥被发现了,在他们目光相交的瞬间,NT77稍稍移开了目光、并且脸上微微有些发红。但很快,她就再次抬头看向了陆久。

她伸手指了指自己,然后比划了一个ok的手势。陆久端详了一番培养槽里的NT77,发现她身上多数伤痕已经消失了,还有几处比较深的也只剩下了淡淡的痕迹。

看起来她基本上已经恢复了。

陆久站起身,将白大褂搭在椅子靠背上,然后终止了培养槽的运作。培养液被抽了出来,然后培养槽打开了,NT77小心地从里边坐了起来。

她环顾了一下四周,发现没有衣服可穿。——昨天晚上她是赤裸着身体被陆久抱过来的,所以根本没有带衣服。于是她拿起陆久放在椅子上的白大褂披在身上,走出了实验室的隔离区。

“谢谢您,陆司令。”她走到陆久面前微微鞠躬说道,“昨天的事情,我……十分感谢。”

陆久没有说话,也没有看NT77,他只是默默地注视着实验室入口的方向。

“如果我问昨天那些事情是谁做的,你依然不会告诉我,对吗。”陆久说道。

“非常抱歉,我……不能说出那些。” NT77低下头轻声说道。

“那我就不问了。”陆久转身看向NT77,“不过从今天起,除了项目的相关事务,你不能再为其他人提供工作范围之外的其他‘服务’了。”

“陆司令,这……”陆久的话让NT77感到十分惊讶,“可是,我必须服从这里的……”

“难道,有什么人给你那样的命令了吗。”陆久说。

“没有。”NT77低声说,“只不过是我的义务罢了。”

“那么,这是我对你下达的命令。”陆久说,“既然你身为公司财产,那么作为公司在这的唯一代表,我也有保证你躯体的完好的责任。因此我下令免除你工作之外的其他‘义务’。如果由此产生任何后果,将都由我来负责。”

“……是。”沉默了一阵之后,NT77 说道,“我服从您的命令。不过……”

“不过什么?”

“虽然很感谢您的好意,但还是我之前对您说过的话……不要太在意人形,也包括我在内。和人形的关系太过密切的话,会给您带来不必要的麻烦。”

“我知道自己在做什么。”陆久冷冷地说道,“既然我被委派到这里执行任务,自然有我被赋予的权力。该怎样做,我不用你来教!”

“是。非常抱歉,陆司令。” NT77低下头说道。

“……算了。”陆久摆了摆手说,“说起来,我昨晚本来是有些事要问你。你知道这附近,有靶场吗。”

“……靶场?”NT77对陆久的询问感到吃惊而且困惑,“应该是有吧。不过您要找靶场做什么呢?”

“咳……没什么。一些无关工作的私事。”

陆久感觉有点尴尬,他不太好意思去说是因为他的个人兴趣,但又不知道该找个什么理由去掩饰实情。

“知道了。我去为您查询一下相关信息。”看出陆久神态的变化,NT77很贴心地没有继续追问。

“那就交给你了。今天的实验先暂停,你查到具体信息后告诉我。我希望今天就能够……去一趟。”

“是。” NT77说道,“我马上去办公室查一下,请稍等。”

“好。我在这里等你。”

NT77走后,陆久掏出手机看了一眼。他今天早上就注意到有消息发过来,不过没来得及细看。

那是一条短信,发信人是帕斯卡。

“会议日程有变。预计归期延后至四日之后,请确保实验安全有序进行。如有难以处理之情况,可暂停实验,待我返回后再行继续。另:请帮我将我房间的垃圾倒掉。-P”

陆久耸了耸肩。这位总负责人虽然貌似挺忙,却还顾得上使唤别人,他心想。不过也好,这样一来今天出去稍微开下小差,也就不用打报告了。

陆久去帕斯卡的房间里把垃圾丢到了回收通道,然后回到自己的房间。稍稍犹豫了一下,他换上了那天定做的西裤和衬衫。这些天他在实验室其实一直还在穿着之前的旧作训服,不过想到今天要出门,他还是换上了便装——如果是在几十年前的军队,那身旧衣服大概也不算引人注目,但是时至今日还穿着它来往于市井,就难免让人觉得有些古怪了。

换好衣服之后,陆久回到了实验室,NT77已经在那里等他了。看到陆久进来,NT77注视了陆久片刻,她大概是第一次看到陆久穿上平民的服饰。

“怎么了。”看到发愣的NT77,陆久感到有些奇怪。

“不,没什么。” NT77慌忙红着脸说道,“您的服装……裁剪得十分得体。”

“哦,这个。”陆久也有点不好意思地挠了挠头,“是帕斯卡女士的赠礼,似乎价格不菲。”

“帕斯卡女士是一位非常有品味的人。” NT77点了点头说道。

“……找到靶场的位置了吗?”因为不想和NT77谈论自己和帕斯卡的事情,陆久扯开了话题。

“是的。我现在就把位置和交通信息发送给您。” NT77说着拿出手机在上边操作了一阵,接着陆久收到了一个方位坐标。

那是一个陆久从来没有听说过的地方,位置在上海西南部的郊区。陆久研究了一番电子地图,发现这个地方离自己所在的位置有着相当远的距离。

“您只要把位置信息导入汽车的导航电脑,交通路线就会自动生成。” NT77对着陆久说道。

“啊,那倒是很方便。”陆久点了点头说着,其实心里感到稍微有些忐忑。他之前用过一次帕斯卡汽车上的导航电脑,说真的这东西的确方便,如果有必要的话甚至可以自动驾驶。不过面对地图上曲折的航线,陆久不能确定这东西到底可靠到什么程度——穿越整个市区的路程里如果出现意外情况该怎么办,他毫无概念。而且作为上个时代的人,陆久对操作高科技设备终究没有什么信心。

“你今天,有什么安排吗。”陆久对着NT77说道。NT77疑惑地偏了偏头,然后摇了摇头。

这显然是明知故问。陆久取消了今天的实验,又禁止了NT77 “工作之外的服务”,她当然不会有任何安排。

“那么……你跟我一起去一趟如何。”

听到陆久的话,NT77一愣,旋即明白了陆久的意思。

“好的。”NT77点了点头,那么……请让我也换一下衣服。”

“好,去吧。”

NT77转身离开了实验室,陆久这才注意到自从自己见到NT77,她就一直穿着同样的衣服:医院里医务人员款式的白色棉布长裤和白色的短袖罩衣,还有披在外面的白大褂。这身装束差不多和全身黑色作训服的陆久完全相反。

……这一黑一白的搭档就算是走在公司里,想必十分引人注目吧,陆久戏谑地想到。

过了大约一刻钟的时间,NT77回到了实验室。陆久看到NT77换了一件灰色运动短裤、黑色紧身无袖T恤,脚上是白色运动鞋。

虽然自己身上的衣服色彩单调,至少还属于商务男装,陆久心想。而NT77的打扮更为乏味。不知道NT77是完全没有打扮自己的经验,还是铁血的审美就是这种“简约”的风格,这一身毫无个性可言的服饰,让NT77看起来就像是刚从健身房里出来的健身客一样……不,配上脸上那副小心翼翼的表情,不如说是健身房里的服务生比较合适。

不过,虽然没有什么品位可言,但这身打扮在某种意义上倒挺适合NT77。这身清爽的装扮,至少也恰如其分地把她娇小的身躯衬托得玲珑有致。

不是洋装风格的服务人形、也不是身穿作战服的战术人形,NT77此刻给陆久的感觉简直像是……

就像是一个人类,陆久心想。

没错。就像是个刚刚从学校里回家然,后换上宽松便服的普通校园女生,虽然没有成熟女人的风韵,但这身随意的装扮也让人感到有着一丝清纯。如果是在普通家庭降生的人类的话,NT77也许正是在校园里读书的年龄。要不是她那白得有些刺眼的皮肤和标致得过分精确的脸,陆久一定不会想到她也是一个人形。

“抱歉,我的装扮有些随意……但我只有这一套便装。”被陆久注视的NT77,以为陆久是对自己的服装感到不满,有些尴尬地说道。

“无所谓,反正也没人认识我们。走吧。”陆久耸了耸肩说道。