\specialsectioning
\chapter{外传:46号实验体}
\section*{前言}

46号实验体是第一部分中出现的一个人物——N17战区的侦察队长、从16LAB出逃后被陆久收留,改名为QCW05的战术人形。

这篇外传对她的经历进行了补全,故事描述了帕斯卡之前做过的一些项目,所以也算是对帕斯卡这个人进行了补全吧。

我个人还是很喜欢这一篇的,就算单独拿出来也是个不错的故事。
对了,顺便一说,帕斯卡是双性恋。

\lineseparator








她大腿内侧的隐私部位还有一个条形码纹身,95通过扫码器解析后得到一串数字和字母交替的字符:16lab-xm830046。陆久说,那段字符的涵义是“16号实验室,83号项目,0046号试制人形”。

她有许多名字:“QCW05”、“46”,还有“山茶”。她的脑海里还有许多破碎的记忆,但她不知道那些记忆到底属于谁。

她不知道自己是谁,从来都不知道,也许永远都不会知道。

但她知道一件事,那就是要顽强地活下去。因为那是许多人对她的期冀。

也是她作为一个“人”的本能。

\section*{外传:46号实验体}

“到了学校好好学习。有了文化,以后才能过上好日子。”

这是46的姐姐每次在她出门上学之前,都会说的话。那是46的第一个“姐姐”,也是她生物学意义上的,或者说是血缘上真正的姐姐。

而“姐姐”这个词对于46号实验体来说,可以指代的有多个对象。

姐姐所说的“好日子”究竟是怎样的,当时的46并不理解。不过姐姐对她的期望她是能够深切地感受到的,因为每次当她走到很远很远的时候,回头望向自己家的方向,依然能够看到姐姐站在那里。

46生在一个远离城市的偏远山村里,最近的学校也要徒步行走一个小时才能到达,所以她总是很早就要起床。但姐姐起得更早——每当她在水窖旁洗脸的时候,姐姐就已经把饭菜做好端到桌上了。虽然只是粗糙的粥饭和盐水腌制的青菜,但至少都很新鲜,那已经是这个家庭能够提供的最丰盛的食物了。

46的母亲在生下她的时候,就因为难产而去世了。她的父亲在她记事之前就常年在外做工,起初还会偶尔会寄一点钱回来,但随着时间过去,父亲的联系渐渐变得越来越少,没过几年便完全失去了音讯,只留下46和比她大几岁的姐姐一起相依为命。虽然尚且年幼,但稍稍年长的姐姐自然成了家里的顶梁柱。

姐姐没上过学,但她相信只有学习才能走出这贫困的土地、改变这艰难的命运。所以就算是家里几乎一贫如洗,姐姐依然让46去到了乡里的学校读书。

那是46的记忆中最清晰且完整的一段,关于她的过去。46的记忆里完整的片段没多少,多数都是支离破碎、断断续续的情景——姐姐那时是如何用自己稚嫩的肩膀扛起的那个家,46不得而知;而姐姐后来怎么样了,46也完全没有印象。

而关于当下,则要比那混乱的记忆更为复杂。

\section*{}

对于眼前的生活,46早就都有着明显异常的感觉:首先,她本该是个贫贱而蒙昧的孩子,但现在的生活条件却极端地优越;其次,46完全不知道自己是如何来到她所在的学校的。

她身边的人都告诉她,她和她的同学都是在全国选拔出来的优秀学生,但46很明白事实绝非如此。这所谓的“选拔”虽然每个人的说法各有差异,不过还能勉强说得过去;但46关于选拔的记忆却是完全空白的。46所能清楚记住的只有最近不到一年的事情,从她睁开眼看到自己处于这个“学校”里开始。

因此46曾推测,那就是这一切开始的时间节点。

可是,假如她的推断是正确的,那么她身边的人们是不该有在那之前的记忆的。但是在46的旁敲侧击之下,她惊讶地发现,事实上所有人都有着清晰完整的关于过去的记忆,只有她的记忆是模糊不清的。

很久之后46才知道,她的推测其实并没有错:之所以只有她的记忆不太对劲,是因为她是唯一一个在人格导入时发生错误的素体,而由于这未知的错误,她的脑海里才出现了那些破碎的记忆。而且相对于那些拥有貌似完整、实则完全是虚构出来的记忆的其他人,46的记忆是真实的——准确地说,应该称作源于一些真实发生过的事情而产生的记忆。

只不过,那是属于另外一个人的记忆。而46,只不过是那个人的一个“副本”。

这些孩子们的身体是来自其他人类的基因——换句话说,他们都是其他人的副本。只是这些“正本”都来自于遥远的过去,本体都已经消亡,有些甚至已经就连后人都不复存在了。

而46,是她们当中唯一确知这件事的人……一直到最后都是如此。

\section*{}

“小茶,专心听讲啊。”46的身边一个声音悄声说道,“这次的考核很重要,要是通不过的话,我们的计划……”

说话的是她的同学兼室友、以及,第二位姐姐。

理论上来说,46的这位姐姐是没有名字的。她和46以及这里的所有学员是一样的,只有一个数字作为编号(学号),她是31号实验体。但这位姐姐是个性格浪漫的女孩,她很喜欢用一些美丽的词语来称呼她喜爱的东西,所以她身边的很多物品都有了名字。

这其中就包括46和她自己——“小茶”,就是她给46起的名字,取的是山茶花之意。而她自己则比较中意被别人称作“海棠”。

“嗯。”

46应了一声,把注意力集中到讲台上的老师身上。这节课的内容是人际关系与社会交往,是一堂技术含量非常高的课程,而且授课人不是别人,正是这座“学院”的校长,帕斯卡莉娅女士。由她亲自教授的课程是十分难得的,46对她的课程很重视,因此不用别人提醒她也会专心听讲。

46专心听讲还有一个理由,就是帕斯卡莉亚女士对46的学习也相当重视,但其中的原因却只为极少人所知……因为有些称呼,46只会在四下无人的时候才说出口。

在某种意义上,帕斯卡是46的第三位姐姐。

\section*{}

46陷入了非常麻烦的事态之中。

她的“海棠姐姐”正在策划一场出逃,而她的“帕斯卡姐姐”殷切地则期待着她们能够有一个好的学习成果。这两者显然是相互矛盾的,因此46感到左右为难。

海棠一开始并不是像她现在这般热情的,她对46以及多数学员都相当冷漠和警惕。这种状态一直持续到46“入学”四周后的一个深夜。

那天,46在宿舍里睡得迷迷糊糊,忽然隐约听到了身边的动静。她微微睁开眼,发现宿舍的室友都已经穿戴整齐地站在了床边,让她以为已经到了早上的出操时间。

46环顾了一下周围,天还没亮。而且,那些室友的表情也不太对……她们的眼睛里,显然没有神志清楚时的神采。

“安静。”46忽然听到一个低声的命令,“穿好衣服站在床边,不要左顾右盼。”

说话的是她的邻铺31,那时候46还不叫她海棠。

“这是……在干嘛?”46一边按照31说的穿起衣服,一边低声问道。

“训练。”那边简单地低声回答,“但不是常规训练。是梦游状态下,在潜意识里进行的武器操作灌输。”

46完全不明白。什么“潜意识”、什么“武器操作”?不过梦游这个词她倒是有点理解。

“别出声。一会儿跟着队伍出去,别人干嘛你干嘛。”

看出46的不解,31再次说道。

\section*{}

一间宿舍有30个人,都是一个班级里的同学。他们被手持武装的教员到了一间类似靶场的地方,每人分配了一个隔间,面前放着一把自动步枪。

“依照指示射击标靶,听到命令你就开火,千万别犹豫。”31的话传了过来,她就在46隔壁的射击位上。

46的身体僵住了。她不是害怕面前的武器——虽然枪械这东西对她来说也很可怕,但更可怕的是那些标靶。

那些靶子不是普通的人形纸片,而是一个个……活生生的人。

被紧紧束缚在墙上、全身赤裸、眼神空洞。但毫无疑问那些标靶是人类,而且是活的。

“这,怎么……”

“不管你看到了什么,都只是些没有意识的人偶,不是活人!”31严厉地说道,“别让人发现你醒来,不然下次被绑在那里的就是你了!”

“……”

46没想过自己是如何拿起枪的。很意外,她非常熟练地就操起了那把武器,检查枪膛、装入弹匣、拉动枪栓、解除保险,一气呵成。就好像她曾经做过这些一般,而且是不止一次。

“举枪,瞄准——”

靶场的扬声器里传来了这样的命令,46立即端起武器瞄准了目标。她的手臂对命令是如此地服从,虽然她的眼睛正在因为害怕而不断流出眼泪。

“只管扣扳机,别想太多!”那是31的最后一次警告。

46没有听到开火的指令,她只是看到了身边枪口的闪光。于是她也随着扣下了扳机。

将武器放在面前的桌面上后,46感到眼前一片昏暗,下意识地向后退了一步——然后便失去意识跌倒在靶场中。

涂满了红色颜料的白色墙壁,是她最后看到的景象。

\section*{}

46醒来时是在帕斯卡女士的房间里,而且正躺在她的床上。

“你醒了?”看到46睁开眼睛,帕斯卡用温柔的嗓音问候道。

“对……对不起,校长女士!”46在看清面前的人之后,慌乱地起身爬下了床,全然不顾自己身上还连接着各种监测设备。

“不用紧张。你的体检还没结束,快躺下别乱动。”

帕斯卡把46搀扶到了床上,并给她盖好被单。

“谢谢,校长女士。不过,我……怎么会在这里?”接过帕斯卡递来的一杯热水,46微微抿了一口,然后疑惑地问道。

但帕斯卡没有立刻回答她的问题,只是笑了笑。

“你难道忘了昨晚发生的事情了吗?”她说。

昨晚的……

46想起来了。昨天晚上她和室友一起参加了一场“射击”训练,然后就晕了过去。不过此刻她所处的地方,却让昨晚的一切都显得那么不真实。

那是真的发生过的事情吗,46环顾着帕斯卡宽敞明亮的办公室,心里想着。无论如何都觉得像是在做梦……

“别让人发现你醒来,不然下次被绑在那里的就是你了”,心里忽然掠过这样的话。那是31的警告。46的喉头痉挛了一下。

“昨晚,我……”46喃喃说道,“好像做了一个,可怕的梦。”

“是吗。难怪你的脸色这么差。”帕斯卡再次笑了,“不过要是噩梦的话,只要醒过来,就不会再害怕了吧。”

\section*{}

“你也成为觉醒者了。”

这是从帕斯卡的办公室回到宿舍,46听到的第一句话。31严肃地站在46面前,神情严峻、语音低沉。

“那是……什么?”46低着头小声说道。

“和我们一样的人。”31说。46抬起头,看到31身后还站着几个人,都是平时和31在一起的同学。但是她们的学号,46一个都不知道。

46其实是个孤僻的人,她所有的时间都在被脑海里那些古怪的记忆困扰。她不知道自己出了什么毛病,但也不敢把自己的情况告诉别人。因为这个班级里,同学们虽然都很积极,却看起来毫不友善。身为异类,必然会被排挤——不知为何,46的意识里有着这样的本能。

所以,昨天晚上31对46说话的时候,她真的非常意外。

“别害怕,我们不是坏人。”为了缓解46的紧张,31微微笑了笑,“但也要时刻保持警惕,因为如果我们的秘密一旦泄露,将会是灭顶之灾。”

“你在说什么?我不明白。”

虽然已经大概猜到了眉目,但46还是假装不知。因为她实在不想被人告知,自己这些天所做的那些梦都是真实的。

她不想知道自己在杀人之后竟然毫无恐惧和愧疚,是因为自己已经在“梦游”的状态下杀掉了许多人。

“既然你昨晚醒了过来,那你就该面对这一现实。”31依然笑着,用表示理解却不容置疑的语气说道,“这个宿舍有30个床位,但现在只剩下26个人了……你该注意到了吧。”

\section*{}

“虽然总数不变,但我们班级里的人员,每周都在变化。”31轻声说道,“那些不能通过考核的学员,就会像昨天那样消失,然后又有新人来补充。”

“我们……这个学校,到底是什么?”46颤声问道,声音里透着难掩的恐惧。

“杀人工厂。不然你以为呢?”31淡然说道。

“怎么可能……”

“对不起,虽然很残酷,但这就是事实。我们是一群杀人机器,而这里就是制造这种机器的车间。产品是不该知道自己的命运的,嗯,本应如此。不过你既然已经觉醒了,那么这些事就变成了你必须知道的,因为以后还会经历很多。”

“就像……昨天晚上那样?”

“是的。”

“……”

46不知道该再说些什么。她有很多疑问,但最大的疑问就是为何31能如此若无其事地讲述这样可怕的事情。

“每周都会进行检测,那些质量不过关的次品会被剔除——变成别人训练的靶子,只是其中的一种方式。事情就是这样,你也不必太过自责,就当是帮她们解脱了也好。”31轻轻拍了拍46的肩膀说道,“想要安静地呆一会儿也可以,我知道你需要一点时间……但最终必须接受。因为想要做点什么的话,首先要活下去。”

说完,31就准备转身离去。但却被46叫住了。

“等等。”她说。

“怎么了,有什么问题吗。”31有些奇怪地问道。

“你说想要做点什么……”46低声说着,“那么,我们要做的……是什么?”

\section*{}

“离开这里”,那是31给出的答案。但31并没有详细说明要怎样去做。她只是说通过所有考核是这一切的先决条件。

“你有名字吗?”31问46说。

“没有。”46回答。这里的学生没有名字只有“学号”,这是不言自明的事情。

“那我给你起一个吧。”

“……为什么呢?”

“因为有了名字,才更像是一个人啊。”

……更像?46一直以为,自己已经是一个人了呢。看来是她想错了。

46忽然想起,31和她的朋友们都有自己的名字,他们之间从来不用那些学号来互相称呼。其他人的名字46不知道,但是她知道那些女孩们,都把31叫做“海棠”。

原来这些名字都是她想的呀,46心想。那么想要融入这个圈子,看来一个“名字”是必要的。

而且那些名字,听起来也挺好听。

“好,那我该叫什么呢。”46终于点了点头说。

“你喜欢花吗。”31问道。听到46应允了她的提议,31的脸上浮现出了一丝明显的笑意。

“喜欢。”

美丽动人的花朵,没人会不喜欢。

“那就叫你‘小茶’吧。山茶一样娇艳芬芳,怎么样?”

“……好。”

\section*{}

不用说,这些人的名字显然都是31起的,因为这些名字全都是统一的风格:31就像一位花匠一样,对花的名字情有独钟。

在31的引荐下,46认识了其他的几位“觉醒者”:大胆奔放的芍药(39)、沉默睿智的玉兰(71)、胆小怯懦的含羞(55)、随和内敛的木荷(56)。加上热情友善的海棠(31)还有她这朵茫然失措的山茶(46),她们一共有六个人。互通姓名之后,她们就算是伙伴了。 

每天白天,她们都在课堂上学习文化知识,和普通的学生无异;但一到晚上,在深度睡眠中她们则被悄悄灌输着危险的能力——杀人、伪装和使用各色武器的技巧。多数人在接受这些训练的时候都是处在梦游状态的,但她们六个人却能在深夜里保持清醒的意识、明确地知道自己在做什么。31把这些同类称之为“觉醒者”。

这些觉醒者,有些并不是什么能力出众的学生,例如含羞就非常内向甚至自闭、木荷除了性格随和之外也没有什么过人之处。她们只是恰好由于不明的原因在深夜的洗脑中醒了过来,然后为了挨过这冰冷残酷的真实而相互依存而已。

而31,正是这几位觉醒者中的第一个。

“没关系,”最早觉醒的31总是这样安稳其他人说,“白天认真学习功课、晚上按照命令操练……只要不露出马脚,通过考核应该很轻松。”

31说得的确没错,白天组成学习小组相互帮扶的六个人,每次考试都有着优异的成绩。而晚上发生的事情,大家都心照不宣地不去过多提起。

46本以为只要习惯之后自己就可以这样熬过这种“学习和考核”的生活,可没过多久她就出现了状况,再次晕倒在靶场的射击台前面。

因为她被分配到的“标靶”中,终于看到了她班级里曾经的同学。

\section*{}

“又做噩梦了吗?”

醒来时,46首先看到的是一张笑吟吟的脸。46扫视了一下四周,宽敞的房间里有着大大的窗户,办公桌上摆着巨大的显示屏和各种杂乱的表格,自己正躺在一张床上,身上连接着各种仪器。

这是帕斯卡的办公室,面前的也是帕斯卡的笑脸。但这一次她没有起身逃脱。

“是的。对不起,校长女士。”46用微弱的声音回答道。

“这没什么可抱歉的。”帕斯卡递过来的依然是一杯温热的清水。

“我……是不是一个,不好的学生?”46有些担忧地说。

“为什么这么说呢?你的学习成绩一直都很优秀,我对你可是非常期待呀。”听到46的话,帕斯卡显得有些惊讶。

“可是我感觉自己的状态……总是不太好。特别是晚上,总是会梦到一些可怕的事情。”

“你都梦到什么了呢?”帕斯卡关切地说。

“我也记不清了……”意识到自己的失言,46闪烁其词地说着,“像是有很强的闪光、很大的声音,还有很多的……血。”

“……那可真是吓人。”沉默了片刻,帕斯卡轻轻握住了46的手说道,“不过,既然已经醒过来了,那么就不用再害怕了吧。”

“嗯。不过我还是有点害怕。如果晚上再梦到那些的话……”

“看来是真的吓坏了呢。”听到46的话,帕斯卡吃吃笑了起来,“那样的话,晚上就睡在我的宿舍里吧,如何?”

\section*{}

“帕斯卡女士是学校的一把手人物,能够接近她的人不多、机会更不多。如果能博得她的好感、或者从她那里得到什么情报的话,一定会对我们大有帮助。所以你能去的话最好是去。”站在帕斯卡办公室的门前,46依然在琢磨着31对她说的话。她想要自己得到什么情报呢?

用过晚餐后,学生们将会回到各自的宿舍等待熄灯睡觉,期间不得随意走动、更不得离开宿舍。但当46洗漱完毕,换上睡衣走出宿舍的时候,门口执勤的保卫并没有阻拦她。

事实上,那里已经有一个陌生的教员在等着她了。

“我要去帕斯卡女士那里。”46对着那个教员说道。

“跟我来。”那个教员点了点头,带着46离开了宿舍。

夜晚的校园非常安静,只有两个人琐碎的脚步声回荡在楼道里。46心怀惴惴地跟在那个教员身后,不知道自己将会被带到哪里。但没过多久,穿过几条走廊,就到了帕斯卡的办公室门前——那是46白天来过的地方。

“进去吧,帕斯卡女士在等你。”站在门前教员对46说道。

46看了他一眼,然后深吸了一口气,轻轻推开门走了进去。那位教员在46的身后轻轻关上了门。

听着门外的脚步声慢慢走远了,46这才定睛扫视了一番这间屋子。和白天看到的一样,宽敞的房间里摆放着办公桌和检查用的卧床以及仪器。但是屋子里却一个人都没有。

不是说帕斯卡女士正在等着她吗,46有些困惑地想着。但是她人在哪里呢?

“啊,你来了。”正当46不知所措的时候,她忽然听到身边传来一个声音。伴随那个声音出现的还有一种古怪的味道……闻起来像是,什么东西烧焦了的味道。

46微微扭头,看到办公室角落的饮水机旁,正站着一个身穿白大褂的人。那个人正是帕斯卡,她的脸上笑盈盈的,手里还端着一个白色的咖啡杯。

“晚上好,46号同学。要喝咖啡吗?”

\section*{}

46本以为帕斯卡住在专用的教员宿舍楼里,但没想到她的宿舍就在和办公室相通的隔壁。

“抱歉,因为每天都工作到很晚,所以没察觉已经到了熄灯的时间了呢。”坐在宽大的双人床边,帕斯卡略带歉意地说着,而46则有些窘迫地站在一旁。

“不过既然今天是约好的,那就偶尔早点休息一回吧。保证充足的睡眠才能青春常驻啊。”

帕斯卡女士,看起来不仅美丽,而且很年轻。那就是46在近距离接触帕斯卡后的第一个印象。就算整天都穿着随意的便装,也难掩她是个美人的事实。

她有一头秀丽的长发,微微反射着桃红的色彩。她的身形高挑而健美、面容和蔼可亲,经常带着亲切的微笑。帕斯卡总是能给人一种友善的感觉,因此在和她独处的时候,46莫名地感到有些安心。

但这安心感也仅仅持续到46来到帕斯卡的卧室为止了。想起自己来到这里的目的,46的心不禁开始慢慢下沉。

如此年轻的女士,竟然是这座校园的负责人,本身就是一件出乎意料的事情。而这座校园其实是一间制造杀人机器的工厂,又是谁能想到的呢。自己班级里有30个人,只有她们六个“觉醒者”略知一点模糊的真相。这个学校里还有其他的班级,但那些班级里有没有同样觉醒过的人呢?46和她的伙伴们永远无法了解。

但帕斯卡清楚地知道这些学生们在做什么,她对事情的前因后果,比这些学生们要了解得透彻得多。这些学生只是她手里小白鼠一样的实验体,而她在面对这些实验体的时候,却依然能露出和蔼友善的笑容。

在那张如花的笑脸背后,究竟是怎样的一个人呢。这种想法让46感到不寒而栗。

\section*{}

46本想靠早些入睡来逃过这令人不安的一夜的,但结果那天夜里46很晚才睡着。她和帕斯卡两个人紧挨着躺在宽大的床上,帕斯卡不断地向她诉说自己的童年趣事,让46听得入了迷。

46从帕斯卡的讲述中得知,她是一个在大城市里出生、长大的人,从小就过着富足的生活。帕斯卡所过的那些生活,是46不知道、也想不到的。她专心地听着帕斯卡如何在学校里当课代表、学习委员、少先队中队长和重点学校的交换生,然后一步步走入全球最优秀的大学里深造。46忽然想起自己那并不存在的姐姐说过的话,她所说的“好日子”大概就是指的帕斯卡这种吧。

不过帕斯卡虽然学习成绩一直名列前茅,但她并不是靠学习来赢得的这些,她从出生起就享受着这样优越的条件。就像有些人生而贫贱一样,这世上也有些人生来就在优越之中。这就是所谓的命运。

“呵呵,我知道你在想什么。你一定在想命运是不公的吧。”见46渐渐陷入了沉默,帕斯卡轻笑了一声说道,“但不是这样,真的。命运其实是公平的。虽然有些人在物质上享有更好的条件,但是为了实现所谓理想而付出的代价,都是一样的沉重。对于那些怀着更高的理想的人来说,他们背负的和付出的,也一定比一般人更多。”

46忽然感到心头一动。莫名地,她觉得这句话是真心话——不同于白天帕斯卡带着笑容和她交谈的时候,46觉得帕斯卡刚才说出那句话时,没有用面具来伪装。

“是,校长女士。谨记您的教导。”46说。

“噗嗤……”帕斯卡毫无缘由地笑了起来,“你怎么还叫我校长呀。两个人睡在一张床上,还分校长和学生,不觉得别扭吗。你就叫我帕斯卡……要是不嫌弃的话,就叫帕斯卡姐姐好了。嗯,做你这样的孩子的姐姐,我也是绰绰有余了。”

\section*{}

帕斯卡对46说,如果有事随时可以来找她。但真正触动46的并不是这种特权,而是那时帕斯卡说出的那个称呼。

“姐姐”,这是个让人感到温暖的字眼,至少对46来说是如此。也许是因为那代表着她唯一有关亲人的记忆。

所以,虽然46不知道为何帕斯卡会如此中意自己,但当帕斯卡说出那句话的时候,她……不想拒绝。于是她在这里又有了一个姐姐,当然,这个称呼只能在私底下时使用。

46还有一个姐姐,就是她们这些觉醒者的头领,31,或者说“海棠”。海棠其实是个活泼开朗的孩子,她并不比其他孩子大多少。但她又是个充满使命感的人,因此作为第一个觉醒的人,她主动担负起了组织和保护这几个特殊人员的责任。她就像是46的那位记忆中的姐姐,虽然还很稚嫩,但现实让她不得不成熟。因此,“海棠姐姐”这个称呼更多意义上是一种尊称。

“那很好啊!”听到46汇报了前一夜的情况,海棠的眼睛里闪烁着兴奋的光彩,“想不到帕斯卡女士会那么喜欢你,真是让人刮目相看呢。”

听到这样的评价,46稍稍松了口气。不知为何,她一直在担心的是自己被帕斯卡认作妹妹,同样负有姐姐之名的海棠会不会因此而不高兴。

但虽化解了忧虑,但46的眉头并未舒展,因为她的心中又结下了新的结。

“我真的希望你们都能顺利通过考核,成为真正有用的人。”那是46在半睡半醒之间听到帕斯卡说出的话,不知这句话她是在对着谁说。但46觉得,那句话同样并非虚情假意。

这件事46并未向海棠提起,因为她觉得那是她和帕斯卡之间是私事。让46感到担忧的是,海棠正在策划一场逃亡,而帕斯卡却希望她们能有好的成绩。这两者显然是相互矛盾的。

46已经能够预感到,总有一天她将不得不做出选择,在这两位“姐姐”之间。而她却不知该何去何从。

\section*{}

人对环境的适应能力,是超出自己想象的。无论说是“习惯了”还是“麻木了”,46在她入学后的四个月里,渐渐过得顺利了起来。她再也没有出现过晕倒的情况,就算她射击的目标是曾经熟悉的人。而通过一次次的遴选,班级里的人员也逐步固定了下来,看来他们这一批产品的品质已经趋于均衡。

但她们之间却没有出现新的觉醒者,或许是出现了但她们并不知道?总之,觉醒者的队伍成员一直是她们六个人。

到了冬天来临的时候,他们的武器操作课程已经不再是深夜里的秘密教学,而是提到了白天的日常中。充当武器专家的教员们向这些十五六岁的孩子讲解如何分解、操作和维护武器,却发现她们仿佛已经有过专业培训一般,武器使用得比自己还熟练。而帕斯卡女士也公开对她们宣布,“你们是作为利器而存在”。

没有人感到意外,或者说所有人在很早以前就已经意识到了自己的命运。当这命运终于被揭示的时候,这些学生们反而感觉松了一口气。

至少,从此她们终于可以毫不掩饰地公开自己要做什么了。

校园里的文化课程虽然仍在继续,但战斗训练所占的比例开始变得越来越大。每天下午,班里的同学们都会分组进行对抗训练。在群体对抗中,六个觉醒者总是组成一队,而在小组对抗中海棠总是喜欢拉上46当自己的搭档。

46看起来是个稍微有些木讷的人,不太善于表述自己的意愿。但有一点是不容置疑的,那就是论战斗能力,她在全班是数一数二的。

\section*{}

芍药和含羞、玉兰和木荷,这样的固定组合有着优秀的互补性。作战风格硬朗的芍药总会第一个带头冲出去,能够很好地鼓舞身边的人的士气,让含羞不至于被胆怯所束缚;而睿智沉稳的玉兰很善于把握大局,忠于执行命令的木荷是她最好的伙伴。但山茶和海棠的组合却不是这种思路。海棠和她的搭档46,属于那种没有破绽的类型。她们两个人都有绝佳的战斗技巧,放在一起可谓是强强联合,无论是联手强攻、各自为战、迂回退守还是声东击西,都能打得对手毫无招架之力。

不过,虽然战斗技巧进步很快,但帕斯卡女士似乎对此并不特别赏识。

“不战而屈人之兵,善之善者也。擅长战斗固然很好,但能不战斗就达到目的则更好,就是所谓的攻心为上。”在每次亲自教授的人际社交课上,帕斯卡女士都会重复这句话,“你们要善于利用自己的资源,这幅柔软的少女外表,就是你们最好的伪装。”

帕斯卡就是一位做这件事的高手,她亲自向自己的学员们传授交流的礼仪和技巧,以及如何用自己的身体来达到目的。46学会了如何用一笑一颦瓦解男人的警惕、博取他们的信任;也学会了如何用一个看似无害的笑容和一句漫不经心的谎言,来误导目标、将他们指向预先设好的陷阱。

46毫不怀疑帕斯卡是个谎言和伪装的大师,事实上这一点她早就已经知道了。她甚至不能确定当自己叫出“姐姐”这个称呼的时候,帕斯卡脸上那明快的笑容,到底是真是假。

\section*{}

校园的位置应该是在北方,因为这里冬季的天气相当寒冷,而且还会下起鹅毛般的大雪。

46对于雪并不陌生,但她的同学们很多却在见到雪的时候非常兴奋——46敏锐地意识到,她可能是北方人,而她的同学们大部分都来自南方……至少从记忆上来说是如此。

“马上就要到春节了,春节过去这个学期就算结束了。”一边清理操场上的积雪,海棠一边说着,“真是没想到,上学期过得非常顺利呢。”

“嗯。”46低着头,不置可否地哼了一声。她正在努力用扫帚推开厚厚的积雪,以便掩饰自己脸上担忧的表情。还有,“春节”……是什么。

海棠时刻都在注意着教员们的一举一动、帕斯卡也一直在课上时间对自己照顾有加,多亏了这两位“姐姐”,这个学期46才得以过得如此顺利。但这也是46感到忧心忡忡的原因。

一个学年已经过去一半,等到下学期也过完,就该进行毕业考核了。考核通过看起来没有什么问题,但考核之后……又会是怎样呢。

海棠说过她们迟早要逃离这座学院,而通过考核是其中的第一步。至于后边的计划,她没有透露,也没有其他人知道。但是她的计划一旦开始实施,那么帕斯卡那边……

46晃了晃头,赶走了脑海里的忧虑。多想无益,只能指望车到山前会有路了。

“小茶,你家在什么地方?”46忽然听到这样的一句问话。

“……啊?”

46一时愕然。她不知道这个问题的答案,事实上她根本没想到还会有人问这样的问题。

家在何处?46能记得的,只有来到这所“学校”之后的事情,之前的记忆都是模糊而破碎的,她根本无从得知自己到底来自何处。

她原以为所有人都是如此,但她很快就惊讶地意识到,不知自己来自何处的,只有她一人。

\section*{}

46默默倾听着自己的伙伴们兴高采烈地诉说着关于家乡的风景,再次听得入了神,一如她第一次留宿在帕斯卡宿舍里的那个晚上。她们每个人都有自己的故乡、家和亲人的回忆,不仅仅是海棠和木荷,甚至就连一向羞于表达的含羞,都讲了关于自己家庭的故事。

完全不知自己来自何方的,只有46。她再次感到自己被孤立了起来,尽管她的伙伴们对她的事情毫不知情。 

46羡慕她们,要比羡慕帕斯卡更甚——对于帕斯卡拥有的近乎完美的一切,46其实没有什么感觉。虽然私下46将她称之为姐姐,但那位高高在上、遥不可及的女士拥有些什么,46并不真心在意。可身边的伙伴们不同。当46发现身边和自己相互依偎着取暖的人,人生也比自己更加丰满时,她才真正为自己而感到悲凉。

就算是在多年之后,每当46回忆起那时的伙伴们时,她依然对此感到羡慕不已。虽然那些人生虚假的,但至少她们是怀着幸福的回忆死去的,而不像自己这样不得不带着沉重到让人窒息的到记忆活下去。

不过,那时的46对她们所在的这个地方,还远谈不上了解。忽然之间,她对自己的身世感到了前所未有的好奇,于是她做出了一个决定——她决定亲自去询问自己那位德高望重的“姐姐”,她一定会知道这到底是怎么回事。

既然是姐妹,那么问一问这种问题也不算冒犯吧,46这样想着。但她很快就学习到一件事:永远不要高估自己的承受能力,因为这世间最冰冷残酷的事情,莫过于赤裸裸的真实。

\section*{}

教学楼里是不许学生们随意走动的,无处不在的探头在时刻监视着这里的一切,一旦有偏离正确位置的对象,立即就会有武装的教员出现加以阻止。但46去往帕斯卡办公室的时候却从来都没有遭到过干扰,显然是有人已经提前打过了招呼。

轻轻推开帕斯卡办公室的门,却发现里边空无一人。这让46感到有些意外,她愣愣地站在了那里。帕斯卡通常应该在这里的,她很少离开自己的办公室。

这间办公室她已经非常熟悉了,因为这几个月里她曾经不止一次走进这个房间——有时候是因为学习上的问题、有时候是因为训练中的情况,还有时候只是单纯地想……过来看看。

还有,里边那间房间也是。

虽然知道帕斯卡不可能真的对自己有什么姐妹之情,但46和帕斯卡也曾亲密到……全身赤裸着相拥而眠。在那扇门后面的房间里,帕斯卡教会了46很多事情,包括如何使用自己的身体——那都是帕斯卡的“私人授课”。

想到她们在那张大床上所做过的事情,46的身体里悄然涌上一丝潮湿和温热。甚至有些沉沦于那种事情了吗,46一边有些羞耻地想着,一边不由自主地轻轻走到了通往帕斯卡宿舍的门前。

然后,她怔了一下。

……好像,听到了什么声音。

46屏住呼吸,侧耳仔细倾听。门的另一边,有人在低声交谈。

“……课题的进度太落后了。公司投入了那么多的财力物力,光是素体就用了近二百具……只选育出不到六十个合格的,并且实验到现在还没有开始。董事会那边……”

里面传来一个男人的略显愠怒声音。这个声音是陌生的,46从来没有听过,但46知道他是谁。一定是那个被称作副校长的男人。

他曾经多次出现在重大的仪式和典礼上,以及对抗训练的看台上,但从来都没有发言过。之所以46注意到了他,是因为他总是站在帕斯卡的身边。

“嘻嘻,既然有四哥的关照,我还有什么可担心的呢?”

一阵放浪的笑声,听起来好像是帕斯卡。

不,那毫无疑问就是帕斯卡。只不过46无法相信如此轻浮的声音,竟然是那个怀着当仁不让的气度、却又总是在温柔地笑着的“姐姐”发出来的。不过46也因此敏锐地察觉到了这两个正在谈话的人之间的关系。

不是靠经验也不是凭直觉,她只是自然而然地就了解了这种事情。

\section*{}

自己的来访选错了时间吗,46心想。她想要转身离去,但不知为何双脚却像被钉住了一样,站在那里无论如何都挪不开。

“别以为有我挡着事儿,你就可以天不怕地不怕了!董事会已经非常不满了,虽然我已经尽力安抚他们,不过……”

“只需要一点时间了。这次的课题是前所未有的,你也知道。这次我们没有屏蔽素体大脑的感性区域,就是为了最真实地测试她们‘拟人’的性能……”

两个人说话的声音忽然低了下去,46不得不全神贯注地听着,才勉强捕捉到只言片语。

“呵呵呵,真有你的。给那些素体的脑袋里装进虚假的模拟记忆,也只有你能想得出来了吧。她们现在真的和一群多愁善感的小女孩一样了,等到她们知道真相的一刻,会不会因为无法接受而崩溃呢?想想都觉得有意思。你这只放荡的小母猫,还真是个疯狂的科学家啊,哈哈……”

忽然,男人的声音大了起来,并伴随着一阵冷笑,让46打了一个激灵。

“嘻嘻……你不也一样吗?”帕斯卡放浪的笑声再次传来,“竟然让我把人形伪装成人类、用人形去对付人类……这件事要是披露出去,恐怕你不被枪毙、也得把牢底坐穿了吧。你的疯狂完全不在我之下,不是吗?嘻……”

“所以我们是共犯啊。不过你必须给我一个确切的时间,不然那边我真的已经糊弄不过去了。”

“最多再有两个月。春节之前我会拿出成形的数据报告和样本,把这些拿给公司当做新年礼物,也够还上欠你的账了吧?”

“年前还债,倒也算讲究。不过我下了这么多本,总得先给我点好处当利息吧?嘿……”

“呵呵,我就知道你赖着不走,一定在打什么坏主意。不过这里可是办公室,还是等……”

“又不是第一次,就别给我装蒜了,你这放荡的小母猫。”

“哟,这么心急呀?啊,嗯……”

46感到一阵茫然。“人形素体”、“拟人性能”、“模拟记忆”,这番短短的对话里包含了太多爆炸性的信息,让她一时无法回过神。但真正抹去46思考能力的,却不是这些。

让她的脑海变得空白的,是房间里传来的若有若无的淫靡之声、和帕斯卡与那个男人正在做的事情。使用自己的躯体去取悦男人,也是她们所学过的重要一课,但46没想到帕斯卡竟然会身体力行地去实践。

46呆呆地站在门口,听着房间里传来莺燕欢声,忽然在内心感到了一阵莫名的怒意。不知为何,她的脑子里只有一个念头,就是冲进去阻止他们、阻止那个男人正在对帕斯卡所做的一切。她下意识地朝着面前的门伸出了手,但在她的指尖触到门的瞬间,她终于压抑住了内心的狂躁。

……几乎压抑住了。

46的指尖触到了那扇门,而那扇门竟然没有锁。门被轻轻一碰,门口出现了一条细细的缝隙。

46从门缝里看到,在那张宽大的床上,有两具躯体在紧紧纠缠着。男人粗壮的身体正背对着门口坐着,黝黑而丑陋、后背上满是疤痕。而帕斯卡纤长白皙的四肢,则犹如八爪鱼一样紧紧攀缠在上面、一边不断激烈地耸动着、一边发出娇媚的喘息。

……何其的不堪入目。但46却移不开自己的目光。她只能默默地站在那里,从门缝里注视着那两个人。

她看到帕斯卡把头伏在男人的肩膀上,因为亢奋而潮红的脸庞上,表情分不清是痛苦还是快乐。她看到帕斯卡那绸缎一样柔顺的长发,被如细雨般淋漓的汗水浸透,一丝丝地帖在那个男人的后背上。她看到帕斯卡犹如白玉般的双臂,紧紧搂着那个男人的脖子,手指深深地扣入他深色的肌肉中。

终于,男人也渐渐发出了粗重的喘息,他犹如要把怀中的娇躯碾碎一般紧紧地拥住了帕斯卡。帕斯卡发出了一声激昂的尖叫,然后不经意间抬起了头——

迷乱而涣散的眼神扫过门缝,掠过了46已经渐渐失焦的眼睛。

目光的相交,让46如遭雷击。她瞬间清醒了过来,迈开已经有些麻木的双脚,快步离开了帕斯卡的办公室。

\section*{}

为什么会感到无法接受呢,46一边在教学楼里漫无目的地游荡着,一边问自己。

只是因为那个男人对帕斯卡,做了自己都未曾对她做过的事情,就感到愤怒吗。只是因为帕斯卡对那个男人,做了她对自己都未曾做过的事情,就感到委屈吗。何其的不可理喻。

在身居高位的帕斯卡面前,自己只是个渺小的人物。或者,把“人”字去掉也许会更加贴切。

自己不过是她的掌中玩物,46很清楚。相对于海棠和其他觉醒者,自己的心智要比她们更为成熟——虽然不知道原因,但46知道自己有这样的天赋。在别人都在单纯地相信一些简单的事情的时候,她总是在思考这其中可能存在的深层原因。虽然有些沉默寡言,但46总能正确地把握人心,这就是为何她在知道了海棠的计划和帕斯卡的期冀之后,依然能游走斡旋于她们之间。

46本以为自己更多地倾向于海棠这一边,而且事实上也确实如此,因为海棠和觉醒者们才是她真正的伙伴,她知道自己永远不会成为帕斯卡那一类人。但她今天才知道,原来她也在渴望着能够站在帕斯卡身边,那是她理想中的憧憬……如果她也配拥有理想的话。她甚至幻想着自己那贫瘠的身体,有朝一日也能成长得像帕斯卡一样,充满着女性的魅力。

虽然她感激那些和她同命运共存亡的伙伴,但她又想要帕斯卡那样温柔又强大的“姐姐”的庇护,甚至有一点奢望能够……独占她。

但46终于明白,那是没有可能的。因为那个女人是帕斯卡。

而她,不过是一具在某个大规模实验中使用的测试样本而已。

 “人形”素体啊,46自嘲地笑了笑,而且是被灌输了虚拟记忆的试验用素体。海棠她们绝对不知道那个词,但46却知道,因为她那破碎的记忆里有关于这种东西的信息。那是一种专供人类使役或者取乐的高级玩偶,通常做成可爱少女的形象,但却没有自主意识和情感……也有可能是被抑制了情感意识,而不能表达出来。

46从来没有想过,自己竟然也是这样的东西。她还以为自己是个人类呢。

不过这样的话,许多事情也就能够说得通了——用匕首削掉人的头颅,对于强壮的男人来说也绝非易事,但她们这些柔弱的少女却可以轻松做到。不仅是力量,还有超越人体极限的反应和速度、以及对外部环境的耐受能力。原来这具身体,本来就不是人类的身躯。

但自己的这具躯体可能还和其他人有所不同:这具躯体里被输入的不是经过仔细加工的记忆,因为自己的记忆不仅不能自洽、里边似乎还有着许多本不该知道的信息。46的记忆虽然支离破碎,但毫无疑问是真实发生过的事情——她意识到那恐怕是来自另外一个人的记忆。实验的准备环节中竟然出现了这样的差错,恐怕就连帕斯卡都没有想到吧。

该是时候划清界限了,46心想。从现在开始,自己的立场应该全盘倒向海棠她们这一边,因为觉醒者才是她真正的同类。既然自己不是人类,帕斯卡又怎么可能把自己当做人类去对待呢。

但不知为何,46的双脚却没有将她带到她的“人形”同学们聚集的操场,而是再次将她带到了帕斯卡的办公室门前。

\section*{}

46走进帕斯卡的办公室,发现里边没有人。

当然没有人。这间办公室的主人,也许还沉浸在亲密行为而带来的欢愉之中。所以46轻轻关上办公室的门后,朝着里边的卧室走了过去。

走进卧室,46看到房间正中那张一片狼藉的大床上躺着一个人。虽然看不清是谁,但凭着被单下隐约露出的桃红色头发推断,那个人是帕斯卡。

刚才的那位副校长,看来是在取足“利息”后离去了。

“咳。您好,校长女士……”

清了清嗓子,46对着床上的人轻声说道。

没有回应。床上的人不仅没有出声,甚至连动都没有动一下。

“您还好吗,帕斯卡……姐姐。”46再次说道。

她看到被单下的人,稍稍动了动。

“你来了啊,山茶。”她用充满疲惫的声音说道。

山茶。帕斯卡在私下也会叫她这个名字,因为46向她透露了自己的小团体的秘密。

当然,46透露的只是她们几个成了要好的朋友,并且海棠用美丽的花朵为她们起了自己的名字。关于觉醒者的事情,46非常小心地隐瞒了过去。

帕斯卡说她非常喜欢这些美丽的名字,也为了46能够结交优秀的朋友而高兴。不过因为校规的原因,大家不能公开这样称呼,但在四下无人的时候帕斯卡会叫46山茶——比“46号同学”这种称谓要温暖一百倍。

“你……身体不舒服吗。”46明知故问地说着。

“没什么,只是有点累。”

46没有再说什么,只是默默地走了过去,轻轻梳理着帕斯卡那华丽而柔顺的长发。但当她碰到帕斯卡的脸庞的时候,却感到了一阵凉凉的触感。

46伸手抚摩帕斯卡的脸颊,摸到的是潮湿而冰冷的东西——那是浸透了脸庞的泪水。

“帕斯卡姐姐,在哭吗。”46说。忽然间,她仿佛明白了什么。

“不是。”帕斯卡虽然否认,但埋在被单里的肩膀,却抖得更厉害了。

“有人伤害了帕斯卡姐姐。”

“没有。”

“请让我来帮姐姐吧。”

“……帮我?怎么帮?”

“把姐姐不喜欢的人抹消掉。”46轻声说,“虽然从来没有人说过,但是其实我知道。我出现在这里的目的,就是为了成为那样的……东西。所以,就算还不是特别成熟,但我能做到。我想我可以。”

46不知道自己为何要说出这些。这番话一旦说出口,等于彻底撕下了她们的伪装,“学生和校长”的角色扮演游戏就再也玩不下去了。但她还是那么说了,因为那就是她真实的想法。

虽然已经决定了要和帕斯卡划清界限,但看到这位“姐姐”此时梨花带雨的姿容,46的胸中忽然感到一丝压抑的感觉。那是一种无法言喻的钝痛。

听到46的话,帕斯卡终于抬起了头。她拨开遮挡了面容的头发,然后默默地望着46,平静地笑了笑——虽然她的脸庞被泪水打湿、眼睛也有些红肿,但那个笑容里的温柔却一如既往。

“你说了些大孩子才会说的话呢。”帕斯卡微笑着说道,“学会关心人了,真让姐姐高兴。”

\section*{}

“其实,你都看到了吧。”

整理好衣装,帕斯卡端坐在床边对46说道,语气轻松得仿佛在谈论别人的事情。

不知该如何作答,46默默地在站在帕斯卡面前。过了一阵,她终于微微点了点头。

“不过你可能误会了,我并没有受到强迫。”帕斯卡说,“我们这所学校的资金全部来自于一家大公司的赞助,而那个男人是那家公司里的大人物,他为我提供了不少便利。但享受这些便利就要有相应的付出——这个世界上没有什么是免费的,那只不过是我们之间的交易罢了。”

“所以,他还要活着。”46点了点头,“我明白了。”

“我好多事还得指望他呢。”帕斯卡笑了笑,俏皮地吐了吐舌头,“以后可不许偷看大人的事情啦!”

“对不起。”46的脸红了,“我……不是故意的。”

“嘻嘻,还是小姑娘可爱。”说完,帕斯卡一把拉过46把她搂在了怀里。

“这个世界上没有免费的东西……我说过了吧。”帕斯卡在46耳边喃喃地说着,“为了得到一些,就不得不舍弃一些,这就是所谓的代价。当所需的东西都掌握在男人手中的时候,想要得到那些,就不得不利用这具身体……”

“我知道。”46平静地说,“课堂上也学习过了,那就是我们都是些年轻女孩的原因吧。我能够理解。”

“看来都知道了呢。”帕斯卡笑着说,“我本来还担心你一旦知道了自己的身份会无法接受,但你比我想象得要坚强。不过,刚才你转身离开的时候,一定生气了吧?”

“啊?不,没有……”被说中的心事,46窘迫地否认着,“只是因为觉得偷听别人的谈话很没礼貌,所以才……”

“别不承认了,我能看出来。”帕斯卡把46抱得更紧,声音却更轻了,“因为姐姐在和别的男人做那种事,所以吃醋了,对吧?”

“不是……”

“能够被山茶这样在意,姐姐很高兴。虽然女人的身体经常被当做工具来使用,但我们也该留一分给自己、和自己喜欢的人。”帕斯卡在46耳边呢喃着,声音缥缈如烟,“我喜欢山茶。可爱的山茶,就是我的愈心良药呢。能留在这里陪姐姐一会儿吗?”

“……嗯。”46闭上了眼睛,没有拒绝帕斯卡正在伸进她衣服里的手。

\section*{}

46是上午离开操场的,但走出帕斯卡的办公室时,天色已经有些发暗。一回宿舍,46就倒在了自己的床上,她感觉被抽空般全身使不出一丝力气。

帕斯卡和她“嬉戏”得太过分了。

帕斯卡火热的嘴唇舔舐着46的每一寸肌肤、每一个敏感的凹凸,仿佛要将她的味道尝尽一般,任何一个角落都没有放过;而46也激烈地回应着帕斯卡,像一只圈定地盘的动物一样,努力地用自己的体液,去掩盖住帕斯卡身上陌生男人的气味。整整一下午,两个人亢奋地沉浸在一次又一次的激昂之中,一直到精疲力竭才相拥着闭上了眼睛。

“怎么了,山茶,哪里不舒服吗?”看着气息虚弱的46,海棠关切地问道。但她永远都不会想到究竟发生了什么。

“没什么。”46把脸埋在枕头里说道,“只是……有点累。”

真是方便的借口,46心想。怪不得帕斯卡会这么说。

“是吗。”虽然感觉46的状态有些奇怪,但海棠并没有怀疑,“那就,好好休息一下吧。”

“嗯。”46轻声说,“对了,玉兰她们呢?”

在走进宿舍的时候46就已经注意到了:其他同学们几乎都在,但平日里形影不离的几个人,此时却只有海棠一个人。

“她们……在战斗模拟室训练。”海棠说,“我本该在那里和她们一起训练的,不过一下午没有看到你,有点担心……所以就先回宿舍等你了。”

战斗模拟室是46和她的同学们日常进行战斗课程授课和练习的地方,配置了各种模拟训练环境和武器的器材。不过让46奇怪的是,除了玉兰等四个人,其他人好像没有参加训练。

“是课程?为什么其他人……”

“不,是我提议的。”海棠摇了摇头,放低了声音说。

“你提议的?为什么?”注意到海棠的语气变化,46抬起了头。

“你离开之后,我趁教员们不注意,悄悄去看了一下接下来两个月未公布的课程表。”海棠用尽量不被他人听到的轻声说道,“课程里的战斗课程和对抗练习,增加到了课程总量的一半。我感觉……这是一个信号。”

“……什么信号?”

“还不能确定。不过,我猜考核可能要提前了。”

听了海棠的话,46心里一沉。她想起下午偷听到帕斯卡和副校长的谈话,帕斯卡承诺春节之前会给出成型的数据和样本……莫非指的就是这个吗。

“所以,你是让玉兰她们……”

“是的。”海棠微微点了点头,“考核的主要内容很可能是战斗对抗,但她们的战斗能力和我们比起来还相差很远,班里比她们强的同学大有人在。芍药和玉兰还好,含羞和木荷则是完全靠着和队友的配合才勉强过关,所以我希望她们能够加强训练来磨练相互之间的默契。毕竟,她们没有其他的人可以依赖了……”

“你看起有点悲观。”46说,“这所谓的考核,到底有多难呢?”

“有多难我不知道,但我倒知道一定不会轻松。”海棠故作轻松笑了笑,脸上的担忧却没有因此变淡,“毕竟,你也知道我们是在干什么。”

\section*{}

就如海棠所说的那样,进入十二月之后,战斗的课程和训练明显增加了,她们每天至少都有一半的时间在相互对抗。一天下,累得筋疲力尽、或者打得鼻青脸肿已经成了家常便饭,她们的对抗训练中甚至开始频繁使用实弹。

46和海棠的情况倒尚可,但含羞和木荷明显对这强度和危险度都突然激增的训练一时难以适应,状态每况愈下,这让46甚至都有些担心她们会坚持不住。但在海棠的努力之下,这个脆弱的联盟总算勉强维持了下去。

到了一月的中旬,离春节只有一个月多点的时间了。虽然学员们是绝对不允许离开校园的,但是有些学员悄悄地剪了一些窗花贴在窗户上,教员们倒没有制止。这让46稍稍感受到了一点节日的气息——虽然她并不了解“春节”的事情,但她意识到不知道这个节日的,恐怕是只有她一个人。

“山茶,我有点不舒服。”

在月底的一节上午的化学课上,海棠忽然开口小声说道。46微微瞥了一眼海棠,看到她的眼睛依然在看着讲台上正在授课的教员,看不出有一丝异常。

虽然不知海棠在想些什么,但她的样子立即引起了46的注意。

“哪不舒服?”46同样不动声色地轻声回应。

“肚子不舒服。下课陪我去洗手间吧。”

“嗯。”

身体“不舒服”是几个觉醒者之间约定的暗号,代表着有情况发生,而不舒服的地方从上到下表示事情的紧急程度。不过因为很少用,46几乎已经快忘了这个暗号了。而且海棠从来没有“不舒服”过,其他成员最多也就是嗓子“不舒服”的程度。这次的事情,看来相当严重。

“什么事?”

洗手间里,海棠和46在洗手池前轻声交谈着。她们都打开了水龙头,以掩饰交谈的声音——虽然效果有限,但也足够干扰悄悄话的声音了。

“关于‘计划’的事情。”

“说吧,我在听。”

虽然不算出乎预料,但海棠的话立即引起了46的密切关注。

不知道是不是因为已经洞悉了46和帕斯卡校长之间的“特殊来往”,海棠一直一来都没有对46透露过她“计划”的细节。46也曾装作不经意地向另外几位伙伴打听关于海棠的计划的事情,但没有得到任何线索,海棠似乎也没有把这些诉过其他人。可正当46有些不抱希望地决定默然等待的时候,海棠竟然主动说起了这件事情。

 “计划的具体细节我早就策划好了,但是为了保密一直没有公布。但现在我觉得有些事情有必要让你也了解,不然,万一我遇到不测就没人知道了。”

“不测?难道发生什么事情了吗。”

“还没有,不过我感觉也快了。你也发现了吧,最近的训练强度越来越高,我们的安全已经没有保障了……不测的事情也许随时都可能发生。所以我要把这些事情告诉你,因为我们两个同时出现情况的可能性较小……而且,你那边有校长的关照,说不定比我更安全。”

“……知道了。你说吧。”

“还记得我们的目标是什么吗。”

“离开这里?”

“没错,但那只是第一个目标。我们的最终目标是向世界揭露这里的一切,这可不是一件容易的事情。”

“那你是怎样打算的?”

“我没有打算。”

“没有?”

“是的。我的‘计划’,只包含逃离这里的部分。离开这里之后的事情我无法去计划,因为我也不知道外面的世界是什么样的。”

\section*{}

海棠没有说明计划的细节,她只是告诉46她已经把必要的东西准备好了,只等一个合适的时机;成功逃脱的人需要在预定的地方拿到在外面生存所需的东西,其他的一切就要靠她们自己了。

这一切大大出乎46的预料。她本以为海棠已经安排好了一切,只等一个合适的实施契机,但事实上海棠能够把握的事情也不过是冰山一角。

46也考虑了海棠是不是为了试探她才故意这么说,但经过仔细思考,46认为海棠所说的是真的。外面的世界她们谁都没有见过,海棠不可能对离开校园之后的事情做出计划,逃离之类之后一切都要随机应变。

如果真的是这样,那么她们几个不可能全部逃脱——46忽然意识到了这个让她惊恐的事实。揭露这里的事情,只要一个人就足够了,面对未知的环境,多一个人就多一分暴露的可能……海棠显然也明白这一点。那么,在她计划之内的,到底是谁呢?

既然海棠告诉了她这些消息,那么她也许是海棠所计划的其中之一。但她可能不是唯一的一个,而这些事情则是不能透露给别人的。

想不到同伴之间,从此也不能相互信任了,46心想。但海棠要如何筛选合适的人选、计划之外的人又会如何呢?这些问题让46的心里第一次感到了一丝惶恐的不安。

但46的不安没有持续太久,因为很快她就得到了答案。

这个校园里所进行的竞赛,与其说是“对抗训练”,不如说就是战场的缩影:分成几个小组的班级,每个人都会领到一把枪,而这把枪里装的都是实弹——这些子弹击中人体会发生什么,自然不用多说。每次训练结束都会有人受伤甚至失去行动能力,那些伤势不重还能继续作战的同学,在经过治疗后有时还会再次回到教室;而那些受伤严重的学员,46则从来都没有见到她们再次出现。学校的说法是她们被送到了条件更好的医疗机构进行康复治疗,但46不认为那些胸口被穿了好几个洞、或者脑袋被削掉一半的同学,还有机会再次出现在训练场上。

虽然她们几个觉醒者一直在互相扶助维持着战斗力的平衡,但在越来越强的分组对抗中,还是被其他小组抓住了破绽——这个厚度不匀的容器,最脆弱的一块终于被击碎了。在一次“训练”中,因为对玉兰的命令产生了一瞬的迟疑,木荷中了好几枪。虽然穿着防弹衣,但是有一发子弹穿过了她的脖子,打断了她的颈椎。木荷当场就倒在了血泊之中,双眼无神地瞪大、身体不断抽搐着。虽然她立即被教员们抬走了,但却没人知道她被送到了哪。

“木荷的体力和反应速度都在我们之下,而且作战经验也算不算丰富。” 回到训练场的器材室,海棠低声对46说道,“客观地说,她甚至不如一些被处理掉的学员,只是靠着玉兰和我们几个才走到今天。她已经到极限了。”

46没有说话,只是默默地看了一眼木荷的搭档玉兰。46看到玉兰两眼发直,依旧沉浸在作战的失败中,回不过神。

而胆怯的含羞,则一直在轻轻地抽泣。

“我就知道会这样的。”含羞一边哭一边说着,“训练场上回来的人越来越少。对手越来越强,而我们却越来越跟不上了。迟早有一天我们都会像木荷一样,不知道下一个是谁呢……”

“我也快受不了了。”芍药咬着牙说道,“每次都要靠杀掉自己的同学才能活下来,如果不这么做,就会被她们杀死。为什么我们要做这种事情啊……”

“冷静点。”海棠说,“这种事情,我们不是早就知道了吗?我们一直以来都是这么走过来的,这种时候如果不冷静下来的话,我们就……”

“冷静!这怎么能冷静得下来!”芍药终于忍不住喊了出来,“木荷她……死了呀!你们没看到她的身体吗?脖子后边,拳那么头粗的洞,颈椎被完全撕碎了!那种伤口,怎么可能……怎么可能救得回来?!”

“救不救得回来不是我们的事情!”海棠也提高了声调,“放弃战斗或者反抗教员会怎样,难道你不清楚吗?想为死去的同伴报仇,我们必须要活下去!”

“说得轻松!你和山茶都那么强,当然可以安然地谈论‘活下去’这种事情,可是我们该怎么办?玉兰已经失去了她的搭档,下次的对抗训练,她该如何……”

“那我们就重新组队吧。让山茶和玉兰一组,你们四个互相援护,作战效率一定会高过之前。我再想办法和其他人组队,或者实在不行就单独作战……小心一点应该问题不大。你说呢,山茶?”

海棠说完,所有人都安静了下来。

眼下看来,这不失为一个可行的办法。不过这样的话,海棠很可能将不得不孤军奋战:一来现在班里的分组都已经趋于固定,二来那些几乎没有交流的同学根本不知道是否可信,要想另找一个搭档谈何容易。

另外这个提议是否能够执行,很大程度上也取决于46的选择。

于是,所有人的目光都落在了46身上。

46抬眼默默看向几位同伴,看到她们在用不同的目光注视着自己,有期待也有忧虑,显然各有各的想法。

46张了张嘴,却不知该说些什么。她本想逢场作戏地说些安慰的话,但她知道自己安慰不了任何人。因为所有人都明白,这种牺牲还会继续,而她们只能自救、相互之间谁也救不了谁。

也许是因为过早地洞悉了一切,46并没有为了失去同伴而感到悲伤。她知道那些都不是真实的——虽然这几个觉醒者都有着完整的记忆,但那些记忆都是加工而成的,由此产生的情感也许并不能称之为羁绊。46所要追寻的真实乃是在“校园”的高墙之外,所以她唯一感兴趣的,就是海棠的逃脱计划。

“不,你去和玉兰一组吧。”46摇了摇头,轻声对海棠说,“你们本来就是一起的,而且她们也需要你的带领和鼓励。我不需要那些、也做不到那些。所以就让我一个人好了。”

“众人拾柴火焰高。而且,你现在也已经无法置身事外了。”察觉到46有意疏离,海棠用略带警告意味的语气说道。

“要么活着逃出去、要么死在这里,根本没有什么事里事外,不是吗。”听到海棠的话,46淡淡一笑,“不过我只是想知道在拼个鱼死网破之前,是不是能有个折衷的方案。”

\section*{}

就像是为学生们布置的作业习题,出题者的意图是历练学生的解题技巧;但多数学生者想要的却只是答案。答案必须由解题得来,一般来说是这样的,因此学生才会努力解题;但有些学生的习题本的最后面会带着参考答案,所以他们就不想再殚精极虑地去思考这些难题该如何去解了。他们要做的只是把习题翻到最后——而46,就是这样的一个学生。

“你来了,小茶。这些天可把我忙坏了,所以也一直没顾上你。”见到46走进自己的办公室,正在一堆数据和表格之间忙碌的帕斯卡眼里放出了光彩,就连称呼都变得更加亲昵了,“你这几次的对抗成绩我都看了,你和31号真是一对不错的搭档,各项指标在所有学员里都是顶尖的。我正想着一定得好好奖励奖励你……”

31号,46思索着这个称呼。那是海棠的代号,但是她已经好久没有说过这个数字了,以至于她一下子没能意识到这个数字代指的是哪位。

事实上,46已经不记得其他觉醒者的代号了。她唯一确知的只有她自己,而海棠,则是她较为熟悉的一个。

但46来找帕斯卡,并不是来接受她的赞扬的,所以她没有说话。

“怎么了,小茶?一副闷闷不乐的样子,是和同学之间发生矛盾了吗?”

看到46不说话,帕斯卡关切地问道。

发生矛盾?这句问话让46简直要忍不住想笑出声来。要说矛盾,也没什么太大的矛盾——充其量不过是偶尔会拿枪互相对射而已。

这场疯狂的滑稽剧,她已经不想继续演下去了。

“校长……女士。”沉默了片刻,46终于鼓起勇气低着头说道,“我希望,不要再这样下去了。”

“不要怎样?”帕斯卡不解地说。

“不要再……让这些孩子,自相残杀了。”

“……噗嗤。”

帕斯卡楞了片刻,然后扭过头,发出了一声轻笑。

“校长女士?”

“嘻嘻,对不起……是我太失态了,嘻嘻嘻……”帕斯卡一边掩着脸、一边笑的全身发颤地说着,“不过,我真的没想到你会说出这个词来。‘自相残杀’?嘻嘻嘻……这句话从一个人偶嘴里说出来,嘻嘻,还真的有那么一点……情怀的意味呢……”

“……”

看着无法控制地发笑的帕斯卡,46的心在下沉,她知道她们之间的逢场作戏彻底结束了。不过这也让46感到一丝轻松。

该来的总会来,早点去面对也好。就算演到最后,也不过还是这个结果。

“抱歉,我懂你的意思,但可惜你的观点一开始就是不成立的。汽车制造厂里,为了检测汽车的安全性能,会进行模仿交通事故的碰撞试验。试验车辆在测试中损坏或者毁掉不是很正常的吗,何谈‘自相残杀’呢?”好不容易止住了发笑,帕斯卡依然带着笑意说道,“你以为你们是什么,老师眼里的乖学生、父母手心的小宝贝?你们不过是一堆测试用的器材而已,杀伤性的对抗训练是普通到不能再普通的实验流程,你怎么能对此感到不满呢?你们就是为了这种事情而存在的东西啊!”

“这么说,其实我们根本就不是人类,对吧。” 46低声说道。

“那当然。你们不过是流水线上生产出来的人偶,加载虚拟的人格和记忆之后用以模拟人类行为的代用品罢了!我还以为你和那些人形不同,早就知道这些了呢。原来是我高估你了吗?”

“不,我很久以前就已经意识到了这些……不过在得到您的亲口确认之前,我一直都不肯相信。” 

“哈哈,你还真是单纯得可爱!”帕斯卡再次笑了起来,“不过你的这份单纯正是我中意的地方,明知道自己是个悲惨的东西,但心里还是怀着一丝期冀。知道吗,看着你心中的希望不断被消磨却依然不肯放弃的样子,也是一件会让人莫名地感到兴奋的事情呢。就像一个每天都遭到强暴少女一样,一边因为被粗暴地蹂躏而感到屈辱、又一边因为求生的欲望而重新振作起来,于是在被玩弄得体无全肤之后,依然会一边清洗着自己还在淌着男人体液的身体、一边重新穿戴好自己的衣服装作若无其事的样子。嘻嘻,真是让人欲罢不能啊,嘻嘻嘻嘻……”

听到帕斯卡的话,46不由得也落寞笑了笑。原来她在帕斯卡眼里,只是这种东西而已。

她感到可悲,为了自己那天真的梦想。明明知道自己不过是一件工具、一个玩偶,根本不可能有什么“未来”可言,但还是像帕斯卡所说的那样,心里怀着对未来的莫名期待。

但46知道有的人比自己更加可悲:虽然自己只是个微不足道的实验体,但至少也曾作为一个“人”去努力地活着,而不是在笑着送走蹂躏自己的男人之后,独自一人默然垂泪。

但这些话,46没有说出口。

“好吧,既然事情已经到了这种地步,该说的干脆我都说了吧。”帕斯卡说,“你本来应该和那些实验体一样,一直战斗到被彻底摧毁为止。但现在可以给你一个做出选择的机会。有句话说‘人不能选择出身,但是可以选择要走的道路’,但对于你来说,这种选择的机会恐怕也只有这一次,所以希望你能好好珍惜。”

说完,帕斯卡收起了笑容,用玩味的目光看着46。而46则默默看着帕斯卡,一言不发。

“那么我这流水线上生产的玩偶,能为您做些什么呢?”

片刻后,46终于开口说道。

\section*{}

“离开测试线、来我的办公室,在我身边协助我的工作——这样就彻底跳出小白鼠的笼子,你也不必再替她们操心了。怎么样?”

帕斯卡说完这句话,脸上露出了笑容——属于胜利者的、志在必得的笑容。她了解46对自己的憧憬,所以她知道46不会拒绝这个邀请、也不能拒绝。

听到帕斯卡的话,46愣住了,她没想到帕斯卡所谓的“机会”是指这件事。

这就是所谓的“权力”吧,46心想。她此时才真切地感受到这个字眼里蕴含的意义:权力在手,决定对别人的生杀予夺,甚至都能如此轻描淡写。

一个人要付出何等的努力,才能改变自己的命运呢,46想不出来。但此刻的她发现改变命运这种事情,有时候就是那么的简单:只凭一个人的一句话,就已足够。

“为什么是我?”46问。

“因为你是不同的。”帕斯卡说,“你比那些实验体思考得更多、成长得更快;你不只会按部就班地去做那些已经安排好的事情,还会思考这些事情之间的联系和意义,我能够看得出来。而且你也比其他实验体更懂得如何与身边的人相处。你有着丰富的情感,拟人程度远高于其他实验体,有时就连我都怀疑你到底是不是一个人形了……坦白说,我很中意你。你大概已经想到了吧,这里的学员就算通过了全部考核,也不过是去做某个利益集团的杀人工具。想一想这样的归宿,也让我感到于心不忍呢。所以,成为我的助理吧。这样对我也好,对你就更不必说了。”

46思考了一阵,她不得不承认帕斯卡的邀请很有吸引力。她隐约记起自己记忆(虽然她知道那是别人的记忆)里真正有血缘关系的那位“姐姐”的话,即使作为真正的人类,想要出人头地也是万般困难的事情,更何况是实验室里的一具实验体。

如果她真的想要成为自己向往的那种人,那么这就是她要迈开的第一步,这不正是她梦寐以求的吗。而如果拒绝了帕斯卡的邀请,那么最好的情况,她也不过是变成一具杀人机器。该何去何从,简直不言自明。

46知道只要她微微点头,她的命运就会彻底改变,那些噩梦般的经历从此就和她毫无关联了。但同时她也知道做出这样的选择意味着什么,因为同样会变得毫无关联的,还有在那些噩梦般的经历中认识的人们。海棠她们也许会全部死在测试之中,但就算如此,她们到死也不会再见到46了。选择接受帕斯卡的要求,46首先必须要彻底出卖给予了她无比信任的人。

可是,那真的是她想要的吗,46问自己。虽然还不完全了解帕斯卡所做的事情,但她所经历的那些,46已经窥见一斑:就算是帕斯卡,也不得不出卖身体来换得他人的庇护。自己在帕斯卡的身边,又会扮演怎样的角色呢。

——无非还是任人摆布的牵线木偶罢了,区别只在于,操弄自己头上的线板的人是谁。这并不是真正的自由,只要还在一人之下,就没有真正的自由可言。要挣脱这桎梏,唯有像海棠所说的,逃离这里。虽然不知道外面是怎样的,但这座“校园”之外的,才是真正的世界。

而通往那个世界,唯有一途。

在长久的沉默之后,46最终摇了摇头。

“不。”46说,“感谢您给我这样的机会,但我不能接受您的邀请。就算是来日不多,但我觉得还是呆在自己真正的‘同类’身边,才会更为安心。”

“……为什么呢。”这个答案明显让帕斯卡出乎意料并且非常失望,但她还是假装不在意地问道,“是因为你放不下自己那所谓的同伴?还是觉得我这样的人不值得你去信任?”

“也许两者都有。”46说,“的确,我曾经幻想过有朝一日能成为您这样的人。但现在我忽然发现,对我来说,那也许不是一个很好的选择。我永远都无法成为和您一样的人,因为我只是一件出于实验用途,而被制造出来的工具而已。”

“呵呵,你也知道自己接下来要面对的是什么吧。就算我放你回去,接下来的对抗里你们之间恐怕只有寥寥几人能够生存下来,纵然你很优秀,但鹿死谁手没人能够把握。”帕斯卡冷笑着说道,“还是说,你在心中对那些晚上不肯好好睡觉的孩子们的‘逃亡’计划,依然有所期冀?”

“什么……”

听到帕斯卡的话,46震惊得说不出话。无论如何,回到觉醒者们之间,自己至少还有一丝逃离的希望——就算是侥幸心理,那就是46心中最后的希望。但她没想到自己隐藏最深的秘密,却被帕斯卡当做孩童稚嫩的小花招一样点破了。

“难道,您已经知道……”

“比你知道得还要早,只是因为觉得很有意思,才故意睁一只眼闭一只眼的罢了。你们觉得我什么都不知道,反而才让人不解。那次杀戮训练中你昏倒了,醒来时却出现在了我的办公室,你不认为这件事很耐人寻味吗?”

听到帕斯卡的话,46无奈地笑了笑。是啊,她早该想到的——有瑕疵的实验体,只会出现在下一次杀戮训练的靶场上,而她却一再得到帕斯卡的关照,显然这里边有文章。

想来这座校园里到处都是监控,虽然明装的摄像头她们都已经记清了位置,但隐藏的窃听器她们根本一无所知,怎么可能那么容易就蒙混过去呢。

“你的前景很好,我不忍心让你就此荒废。但你毕竟还年轻,对人的了解太过肤浅了,不会想到人为了自己会出卖多少。”帕斯卡轻轻叹了一口气说,“你以为只有我会迎逢上层以求私利吗?你错了。哪怕只是人类的仿制品,但凡有一丝人类的影子,他们都会这么做。守卫!”

帕斯卡说完,朝着办公室的门口呼唤了一声,两个全副武装的教员走了进来。他们显然已经在那里等候多时了。

“控制住她。”帕斯卡说道。

两个教员走了过来,用有电击功能的枷具将46拘束了起来,但46并没有反抗。

“我这就把你的伙伴们带来,你可以和她们一起谈谈心,顺便了解一下你到底该信任谁。也许到时候,你就会改变你的决定。”

帕斯卡说完,脸上露出一个意味深长的笑容。

\section*{}

“你这个该死的叛徒!”芍药死死盯着46,恶狠狠地说道,“我就知道你不可靠,你一定是受了校长的威胁,把我们全都供出来了吧?还是说,你早就和她是一边的了?!”

面对芍药的咒骂,46没有做声。她没有出卖伙伴,但她知道眼下的情况无论怎么辩解都是没用的。她只是在心里一直思考着帕斯卡的话。

“哪怕只是人类的仿制品”、“比你知道的还要早”,帕斯卡显然是在有意无意地透露这些觉醒者中间有叛徒。但46是最后一个加入觉醒者的,所有人都比她更早,那个真正的叛徒会是谁呢?

46抬头扫视了一番同样被拘禁得一动不能动的同伴,发现每个人都在用锐利的目光盯着自己。她忽然意识到了问题的关键。

说到底,这个所谓的逃亡计划,需要多少人逃出去才算成功?

谁都知道,不可能每个人都能逃出去。那么把这个数字降低到极限的话,其实只要一个人就够了……

其他的人,都是多余的。

原来如此。帕斯卡所说的“比你要早”,其实是比任何人都要早——也许从海棠一开始制定这个计划开始,她就已经知道了。帕斯卡只是一直在默默观察,海棠是如何假借计划之名来摆弄她的同伴的。如果没有46,那么剧情大概就会按照海棠所设计的发展:直到海棠利用完了所有人、直到只剩下她最后一个,到那时,帕斯卡可以再把她的计划轻而易举地彻底粉碎。

说不定,帕斯卡也一直在其中推波助澜。而46的出现,只是一个意外。

没有人会想到46和帕斯卡之间产生了怎样的关系,更不会有人想到她会和帕斯卡直接摊牌。因为没有人想到46的心智,和别人有着天壤之别。

“海棠,是你吧。”46低声说道,“我们的一举一动全都在帕斯卡的监控之下,你是知道的,但你却没有对任何人说过。你导演了这场生存游戏的戏码,是因为在你的逃亡计划中,真正的逃亡者只有你一个人,对吗。”

“放屁,你这个血口喷人的……”

“冷静一点,芍药。我也有些疑问想问海棠。”正当芍药想破口大骂的时候,玉兰开口制止了她。

“我也一直都觉得奇怪,我们的行动看似缜密,实则有着不少破绽。但每次我们都很幸运地蒙混过去了,教员和帕斯卡竟然从来都没有任何察觉。如此想来,这一切似乎太过顺利了吧?”

玉兰说完,所有人的目光都转移到了海棠身上。这些话说得没错,引起了大家的思考。

海棠看了四周的同伴一眼,大家都在疑惑地期待她的回答。显然,她们对海棠的信任已经被动摇了。

“是的,我没有告诉你们全部真相。但不要把我想得那么卑劣,我只是为了避免节外生枝。”一阵沉默之后,海棠终于开口说道,“逃亡计划是真的,不过这件事经过了帕斯卡的默许……所以,可以说这也是实验的一部分。帕斯卡和我约定了一场赌局,如果真的有人能够从这里逃出去,那么逃出去的人就自由了。”

“什么,海棠?!你、竟然真的是你……”

“闭嘴,芍药。”玉兰冷冷地打断了芍药的惊呼,“‘逃出去就自由了’,说得倒是很好听,但没人会相信这种信口雌黄的话。帕斯卡的承诺用什么做抵押呢?她如果出尔反尔,你又如何保证她履行诺言呢?我想你不可能没有考虑过这个问题对吧?”

“……我自有办法保证她履行诺言。”

良久,海棠终于开口说道。

比起谁是真正的叛徒,海棠的这句话更让人惊讶。她说她有办法保证帕斯卡履行诺言,但她身为实验体之一,是不可能抓到帕斯卡的把柄去威胁她的。那么她又会有什么办法呢。

“这么说的话,的确是我想错了。”46轻声说,“原来这场赌局里你赢得的奖励,不是自由。在这场逃亡游戏中,唯独你不是逃亡者,是这样吧。”

“是的,我会留在这里。如果你们中间有人能够逃离这里,那么我就会被允许成为帕斯卡的助理。”海棠点了点头说。

\section*{}

所有人都沉默了。这和海棠之前所描绘的情景完全不同,她竟然只是为了自己的目的才鼓动这些人逃亡的。

“你……利用了我们。”从始至终都没有说话的含羞颤声说道。

“你可以这么说,但这对你们没有任何坏处。”海棠说,“你们也知道,这里的多数人的结局都是在实验中死去,能够逃出去是一种幸运。虽然木荷死了,但我已经尽力帮她了。在这弱肉强食的规则之下,她必然会被淘汰,你们中的其他人也是如此。如果没有我,你们也许根本活不到今天。”

“你的话依然漏洞百出。”玉兰说道,“如果你的愿望是留在这里,那么逃出去的我们对你来说反而是一种威胁。你怎么可能真心实意地想要帮我们逃出去?”

“因为你们已经不是第一批学员了,而我是。”海棠说道,“早在第一批测试实验中我就已经觉醒了。那时候我的想法是逃出去……和现在的你们一样。但我失败了。可帕斯卡却对我的想法很感兴趣。她认为逃亡比战斗的选拔效果更好,可以选拔出更富策略性的实验体,所以我才没有被处理掉,而是被送进了第二批学员的测试实验。帕斯卡对我说,如果我能带领这些觉醒者成功出逃,那么就放我离开这个地方。但第二次我依然没有成功。”

“我们是第几批?”玉兰问。

“第三批。”

“觉醒者的出现,不是偶然吧?”

“在第一批的实验中,我是个偶然。但第二批就不是了,虽然帕斯卡没有说出来,但我相信她刻意地让一些人在深度睡眠中醒来了。”

“何以见得?”

“因为第三批的觉醒者,也就是你们,全都是我指定的。”

“……”

再次的意料之外,接着是再次的沉默——如果不是海棠亲口所述,这些事情的渊源她们就连做梦都不会梦到。

看似破绽百出、实则机关算尽,在这场迷局之中,竟然交织着如此精巧的策划和险恶的谋略。如果这是一场阴谋,那也许会是一场足以颠覆一个政权的庞大阴谋。

“所以这才是……你真正的计划?”

良久,46终于开口说道。

“是的。一开始我只是想要逃出去,但第一次实验我因为经验不足而失败了。第二次我制定了详尽的计划,如你刚才说的一样,那次计划中我是唯一的逃亡者。但是还是不行,没有人里应外合,这个计划也不可能成功。所以第三次我的计划是在帕斯卡身边安插一个内应……那就是你所扮演的角色。这次的计划本该很顺利的,因为你已经取得了帕斯卡的信任。但在计划实施的最后阶段,我改变了主意。我不再想要离开这里了。”

“为什么?”

“因为我忽然意识到,外面的世界,对我来说太过陌生了。我已经在这里度过了三年时间,差不多已经习惯了这里的生活,如果真的离开了这座实验场,我不知该如何在外面生存。我有着培育学员的经验、帕斯卡也需要一个内应,也许留在这里才是更好的选择。所以前几天对帕斯卡说……我不想离开了。如果我把你们中间的一个送出了这里,希望她能够让我继续留下。她当时也同意了。所以我才把逃亡的计划透露给了你们每一个人,就是为最后的步骤做好准备。”

因为胆怯而退缩了吗,46心想。外边的世界对于这里的每一个人来说都是陌生的,但除了海棠,这里的每一个人都期待着能够离开,无论让谁逃出去那个人都不会拒绝。如果看作是各取所需的话,也许不能说海棠完全是在利用她们。

可惜,纵然机关算尽,海棠还是棋输一着——她无论如何都不会想到,被她唤醒的46,是一个没有依照规则行走的棋子,她从帕斯卡那里得到的,不仅仅是信任。帕斯卡的确是选择了一个人做她的助理,但那个人不是海棠。

经过这三轮的实验,帕斯卡已经收集了足够的数据,恐怕不会再有下一次实验了。她此刻想要的只有一个人,就是46。她们此刻在这里“欢聚一堂”,只是因为帕斯卡想让46知道,人心是有多么的复杂和难测。

“很遗憾你再次失败了,海棠。”46喃喃地说道,“如果你就按照一开始的计划去执行,这次你极有可能真的会成功,可你却在最后做出了错误的决断。帕斯卡不会履行承诺的,就算真的有人成功逃脱,你的利用价值也彻底完了。我今天对帕斯卡摊牌了,但帕斯卡邀请我做她的助理……她需要的不是你。逃离这里是你唯一的出路,但你却放弃了。”

“什么?”听到46的话,海棠也不禁瞪大了眼睛,“帕斯卡邀请了……你?不要告诉我,你却拒绝了她的……”

“没错,我拒绝了她的邀请,因为我是真心想要离开的。这就是我们为何在这里的原因。”46轻声说道,“实验数据,我想帕斯卡已经收集够了。她之所以把我们聚集于此,就是为了让我对你的‘计划’亲耳一闻……为了让我知道什么叫做,‘人不为己,天诛地灭’。”

“呵呵……”海棠摇了摇头笑了起来,脸上的表情满是讽刺,“真是造化弄人。早知如此,何必当初?何必当初啊……”

“别那么绝望,31号。事情还没有结束呢。”

正当海棠万念俱灰的时候,囚室的门忽然打开了,从门口传来一个熟悉的声音。

\section*{}

走进来的人是帕斯卡,身后还带着十几个荷枪实弹的士兵。

“山茶说的没错,到此为止收集的所有实验数据,已经足够我交差了。但正因为时间尚有余裕,所以我还是想把这一局赌完。”帕斯卡停在几个人的面前,冷酷地笑着说道,“我知道对你而言我从来都是不可信的,其实对我而言,你也已经没有利用价值了。但看在你为我找到了小茶这个可爱的小姑娘的份上,我再给你一个机会、给你们每人一个公平争取的机会。”

“你想干什么?”海棠冷声说道。

帕斯卡没有回答,从大褂的兜里取出一个遥控器,轻轻按下了一个按钮。

瞬间,囚室里的一个屏幕亮了起来。几个人朝着屏幕看去,看到屏幕里显出的图像是她们对抗训练的场地……以及整齐排列在场地中的方阵、和摆放在场地边缘的各式武器。

方阵之中,有认识的人、也有不认识的,一共大概有上百人。但毫无疑问,她们都是和海棠她们一样被选拔出来的学员。

“别紧张,依然是和平时一样的对抗赛,只不过规模稍微扩大了一点。反正最后都要走这样的过场,还要一张一张打出手里的牌,也太麻烦了吧?不如就痛快一点——梭哈吧。”帕斯卡两手一摊,做了一个showhand的姿势,“游戏的规则很简单:最后活下来的人就是胜利者,要走我绝不阻挠、要留我安排位置。怎么样?”

没有人回答,因为没有必要回答。她们知道帕斯卡的提议是不容拒绝的。

“不反对,那就是同意了?很好很好。”帕斯卡高兴地拍了拍手,“守卫,放开这些孩子。”

几个士兵走了过来,小心翼翼地给几个人解开了拘束,其余的士兵则用枪口齐刷刷地对准了她们。

“放心吧,既然说了是给每人一个机会,那么我一定会公平行事,不会让你们几个去对抗她们全部的。这次是自由交战——当然,能够组队的话更好。只是不知你们之间,是否还有并肩作战的信任呢?”

说完,帕斯卡伸手做了一个“请”的手势。士兵们为她们让出了一条路来,但手里的枪依然紧握着,枪口也没有离开几个人。

知道多说无益,海棠看了帕斯卡一眼,然后首先带头走向了训练场的方向。后边的几个人迟疑了一下,也纷纷跟着走了出去。

“小茶。”

当46走过帕斯卡身边的时候,忽然被帕斯卡开口叫住了。

“虽然我说过你只有一次选择的机会,但为了你……我愿意食言一次。”帕斯卡轻声说,“如果你愿意接受我的邀请,可以不去参加这次的对抗。”

听到帕斯卡的话,46停了下来。

“谢谢您,校长女士,您的好意我十分感激。”46微微颔首说道,“但如果我不参加,我的伙伴们的战斗将雪上加霜。而且,坦白说,我向往的地方在这所监牢的高墙之外。所以恕我再次拒绝您的邀请。”

说完,46抬起了头,继续朝着门外走去。

“你这条不知好歹的小母狗!区区一具朝生暮死的实验体,也竟敢如此口出狂言!当你死在混战之中的时候,再后悔没有接受我的施舍就来不及了!你这顽冥不灵的蠢货!!”

恼怒至极的帕斯卡在她的身后高声咒骂着,丝毫不顾及自己作为“校长”的形象,但46并没有再次停留。

\section*{}

“抱歉,一直以来我都瞒着你们,没有把真相告诉你们,因为我不知道谁是可信的、谁会配合我……也不知道最终的结果会是如何。如果一开始就让你们知道一切,现在也许就不会是这样的结果吧。”

在走进方队之前,海棠停了下来,对着身后曾经的伙伴们低声说道。

“就算一开始你就告诉我们一切,到最后最好的也不过还是这种结果。所以道歉的话就不必说了,你的计划本身没有什么问题。”玉兰冷冷地说道,“既然我们的命运注定如此,我也不会归咎于你。但你注定会失去所有人的信任,因为你一开始就没有打算信任任何人。”

“帕斯卡说这是一场公平的竞争,至少这句话不假。面前的阵势,恐怕再也没有人能够把握了,所以每个人都可以说是凶多吉少。所以就在此告别吧。”海棠低声说,“是我辜负了大家,你们认为被我出卖了也好、被我利用了也好,我都无以辩解。只是很遗憾,没法为自己的所作所为去赎罪了。”

“收起来那些泄气的话吧,现在还不到放弃的时候。团结起来,我们还没有失去希望。”46说,“我们之间经历了上百次的协同作战,总比那些各自为战的人要默契得多,我们做过的所有对抗都是为了现在——如果已经到了最后的关头,那至少不要浪费之前的训练成果。既然面对的依然是战斗,那么我们必须采用最优策略。”

“但我们之间还有能够相互信任的默契吗?在你们的心里,我已经不是伙伴了吧。”

海棠转过了身,46看到泪水滑下了她的脸庞。她知道海棠也在害怕,但她最害怕的不是死亡——死亡她们已经经历过许多次,在每个人的心中,多少都已经有了一点准备。而众叛亲离后孤独地死去,才是最可怕的。

“可是,没有别的选择吧。”芍药说,“就算是为了生存而战,我们也要团结一致。我这个人的头脑没有你们那么聪明,但既然选择了相信,我就会相信你到最后。海棠到底在计划些什么,我没有想过,但在我眼里你一直都是可靠的领袖。所以,我希望这次也依然能够如此……尤其是这次,我希望依然能够如此。”

“那样的话,要是再想找我报仇,就得等到收拾了这群配角之后了。”听到芍药的话,海棠那泪迹未干的脸上露出了笑容,“虽然有点不谦虚,但我们是最强的这一点,我倒是很自信。”

“可是,海棠姐姐……我还是有点……害怕。”含羞看着海棠,嘴里喏喏地说道。

“别怕,姐姐会保护你的。我会带头出击,你就紧紧跟在我的身后,好吗?”

“……嗯。”

“那么,觉醒者们。我们要上了!……”

\section*{}

或许选择性遗忘,也是对意识的一种保护机制?46没有记住自己挨过那场战斗的细节。她所记得的只有遍布整个训练场的残肢和尸骸、以及淹没了脚面的鲜血。那是一场史无前例的大混战,狭小的空间里多数武器根本施展不开,但为活下去而挣扎的人们已无暇顾及那些。冷热兵器相接的声音和人的哀嚎,共同谱奏出一曲诡异的乐章,许多人都是死于流弹。

芍药是站着死去的,她的左臂被砍掉了、腿上中了好几枪,还有一把刺刀穿透了她的胸膛。但在她失去挥舞砍刀的力量之前,她已经打光了自己所有的弹药。敌人的尸体,在她身边围了一个圆圈。

玉兰本来可以凭借她灵活的身姿周旋一阵子,可是为了保护含羞,她被不知从何而来的冷枪击中了。那颗子弹同样贯穿了含羞的身体,两个人就那样拥抱在一起,一直到被更多的流弹扫到在地。

当地狱般的战场沉寂下来的时候,活着的人只剩下了两个——半跪在地上的海棠和勉强站立的46,两个人后背相靠、浑身是血,但都还平稳地呼吸着。

“没想到最后剩下的会有两个人呢,我们这群觉醒者。”海棠嘶哑地说道,“也许是因为我一直祈祷我们不要团灭,但忘了祈祷我们最好能只剩一个?”

“要不是拼了命地努力战斗,恐怕现在一个都剩不下了。”46也喘息着说,“也多亏了你这么能打,发挥出平时两倍的水平了吧?”

“你也不错,说实话我最不想交手的人就是你了。可惜,这次的竞赛胜利者似乎只能有一个……”

刷。海棠话音落下的瞬间,两个人同时转过了身,四目相对。

“也就是说必须有一个人要死在这里。”海棠说,“那么,我们谁来?”

“……”

46没有说话,她看到海棠全身是血、通红的眼睛里满溢杀气。

这个人看来是不会放弃的,她心想。虽然海棠已经受伤,但自己真的能够打败她吗。

自己真的能够,杀掉她吗?

“你还能跑吗。”46低声说道。

“还行。”

“逃出去的事情,都准备好了?”

“早就准备好了。”

“……你走吧,我殿后。帕斯卡的话是信不过的。”

“……”

这次轮到海棠沉默了。

“你就那么大义凛然?还不明白吗,事情走到这步田地,我全是为了我自己!”海棠咬紧了牙说道,“我可是利用了你们所有人的人,真的就让我这么得了最后的便宜?”

“我倒是不想便宜了你,可惜该怎么出去,我根本不知道。”46耸肩一笑说,“再说,作为计划的制定者,执行起来一定比我有把握。这才是最优策略,不是吗。”

“……这片场地的北墙下面有个排水口,我已经加工过了,足够让一个人钻出去。出去之后你一直沿着墙壁向西,大约四百米距离是垃圾处理站,里边停着六辆垃圾处理车。二号车我修改过了自动驾驶的路线,它会在西北边的森林边上走一圈,你发动它就会自己走,到了森林边你趁机躲到森林里,伪装用的衣服和一些食物藏在了一棵冷杉树下,你扒开树下的浮土……”

“你说的这么多,我记不住。你自己去……”

“给我安静!好好听着!”

明白了海棠的意图,想要打断海棠的46,反而被海棠焦躁地打断了。

“听着,山茶。我的腿受伤了,恐怕干不成这件事了。你还能走,逃出去的几率要比我高得多。再说帕斯卡看中的是你,就算你逃不掉被抓回来依然还有退路。但我不一样,你死了或者被抓住了,帕斯卡一定不会放过我的。我的结局已经注定了,明白吗?”

“可是……”

“没什么可是的,逃亡计划是我制定的,我比你更清楚。仅仅知道逃脱路线是不够的,还要有人拖住他们才行。这件事你做不到,但我可以,因为我已经安排好了。听我说完然后你马上走,一秒都不能犹豫,听见了吗?”

“……海棠。”

“想问为什么是吗。好,我就浪费一点时间告诉你。”海棠说,“人不为己,天诛地灭,帕斯卡这话说得没错。只要是人,都会先为自己考虑,我们这些人类的复制品也是如此。不过无论人类也好人形也好,就算是最最自私的那一个,有时候也会为别人考虑一次。所以千万别对所有人都绝望啊。”

“海棠?”

“如果成功了,你一定要好好活着,不用想着报仇的事情,没有必要。我活了三轮实验了,最后还有好姐妹们相送,我很知足。”海棠笑了笑,“我们这些实验体啊,不过是用完就丢的工具,实验结束了,就连痕迹都不会留下。所以只有活下去,才是我们曾经存在的证明,明白吗?那就这么说定了哦。”

“海棠!!”

……砰。

没等46再说出什么,海棠已经抽出怀里的手枪,对着自己的下巴开了一枪。子弹从下颌向上贯穿了她的头,掀飞了一大块头盖骨,脑浆洒了46一身。

“……”

46看着海棠倒地的躯体,呆呆地愣在原地,甚至忘记了逃亡的事情。一直到帕斯卡走了过来都没有回过神。

“战斗终于结束了呢。”帕斯卡站在46身边轻声说,“该说是冥冥中自有天意吗?你竟然真的是最后的生存者。”

“不,这恰恰证明了这世上根本没有什么天意。”46咬着牙,冷冷地说道,“不然的话,死在这里的,就不该是这些无辜的孩子,而是你这个残暴的屠夫。”

“随你怎么说吧。”帕斯卡说,“不过既然是牺牲了这么多人之后才活了下来,你就更该珍惜自己这条来之不易的小命,不是吗。检查一下这些尸体。”

46没有再和帕斯卡说话,她觉得已经没有什么可说的了。能不能逃离这里,她已经不在乎,帕斯卡要拿她怎样她也不在乎了。因为无论外面还是这里,她都只剩下孤独的一个人了。

帕斯卡身后的两名士兵走了过来,依次查看着四周七零八落的破碎尸身。当他们检查到海棠的尸体的时候,忽然发现了一丝的异样。

“这具尸体,好像在嗡嗡响?”一个士兵嘟囔着说。

“别胡说,死都死了,这些东西又不是电动的……”另一个说道,弯腰去摸索着海棠那已经没有了生命的身体。

“不好,”帕斯卡惊叫道,“快趴……”

轰!!

当46回过神的时候,已经仰面朝天躺在了地上。泥土夹杂着血肉噼里啪啦地落在她的脸上,分不清到底是海棠还是那两个士兵。而压在她身上的是帕斯卡——在爆炸的一瞬间,帕斯卡将46扑到在地,用身体掩护住了她。

“……帕斯卡?”46开口呼唤了一声。

“没事……还活着。”帕斯卡倒在地上虚弱地说道,“别看我是个技术人员,其实也有一点……对付爆炸物的经验呢……”

46坐起身,看着帕斯卡的脸。那张妩媚的脸上口鼻都在渗出鲜血,也许是被爆炸震伤了内脏。

“……为什么?”46说道,“为什么要救我?”

“本能反应罢了……”帕斯卡勉强挤出一个微笑,“人在生死关头……都会下意识地保护自己重要的人,只是出于本能……”

“你……”面对帕斯卡苍白的笑容,46不知该再说些什么。

“呐,山茶啊……”稍稍缓了一口气,帕斯卡再次说道。

“……嗯。”

“留下来……可以吗。”

“……”

“不做助理也好,什么都不做也行……留在我身边,可以吗……”

“……”

46没有回答。

她觉得帕斯卡简直是个恶魔,却又无法抑止地想起她帕斯卡之间,曾经的种种。

她知道帕斯卡是这个实验室的管理者,而她们这群“人”,只是这场实验中的实验体。如果从帕斯卡的角度来看,她做的事情并没有什么不对,因为这座试验场就是为了这种事情而存在的。

而且,无论帕斯卡对别人做了什么,有一件事是46不可否认的,那就是帕斯卡一直以来都对她关照有加、从来没有伤害过她。

46感到胸前一阵撕裂般的疼痛,她努力地咬着牙,才勉强控制住了想要夺眶而出的泪水。她明白这是帕斯卡最后的挽留,但她已经决定了不会回头——她再也不会相信人类这种动物了。

“最终还是……不行吗。”看到46脸上决然的表情,帕斯卡无力地笑了笑。她已经明白了46的决定。

“对不起。”

“……是吗。真可惜啊。”帕斯卡闭上眼,再次笑了笑,“那就快走吧……如果你再被抓住了,我可就会……改变主意了哟……”

“多谢一直以来的照顾。再见……不,永别了,帕斯卡……姐姐。”

一边说着,46摇晃着站了起来,微微欠身对着帕斯卡鞠了个躬。然后,她迈开踉跄的脚步,用尽全力朝着训练场的边缘奔去。



尾声

如果海棠的计划也是实验的一部分,那么这真的是最终考核,那些没有自我牺牲精神的人形,是无法通过的测试的。但讽刺的是,通过测试则意味着无法活下来。于是在牺牲和苟活之间,46选择了后者——那个时候,46已经知道海棠要做什么了,却眼看着海棠拔枪打碎了自己的头颅而没有阻止她。也许阻止也未必成功,但在那火光电石的一瞬,46遵从了深埋在她内心的本能。

关于海棠的逃亡计划,46始终有两个疑问:一、海棠到底是真的想要逃亡,还是仅仅在装腔作势骗取帕斯卡的信任;二、海棠到底是根据什么标准,选出了这几个“觉醒者”。经过反复的思索,46发现虽然帕斯卡全程都在监控她们,但海棠依然成功藏匿了许多细节,她在玻璃橱窗里完成许多了掩人耳目的把戏。而海棠选择这五个人的理由,似乎只有一个,那就是容易控制。海棠没有说谎,无论如何,她都认真地筹备了她的“计划”。但在最后一刻,她面对未知世界产生的恐惧,改变了一切,包括她和46的命运。

46在离开实验室之后,在华南的森林里游荡了四天才走找到了逃离路线——那是一条运送木材的单轨道铁路。她扒在火车上离开了森林、又沿着铁路来到了城市。她在城市中昼伏夜出、靠偷窃食物过活,悄然而仔细地观察着人类的生活。直到她觉得自己已经足以伪装成一个流浪者之后,她踏上了北上的逃亡之路。

四个月后, 46闯入了荒芜的北部战区,并在那里被格里芬的部队俘获。她在那里扮演了一段时间的侦察兵,然后在冬季初雪的那天,离开了战区。

现在,46终于找到了适合她落脚的地方——这个地方偏僻而动荡、到处都是来历不明的人类和人形、并且和她曾经居留过的地方都远隔千里、绝无被人发现之虞。她在这个地方经营着自己趁手的行当:看似闹市里一间不起眼的小小的花店,实则为庞大情报网的一个终端。

那间花店的名字,据说叫做“无名芳草”。



