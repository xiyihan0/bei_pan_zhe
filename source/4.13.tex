\chapter{背叛者(十三)}

\section*{前言}


当陆久决定背叛的时候,他知道自己背叛的不只是一个人,而是整个世界。所以,他已经做好了清除每一个挡他路的人的准备。


就连作者都没有想到,陆久竟然也是个翻脸不认人的人呢。

\lineseparator

陆久离开了安洁,向着他原本计划的方向而去。达成他的目的需要很多东西,而得到这些东西付出的代价,陆久已经不在乎了,既然做出决定,就没有回头的打算。

没人能想到,一个人的转变如此之快,竟然只是因为一个念头。经历了思想上的变革之后,现在的陆久目标明确、意志坚定,已经和以前那个随波逐流的蠢货截然不同。

就连陆久都不相信这一切是真的,他竟然还能成为这样一个人。但也许这才是真正的他自己。

正文在下一页。

\section*{}

陆久摸了摸自己的后脖子,因为他感觉脖子上有点冷。当然,他的脖子并不冷,因为他穿的防护服很保暖而且是封闭式,他也摸不到自己的皮肤。但他还是想摸,因为这种感觉就像是神经性的瘙痒感——虽多半是心理因素导致,但不去摸一摸就会一直难受。

这种感觉从陆久转背向叶戈尔的那一刻起就有了,而且持续了好久。陆久知道叶戈尔一定举起了枪,他有种强烈的预感,只要自己回头看或者做什么多余的动作,叶戈尔一定会开枪。但陆久只能在心理默默祈祷那些祝福自己好运气的话能在此时应验,因为叶戈尔要取他的小命太容易了,而他没有任何反抗或者逃脱的可能。

幸运的是,陆久真的没吃子弹活着离开了。

陆久猜测安洁应该是成功撤退了,因为他没有听到爆炸声——虽然就算是引爆炸弹,也未必能把大楼炸塌,但拼个鱼死网破的时候安洁一定会这么做的。但她没有引爆,那么说明“鱼”还没死。

安洁让那两个人形在二楼三楼装上没有炸药的引信,是为了虚张声势,让叶戈尔的人探测到大量的爆炸物信号。叶戈尔看似穷凶极恶,但他还没有完全失去理智,因此,安洁把赌注押在了叶戈尔不能承受更多的人员伤亡上。她是想把叶戈尔的人都骗进大楼,然后再用引爆大楼来威胁他……说不定,还吹了个大楼里有脏弹的牛皮。这一计能骗过被复仇蒙蔽双眼的叶戈尔,但陆久却看得很明白,他也知道自己该怎样配合安洁。他那番临别的“劝诫”,正是给叶戈尔的心理暗示。

人在心理上的弱点是最容易被利用的,陆久心想,安洁这一定很了解叶戈尔怕什么——这个叶戈尔在某些方面,和自己很相似。陆久知道自己其实还不如叶戈尔,叶戈尔至少有为了胜利血流成河的决心,而自己则会在胜利和流血之间反复权衡。

想到这里,陆久自嘲地笑了笑。据别人所说,过去的他绝不是优柔寡断的人,那个“阿虎”杀起人来可是连眼都不会眨,而此刻的他,却在为了儿女情长而四处奔走。陆久知道,真正把命运掌握在手中的人,是要把大局置于一切之上的。比起这个乱世上真正的枭雄,自己差之甚远了,他不过是有点战斗经验、会在战场上耍点小把戏罢了。但陆久却觉得这样也不错,当他不再扮演那个被他人强加的角色时,他心中一直以来的压抑和绝望也随之渐渐消失了。

陆久和安洁分别的地方离安全区域本来已经很近,可惜他要去的是城市的另一端。靠着身上的防辐射服,陆久直接穿过了辐射较高的区域,回到了和安洁初次会面的地点附近,因为他把自己的摩托车藏在了这里。骑上摩托车,陆久向着原爆点疾行而去,路上他遇到了不少感染者,这些悲惨的家伙有些是士兵、有些是当地没有逃离的居民。它们发觉了陆久,便呜咽嘶吼着向陆久袭来,但轻便迅捷的摩托车很快就把它们甩在了身后。

但陆久没能到达原爆点,因为在接近原爆点的路上探测辐强度的指示器发出了警报,这是之前没有过的情况。液晶屏幕上面没有单位的数字陆久读不懂,但当屏幕发出红色光芒时,那个指示器开始滴滴作响,在陆久又前进了一段之后,警报声终于连成了一个不间断的刺耳的声音。

虽然看不懂数字,但陆久也知道那个小玩意是在提醒他“再往前走,就离死不远了”。无奈,陆久只好停止了前进。他往回退了一段距离,然后找了个比较高的大楼。随着陆久的爬高,探测器的警报声渐渐停息了,这让他感觉稍稍安心了一点。但走上楼顶,对着自己的目的地稍稍瞭望了一阵后,陆久的心情再次沉重了起来。

SOG小组曾经拦在AR小队和军方之间,那么她们出现的位置应该就是军方旧驻地的附近。根据安洁提供的地图,那个位置离陆久现在所处的方位仅有三十公里。但随着安洁的四处逃窜和军方的不停追击,现在军方的主力已经转移到了城市外缘,将SOG小组离开城市的路阻断了。

陆久在高高的楼顶望向远方,所见之处尽是废墟,坍缩液炸弹摧毁了城市差不多三分之一的面积。想穿过那片区域是不可能的,且不论道路是否能够通行,那边的辐射比这里更不知要强多少倍,即使有防护服依然会要了他的命。而另一侧则驻扎着军方的主力,同样是过不去也绕不开。

陆久要去的地方,在原爆点的辐射带和军方的营地之间。辐射的强度人类是必然承受不住的,军方的力量他也无以抗衡。就如416所说,想进入那片区域,除非长出翅膀来飞进去。那么陆久只有一个选择,那就是找个地方躲起来,等军方转移之后再去碰碰运气。

陆久盘点了一下自己身上的储备:口粮是两天份的,但在高寒地区没法省吃俭用,不吃饱晚上就会冻死。而水也不多了,最多坚持三天。而军方什么时候会移动、会向哪里移动,陆久无法预测,他同样也不知道V到底在哪。他更不知道潜伏这几天中会遇到什么事情,如果被感染者包围或者攻击,他甚至没有自保的能力。这个选择,希望实在是太微茫了,陆久不相信自己会有那么多的好运。

那么是否有人可以求助呢?格里芬是不可能的,且不说自己正被格里芬通缉,和军方周旋他们已经是精疲力竭;帕斯卡就更不必说,陆久倒是不在乎厚着脸皮去求她,但从营救安洁这件事来看,帕斯卡手里没有能够对付军方的筹码。安洁也一样,她能逃命就已经是万分侥幸了。

陆久掀起面罩,深吸了一口气。极北之地的冬天清晨,空气冰冷得刺骨、而且还漂浮着许多辐射尘,但陆久已经顾不了这些。他的心中焦灼如焚,因为他意识到自己已经走进了死胡同。他打开通讯器,试着接通V之前用的频段,却只收到一片沙沙声,不知是V换了频段,还是单纯地因为辐射干扰太强了。

为什么总是这样?陆久很想问问什么人,但他身边没有人让他问。他知道即便问了,也不会没有人能回答。

人的力量终究是有极限的,想到这里,陆久叹了口气。在最后分别的时候,他对V说必要的时候他会派人去支援,但如今看来这支援是不会到达了。

这样的结局也没有什么可抱怨,自己已经尽力,但现实却是无法逆转的绝境。枉费皮尔斯几次为他们牵线,这条线如今却像是飘摇的蛛丝,行将消失于风中。

对了……皮尔斯。陆久忽然想起了点什么。

他倒不是指望皮尔斯会帮他,因为要调动皮尔斯的运输机非格里芬出面不可,但以陆久现在的身份,要是让格里芬知道皮尔斯假公济私地帮他,那可不是闹着玩的。陆久是想起皮尔斯进而想起了一样东西,而有了那样东西,他说不定真的能如416说的那样“飞进去”……

陆久曾经长期“租用”皮尔斯的一架私人飞机,如果出动那架飞机的话,是不需要开格里芬的发票的。只不过陆久要去的地方可没有降落用的跑道,这一点皮尔斯绝对不会不明白。虽然皮尔斯帮了陆久许多次,但那架飞机是皮尔斯的心爱之物,纵然陆久把V当做自己最亲密的人,但战术人形终究不是传统意义上的人类。如果皮尔斯并不认可陆久的观念,那么他会把飞机交给自己吗?陆久不敢肯定。

但陆久还是决定去见一见皮尔斯,因为这是他能抓到的最后一根稻草了。

\section*{}

陆久再次折回,向着城市之外格里芬的驻地而去。这一路上费了不少周折,好几次走进了死路,险些被蜂拥而至的感染者堵住。当陆久终于钻出城市的时候,已经将近中午。陆久感到一阵头晕、还有点恶心,他感觉自己的血压有些低。

是因为从昨晚到现在一直都在进行高强度的运动,但却一口东西都没吃吧,陆久心想。也许是,但原因肯定不止如此。他昨晚大概有三个小时的时间,没有任何防护地暴露在辐射环境中,身体必然受到了一定程度的伤害。他感到眩晕和恶心,很可能是因为放射病的初期症状贫血。

希望能多撑一阵,至少撑到把SOG小组解救出来……不,陆久心想,不仅如此。把SOG小组解救出来的希望不是那么大,因为想要见到皮尔斯并得到他的支持,本身就面临着重重阻碍;而SOG小组究竟在哪、所处的环境如何还是未知的,要去那片区域来个杀进杀出,简直难于登天。但陆久想要的比这更加奢侈,他希望自己能健康、完整地见到V,然后带着那个女孩逃出生天。否则的话,他做的一切就没有意义了。

陆久吃了一些军用口粮,他这是第一次对那东西感到满意,因为那难以下咽的味道让陆久感觉自己还活着。接着,陆久喝了几口全靠自己的体温加热才没有冻成冰块的冷水,然后又再次向着皮尔斯的营地而去。

陆久很了解格里芬军事基地的结构:南部军团指营地、北部军团营地和中心指挥部虽然各自分开,但总体上还算处于同一区域,而皮尔斯所在的空中勤务基地则是在一段距离之外的独立区域。空勤基地是有自己的守卫部队的,但格里芬也派遣了一些部队共同承担防御任务,这些格里芬的部队属于中心指挥部。但在格里芬开始撤离后,这些防卫部队到底是什么状态,陆久无法确定。他不想和这些士兵对峙,因此希望防卫部队已经撤了。但这也未必是好事,因为皮尔斯处的防御薄弱,也可能会导致一些危险的情况。而最坏的情况则是空勤基地已经先于南部军团撤离……那陆久可真要叫天天不应、叫地地不灵了。

但当陆久怀着忐忑的心情来到空勤基地时,发现情况和他之前所想到的都不同——防守部队不仅没有撤离,反倒是增加了许多,甚至在空勤基地外围建立了另外的营地。而营地里的士兵,根据陆久观察,不是中央指挥部的警卫团、也不是北部军团的士兵。

这比陆久之前想过的所有不利情况加起来还要糟:挡在他和空勤基地之间的,是南部军团的临时指挥所。

陆久默默地看了那片戒备森严的基地一阵,还是硬着头皮来到了岗哨前。如果守卫没有一见面就请他吃枪子儿,那么他也许还有一丝斡旋的机会。

“什么人?停止前进!”岗哨里执勤的人形警戒地端起了枪,指向陆久说道,“表明身份,否则我就要开枪了!”

我还真是个有的身份的人,陆久心想,你要是知道了准会大吃一惊。

“我是北部军团的军官。”陆久说,“我要去里面的空中勤务基地。我不记得这需要你们的许可。”

“因为特殊原因,这里的保卫工作已经被我们接管。”战术人形哨兵说,“你是北部军团的军官,为什么没有穿制服?”

“你无权询问这个问题。”陆久装腔作势地说。

“如果是那样……我需要核实您的身份。”哨兵见陆久态度强硬,语气软了一些。

“好吧,快点。”

哨兵用扫描仪照了一下陆久的脸,他的信息立刻显示了出来:

姓名:陆久

单位:北部军团指挥部

职务:军团总指挥官

军衔:上校

一如既往,陆久的近照也投射了出来,而且是全息影像。

……军衔竟然晋升了啊,陆久心想。想必是军士长的军衔和战区司令太不匹配,所以他才跃级升成了高级军官。鸡犬升天,何其讽刺。

不过看来格里芬这次真的是仓皇撤离,数据库中的资料都没来得及更新,自己竟然还是在职人员。

“长官好!”哨兵立即挺身向陆久敬礼,“欢迎莅临南部军团临时指挥部。请问有何指示,我可否代为传达?”

“不用劳烦,我要去空勤基地。”陆久抬手回了个礼,“我可以走了吧。”

“这……还请您暂且留步。”哨兵再次拦住了想要蒙混过去的陆久,“本岗哨目前仅允许南部战区人员通行,我必须向上级请示后才能将您放行。请稍候,我马上汇报。”

陆久看着那个哨兵,心里叹了口气,知道再多说也没用了。如果是人类士兵,那么现在绝对不会再阻拦陆久,但战术人形对岗位职责的忠诚是到了死板的地步的。这忠诚曾经正是陆久对她们欣赏的地方,但此刻却成了阻碍陆久的东西。这就是命运啊,陆久暗想。

哨兵汇报完毕,告诉陆久执勤的军官马上就到。过了没有五分钟,一辆越野车停在了岗哨旁边,上面下来三个人。

陆久定睛一看,无奈地笑了笑。他笑的是带头的人他恰好认识、而且还有点交情;而无奈的是这次他没法再蒙混了,而凭他的这点交情,不知是否足够让那个人放他一马。

“你好,陆司令。”为首的人走到陆久面前开口说道,“没想到会在这里见到你……应该说,我没想到还会再见到你。”

“我也没想到。”陆久说,“你为什么会在南部军团的营地里?”

“北部军团撤离之后,剩余的人员都暂归南部军团指挥了。而南部军团的警卫团,现在都归佩瑞特少校指挥。”

陆久面前的,是一个身材高挑而丰满的银发战术少女。她正是陆久旧日手下的副官,名为 PK的战术人形。

“久违了,PK。我要去空勤基地,事态紧急,没时间叙旧了。让我过去。”陆久直截了当地说,他没功夫和PK闲聊了。

“这不可能。你现在是被格里芬通缉的人,我本应该把你抓起来。”PK压低声音说,“请马上离开,看在曾经的情面上,我可以假装你没有来过。快走吧!”

“Vector和她的SOG小组被困在了辐射区,我现在需要里面那位管飞机的准将先生的帮助。”陆久也沉声说,“我会不动声色地过去,没人会知道这件事的,拜托你了。”

“不行。这个营地是通向空勤基地的唯一通路,你只要进去,毫无疑问上级会知道是我放你进去的。”

“PK,被困在那个地狱一样的地方的是V,是我的……伙伴。”陆久咬牙说道,“相信你没有忘记自己被困于敌阵时,是佩瑞特委托我,把你救了出来。如果你能理解那时佩瑞特营救你的心情,你就该体会我此刻的心情!”

“正因蒙你搭救,我这次才让你离开……!”PK涨红了脸,“我没有忘记你的恩情。如果只是我的话,我一定报答你,无论是接受怎样的处分都没关系……但这样做会牵连到佩瑞特。所以,我不能放你过去,请原谅!”

“V现在可能已经命悬一线,我没时间和你废话,孩子。”陆久冷冷地说道,“我必须从这里过去,你阻拦不了我,任何人都不行。先礼后兵,我希望你能听进去好话,但也准备好了动武。”

“如果你能做到的话……倒不如说,我希望你能做到。”PK说,“但我不会手下留情的。”

陆久拔出手枪对准了PK,PK也拔枪对准了陆久。PK身后两个不明情况的战术少女见状,立即也将武器对准了陆久。陆久知道自己不是对手,论正面战斗,就算五个他也不是这几个战术少女的对手。但PK毫不退让,而陆久又必须去空勤基地找皮尔斯,不拼上老命,这道关无论如何都过不去了。

“给我住手,把武器放下!”

正当两个人剑拔弩张、冲突一触即发的时候,忽然一个声音高声喊道。在不远处说话的,正是佩瑞特。

“你在用枪指着谁,PK?”佩瑞特对着PK严厉地呵斥道,“你不认识自己面前的人吗?还不退下!”

“……我当然知道,他是格里芬第一号的通缉犯。”PK咬着嘴唇说,没有后退、也没有放下枪。

“放肆!陆司令是什么样的人,难道我没有告诉过你?他是我曾经的长官、也是你的恩人!你怎么能用枪指着曾经救过你命的人?!”

“可是——”

“没有什么可是。我们的枪口,不该对着支持和帮助过我们的人!”佩瑞特走了过来,把手按在PK持枪的手上,“他不是什么通缉犯,只是个想去见自己爱人一面的男人。把枪放下。”

PK犹豫了一下,终于放下了枪。陆久于是也把枪放了下来。

“我听说您的事情了。但是很抱歉,由于立场原因,我帮不了您什么。”佩瑞特对陆久说,“您如果想要过去,我不会让任何人阻拦您。我能做的只有这些了。”

“我知道,没关系。”陆久说,“不用责怪PK,她只是在尽自己的职责。她说得没错,我是个被通缉的人,我们现在已经不是同袍了。我这幅狼狈的样子,如果还自称是士兵,那实在是丢军人的脸。”

“让军人的名节蒙羞的,是我们这些像傀儡一样,不知在为何而战斗的人。”佩瑞特摇了摇头,“就算是现在,我也很敬仰您为了重要的人而不顾一切的勇气。我所做过的最英雄主义的梦,都是您践行的道路……而我却发现自己此刻除了叹服以外,只能袖手旁观。这让我感到羞耻。”

“你能和PK在一起,我其实也很羡慕,我只是没有那样的幸运罢了。”陆久说,“你没有必要愧疚,我也不希望任何人遭遇同样的痛苦。你能珍惜自己拥有的东西就很好。”

“我会珍惜的。”佩瑞特说,“我也会祝福您能找到V小姐,然后平安归来。”

“愿能借你吉言。”陆久笑了笑说。

\section*{}

穿过南部战区的临时指挥所,陆久很快来到了空中勤务基地的大门前。这次他进门的时候就连问都没人问,看来里面的人已经把警卫任务完全交给了格里芬。

进入基地,陆久直接去了皮尔斯的办公室。办公室没锁门,但皮尔斯和因菲尔德都不在,这让陆久有些吃惊。陆久在屋里停留了片刻,决定直接去开飞机,因为他不知道皮尔斯到底去哪了、什么时候才回来,他没时间再等了。

陆久来到五号机库,那架A-12“雷霆”攻击机就在库房里静静地停着。这架飞机的速度不算快,但火力十分凶猛,可惜在退役之后因为没有弹药补充,沦为了陆久的座驾,标志性的730机炮也被罩上了整流罩。

陆久深吸了一口气,朝着飞机走去。但在他走到飞机下面之前,听到有人叫了他的名字:

“陆久。”

陆久回头,看到呼唤他的正是皮尔斯本尊。皮尔斯一手拿着半瓶威士忌、一手夹着雪茄,正靠在机库的门后不知在做什么。

“我去你的办公室了,但是没找到你,所以就直接来这里了。”陆久说。

“我懂。你就是个直奔主题的人。”皮尔斯笑了笑,然后扬了扬手里的酒瓶,“来点吗?”

“不了,谢谢。酒后驾驶是违反交通法规的。”陆久说。

“没人会去天上抓你。据说毛子飞行员们上天前,都喜欢先喝几口,这样躲起导弹来更溜。真不喝吗?这是我从大不列颠带来的最后半瓶了。”

“不。我没时间喝酒了。”

“难得来一趟,干嘛那么着急?咱哥俩以后怕也是没多少相聚的日子了,趁着现在好好聊聊,以免以后没了念想啊。”

皮尔斯说话有些唠叨,陆久感觉他已经喝了不少了。但陆久也不能确定皮尔斯到底是喝高了,还是确实想和陆久谈谈,因为上次见面的时候皮尔斯似乎遇到了一些问题。但此时陆久心中正如水煮,实在无心和皮尔斯闲聊。

“我直接说吧,皮尔斯。我要用这架飞机。”陆久说,“我那出了点情况,如果你还不知道,我怕也没时间和你细说了。我赶时间。”

“我知道,知道得很详细。所以你不用解释。”皮尔斯说,“我想和你谈谈是因为,你不能开走这架飞机。听着,不是因为我舍不得借给你。”

陆久看着皮尔斯,然后叹了口气。“不是舍不得”,这句话陆久喜欢听。但皮尔斯先说的是“不能开走这架飞机”,那么无论原因为何,这都是个坏消息。

“你如果了解了情况,就该知道我是去干什么。”陆久说。

“我知道。我担心的是,你自己不知道自己是去干什么。”皮尔斯说。

“以什么标准衡量,我会不知道自己是去干什么?”

皮尔斯看了陆久一阵,也叹了口气。

“好吧,老陆。”皮尔斯说,“大道理我不讲再了,因为克老板一定已经讲过了。总之,我认为你不能对他说的那些充耳不闻。”

陆久瞪大眼睛看着皮尔斯,不知道是该生气还是该发笑。他简直不敢相信自己的耳朵。他揣测了几十个皮尔斯不让他开走飞机的理由,唯独没有想到是因为“大道理”。陆久还以为皮尔斯不是个会执着于这些刻板的东西的人呢,但他忽然意识到自己错了。仔细回想起来,皮尔斯其实没有做过任何叛经离道的事情,更从来没有违反过命令。他看似风流倜傥,但却并非玩世不恭,反而是个循规蹈矩的人。

“你的意思是说,我应该为这个操它个妈的世界考虑考虑?”陆久冷冷地说道。

“你看,你别这么开口就怒火冲天的。”皮尔斯说,“我知道你不喜欢大道理,我也不喜欢。但大道理总是有道理的,你必须考虑考虑。”

说道最后一句的时候皮尔斯加重了语气,重音落在了“必须”上,显然是在暗示陆久不考虑就会有后果。

“当我手下的姑娘为了获取‘伞’病毒样本而自决的时候,谁为她们考虑过?”陆久说,“当他们派Vector带着仅仅三五个人去吸引军方火力、为AR小队争取撤离时间、然后像垃圾一样被丢弃在那里的时候,谁他妈为她们考虑过!?”

“冷静点,兄弟。”皮尔斯淡淡地说道,“我不想说这句话,但你得知道——她们是战术人形,不是人类。”

陆久看了皮尔斯一阵,胸中的怒火突然消失了。他感觉自己的心犹如白热的金属被丢进了冰湖,还没沉到底,就已经冷透。陆久终于意识到了他和皮尔斯之间的隔阂,虽然他一直装作视而不见,但那沟壑之大就连意识形态之争都相形见绌——把战术人形当做人来对待的只有他陆久一个人,而他竟然还一厢情愿地以为皮尔斯会理解他。一直掩耳盗铃到现在,这是何其的可笑。

“皮尔斯。无论如何,即使她们的人全都是假的,但我心中的情感是真的,这一点你得相信。”陆久说。

“我相信,而且我理解。”皮尔斯说,“我也曾遭遇过类似的困境,我的副官也面临着强制退役的命运。但我们得看得更远一点——”

“你理解个屁!!”陆久愤怒地说道,“你生来就在优越的环境中,不曾经历过失去,怎么可能理解我的感受?”

“哈哈哈哈哈……你说得没错,我这样的纨绔子弟,怎么会在乎区区一个人偶?就算是真正的女人,我也要多少有多少!”皮尔斯大笑了一声说,“但我的朋友可是不多,我是真心为你好。克老板可是放话了,只要我搞定你,不管是活着还是死掉,他都会向我父亲说情,恢复我的飞行。但我没有在你一踏进我的基地就干掉你,难道我还不够意思?”

“克鲁格已经被捕了。”

“他不会永远呆在大牢里的,这一点我比你清楚。别说他了,陆久,好好想想我说的话!”皮尔斯说,“我知道你干了点什么。好好想一想。就因为你把战术人形的‘作战训练’内容传播了出去,现在外面的风暴已经开始酝酿了,许多人都对战术人形产生了抵制情绪、一些原本在暗地里活动的人形权益组织,也得到了舆论武器,并开始登堂入室。民用人形服务人类差不多四十年了,这本已是一个趋于成熟的产业,但你却动摇了它的根基。你可能没有意识到,为什么军方忽然缩了回去,那是因为这里的战斗吸引了全世界的眼球,一场大变革一触即发。在你所不知道的地方,许多人都在悄悄盯着这里,那些人就是潜伏在各人形制造公司、私人军事公司、保全公司里的人形同情者们。克鲁格说得没错,你不顾自己的小命去救一个民用人形,不仅会给其他指挥官带来极大的压力,而且会破坏人类相对于人形的绝对优先权。你一旦做出这样的榜样,将会引发全人类社会的巨大动荡!”

“我知道。”陆久平静地说,“要不然我怎么浑水摸鱼呢。我这招还是拜你所赐,这不就是你所说的那一招吗?我已经‘把水搅浑’了。”

“你把水搅浑了,而且是把大家的水全都搅浑了,但多数人不想蹚这塘浑水。”皮尔斯说,“冷静下来,陆久,再好好想想。你这样做,所有人都会知道你是始作俑者了。会有一些人从这件事里受益,但将有许多人因此而失去自己原本的生活。那些受益的人不会感谢你,但那些受害的人绝对会视你为仇敌!你觉得到时候,你还能优哉游哉地生活下去吗?”

“我是靠枪活到今天的,我从来没有指望自己能得善终。”陆久说,“这个世界的战争从来没有停息过,因为人类的就是如此,自从树下下来就一直相互争斗不休。帕斯卡在谋划着她的弥天阴谋、克鲁格想实现他的千秋大业,你当然不会做个苟且偷生的普通人,我毫不意外。这个乱世之中,一个普通人或者一个人形的生死,没人在乎。但那些只想安静地度过一生的人的命运,就一文不值了吗?我并没有太多要求,只是想和自己喜欢的姑娘在一起,过普通人那样平淡的生活。但我知道这不可能,我这样的人是不配拥有那种生活的,Vector大概也不行,民用人形的生存环境每个人都清楚。无论我被怎样对待,我都可以忍受,可Vector没有理由一生都被人利用、到最后还被当做炮灰丢弃。她是我陆久喜欢的姑娘,这对你们来说——格里芬也好,帕斯卡、克鲁格也好,都不是什么值得在意的事。克鲁格甚至就在我的面前把Vector丢进了火坑,因为在他心中Vector只是一件耗材,她的生死,就和我的意愿一样,无足轻重。这让我想明白了一件事:我没有义务为这个世界负责。如果我必须把这世界搅个天翻地覆才能趁机救出Vector,那我很乐意这样做。凭什么你们这些人可以高高在上、旁若无人地追求自己的理想、抱负、功名利禄,我们却要像玩偶一样被操纵、被践踏,最后连一丝怜悯都祈求不到呢?自己看看吧,皮尔斯——沙文主义、唯我独尊,你们这些‘现代’的人类都已经变成什么样了。这自私卑劣的物种如果还不反省一下自己所作所为,那还有什么存在的价值?只为了荼毒这个世界吗?”

“就算你这么说,统治这个世界的,依然是你和我这样的人类。”皮尔斯说着把手里的酒瓶再次递向陆久,“不管民用人形还是什么,那些事我们可以再想办法解决,何必要为了它们而和同族反目呢。来,喝一口,就像我们从前那样,怎么样?”

但陆久却一把推开了皮尔斯的手,他手里酒瓶险些滑掉。

“你不喝就算了,别浪费了这好酒。”皮尔斯把剩下的酒一口气全喝了下去,然后擦了一下嘴角说道,“对我来说,好酒就和好朋友一样,都已经不可多得了。”

“我不知道你是喝醉了还是怎么回事,不过看来我们对彼此的理解,从一开始就只是一厢情愿。”陆久说,“我们不是同一个世界的人,我以前还一直以为这句话只是一种修辞方式呢。既然我们无法达成共识,我也不想再浪费时间和你说这些废话了。”

说完,陆久大步朝着战斗机走去。

砰!!

陆久停下了脚步,因为他感到了耳鸣。一把.357口径的柯尔特“蟒”式左轮手枪,在距离他的脸庞两三步远的地方开了火,枪口喷出的火焰烧焦了他的鬓角。

“后退。”皮尔斯用手枪指着陆久说道,“继续一意孤行,你不仅是死路一条,而且会身败名裂、万劫不复。那样的话,我还不如让你死得体面一点。”

“皮尔斯。”陆久用低沉的声音说道,“不要这样。”

“你是在向我求情,还是向我手里的家伙求情?”皮尔斯嘲弄地说道,“你该知道这东西是干什么的吧。它不就是因为人们厌倦了说教,才被发明出来的吗?”

“唉,你说得对。这也是没办法的事。”陆久叹了口气,然后笑了笑,“知道吗,皮尔斯,我一直都非常感激你把Vector带到了我的身边。你帮了我很多次,无论如何,我都得谢谢你。你是我真正的朋友。”

“不客气。都是我该做——”

皮尔斯突然向后退了一步,因为他感觉有人推了自己一下。这让他感到十分惊奇。

陆久的手枪,明明还插在胸前的口袋里。皮尔斯很警惕,一直时刻注意着陆久的手,防备着他去胸前拔枪。但陆久并没有将手向那里伸。

陆久只是——

那只垂下的右手,只是做了一个非常不显眼的动作:揸开手掌,用大拇指勾了勾自己的衣摆。接着,皮尔斯的胸前中了三枪。

他甚至没有听到枪声。

皮尔斯倒在了地上,感到一切就像是幻觉。他挣扎着想爬起来,却全身都使不上力气;他试着抬手举起手里的枪,但却无论怎么都抬不起胳膊。

皮尔斯躺在地上,不解地看向陆久,过了片刻才看清他手里那把P7袖珍手枪。

……毒蝎尾后针。

皮尔斯自嘲地笑了,他明白了陆久的枪是从哪来的,那一定是藏在背后的应急武器。他记得克鲁格提醒过自己“小心背后”,现在想想克鲁格说的小心背后,不是指皮尔斯的背后,而是指陆久的背后——作为“阿虎”的老战友,克鲁格一定知道,陆久会在背后藏着手枪。

陆久走过皮尔斯身边的时候,踢开了他的左轮手枪。皮尔斯没有试着去拿,因为他知道自己如果乱动,陆久很可能会给他再补几枪。所以他只是咬紧牙关,保持住了最后的意识。

当陆久登上飞机之后,皮尔斯才掏出自己的手机,用残存的知觉拨了一个号码——

在那架A-12e战机的油箱里,藏着两公斤的TNT炸药,并安装了遥控引信。只要皮尔斯拨通电话就能引爆,整架飞机都会被炸成碎片。

但皮尔斯终究还是没有按下拨号键,因为他舍不得自己的飞机。

\section*{}
\paragraph*{作者按:}\mbox{}\\

这一部分最初的构思,本来是陆久和皮尔斯大战了一场,然后皮尔斯被陆久重伤而死的。为的是衬托出陆久吃了秤砣铁了心之后,到底有多么无情无义。而且里面还有那种“为了友人终于醒悟而怀着欣慰的微笑死去”的烂俗情节。

不过回头想想,皮尔斯就这么死了,实在是太不值得了。皮尔斯是个好人,他应该有一个更好的结局。虽然不是太好,但也该不是这么一个纯工具人。因此后来在写皮尔斯的时候,给他加入了大量的个人剧情,然后铺垫出了一个另外的结局。

虽然皮尔斯没有死成有点遗憾(?),但我还是对这样的安排感到满意的。纵观全篇,我没有为了制造话题而故意写死某个角色,也算守住了自己“讲一个温暖人心的故事”的初心。

虽然这个故事本身也没什么话题可言就是了……