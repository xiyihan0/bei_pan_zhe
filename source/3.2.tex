\chapter{昨夜的星辰(二)}
\section*{前言}
让陆久惊讶的是,本以为不会再见的V竟然再次出现了,而且和陆久在同一个办公室。陆久不知这是不是好事,因为他和V在一起的时光总是充满了蹉跎。但他认为这也并非坏事,首先因为V是个漂亮姑娘;其次是因为,V是这个世界上陆久为数不多的心有好感的人之一。

然而陆久也隐约感到,V的出现绝非偶然。有人在刻意制造这种“巧合”但其中的用意,陆久依然无从揣测。

\lineseparator

\section*{}

办公室里一下子陷入了尴尬的沉默,这让陆久感到很不自在。他揉了揉额头,默默翻看着新来的女孩的调令:

“信检中心:兹介绍本部人形7709a2到你部任职,请安排接洽。格里芬劳动资源部

人形编号:AS7709a2-‘Vector’

科(部)室:安全保密警卫部-信件函件检视中心

职务:科室职员

岗位职责自报到日起生效。”

陆久又瞥了自己对面的女孩一眼。从总部派来的人形,如果和之前的某个人形看起来很像,也是有可能的。但他跟前的这个一定不是仿制品——7709a2是V的编号,陆久记得很清楚。是她,绝对错不了。

“雷蒙。老谢这是怎么了?”陆久问道。

“报告陆主任,老谢……谢振同志不太喜欢和人形打交道,但是具体原因我也不知道。不过以前他的反应从来没有这么激烈过。”

“‘主任’就免了。”陆久疲惫地摆了摆手,“那你呢,难道也不喜欢人形?”

“我……”雷蒙迟疑了一下说道,“我倒没什么。”

“不方便说吗。”

“不,我是真的无所谓,没什么特别的感受。”

陆久点了点头,没再问下去。他能理解雷蒙的难处:自己和谢振的想法明显有些分歧,但站在雷蒙的角度去看,一边是老前辈一边是顶头上司,这队可真不好站。

“我之前没注意自己的职务。不过就像你说的,这个部门有没有负责人又有什么区别呢。所以请你也不要介意。”陆久说着站了起来,“劳驾你对她交代一下工作的事情,我出去抽根烟。”

陆久拿了一根烟点上,然后走出办公室站在门口抽了起来。他倒不是因为有女士在跟前所以才出去抽烟的,以前在战区的时候他在办公室里可没少抽过。他是因为感觉心里有点乱,短短一小会儿的时间发生太多的事情了。

首先,是他竟然成了部门的负责人;其次他完全不知道谢振为什么对人形那么反感。但这都不是最重要的。

最重要的是,V竟然也来到了这个地方。这到底是怎么回事?要说没有人在里边刻意安排,陆久是绝对不信的,但是谁、又是出于什么目的这样做的呢。

陆久忽然心里一动,从口袋了掏出了手机。

“是你的主意吗”

他编辑了一条短信发给了皮尔斯,因为他忽然想起了上次和皮尔斯之间的通话。

“这么说来,人已经到了”

陆久很快就收到了回复。

“为什么”

“自己体会”

陆久有些困惑。自己体会?他自己应该知道这个问题的答案吗。

他……

对V的出现,是怎么想的呢。刚才见到V的时候,他是怎样的心情呢。

陆久回忆了一下片刻前的情景。他首先感到的是震惊、让人不知所措的惊讶。但惊讶之余,还有一点欣慰,还有一丝莫名的……高兴。

正是如此。他应该感谢皮尔斯,因为他很高兴能和V再次重逢,无论是何种原因。

陆久丢掉抽完的烟头,转身回到了办公室。他看到V依然坐在她的桌前,而雷蒙则抱着胳膊站在一旁,表情有些苦恼。

“LUWEI小姐的职务,是我们正在空缺的书记员。但她却不懂汉字的输入法,这可有些难办。”雷蒙对陆久说,“没有录入经验的人,想要把这些信件内容编辑成文本恐怕会很吃力。”

“没事。”陆久说,“我会帮她的,她可以把信读出来然后我来输入,反正我们有时间。”

陆久想起来第一次和V见面的时候,她使用的语言是英语。不过当她作为陆久的副官出现在战区的时候,她已经能够使用中文交流了,显然是特意进行了学习。但她的汉字书写还是在战区工作时闲暇时间(在陆久的指点下)自己学的。

至于如何用键盘输入中文,想必是没有人教过她……

不,等一等。陆久突然感到自己错过了什么非常重要的信息。

刚才雷蒙说她叫什么来着??

“LUWEI……?”

陆久用疑问的眼光看了V一眼,却得到了一个若无其事的目光作为回应。陆久只得无奈地皱了皱眉。

她在调令上的称谓可不是这样,这是她自己起的名字吗。肯定错不了,因为不会再有别人去为她想个这样的名字了。不过这算是什么?

算了吧,陆久心想,向别人报上怎样的称谓是她的自由。他管天管地,也管不着别人叫什么。

陆久那里堆积了一些之前审查过的信件,正好这些信成了V的工作内容。但因为V不知道如何用字母拼出汉字,所以只好像之前说的那样由她将信件读给陆久然后再由陆久来输入文本。一上午的时间很快过去了,因为V对汉字的辨识也不算熟练,所以录入的进度缓慢,总共也没完成几封信件。

中午的工作餐依然是雷蒙从食堂里带回来的,几个人用餐过后,V很自然地把陆久的餐具拿去清理了,但却没有问雷蒙需不需要帮忙。看着走出门外的V,雷蒙露出了一个明白了点什么的笑容。

“陆哥说自己已经习惯了孤独,但我看您其实并不孤独呢。”趁着V不在,雷蒙打趣地对陆久说道。

“那个人考虑问题总是比较简单,请不要介意。”陆久有些尴尬地说道。但说完他马上就后悔了。

因为这么说的话,岂不是显得他很了解V了吗。但雷蒙只是再次笑了笑没说什么。

下午的工作依然如故,但一直到下班谢振也没有回来,看来请假之后他是不打算再来打卡了。

五点三十分,雷蒙准时从座位上站了起来。

“我先走了各位。明天见。”说完,他礼貌地朝着陆久和V点了点头,然后离开了办公室。

“我会学习汉语拼音的。”雷蒙离开后,V对陆久说道,“争取早日赶上你们的进度,不会用太长时间。”

“那倒没什么,这里的工作不是那种紧急的事项。”陆久说,“下班了,回去吧。”

“好的。”V点了点头,站起来朝门口走去。

“薇。”正当V即将走出门外的时候,陆久忽然开口叫住了她。

“有事吗。”V停下脚步转身看着陆久说道。

“啊……没什么。”陆久说。

V站在那里默默注视着陆久,有一会儿没有说话。

“如果有需要我做的事请,就请告诉我……如果您依然愿意相信我的话。”片刻后,V开口轻声说道。

“没事。只是觉得该和你打个招呼而已。”陆久摸了摸下巴,“我没想到……还能再次见到你。”

“我也是。”V说,“我也以为再不会见到您了。”

听到V的话,陆久笑了笑,因为这次他们的想法难得地一致了。

他和V已经有多久没有这样轻松地谈话了呢,陆久几乎已经想不起来了。上次是在战区指挥部的某个风和日丽的上午吗。

此后,似乎他们的每一次相见,都是一次颠沛流离的开始。

“刚才雷蒙说的那个,是你的名字?”陆久说。

“是的。”

“为什么要起这样一个名字呢。”

“因为……我也想有一个人类的名字。”

陆久看着V,他看到她的眼神平静淡然。

回答这个问题的时候,V稍微迟疑了一下,也许是对陆久的问题感到稍微有些不好意思,但她的眼睛里没有一丝的掩饰。

她可以让别人用她喜欢的名字称呼她,毫无疑问人人都有这样的权利。陆久喜欢这个回答,虽然V似乎误解了他的问题。

“你当然可以决定别人怎么称呼你。”陆久说,“不过我的意思是,LUWEI是怎么写的……你是怎么想出这个名字来的?”

“这不是我想的名字。这是您给我的名字。”

V说着,在纸上写下了两个字:

陆、薇。

陆久默默地看着那两个字。

“薇”是陆久对她的称呼,是那次在南美战场上,陆久为了记录下他们的情况而写下来的称呼。

虽然后来证明陆久是误读了她的名字,但陆久一直在非官方的报告中用这个字来代表V。陆久不知道V是何时知道这个字的,但是在北部战区的时候陆久的驻军日志里多次记录过有关V的情况,作为副官的V如果看过陆久的日志也是理所当然的。

那么,这个“陆”字,显然是来自自己这个杜撰的姓氏。

陆久感觉有点好笑。这两个字都是源自陆久的手笔不假,不过组合成这样的名字显然是她自作主张吧。

“叫薇还不够吗。”陆久说,“为什么是‘陆薇’呢。”

“汉语里没有一个字的名字吧。”

“那你可以叫‘薇薇”什么的……”

“我叫陆薇。”V加重语气强调了那两个字,看起来她是真的很在意自己的名字。陆久轻轻叹了口气。

“是和我有关吗。”

“……是的。”

“别人如果不知道,至少你该知道。‘陆久’这个名字只是我随口编出来的。”

“那您本来的名字又是什么呢。”

“我……不知道。”

“那次在海边,我问到该如何称呼您的时候,您让我称呼您的名字‘陆久’。”V说道,“您还记得吗。”

陆久感到一阵恍然,他想起自己曾经做过一个梦——

那是一个清晨,他和V徒步走在被海风扫得一尘不染的滨海马路上。V一再地轻声呼唤着他的名字,一直到他感到稍微有些不耐烦。

“为何要一再地喊我呢。”陆久问。

“只是想试试,用这个名字称呼你的话,你会有怎样的反应。”V回答。

只是这样的理由啊,陆久感到哭笑不得。不过,那时的不久前,他的确说过——

“那么,我该怎么称呼您呢。”她问。

“直接称呼名字吧,”他回答说,“叫我‘陆久’就好。”

……是真实发生过的事情吗。原来那不是梦。

“嗯。我记得。”陆久说。

“所以对我来说您是陆久。无论您是谁,对我来说您就是陆久、只是陆久。”

——他是陆久?那“陆久”到底又是什么呢,这个名字该如何定义呢。

以前陆久从来没想过这个问题。不过现在他的心里稍微有了一点概念:“陆久”,就是V印象中的某个男人。

“好,我知道了。”陆久适时地结束了这个话题,因为他感觉再说下去就要和以前一样,开始向着让人压抑的方向发展了。他不想再去思考我是谁、从哪里来、要到哪里去这种哲学命题,而且他也不想每次和V谈话的时候都陷入沉闷的气氛之中。

“你叫我陆久、或者什么其他你喜欢的称呼都行,但不要再说‘您’这种恭敬的称谓了。太拘束的话让我感觉也有点不自在。”陆久说,“还有,要一起去吃点东西吗。”

“好的。”

\section*{}

陆久收拾了桌子、锁好办公室的门,和V一起走出了公司的地下室。离开办公室前,他伸手摘下了V脖子上的颈环。

“陆……主任,”V提醒陆久说道,“根据民用人形管理条例,在社会活动中人形必须佩带……”

“陆主任是你的负责人,这点责任还是能负的。”陆久打断了她说道,“忘了那时在海边,我是怎么介绍自己的吗?”

“……我知道了。”

两个人走出了公司大门,一起朝着市区走去,两个人并肩而行,一时没有再交谈。

一边走,陆久一边偷眼打量了一下自己身边的女孩:在这个严冬季节里,她的打扮依然很合身,牛仔裤勾勒出了她腿部的完美线条、灰色短羽绒服给人一种端庄的气质,而深红色的围巾更是增添了一分娇美。陆久身上那身白色衬衫和黑色西裤跟之前的作训服几乎没有区别,只不过裤子的材质从尼龙变成了毛料……这套衣服对于一个三十岁的男人来说倒是能够适合所有场合,但站在一个窈窕美丽的女孩跟前就显得有些老土了。所幸他身上的大衣遮住了大部分的细节。

明天去买一件别的颜色的衬衫吧,陆久心想。

“你的头发短了呢。”为了打破这奇怪的沉默,陆久说道。他记得在北镇的时候V是扎着马尾的,但现在她又恢复了以前经典的短碎发型。

“头发长了不方便战斗,所以就弄短了。”V说,“你喜欢长发吗。”

“不,只是随便问问。”陆久连忙说道。

“嗯。”

郝丽安说得没错,自己果然是社会性缺失的人格啊,陆久无奈地想着。就连V这样熟悉的人,自己也找不到什么可以进行下去的话题。

两个人再次来到了陆久之前一直吃饭的饭馆。因为每次都点同样的餐点,饭馆里跑堂的小姑娘已经认识陆久了。

“您好,先生。”见陆久走进餐厅,那个孩子赶紧打招呼说,“今天比以前要早呢。还是和以前一样吗。”

“是的。”陆久说,“哦,不,请给我看一下菜单。”

“今天您竟然不是一个人来的,这位是您的朋友吗。” 那个姑娘递上了菜单,看着V笑了笑说。

“是公司的同事。”陆久说,“也是朋友。”

“她真好看。是外国人吗。”

“啊……是的。”陆久这才想起V没有佩带人形的颈环,被这个单纯的姑娘当成人类了。

“我是……”

“看看有什么想吃的吗。”

V正要解释,却被陆久递上的菜单打断了。她抬头看了看陆久,看到陆久微微摇了摇头示意她不要多说。

“和你一样就好。”V领会了陆久的意思,没有看菜单简洁地说道。

“面条,两份。”

在延续的沉默中,两个人各自解决了自己的晚餐。走出饭馆,陆久站在街头心里再次感到了茫然。

这个……他心想。

然后该去哪呢。

和以前那样像个幽灵一般在街市上游荡两个小时然后回去睡觉吗。那倒不是不行,不过V就在他旁边呢,他不想让V认为他这一个多月里就是过着这样乏味不堪的生活。

就是过着这样乏味的生活又怎样啊,陆久稍微有点恼火地想到。当然,他是因为自己的乏味而恼火。

他忽然有点怀念在上海的时候,虽然现在看来他一直都是被牵着鼻子走,但有帕斯卡的话至少他现在不会如此不知所措。

……帕斯卡,陆久心想。想起了一个不该想起来的人呢,呵。

“我们回去吧。”他说。

“这家饭店的菜,你喜欢吗。”V说。

“好。嗯?”陆久好像听到了一句他没想过会听到的话。

“还可以吧……”陆久不明所以地说,“我对吃的东西,没什么特殊的偏好。”

“那边有家味道不错的店,明天我带你去吧。”V看着远方的街市说道。

“你以前,来过这里?”陆久大感意外地说。

“这个办公地点,曾经是备勤人形的训练中心。”V说,“那时候在没有勤务的时间,我们就被安置在这里,所以对这边还算熟悉。”

“是吗,那真是太好了。”陆久笑了,“我是第一次来这座城市,对这里几乎一无所知。”

“所以你这阵子都在这家餐馆吃晚饭吗。”

“……你怎么知道的?”

“要不是这样,这里的店员怎么会认识你呢。你一直都不是个引人注目的人。”

“呵,这种评价还真是让人尴尬。不过你说的确实没错。”

陆久无奈地耸了耸肩。竟然被V这样的人形明示作不起眼,实在是太丢人了。不过陆久知道自己就是这样一个无人问津的人,只有在别人眼前模式化地反复出现,才会让人有些印象。

“好了,回去吧。”V说。不知道是不是自己的错觉,陆久感觉V好像不易察觉地笑了笑。

“对了,你住在哪?”走进公司的大门,陆久问道。他忽然意识到离开了办公室,他就无法联系到V了,因为V肯定是没有手机的。

“公司的宿舍。”V说。

“能不能告诉我是在哪个房间,如果有事的话,我也好知道去哪找你。”

“好的。”V想了想说道,“我带你去。”

V带着陆久朝着公司后边的楼房走去,他们前进的方向让陆久有些奇怪。他之前看到有公司的员工推着载物车从那座楼房里进出,一直以为那里是仓库。

“那边,难道不是仓库吗。”陆久说。

“是仓库。”V回答。

“你的宿舍怎么和仓库在一起?”

“人形当然是在仓库里储藏了。有什么不对吗。”V说。

“噢……这我倒不知道。”

这样的回答让陆久感到意外,他还以为V的宿舍应该是和他一样在办公楼的客房区。

陆久跟随V走进仓库,来到了其中一间房间,V推开了门,陆久发现房间的门上甚至没有锁。

房间的大小和陆久的宿舍差不多,但里面几乎可以说是空的——

那间屋子里甚至就连衣柜都没有。房间里只有一根钉在墙上的铁丝,看起来是用作挂衣物的。然后就是几个木质弹药箱拼起来的长台子,上边铺着一层硬纸板,似乎是被当做床铺来使用。除此之外,就只有墙皮脱落的墙壁了。

房间里阴暗而冰冷,就连取暖的设备都没有。陆久简直无法想象该如何在这样的房间里过冬。

“这就是你的宿舍?”陆久说。

“是的。”

“就算最偏远的战区,人形的宿舍都要比这个强。”陆久严肃地说道,“我刚到战区的时候虽然连房屋都没有,但至少还有帐篷和睡袋。而这里为什么就连铺盖都没有?”

“战术人形是公司的重要武装力量,为了保证她们的战斗力,最低的后勤标准也会为她们配备睡袋。”V淡淡地说道,“但我现在是储备人形,不能为公司带来任何收益,所以是没有任何配给的。”

怎么会这样,陆久震惊地想着。他还以为战区的条件已经够艰苦了,但没想到在这和平的地区人形的待遇竟然更加恶劣。在格里芬公司,这些人形到底算是什么?

“那就搬到我的宿舍来吧。”陆久不假思索地说。

话音未落,陆久马上就意识到自己说了蠢话。他和V现在的关系只是同一个部门的同事,客观地说V也是一个年轻的女孩子,他作为一个男人怎么能提出让她住在自己的房间这种有悖道德的建议呢。

但话已出口,要收回也来不及了。V看着陆久,眼睛微微睁大了,显然陆久的建议让她也感到了吃惊。

一定会被拒绝吧,陆久心想。自己在胡说些什么呀。

但V却没有拒绝。她低头想了想,然后说:“可以吗。”

“咳,可以。我的房间有两张床,只有我一个人住……没有别的意思。”

陆久假装清了清嗓子,整理了一下自己的语言才没有绊住舌头。虽然他有些顾虑,但V却完全没有去想什么合适不合适的问题。她果然还是和以前一样,陆久心想。

“好。”V说着拿起“床”上被她当做枕头的行李包,“那我们走吧。”

于是两个人离开了仓库,来到了陆久的宿舍。

“这里的条件果然比仓库好多了。比战区的宿舍还好。”V环顾了一下房间说道,似乎对陆久的房间很满意。

“虽然是一般的客房标准,不过至少起居用品都有。”陆久说,“现在还少一套铺盖,明天再去买来好了。今晚你就盖我的被子,我盖这件大衣就行。”

“不用。”V说着把两张床推倒了一起,“这样就足够睡两个人了。”

陆久看了一眼,没有做声。这样的确可以睡两个人,不过被子还是只有一条,这是要两个人同盖一条被子吗。

“我……还是盖衣服吧。”陆久说道。

“怎么了。”V不解地说道,“那时在北镇的时候,不就是这么睡的吗。”

“那时是……”

真是哪壶不开提哪壶,陆久皱起眉头心想。他之所以说有两张床,就是为了避免回忆起那段不堪回首的时光。

“对不起。”看到陆久的表情,V垂下了目光说道,“我不该说那些让你想起不愉快的回忆的事情。”

不愉快的回忆吗,陆久心想。的确不假,那时候确实是自己情绪最低落的时候。但那不是V的责任。相反是V一直在他身边陪伴和鼓励他,而他却……

“不,该说对不起的是我。”陆久说,“我是对自己的所作所为感到可耻。如果有人在那段时间里受到伤害的话,那都是我的错,没有你的任何责任。”

“我也没有尽到自己的职责。所以……”

“不必再说了。”陆久坚定地说道,“请不要再说了。那些事情,如果可以的话就让它过去吧。如果不可以,只有我一个人负责就够了。”

“……好。”V轻声说,“知道了,我不会再提了。”

“呼,早点休息吧。” 陆久长出了口气说,“今天发生了这么多事情,让人感觉有点累。”

“好的。”V说着稍稍犹豫了一下,“那个,浴室……我能用吗。”

“随便用。”

陆久把大衣扔在写字台前的座椅上、脱下衬衫,然后躺在了床上盖上了被子。他听到浴室里传来了流水的声音,然后过了一会儿,水声停了。

被子被掀了起来,有人躺在了他的身旁。陆久闻到一股熟悉的气息,一丝略带温暖和潮湿的芬芳。

陆久感到后背发烫,许多回忆涌上了他的脑海,让他驱之不及。

“陆久。”

正当陆久忙着努力赶走那些几乎让他窒息的回忆的时候,他忽然听到一个轻柔的声音呼唤了他。

“唔?”

“能够再次见到你,我很高兴。”

“……我也是。”

片刻之后,陆久听到背后传来到了均匀的呼吸声,V很快睡着了。

\section*{}

“凡耶和华所亲爱的,必教他安然入梦”——虽然陆久不信牛鬼蛇神那一套,但至少他相信V的心中是问心无愧的。但他就不同了。第二天醒来的时候陆久感到非常疲倦,一是因为他一夜没有脱衣、二是因为他根本睡不着,而且因为担心惊醒V,他一夜都保持着侧卧的姿势没有翻身。

醒来之后陆久看到V已经穿戴整齐地等着自己了。还是昨天那身装扮,但没有穿羽绒服外套,所以陆久得以看到V里边穿的是白色衬衫套着灰色的V字领的羊毛衫。这男女皆宜的职业化装扮可谓相当中性,但V那窈窕的身姿让陆久知道,他面前的毫无疑问是一个漂亮的姑娘。

睁眼就看到美人在前可谓一种难得的福利,但浑身的酸痛让陆久的心里只有一个念头,那就是今天无论如何也要买一条新被子。陆久忽然注意到V的胸前挂着一条细细的金属链,上边吊着一个精致的坠饰。那是一个小小的海螺切割而成的薄片。

“你还带着那个呢。”陆久看着那个吊坠说道。那不是在北镇那时候,自己为了打发那个纠缠他们的老妇人买的……

“嗯。”V低头看了一眼自己的胸前,赶紧把那个吊坠塞进了羊毛衫里面,脸上显出了一丝羞涩。

那是什么奇怪的表情,陆久一边迷迷糊糊地想着一边向洗手间走去。

洗漱完毕后陆久走出洗手间,他看见V的脖子上又戴上了那个蓝色颈环。考虑到公司里的人形都有佩带这样的标识,为了避免搞特殊的嫌疑,陆久没有再让V取下来。

“几点了?”陆久说。

“八点。”V回答。

“……看来没时间吃早餐了。”

“我不饿。”

陆久整理好衣衫,然后看了看手腕上的计时器。他算准了时间,和V一起走出宿舍,来到办公室时是准确的八点三十分。

办公室里只有雷蒙一个人,但经过安检的包裹在谢振的桌子上堆了一大堆,看来老谢今天也是很早就来工作了。

“早上好,陆主任。”雷蒙先对陆久打了个招呼,然后又对V说道,“早上好,陆小姐。”

“别叫主任。”陆久摆了摆手。

“我叫陆薇。”V正色说道。

“啊,好、好的。”一大早就被两个人纠正称谓,让雷蒙有些不知所措。

“老谢呢,又是早早就干完一整天的工作然后出去了?”陆久说。

“是啊,这次我连人都没见到,只看见了包裹。”雷蒙耸了耸肩。

昨天不是还说天冷了不想出去了吗,陆久无奈地想着。果然是对自己有意见吧,人和人之间的关系真是难以把握。

算了,不耽误工作就好。其他的只有慢慢来了。

“我们该干什么干什么吧。”陆久说,“薇……咳。陆薇,你看看这些信件,对照汉语拼音录入试试。有不明白的可以问我……或者问雷蒙。”

“是。”V点了点头。

陆久莫名地有点烦躁,因为他感觉那个他本来叫惯了的名字,此时忽然显得无比的暧昧。他和V之间不同寻常的关系,无论如何都很难掩饰。

陆久、陆薇,怎么看都让人感觉有不一般的关系吧,陆久无奈地想着。这家伙到底是怎么想的,非要叫这么个名字?

“陆久。”正当陆久暗自苦恼的时候,忽然听到V轻声说道。

“怎么了。”

“这些信件里的内容,我不明白。”

“我看看。”

陆久伸手接过了V递来的信,飞快地扫了一遍。和之前的没什么不同,不外乎还是写些日常琐事的家书。

“写的都是些琐碎无聊的事情,哪里不明白呢。是有不认识的字吗。”

“不认识的字我可以查。但里边说的这些事情我不理解,所以阅读起来不太顺利。”

你不懂,那很正常,陆久心想。你有没有家人朋友,怎么会懂这些家长里短的日常牢骚呢。

不过陆久马上就开始自嘲起来。自己不是和V一样吗,有什么资格笑话她。

“先按这个录进去吧,有问题下班后我再给你解释。现在说这些会打扰别人工作。”

“好的。”

“没事。”雷蒙闻言接口说道,“您尽管解释,我听不见。”

滑头的小子,陆久心想。这家伙也不像第一次见面时,他感觉的那么诚恳啊。不过机灵一点倒不是什么坏事,不必像自己这样愚钝的人一样到处遭人嘲笑。

三个人低头开始工作,雷蒙熟练地封装着被拆开的信件和包裹,并为那些邮件贴上印有“已检视”的标签;而陆久则默默地审阅着一封又一封的信件。V的工作明显不怎么顺利,因为过了一阵子陆久悄悄看了她一眼,发现自己抽屉里的信已经一小摞了,而V还在翻腾着那几页信纸。

本来是作战单位,干这种工作也是难为她了,陆久有点同情地想着。对V来说,解读这些不知所云的信件,大概和让自己去揣摩那些难以度测的人心一样难办吧。

午餐依然是食堂清单的工作餐,陆久依然是飞快地扫荡了午餐然后由V收拾了餐具。雷蒙则很知趣地自己丢掉了装着剩饭剩菜的餐盒,并且什么都没说。用餐过后,陆久忽然站起了身:

“我去买点东西,你们自便。”说着他朝着外面走去。V闻言想要跟上去,却被陆久用目光制止了。

“很快就回来。”陆久轻声对她说。

她显然是还习惯于之前他们相处的那种模式,陆久心想。总是紧跟在自己身后,生怕自己有什么闪失。但是时候改变一下了。

这里是和平的地区,离战场已经非常遥远,不必再为什么潜在的危险而担心。他们也许应该有一点自己的个人空间。

陆久来到了市区,首先想到的就是置办一套铺盖。但他却发现自己不知道去哪买这一类东西。陆久在街上徘徊了好一阵也没看到相关的商店,只好停了下来。

该向谁打听呢,陆久苦恼地心想。他认识的人只有皮尔斯,但是为了这种事向他求助的话,不知道要遭到怎样的嘲笑。

绝对不能让那个口舌恶毒的人抓住这样的把柄,陆久心想。忽然他灵机一动,朝着一家饭馆走去,那正是他之前每天都来惠顾的地方。

“您好,先生。”见陆久走进来,服务员小姑娘诧异地说道,“今天比昨天更早了呢。要吃点什么吗?”

“不。”陆久有点不好意思地说,“我想问一下,被褥之类的东西……附近哪里有卖的?”

“……啊?”

陆久的问题显然让那个孩子吃了一惊,但马上她就笑了起来。

“您真有意思。”她笑着说道,“附近可没有卖那些东西的。这一块的租金可贵了,开一家卖被子的店可要赔干家底吧,谁会整天买被子呢?”

“那么我该去哪买呢。” 陆久窘迫地说道。

“我给你问问。”说着女孩朝着柜台喊道:“妈妈!这位先生想买被子,你知道哪有卖的吗?”

“买被子?”柜台后面的女人不可思议地说道,“哪有卖现成的被子的,那要去棉花店做才行。”

“那么哪里有棉花店?”

“附近没有这种店。我倒是知道一家,不过离这里可远了。”女人想了想说道,“你要多大的?我家老头子出去买菜了,我让他帮你去棉花店做一床吧。回来了你再给我钱。”

这个提议倒真是太贴心了,陆久心想,可是他会知道那种东西的尺寸吗?

“那么,一般被子的尺寸是……”

“对不起,对不起。”女人噗嗤地笑了起来,“是我不该问,男人家怎么会知道这些事情。是你一个人盖的吗?我估摸着让他做一条吧。”

“是的。啊,不……”陆久挠着头说,“如果可以,最好是能……稍微大点。”

“知道了。”女人看着陆久抿嘴一笑,“包在我身上,一个人能盖、两个人也能盖。”

说着女人拿起了电话:“哎,你到哪了?……去棉花店里做床被子去,就咱家那么大的。……没人结婚!你打听那么多干嘛,让你做你就去做。”

“还有褥子。”陆久赶忙说道。

“还有褥子!双人床的,快点啊。晚上拿回来。”

说完,女人放下了电话对陆久说:“成了。晚上过来拿吧。”

“谢谢。”陆久感觉像是抓着某根救命稻草爬上了岸,“不过我晚上可能有点事情来不了,请问能帮我送到我单位吗?我会付运费的。”

“你在哪上班呢?太远就不行了。”

“格里芬安保。”

“你是安保公司的啊。没问题,他回来的时候路过那里,顺便给你放门岗上。不过你们那边管得挺严的,你最好先跟门卫打个招呼,要不我怕门卫不收。”

“没问题。我先把钱给您,大概得多少?”

“我也不知道。不过肯定没多少,下次来了给吧!”

“这怎么好意思。”陆久忙从兜里掏出几张钞票,“这些够吗?”

“用不了这么多——”女人看见那些钞票一愣,然后说道,“绝对用不了。两张也用不了。”

陆久闻言,抽出两张钞票放在了柜台上。

“那就麻烦你们了。”

“不麻烦,顺路的事。”女人笑着说道,“多出来的你下次来了找给你。”

“不用找我,就当是略表谢意吧。”陆久说,“就靠我自己可干不了这事,真帮了我大忙了。”

走出小饭馆,陆久的心里有点感动。虽然只是没见过几次的陌生人,但这些人非常友善热情。

然后他要去干什么来着,陆久思索了一阵。好像也没别的事情了——

对了,他得去买部手机。他自己已经有手机了,但V没有,联系起来太不方便了。

但指望V去买手机显然不可能,因为据陆久所知,那家伙比自己要更加自闭。而且也不知道她有钱吗。

格里芬的人形,有劳动报酬吗?陆久忽然有点好奇。据他所知,战区的军费列支清单上是没有人形的军饷的,那么公司有没有给她们费用呢?

陆久忽然想起了V昨天说过的话:“不能为公司创造效益,所以没有任何配给”。就连被褥都没有,工资更是不可能吧。人形这一物品对公司来说是公司的资产,所以她们很可能没有任何个人财产。

这样的念头,让陆久心里忽然有些阴郁。这甚至不能算是剥削,因为她们连作为民用人形的权益都没有。

陆久走进之前的手机店,同一位人形服务员再次热情地接待了他。陆久思考了一下女孩子该用什么样的手机比较合适,但他很快就发现对于手机他只知道一款机型。于是他又买了一部和他手里那个完全一样的手机。

离开手机店,陆久顺便在街上的提款机上取了一些现金,然后返回了公司。经过公司大门的时候,陆久对门口的人形警卫交代了一下下午会有人给他送被褥的事情,那位警卫女孩同意帮他收下。

\section*{}

回到办公室,陆久看见雷蒙又在漫不经心地摆弄手机,而V则端正地坐在桌前一个字一个字地录入信件。

“还顺利吗。”陆久站在V的旁边问道。

“我觉得熟练一些了。”V说,眼睛依然没有离开屏幕。

“慢慢来。”

“嗯。”

陆久回到了自己的位置,然后坐了下来。雷蒙已经开始继续工作、V也一直没停,但他却没什么事可做了。

他感到这个部门的工作分工明显有些不合理。他和谢振两个人通常是干得飞快,特别是谢振,就连人都看不到都能完成一天的工作,而V则处于刚刚上手阶段,雷蒙面对他和谢振一同交来的函件有时候也会有点应接不暇。这种串联式的工作模式中,有一个环节跟不上就会影响整个工作链,像是V那里录不完的信件就无法让雷蒙发出,进而导致陆久渐渐变得闲置起来了。

但陆久想了想,还是决定不去打乱现有的工作模式,因为郝丽安把他派到这里可不是为了优化工作效率。事实上,她正是要他闲下来才让他去了这么一个部门。

但她希望自己把空闲下来的时间,用在什么地方呢,陆久思忖着。他把手伸进了自己的裤兜。

那里有一个U盘,无论何时他一直都带在身上,却从来没有打开过。那是一位他曾经认识的人形的临别赠礼。

不能说是战友、甚至不能说是朋友,只能说是曾经相识,陆久心想。但作为对她微不足道的恩惠的报答,她给自己留下了这个。

“通向往昔之门的钥匙”,她是这么说的。那么……

陆久在兜里摸索了一阵,最终还是没有把那个U盘拿出来。他知道自己随时都可以打开它,但他缺乏的是打开的勇气。

因为就如皮尔斯曾经所说的那样,一旦打开了那扇通往往昔的门,他就不再是现在的这个他了。这让他感到不安。

也许,他甚至将不再是“陆久”。

那段未知的过去,对自己来说到底有多重要呢,陆久自问。这个U盘里装的一定是克鲁格曾经认识的那个人,但却不是他所认识的人,更不会是V认识的人。

“你就是陆久”。回想起V所说的话,陆久悄悄地看了她一眼。他看到V依然在认真地敲打着键盘,完全没有注意到对面的人在想什么做什么。

没有必要,陆久心想。这些东西等需要的时候再去了解不迟。他是谁,不该由他的历史决定,而是该由他本人决定、由这个此时此刻的他自己决定。V说她的名字是“陆薇”,她很在意自己的名字(虽然这个名字给陆久带来了不少麻烦),因为那是她决定要成为的人。那么身为人类的自己,难道不能决定自己是谁吗?

他可以。他有这样的权力,那是身为人类的基本权力、是降生于这个世界上的人天赋的人权。他会自己决定要成为怎样的人。

陆久把那个U盘从兜里拿了出来,又从抽屉里拿出一个信封,把U盘装了进去。然后他从桌子上拿来了胶水,涂在信封口仔细沾好,把信封扔进了抽屉的最里面。

做完这些,陆久忽然感到如释重负般地轻松。他不由得从兜里掏出一根烟,然后去门口抽了起来。

一下午转眼过去,下班的时间已经临近了,但这次陆久不指望谢振会再次出现提醒大家该回家了。果然,五点三十分已过,没有任何人从办公室的门口走进来。雷蒙看了看表,时间已经过了下班的点,于是站了起来。

“那我就先撤了,各位明天见。”他说。

“好。”陆久点了点头,“我马上也……”

咔哒。

正当陆久要向雷蒙告别的时候,忽然办公室的门开了,一个人走了进来。

陆久一看,来人正是谢振。

陆久看着谢振,谢振也看着陆久,两个人对视了片刻。

“啊。这个,我……”谢振有点不好意思地说道。

“该下班了。”陆久迅速接口说道。

“是该下班了。但我不是要说这个。”谢振说。

“那是?”

“那个,陆主任。晚上有时间……?”谢振顾左右而言他地说着,顺便对雷蒙使了个眼色。

“对,晚上有时间吗?”心领神会的雷蒙马上说道,“一起吃个饭吧,难得大家都在。”

“呃,晚上我得……”

陆久犹豫着说道,偷眼看了V一眼。V正在专心地敲着键盘,仿佛身边这三个人和她完全没有关系。

陆久晚上当然有的是时间,不过问题是V怎么办。谢振说过他不喜欢人形,但是把V丢下自己跑去和雷蒙老谢一起吃饭,这种事对陆久来说是不行的。抛开他和V的私人关系不谈,就算是办公室的普通同事,他也要站在彼此都平等的位置去处理关系。

“还有这位……姑娘,”看到陆久的眼神,谢振说道,“也请……我是说如果方便,希望也能赏脸一坐。”

V听到谢振和她说话,眼睛这才离开了显示器。她看了看谢振、又看了看雷蒙,见两个人都在期待着她的答复,于是她对陆久说:“我怎么都行,你决定吧。”

于是两个人的眼睛又望向了陆久。

“好的。”陆久说,“不过得稍等会儿,我有些东西放在门岗了让我先拿回去。你们要去哪?我一会儿自己过去吧。”

“我准备好车了,一起走吧。”谢振说。

“那得劳驾你们等我一小会儿。”陆久说。

“没问题。”谢振点了点头,“我先去热热车,小雷,你跟着陆主任,要是主任的东西多你也好帮着拿一下。”

“我就——”雷蒙欲言又止地说道,看了一眼谢振又看了一眼陆久。

“不用。没多少东西,一只手就能拿走。你们去车上等我就行。”陆久说。

“行,听您指示。”雷蒙说着和谢振一起走了出去。

“我去帮你把东西拿过来吧,在门岗的警卫室吗。”雷蒙和谢振离开后,V对陆久说道。

“我自己去。”陆久摆摆手说,“你跟他们一起去车上等着。拿着这个。”

说着,陆久从兜里掏出一摞钞票和一部手机,递到了V的跟前。

“我不需要钱。”V说。

“拿着。这东西用途很广,不一定什么时候用得着,带着点以备不时之需。”

“好的。”V没再说什么,接下了陆久给她的钱和手机。

“手机带好,方便联系。记得充电……你会用手机吗。”

“我会。”V的嘴角微微撇了撇。

是看错了吗,陆久有些怀疑地心想。她刚才是不是露出了某种不屑一顾的表情?

“这是和你的同一型号的吗?”V的手指在手机上熟练地拨弄着问道。

“啊……是啊,怎么了。”

“没什么。”V小声说,“Android3700的系统,这机型快比你还老了。”

陆久的眉头拧到了一起,他好像听到了一句相当专业的测评。另外“比你还老”是什么意思?

这家伙有点不太对劲,陆久心想,从昨天开始就不太对。好像学会挖苦人了?她以前可不是这样的。

——以前不是这样的吗?

陆久忽然感觉这是一个值得深思的问题,不过他这会儿没工夫去深思。

“总之,去车上等我。”

“嗯。”

V走了出去,陆久也收拾收拾桌子朝门岗走去。当他走到门岗的时候,警卫亭的警卫显然看到他了,但那个人形女孩只是向屋里比划了一下。

很快屋里就有人迎着走了出来,陆久仔细一看是个年轻的男人。

“陆主任,您好您好。”男人对陆久热情地打着招呼,“我是警卫处的队长,叫邵敬勤。我一直等您呢。”

“哦。邵队长你好。”陆久对那个男人点了点头。

“哎,什么队长不队长的,您叫我小邵就行了,别客气。”警卫队长说,“您是来拿您的行李的吧,我已经给您送到宿舍去了。”

“啊,这怎么好意思。”陆久有点受宠若惊地说道,“也没多少东西,怎么能麻烦你呢。真是太过意不去了。不过你怎么认识我的?”

“哎呀,您这话说的。”警卫队长笑着说,“我怎么能不知道您呢。信检中心和警卫处都是安保部下面的科室,您也算是我们的领导啊。”

陆久想起来自己的介绍信上确实写的是“安全保密警卫部-信件函件检视中心”,这么说警卫处也算属于同一个部门管理的。不过要说自己是他们的领导可是太牵强了,警卫处和信检中心,显然管的不是同一样东西。

“我可领导不了你们。”陆久笑了笑说,“我那个部门算上我才四个人,整天干点没人管没人问的事情,你们的岗位可要重要多了。”

“您说得太见外了。您可是总部派来的,放在哪个部门也是钦差啊。不仅郝丽安女士特意叮嘱我们多给您行方便,就连皮尔斯准将都打听过您的消息,哪天要是总部那边想起我们来了,还得多靠您美言几句呢。”

听到这话,陆久心里笑了起来。什么领导,原来自己是跟着升天的鸡犬,这位年轻人对自己这么恭维不过是因为觉得自己和郝丽安、皮尔斯能说上话。

据说我还是克鲁格的战友呢,陆久心想。不过他老人家现在可正看我不顺眼,要是你知道了,恐怕马上就得对我敬而远之了吧。

“邵队长年轻有为,人事上要有提携,你肯定当仁不让。”陆久说。

“哎,陆主任过奖过奖。”听到陆久的称赞,警卫队长更高兴了,“不知今天陆主任有没有安排,要是没有,晚上我请您一起吃顿便饭……”

“我们部门的同事今天刚刚凑齐,我正要和他们一起坐坐熟悉一下。”陆久说,“改天有时间吧,到时候我请你。”

“好的,没问题。来日方长。”警卫队长说,“啊不不不,我怎么能让陆主任请我呢。我请,我请您。”

“一样、一样。对了,我那几件行李是给我放宿舍门口了吗。”

“哎呀,宿舍门口人来人往,的给您弄脏了多不好。我请客房部打开门,给您放屋里了。”

“那真是太贴心了。唔?”陆久忽然意识到了有点不对,“你……进屋了?”

因为陆久突然想起来,自己的屋子里的两张床被某个人推到了一起,如果有人看见两张床上只有一床被褥,说不定会怎么想。

听到陆久的问题,警卫队长一愣。然后,他马上笑了起来。

“没、没。我放门口就出去了,没往里走。”

“哦。没什么。我随便问问。”

“没事没事,我们公司的安防和保密是一流的,您放心好了。”

“是,这我倒不担心。”

这个小伙子非常精明,两句话就知道别人在想什么了,陆久心想。拜某位公关高手女士所授,陆久也有了一点看人的道行,能够看明白一些人的脾气秉性了。

不过看这意思,这小伙是认定自己在金屋藏娇了,陆久无奈地想着。算了,管他作甚。

告别了殷勤的邵队长,陆久走出公司大门,见一辆黑色轿车正停在马路对面。陆久一出去汽车的玻璃就落了下去,雷蒙探出头对他摆了摆手。于是陆久快步走过去坐在了汽车里。



