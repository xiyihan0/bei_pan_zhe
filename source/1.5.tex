\chapter{新世界(五)}


\begin{QuoteEnv}[85 Evo]{}
布满弹坑的装甲运输机前,强烈的气流吹乱了副官的头发。她带着她的指挥官回来了。

她一只手扶着着指挥官架在她肩上的胳膊,另一只手还紧紧握着枪,在全军将士的注目和军礼之下,一步步地缓缓走向营地。

而指挥官则几乎半残。他全身都是血、一条腿和肩膀中了好几枪,单靠自己已经无法行动,被他的副官架着依然步履蹒跚。但即便如此,他还是在努力抬起那只尚能活动的手,向对他们敬礼的士兵们回礼。

这一幕被记者拍了下来,他们把这张照片取名为“老兵归营”。

指挥官和他的副官被描述成一对历尽磨难的战斗伴侣,他们走向营房的背影,仿佛在亲朋祝福下走向神坛的伉俪。他们一个眼里全是忠贞与果敢、一个脸上写满了对明天的期待,那两张面容在很长一段时间里都被当做指挥官和战术人形之间相互扶持的象征。

但事实上并非如此。

那天当陆久和Vector走下飞机的时候,陆久的脸因为伤口的疼痛而扭曲着,Vector则因为架着陆久难以保持平衡而步履维艰。Vector的脸上虽然毫无表情,心里却充满了深深的忧虑;陆久也不是在向任何人回礼,只是想要伸手擦去额头流下的汗水。

他的目光也没有什么果敢和期待。在他的眼睛里,只有仿佛要燃尽一切的怒火。
\end{QuoteEnv}




“各单位,汇报状态。”陆久对着对讲机低声问道。

“突击队全员就位,等待战斗指令。”传来95的回应。

“远程火力支援小组在线待命中。”“目标锁定、保险解除,随时可以介入。”传来“静默”小组的回应。

“是,长官……各项器材部署完毕。侦察连就绪,等候敌人进入火力范围。”传来一个陌生的声音,说话的应该是出勤在外的05的代理队长。

眼前的情况是:铁血的大军正在围攻N21战区的指挥部,而前来应援的陆久部队已经悄然就位,正准备对敌人发动突袭。不过看起来指挥部的防守并不薄弱,只要稍加调度,铁血以步兵为主的部队,对指挥部坚固的外围防御构不成太大威胁。

这形势好像并没有N21战区传来的求援信息里说得那么严峻。再有最多两个小时天就该亮了,到时候这些铁血的步兵就是机枪下的靶子,他们根本没有可能攻下N21指挥部。

这是怎么回事?陆久感到有些奇怪。

从05离开指挥部的第二天起,陆久就下达了动员令,命令战区的士兵随时备战。但在那之后的一周里没有任何敌人的动静。今夜凌晨刚过,陆久的指挥部突然收到了公开频道里的求援信息,信息正是来自毗邻的N21战区,是战区指挥官亲自发出的求救信号:

“我们遭到铁血的猛烈进攻,情况危急……”信息里是一段模糊不清的音频,“……向附近的战区请求支援。重复,情况危急……请任何收到此信息的友军,火速支援……”

作为相距最近的战区守军,陆久立即做出了反应。他首先通知95整队,又点了三个战术中队,亲自率军赶往21战区增援。虽然信息里没有说发生战斗的具体位置,但既然是总指挥官发来的求援,陆久决定先去21战区的指挥部一探究竟。

他当悄悄从一侧接近21战区指挥部所在的位置时,发现这里果然正打得不可开交。但是等他部署好兵力的时候,却发现情况并没有想象中那么严重。于是他先观察了一阵。

陆久低头看了下手腕上发出微光的计时器。距离接到N21战区的求援,已经过去了四个小时。

无论如何不能再等了。按兵不动,岂非白跑一趟。

“等我信号,按照计划行动。”他对着对讲机说道,然后回头对着后面说:“所有人员,准备战斗。”

他的身后响起一片拉动枪机的哗啦哗啦声——在那里,至少有六十名荷枪实弹的战术人形……还有陆久的副官兼保镖。

看到武装的V,陆久有些无奈。他本想让V留在指挥部防范意外情况的,但是看到V的表情,他放弃了这种想法。出于对他的安全考虑,V一定不会同意他亲自上阵,在这一点上他们两个恐怕都不会改变立场——所以带上V是唯一的折衷方案。V显然也明白这一点,因此在陆久向95下达整队命令的时候,她虽然并没有出声反对,但却穿上了作战服。

她服从过自己的命令吗?陆久思索着。大多数时候是服从的。但是事关她的“原则”的时候,她从不妥协——无论是过去还是现在,这一点从未改变。

“重复一下作战计划:主力部队负责正面战斗,突击队准备侧袭。侦察连埋伏在敌人的退路上,远超支援小组见机行事、定点清除。”陆久对着对讲机说道,“现在——开始进攻,自由开火!”

深夜里,稀疏而杂乱的枪声,瞬间变得密集了起来。

正在和守卫部队胶着作战的铁血士兵,忽然遭到背后的猛烈火力。他们慌乱地散开寻找掩护,但却是徒劳的,腹背受敌的情况下没有任何有效的掩护物。

仅仅二十分钟,铁血的士兵就被歼灭了一半,剩余的也全部被火力压制在原地无法行动。

“各战术梯队推进,注意友军。突击队,上!”陆久低声说道,同时朝天空中发射了一颗绿色信号弹,向21区的友军标明自己的位置。

95带领六个突击队员从侧面冲了出来,开始攻击那些被压制的敌人。她们分成两个小组相互掩护交替前进,一边跑动一边射击,将躲在火力死角中的敌人逐个清除。

局势已经一边倒——阵脚大乱的敌人完全顾不上集结,只是一边向周围胡乱地放着枪,一边向着火力稀疏的方向后退。陆久的主力部队和突击队从两个方向继续追击着敌人,但是移动速度却渐渐放缓。

拉开了一点距离的铁血部队停止了开火,加快了撤离速度,但他们不知道自己已经被赶进了陷阱。

“侦察连,准备收网。”

“是,长官……不会让他们跑掉的。”

随着一片沉闷的爆炸声,急促的枪声再次响起。大路上的铁血士兵被埋在路边你的遥控炸弹消灭了,逃入树林的也被早就埋伏好的侦察兵们全部肃清。

“都搞定了吗。”陆久说道。

“嗯,看来差不多……”95在对讲机里说着,“啊,注意,七点钟位置有情况!”

陆久立即转身朝侧后方看去,只见之前一直停放在那里的一辆铁血的轻型装甲车突然动了起来,并开始加速逃离。

因为它一直没动静,所有人都以为那是一辆净空的载具,没想到里边还有驾驶员。

不好,陆久心想。虽然装甲车正在朝他这边驶来,但是战术梯队装备的都是轻武器,无法对这辆车造成什么阻碍。

“后退!”陆久高声喊道,战术人形们纷纷朝周围散去。陆久知道他们无法阻挡那辆装甲车,为了避免受到冲撞他下达了回避的指令。

反装甲武器在突击队手里,但是95的位置离得太远了……虽然她们在努力追赶,但只靠徒步,是不可能追上那辆装甲车的。

但陆久看到那辆装甲车跑到半路忽然停了下来,然后发动机上面的散热窗里冒出了一股浓烟。

……故障了?不可能吧。陆久感到一阵惊讶,他端起突击步枪朝着装甲车靠了过去,想要查看一下情况。

“哎呀哎呀,这一枪居然没能摧毁目标。85你的枪法还是不行啊,等着司令训你吧。”

“我明明打中发动机了,怎么说枪法不行?要怪也该怪你,刚才给我的不是穿甲弹吧?”

对讲机里传来一阵吵闹的声音,原来是“静默”小组出手了。

安静,别占用公共频道……还没等陆久把这句话说出口,对讲机里再次传来了另一个人的声音。

“司令……低下身!”

陆久听到95急促的呼喊,立即伏下了身。

只见远处的95手中火光一闪,一团火球飞向无法移动的敌军车辆。接着一声闷响,那辆装甲车熊熊燃烧了起来,明亮的火光照亮了夜晚的战场。

在灼热的烈焰之前,陆久不由得往后退了几步。

……“虹”式反坦克火箭筒吗?

这种单人反装甲武器虽然小巧轻便而且威力巨大,但是只能一次性使用。如果一击不中的话,那么自身就很危险了,因为发射时的巨大火光和火箭的尾迹会立即暴露射手的位置。

陆久望着那辆烧成一个火球的战车,不由得微微一笑。这姑娘,还真是自信啊。就像和以前的自己一样。

不过还没等陆久在心里夸赞完95,他就被一个人拽到一边、然后按在了地面上。然后陆久看到了刺眼的光芒,一股滚烫的热浪包围了他。

那辆装甲车发生了殉爆。

“司令同志,请注意自己的安全。”陆久听到一个淡淡的声音在身边开口说道。

不用说,陆久也知道是谁。

“……啊,多谢。”陆久诚心地说道。如果不是自己的这位“贴身护卫”,自己现在大概已经躺在担架上了。那阵冲击波可不是闹着玩的。

“嗯,这次总算是最后一个了吧。”伴着一个轻快的声音,95面带笑容朝着陆久走了过来,看来她对刚才那一击相当得意。

但当她看到陆久身边V的脸色时,她知道自己捅了娄子——V的面色冰冷,显然是生气了。

她这才意识到刚才自己的攻击有多危险。

“95同志,请注意危险距离。”V冷冷地说道,“如果陆久司令出了三长两短,我们都不好交代。”

“是……对不起。”95低下头说道,“我没想到司令会……对不起,是我的错。”

“算了。要说错的话,应该是我的错。我不该傻站在一辆起火的车辆旁边一动不动。”陆久开口结束了这场争端,“你们尽了自己的职责,应该反省的是我。我向你们道歉。”

“不……”

“没有……”

V和95异口同声地说道,然后相互看了一眼。陆久摆了摆手示意到此为止,然后向着N21战区指挥部的大门走去。

在那边,陆久的主力部队已经在门前整队待命,而N21战区也正在打开大门迎接友军。

“抱歉……V。”95对着V小声说道。

“不,我才是……”V轻声说道。虽然她还想说些什么,但见陆久已经走远,于是打住话头快速跟了上去。

\section*{}

95看着向着陆久跑去的V,心里感到一阵失落。

V的责备有着充分的理由。那辆装甲车如果逃离,也许会带回去一些关于援军的情报,但再怎么说也不会比陆久的人身安全更重要。

她急切地想要在陆久面前证明自己,却因此犯了危险的错误。

关于V和陆久以前的事情,95从05那里听说了一些——在某次战斗中,V曾经救过陆久的命。虽然面前的这个V已经不是“以前”那个V了,但陆久依然是那个陆久。所以她知道,在陆久的心里,V有着她独有的位置。

而虽然95和陆久也共同经历了很多危险的战斗,但每次都是陆久把95拉出油锅,这一点是95无法否认的。

她并不是没有舍身忘死的觉悟,只是她没有机会——以后恐怕更不会有了。所以她只能用战斗来证明自己。

95苦笑了起来。

今天,是V副官在危险中保护了司令,而制造这危险的却是自己。多么蹩脚的表演啊。

可自己说到底也不过是个战斗人员,有什么办法呢。自己能做的只有战斗。

这不公平、太不公平了,95心里甚至有些哀怨。这样自己永远都赢不过V的。难道真的只有像05说的那样,求V高抬贵手、让自己一步了吗。

“队长?”

95身边响起一个声音,把她拉回了现实。是她的一个队员过来叫她了。

“哦,没什么。”95赶紧收拾起自己的慌乱,“让大家整队——战斗也许还没有结束。”

“我想,嗯……”那位队员有点犹豫地说着,“其实,您不必过分自责。司令如果稍微躲避一下,也不会……”

“不。”95打断了那位队员,“我不该贸然攻击司令身旁的目标。无论他在做什么,这一点都是我的失策……如果不是V副官,现在真不知该如何去弥补。整队吧。”

“是。”

……

“您能派出援军实在太好了,陆司令。来得真是太及时了。这一会儿我们已经焦头烂额,如果不是你们赶到,事情可能就很难收拾了。”

当95带领突击队赶到陆久身边的时候,陆久和V已经站在N21战区军营的里边了。指挥部的门前,战区的总指挥官亲自出来迎接了陆久的部队。

95打量了一下眼前的人:中等身材的男人,站在高大的陆久面前,甚至可以说有点瘦弱。他看起来很年轻,穿戴着整齐的军服,脸上还戴着一副精致的方型眼镜。

这模样大概很容易被当做公司里办公室里的文职人员,如果不是他穿着指挥官独有的猩红制服的话。不过对于95来说就另当别论了——95曾经见过他。正是他的车队,在当时还是17号区域地方发现了被遗弃的95。

高尔顿·佩瑞特,N21战区的承包者,公司最年轻的指挥官之一。作为闻名全国的高材生,他20岁就从中央军校毕业了。毕业后他并未继续在部队服役,而是选择了(另有一说是受国家指派)进入民营保全公司实习。派瑞特曾饱读全球战史,对从古代到现在的种种战例都有着深刻的见解,可谓理论基础雄厚,这样的人才公司非常重视。为了弥补他尚且年轻、缺乏实战经验的弱势,他进入公司的第二天就被派到了N14战区学习。在战区学习了一个月后,他又被派往21号区域——也就是如今的N21战区,开辟新的疆域。之后借着公司的大力支持,N21战区发展十分迅速,不到一年的时间就在北部战线站建立了稳固的据点。

所以如果从时间上看,派瑞特其实比陆久到达战区要早。不过作为军营后辈,科班出身的派瑞特对陆久十分尊敬,甚至可以说有几分仰慕。他在私下曾不止一次表示希望能向陆久学习战斗经验。

有几次陆久和派瑞特在线上会晤的时候,95曾在旁边听过他们的交谈。派瑞特对陆久的态度十分谦恭,95能看出陆久也对这个年轻的军官有几分欣赏。

当然,关于95 的事情派瑞特是不知道的,陆久也从未提起过。派瑞特只知道自己曾经向公司请示,希望能向21号区域派出一些增援和补给,但他并不知道当时接收增援的具体是谁。而陆久到达21号区域,则是在那之后一段时间的事情了。

“没那么夸张吧,少校。”陆久摇了摇头,“据我观察,铁血的兵力并不能威胁贵部的安全,如果坚持到天亮,他们不过是一群机关枪下的靶子。情况没有我想象中那么紧急,可是您似乎非常担心,为什么呢?”

“的确,铁血的兵力远不足以攻陷我们的军营,仅仅是外围防御他们都过不去。但是我们的危机不在此处。”派瑞特严肃地点了点头说,“还记得夏末的那次战斗吗,在我战区的边缘经过的公司车队……他们曾经带了一些铁血的设备回去。”

“嗯,我记得。”

陆久的表情严峻了起来,点了点头示意派瑞特继续说。

“两天前,我们的一个侦察小队在战区边缘的d3区域发现了铁血的营地,看到他们在部署一些……用途不明的设备。虽然还不知道具体作用,但我推断这些东西和我们之前了解到的危险的‘秘密武器’有关。出现这种东西我自然不能坐视不理,所以我派出了兵力,希望能在铁血站稳脚跟前摧毁他们的营地。可是在昨天傍晚的时候,我们的部队突然和指挥部失去了联系,直到现在状况不明……我怀疑这也许就是那些未知设备造成的。之后我了解到,那个营地附近出现了铁血的大股部队。如果我们不及时支援她们,这支部队将会有全军覆没的危险,而且那件所谓的秘密武器可能会威胁到整个N21战区。而此处的这些铁血的部队并不是来进攻我们,而是围困……他们的真实目的是阻止我们的增援。”

“秘密武器……我也从一些人那里听说了这件事。这东西的存在被确认了吗。”

“是的,我们部队的失联大概就是这件武器造成的,它能在一定范围内屏蔽我们的通讯信号。而且,我想这只是它全部功能的一部分。”

“难道是那个传言中的‘伞’吗。”

“……伞?”

“是的,在南部的某战区曾经出现过这种武器……结果就是造成了一个战区的失守。16号实验室曾经派出了几个专门针对这一武器的试制人形,但也在这件武器的影响之下遭受了损失。因为公司透露的相关情报很少,所以北部战区几乎没有人详细了解这件事情。”

“就连16号实验室专门研制的试制人形也未能发挥效果吗?仅仅是一件武器,就让一个战区沦陷了……”派瑞特少校喃喃地说道。

“所以现在眼下的形势已经相当严峻了。也许,整个北部战区都正在面临危险。” 

“如果是这样的话,我们必须把这件事情上报公司!”

“来不及了。等公司讨论出对策,整个北部战区说不定都已经乱作一团。我们必须先救出你被困的部队,自己来弄明白到底发生了什么。你被困的部队有多少人员、装备如何?”

“一共有约70名人形,都受过良好的训练而且装备精良,如果是防守作战的话暂时应该没有危险。只是那些人形的自律模块不足以组织大规模的反击,面对铁血的围困不知道能坚持多久。”

“不必担心,交给我来处理。你派出部队牵制住敌人的外围力量,我设法指挥被困的部队突围、里应外合。”

“但是现在通讯信号被屏蔽,根本传达不到被困的部队,您要如何指挥?”

“现场指挥。”

听到陆久的话,派瑞特沉默了一阵。

“‘现场指挥’。整个北部战区的二十七名指挥官里,能说出这句话只有一个人,那就是您。”派瑞特看着陆久严肃地说道,“您是一位真正的战士。就算让我为您擦枪,我也倍感容光……您是我们所有人的榜样。”

“千万别这么说。”陆久笑了笑,“在元帅同志眼里,我可是个坏榜样。”

“我将派出N21战区全部兵力支援您。另外——请带上这个。”派瑞特从腰间抽出一把手枪递给陆久,“我战区的人形士兵们都认识这把枪。向她们出示这件武器,她们会听从您的指挥。”

陆久接过那把手枪放在眼前仔细端详了一阵。那是一把精致的马卡洛夫9毫米工艺手枪,不锈钢镀铬的银色枪身、手柄上是红色木质防滑片,整支枪都雕刻着精美的花纹。

真是件不可多得的艺术品,陆久在心里赞叹。相比起来,自己那把消光漆都磨掉了的老式军用手枪是在是不上台面。这位年轻的派瑞特少校显然是个十分有品味的人。

“是把好枪。”陆久点了点头,把手枪收进了裤子上的口袋里。

“手中玩物而已。”年轻的少校谦虚地说道,“虽然不能和阅枪无数的陆司令相比,但是如果陆司令不嫌弃……”

“不。”陆久打断了派瑞特的慷慨之词,“既然被谬冠‘战士’之名,我自然明白枪意味着什么。你的人,我会连同你的枪一起带回来。”

“这一点我毫不怀疑。”派瑞特站直身子向陆久敬了个礼,“全都仰仗您了,陆司令。”

陆久点了点头,也抬手回了个礼。

“那就开始行动吧。”陆久转身对着身后自己的部队说道,“侦察连在外围警戒,V,你带领战术梯队留在这里协助防守基地,配合派瑞特少校的调遣。95,带上你的人跟我走。”

“报告!”还没等陆久部署好作战任务,忽然有一位人类文职人员跑过来打断了他。那位年轻的尉官跑得气喘吁吁,似乎有什么紧急的情况要汇报。

“报告……报告少校。我们收到一条来历不明的讯息,请您过目!”

派瑞特接过文员递过来的纸,上边只有一条简短的信息。

“发信人是‘05’。这是谁?”派瑞特有些不解地说道。

“我的人。”陆久说。派瑞特立即把那张纸递给了陆久。

“造成干扰的发射器的位置不在d3区域,而是在离它16公里外的h6区域。d3区域的只是增幅器,仅仅破坏它是没用的,铁血很可能不只部署了一部增幅器。只有摧毁发射器,才能彻底消除‘伞’的信号屏蔽。” 陆久接过来看了一会儿,然后说道。

“不过,这个消息……可靠吗?”派瑞特有些疑虑地说道。

“据我经验,这个人的消息还没有出过差错。”陆久说。

“那么我们必须兵分两路了。”派瑞特点了点头,“好,那我就抽出一支部队去d3区。”

“不必,你继续按照刚才的计划执行。”陆久摇了摇头,“薇,你带领甲、乙两个梯队去进攻h6区域。根据05的情报,那边为了最大限度地隐蔽部署的兵力并不多,这些人足够了。丙队和侦察连在此协助防守,我和95设法渗透进d3区。”

听到陆久的指示,V稍微犹豫了一下。她依然在顾虑陆久的安全,不过她知道陆久的指挥权是绝对的,战场之上他的决定不容违抗。而且95和陆久协同作战的经验丰富,他们在一起应该会得心应手,不须过多担忧。

所以在沉默了片刻后,V点了点头接受的命令。

“不,我想还是我去h6区比较好。”一直站在旁边的95这时忽然开口说道,“首先,V副官没有指挥部队的经验,如果出现意外她可能难以准确把握形势,还是在陆司令身边让人放心。其次,我去h6区只带突击队就够了,能够节省一部分兵力用于营地防守——光靠一个梯队防守基地,恐怕太过单薄了。”

95显然是看出了V的心事才这样说的。她的提议得到了V的感激的目光。

“你说得的确有道理,不过这样行吗。”陆久微微皱眉,“只带突击队,会不会兵力太少了?敌人虽然人数不多,但至少比你手里的人多出三倍以上。”

“我会避免正面作战的,力争悄悄靠近、一击脱离。”

“……真的不是在逞强吗。”陆久严肃地看着95说道。

“请您放心吧。”95笑了起来,“在和司令一起战斗那么多次之后,我也想自己立一次战功呢。就给我这次机会吧。”

陆久看了95一阵。

“好,那就交给你了。”陆久说道,“但是记住——有把握再出手,不允许投机冒险。如果必要,随时呼叫营地中防守部队的支援。”

“明白。”95敬礼说道。

“开始行动。”陆久说着转过了身,“战术梯队在此防御基地,侦察连跟我来。”

“集结所有人员和载具,向d3区域进发!”派瑞特少校也下达了命令。

“……V。”趁着陆久对侦察连讲解战术的空隙,95忽然对着正要离去的V小声说道。

“嗯?”

“司令一向不爱惜自己的身体,记得多照顾他。”95说。

“什么?……”V满心都在想着作战的事情,忽然听到95这毫不相干的话,一时间不知该如何回应。但95似乎并不在意。

“还有,别忘了闺中密友之间的约定。”95说完笑了笑,转身向着整齐列队的突击队走去。

\section*{}

“闺中密友的约定”吗。V一边走,心里一边想。她倒是没有忘记,而且也不可能忘记。

可是,95为什么忽然在这时候说起那些呢。V心头掠过一丝阴影。

……该不会出什么事吧。她心想。

不会的,陆久交代得很明白。95如果按照陆久的命令去做,突袭成功与否姑且不论,但一定不会有什么危险。

该担心的反倒是自己这边——就算95那边成功恢复通信,但是想要靠一个小队就杀进铁血的重围之中,绝非易事。

这位“战地指挥官”,是否真的像传言中那么神勇呢。V的心中忽然有些期待,她很久没有和陆久一起作战了。

不对,V微微皱起眉头。她这是第一次和陆久一起战斗。

铁血的基地设在一座废弃的气象站之中。N21战区和N17战区的环境类似,这里没有大规模的城市。但是在末世战争之前,这里曾经有一座为了采集气象数据而建立的气象站,虽然荒弃已久但是那群破败的建筑依然耸立在N21战区的边缘。

最近,这座早已人去楼空的气象站再次热闹了起来。

四个侦察队员跟着V和陆久,搭乘三部全地形摩托车一路行军,半个小时后到达了他们的目的地。但是让V感到奇怪的是,他们并没有径直朝着气象站而去,而是来到了距离气象站外围几百米远的地方,一段干涸的河道里。几个人把摩托车放在河床用伪装网盖好,然后沿着河床走了约二百米,在一处暴露在外的残破管道口停了下来。

“我们从这里进去,”陆久说道,“这段管道应该能直通气象站内部。”

“这是什么管道?”V问道。

“地下排水管。以前的建筑都有这种东西。”陆久说着,拨动手腕上的计时器,投射出一幅地下管道网的全息地图。

“您……从哪得到这些资料的?” V稍稍有些吃惊。

陆久抬头看了V一眼。

“公司的数据库。”

“您一开始就计划好了?”

“当然。”

没想到陆久居然连地图都准备好了。陆久没有提过他计划的一个字,无论是对他的部队还是对派瑞特少校,V本以为陆久是打算等了解情况后再审时度势,但没想到他的作战计划竟然早就制定好了——而且是在派瑞特少校介绍清楚情况之前。

陆久已经预料到了可能发生的情况,并且做好了相应的对策,他早就详细地了解过了N21战区……也许不止是N21战区,北部多数战区的大概情况,陆久大概都已经了然于心。

——知己知彼,百战不殆。

也许这就是派瑞特少校一个字都没问,就把事情交给陆久的原因吧,V心想。陆久果然并非徒有虚名,他在战斗方面有着丰富的经验,甚至称得上是个战略家。

“这次渗透行动不需要那么多人,我带一个人进去就够了。你们几个负责守卫这个入口,不要让任何不明目标接近。”没有注意到身边副官的走神,陆久对侦察队员布置着任务,“95她们的行动还不知道情况如何,如果那边不能成功摧毁干扰器,那么用不了多久我就会被屏蔽和外界的联络。到时候你们负责接收这段时间的通信,并把所有情况详细记录下来。”

“明白。”侦察队员简洁地答道。

“分成两个小组远距离监视,该怎么做05应该都教过你们。2小时后如果我还没有消息,你们就把情况汇报给派瑞特少校;4小时后要是还没有消息,无论派瑞特少校的人在做什么,你们立即返回N17战区营地,等待05回来并把情况告诉她。天马上就要亮了,马上行动。”

“是!”四个侦察队员领命后,立即各自分散隐蔽了起来。

陆久将突击步枪背在肩头勒紧背带,然后取出手枪把消音器拧在枪口上。

“我们走。”他对着V说。

V点了点头,也在冲锋枪上装上了消音器,紧跟在陆久身后。

两个人进入管道后,走了很长一段直路。一开始V还能看到入口处传来的微弱光芒,但在走了几分钟之后,入口终于化作了一个模糊的光点、最后彻底看不见了。两个人陷入了一片黑暗。

咔哒。传来一阵轻响,V的眼前忽然亮了起来,是陆久启动了手枪上的下挂照明。

“跟紧点。虽然这里不太可能有铁血,但是如果有的话,可是敌在暗处我在明处,我们会很被动。”

“明白。”V轻声回应。

她感到有些冷。下水道里常年没有日光,本身就不高的温度随着他们的深入而变得越来越低,一阵阵夹着霉烂气味的冷风吹得她身上发凉。V拉紧了自己的外套拉链,发出一声尖利的轻响。

“冷吗?”V听到前边的陆久说道,他可能听到自己拉拉链了。

“稍微有点。”V说。

陆久停了下来。他解下自己的挂带,然后脱下外套递给V。

“穿上这个。”陆久轻声说。

“不用,也没那么冷……”V稍微有些慌乱地说道。陆久这突如其来的关照,让她有点措手不及。

“穿上。”陆久再次说道。

虽然他的语气没有一丝改变,但不知为何V感觉这句话仿佛如命令一般让她无法拒绝——也让她不想拒绝。

“……好。”V轻声应道。这如果是所谓的好意的话,索性就接受吧,她心想。

她解下自己的武装带,然后接过陆久的外套套在身上、接着整理好身上的装备。她看到陆久也把他自己的背带挂在了内衬的防弹衣上,然后转身开始继续前进。

V拉上拉链,她闻到一阵混合着古龙水和汗水的气味。外套很温暖,上面还带着陆久的体温。

……这是?

V感到自己的心跳有些加速,她不知为何自己会感到有些紧张。这感觉就像是……被什么人,拥抱着一般。

虽然这间外套并不能防弹,但V却感到了莫名的安全感。她又将衣服拉紧了一些。

他们又走了一阵,在走过一个拐角的时候,陆久突然停了下来。紧紧跟随的V差点撞到他身上。

“怎么?……”V轻声问道,但她话还没说完,就停了下来。

她顺着陆久的目光望去,眼前骇人的景象让她倒吸了一口凉气。

陆久枪上的照明亮度不高,无法照亮面前的整条通道。但是在那微弱的光线下,地上那些恐怖的影子还是让V感到寒毛倒竖。

他们来到了一个空旷的区域,那里横七竖八地散落着很多破碎的骸骨,也许有上百具。显然,他们都曾经是活生生的人类,但是陈尸此地已经不知多少年了。陆久调亮了手电的光线,仔细检查着那些尸骸。

那些尸骨或坐或卧,姿态各有不同,向陆久和V沉默地展示着他们生命的最后时刻。就连向来冷漠的陆久也不禁停下了脚步,无言地看着眼前的一切。

“他们……都是人类吧。”V在陆久身后小声说道。

“是的,”陆久说道,“而且看起来是战前的人类,不知已经死了多久了。”

“他们为什么会在这里?”

“为了躲避战争吧,我猜是这样。现在已经没有人能说出答案了。”

“这里,到底发生了什么……他们为什么不离开?到底是谁杀死了他们……”听到陆久沉着的声音,V的心里没有之前那么恐惧了,小声自言自语地说道。

“这个问题倒容易猜得多。他们一定是畏惧外边的不明的危险而不敢外出,然后在这里耗尽资源后,自相残杀而死的。”陆久冷冷地说道。

“……怎么会?”听到陆久的话,V感到无比的吃惊,“你怎么知道,又为什么如此肯定?”

“因为他们是人类。人类就是这样一个怯懦、自私而残忍的种族,他们比任何物种都要害怕死亡,而且为了活下去会不择手段。他们会利用一切可以利用的东西,甚至自相残杀……如果他们从这世界上灭绝,唯一的可能的原因,就是他们自己。”

V在陆久的声音里听到了冰冷的寒意。她不由得想起她听到95对未来的描述,那副灰暗的光景也曾让她不寒而栗。

他为什么会这么想。如果之前的95那么说,是因为她对陆久的愿景早已看破,那么陆久为何又会对人类评价如此之低呢。

那他这样奋不顾身地战斗,又是为了什么呢?

“您为什么……会这么说?您也是人类啊。”

“我只是个来自往日的阴魂。”陆久转过了身,将手枪的枪口垂向了地面。他的脸陷入了阴影之中,V看不到他的表情,“现在的人类,和我的那个时代已大不相同:他们那些丑陋的恶习不仅没有一丝改观,还摒弃了很多本该流传下来的美德。我一直在努力说服自己去相信这个新世界很美好,但我知道那不过是一厢情愿的想法。这个自欺欺人的梦,总有醒来的一天。”

“您并不是在自欺欺人。您的气概感染着基地的每一个人,无论是人类还是人形,这是我来到基地一段时间后才发现的。他们都把您当做榜样,您给了他们战斗的理由、甚至生命的意义。很多人……至少我和95,还有营地的战术人形们,都认为你所做的一切是很好、很有意义的。”

这不算是谎言吧, V心想。虽然95的想法并不尽然如此,但是她对陆久的忠诚是毋庸置疑的。所以,V毫不怀疑95也是赞同陆久的。

但陆久只是发出一声嘲讽的轻笑。

“‘生命的意义’吗。”陆久举起手电照向前面的白骨群,“看看他们。总有一天,我们也会像这些人一样。当一个人生命停止活动以后,他的名字会被多少人记得、又会被记得多久呢。这里面的每个人都有自己活着时的故事,但如今有又谁会知道。生命只是一个繁衍生息的过程,其本身,也许并没有什么特殊的意义。”

“怎么会被……”陆久的话让V不知该如何反驳,“人类的生命如此宝贵,怎么会没有意义……”

“其实我知道,我和你没什么不同,区别只是在生物学上而已。”陆久的声音低了下去,并透出了一丝倦怠,“如果你觉得自己是一部战争机器的话,我也不过是……一部来自上个时代的战争机器罢了。要说生命的意义的话,大概也不过如此吧。”

V沉默了,她无法相信这些灰暗冷酷的话是从陆久的口中说出的。她再次想起那天从小镇上回来时和95的对话,那时的情景和现在如出一辙。

仿佛一张面具一般,陆久和95,都在扮演着别人心中那个理想的“自己”。他们为了回应别人的期待而努力地去演绎着,却始终不肯把自己的真面目示人。

——不,也不能说他们完全没有袒露过自己的内心。至少他们都曾经在自己的面前敞开过心扉。只是,为什么是自己呢?

V感到困惑。她自认为不是个好的倾诉对象:她性格孤僻而自闭,不懂如何理解别人的心思,更不懂如何去安慰别人。为什么陆久会对自己说这些呢。

“无论您怎么说,我的看法不会改变。只要我存在一天,我就会坚守自己的职责和任务。”V平静地说道,“而且,我认为您是一位了不起的指挥官。您拯救了很多人形和人类,您所做的一切,都很有意义。”

“谢谢。”

“没有什么值得感谢的,这是我的义务。”

“不。我是要谢谢你……救了我很多次。”

听到陆久黑暗里的轻声低语,V感到胸口一紧。她知道陆久指的不是自己,自己从来没有救过陆久。他说的是另外一个人——那个真正让他在意的自己、曾经的“自己”。

陆久说过不会谈论他们的过去,他却打破了自己制定的规则。

V忽然感到一阵难以言喻的空虚。她感觉心中仿佛打开了一个黑洞,正在吞噬着关于她的存在、吞噬着那些构成她的并不多的过往。她的心里涌起很多影像,不仅有陆久那天晚上对自己的告诫,似乎还有更早的难以辨识的场景。

她这时才明白陆久的话,明白了那时陆久为什么会说“当我说出关于你的过去的时候,也许我们就无法再保持现在的关系了”。陆久正在向着她致谢,她却不是那个应当承受这感谢的人。

在这寂静的黑暗之中,就连时间都仿佛停滞了一般,除了阴冷的风拂过皮肤之外,感受不到一丝外边世界的气息。这让V感觉恍若隔世。

但是她明白,陆久对自己如此坦诚,只是因为,他把自己当成了理想中的另一个人——她本以为自己是真切地守候在陆久身边的,结果其实自己却是一个早已消亡的人的影子。

还以为自己能替代什么人,结果不过是个替代品。V自嘲地想着。不过就算是这样,自己也有不同推辞的责任,那是她从公司接受的命令、也是她存在的意义。

V把枪背在肩头,然后仿佛确认目标的存在一般,朝着陆久的位置伸出了手。虽然在微弱的光亮之中陆久的身影甚至都有些模糊了,但V还是摸索着拉住了陆久没有握枪的手。

那只手温暖而干燥,握住的时候给人一种莫名的安心感,完全不像是一个正在踟蹰不前的人的手。V犹豫了一下,然后稍稍用力握紧了那只手。

“您说的那些事情,我并不知道细节。如果以前曾经有人挽救过您的生命,我由衷感到欣慰。”V轻声说道,“不过如果有机会的话,我也愿奋不顾身地去那么做,如果您肯相信我。”

当V说完这句话的时候,她感到陆久的手轻轻地颤了一下。陆久似乎是想要握住V的手,但是片刻之后,他轻轻抽回了手。

“抱歉,我不该说这些毫无意义的话,我们都已经不是过去的自己了。”陆久轻声说道,“怀疑当下、沉湎往昔,不会让任何事情变好。请忘记我刚才的话吧。”

“但是,如果那是您所期望的,那么我……”

“不。”陆久稍微提高了声音,“不……那不是,我期望的。完全不是。”

“……是吗。”

“是的。走吧,我们还有仗要打。”没有等V再说什么,陆久再次举起枪朝着那散落着一地枯骨的下水道走去。他踩在那些早已腐朽的骸骨上面,仿佛踩在树枝上一般,发出一阵咔嚓咔嚓的声音。V也快步跟了上去。

“感觉到了吗,气流更强了。还有隐约的枪声……”陆久说,“出口应该就在附近,希望一楼还没有被铁血攻陷。” 

\section*{}

大约二十分钟后,陆久和V找到了他们的目标,一个能够钻出去的下水道口。通常来讲这么大的洞口都是设在室外的,但是陆久仔细搜索地图后很幸运地发现在气象站主建筑一楼的某个洗手间里,也有一个这样的下水口。

当满身灰土的两个人地钻出下水管道的时候,迎接他们的是几支黑洞洞的枪口。但幸运的是手持那些枪械的不是铁血的人形,而是几个慌乱的N21战区士兵。

“你们是……什么人?”陆久身边一个身形矮小的人形,不知所措地问道。陆久举起双手示意没有敌意,顺便扫了这些人形一眼。

这个小小的卫生间位于气象站主建筑的外侧,里面有五个人形正在通过窗户和外边的铁血部队交火,但也许是因为位置不重要的原因,受到的火力压力还不算大。她们的装备相当丰富,各色枪械一应俱全,看来的确是有备而来,可惜现在失去了指挥官的指挥,这些人形已经阵脚大乱,只是在盲目地各自为战。

“别开枪,我是友军。”陆久简单地答道。

“你是人类?”

“是的。”

“嗯……”那个人形少女显然没有想到地底下会钻出一个人类,一时不知该如何处理,“我们必须保护人类。不过,现在是在铁血的包围圈中间,我们连自己都保护不了了……你去里边的房间呆着,那里还安全一点。不要四处走动,这里很危险。”

陆久感到哭笑不得。他当然知道这里危险,但他不需要这些人形的保护——再说这算是哪门子保护?用不了两小时这里就会被彻底攻陷,到时候所有人都是死路一条。

而且,这种情况下首先要做的,怎么想也该是问清来意吧。

看来N21战区的战术人形的想法比较单纯。

“不用了,我奉命来此是来带领你们突围的,我有调动被困部队的指挥权。我现在要了解战斗局势、人员装备和弹药补给的情况。”

“你是个指挥官?”那个人形的眼里放出了光彩。

“是的。”

“不,不可能。指挥官应该呆在指挥部。”那个人形摇了摇头,“况且,我们不能接受不明的人员的命令。”

“我是来自N-17战区的指战人员,名字是陆久。”陆久无奈地说道,“我受派雷特少校委托渗透入此地指挥各位防守和突围,少校已经派出大部队牵制外边的铁血大军,我们要和他里应外合。”

“可是……我不认识你。”那个人形依然没有打消疑虑。

“我听说过陆久。”旁边有个人形一边开火一边回头说道,“他是个非常出名的指挥官,少校经常提起他。但是不知道这个人是不是真的。”

“不可能吧?指挥官怎么会带着枪亲自出击?那可太危险了!”立即有另一个人形提出了反对意见。

“就是,我们又没有见过他。再说就算他是真的,没有少校的命令……”又有声音附和说。

陆久感到一阵头疼,他从来没想到过和其他部队的人形沟通是如此困难的事情。

“少校的枪。”V在旁边悄悄提醒陆久说道。

对了,把那东西给忘了,陆久一拍脑门。他从裤子上的口袋里取出派雷特少校给他的手枪,然后高高举起:

“如果你们还有疑虑,请看看这个,你们应该认识吧。”

人形少女们暂时停下了射击,的目光都集中在了陆久的手上。

“那是少校的枪!”陆久身边的小个子人形大声说道,“你从哪拿到的?!”

……不仅单纯,而且好像还有点迟钝。

面对这样的反应,就连V的表情都有些失望,陆久也终于失去了全部的耐心。

“没时间说这些废话了。”陆久叹了口气说。

“什么……”

“所有人员列队,稍息——立正!”陆久大喝了一声。听到这突如其来的命令,人形少女们下意识地站成了一排。

“我是贵部负责人授命的临时指挥官,我现在宣布由我来接收此地的指挥权、并负责指挥作战。”陆久严厉地说道,“你们如果想要活着回去,就一字不差地按照我说的做,听明白了吗?”

“可是,你还没有……”陆久身边的人形犹豫着说道。

“听明白了吗!?”

“明白,长官!”

“我现在需要马上了解战斗的情况。”陆久对着身边的人形说道,“你叫什么?”

“我叫……74u。”那个人形惶恐地回答道。

……AK74u吗,陆久有些诧异。想不到派瑞特少校手下的名字,竟然和自己品位类似呢。

“很好,74u。你们目前有多少人、配置了什么装备、主要交火位置在哪?”陆久打开了全息地图,投射出气象站内部的立体影像,“以及在这些被困的人员里,有在指挥战斗的人形吗?”

“具体情况我也说不清,但是有一位拥有D级指挥权限的人形,目前在负责调度……”74u回答道。

D级指挥权限,陆久心想。战术人形的指挥权是根据其配置的指挥模块授予的,这个等级的指挥权限只能带领十人以下的小队,要调度70名人形显然远远不够。

“马上去找她,让她来见我,如果有其他配置有指挥模块的人形也一起叫过来。其他人,回到自己的位置上去。”

“是,长官。”74u敬了个礼立即跑了出去,其他战术人形也纷纷归位。

片刻后,74u带着两个战术人形走了进来。陆久稍微打量了她们一番——两个高大的战术少女,都是欧洲人的形象、银色头发。其中一个骨架较宽,手里没有武器;另一个略显纤细,手里拿着一把和她一样纤细的SVD-M狙击步枪。

“您好,支援火力中队PK参上,这位是侦察小组的SVD。”没带武器的战术少女说道,“74已经向我简单说明了情况,您就是前来支援的友军吧。”

“是的。”陆久微微点头,他感觉这个人形的讲话方式比74u清楚多了。

“请问您有公司或者我战区营地负责人颁布的授权文件吗?”

“文件没有,手枪倒是有一把,不知道算不算数。”陆久说着再次取出派瑞特少校的手枪,递给自称PK的少女。

“这是少校的佩枪。”少女接过枪看了看,抬起头说,“毫无疑问您是少校信赖的人。请问您的名字是 ‘陆久’吗?”

“正是在下。”

“久仰大名,陆司令。”少女将手枪双手捧起交还给陆久,神色肃然地说道,“少校经常说起您,我一直很期待能和您见面……但是没想到会是在这样的场合。”

“……你是?”陆久有些纳闷,这个少女似乎认识自己,但自己却对她毫无印象。

“啊,您没有见过我。我是少校的副官。”

少校的副官?陆久感到吃惊。派瑞特少校可没有说自己的副官也被困在了这里,至于74u汇报的信息里遗落了这样重要事情,倒是其次了。

陆久开始理解派瑞特少校为何发出求援信号时为何那么焦急了,不过就他隐瞒了自己的副官也被困敌群这一点来看,派瑞特少校还是能沉得住气的。

没有想到,这个年轻的指挥官倒颇有几分度量。陆久内心暗自赞许。

“幸会,PK副官。”陆久说道,“这位是我的副官,名字是V。”

听到陆久介绍自己,V朝着PK微微点了点头。

“哦,您就是V副官。想不到……”听到陆久的介绍,PK的神色似乎有些惊奇,但是马上又严肃了起来,“唉,21战区也算是北部的一座大型要塞,但此刻竟然要在这种地方和二位会晤。真是招待不周,还请包涵。”

“不用客气了,现在也不是寒暄的时候。”陆久说道,“请问眼下战况如何?”

“是。我军现在被困在建筑之内总人形数76名,现存72名,其中57名尚有战斗能力、15名伤员里10名尚有行动能力。敌军人数大约有700名,将建筑团团包围,并在一楼六个出口处和我军猛烈交火。我已将自己所率领的机枪班分散部署在所有出口处形成火力点,依托建筑和临时工事,暂且将敌人的进攻阻挡住了。不过因为我无法指挥所有人形,所以现在有很多人形的作战效能无法发挥,而且机枪班的弹药供给已经捉襟见肘,最多再过半小时她们将无法提供持续的火力支援。届时敌军将涌入建筑,我们恐怕就很难对抗了。”

“知道了。我已经派出突击队去解除此地的通讯屏蔽,但是行动进度和能否成功目前还未知。另外,派瑞特少校已经倾其所有军力在外围组织攻势来牵制铁血的部队,但是等他们全部就位并击退铁血的进攻,可能还需要两三个小时的时间。这期间我们必须死守阵地,在少校进攻的时候配合他。”

“是吗,少校他……”听到陆久说少校带领大军前来解围,PK的眼里闪过一丝欣慰,“不过,两三个小时我们能够坚持过去吗。虽然有陆司令的协助,但是一旦失去机枪组的支援火力……”

“没有问题的。”陆久严肃地说道,“建筑物内空间狭小,只要合理部署,轻武器的火力也足够招待敌人了。我们现在的任务就是把人员组织起来,还有最重要的是——给她们胜利的信心。”

“有您指挥战斗的话,我已经有这样的信心了。”一直神情严肃的PK终于微微笑了笑,“请告诉我您的战斗方案,由我去传达吧。”

“很好。”陆久在墙上分层画出了气象站的草图开始作战部署,“我们现在需要加强交火地点的工事,让持有自动武器的人形也投入战斗,发挥她们的近距离火力优势。机枪班加强压制为她们争取一点时间,把建筑物内可用的掩护物堆在这里、这里和这里……然后在这些走廊里部署一些机动小组,留意可能渗透的敌军。二楼的窗户前布置装备了半自动武器的人员,三人一组相互掩护压制并干扰楼下的敌人。伤员置于建筑内侧的房间。楼顶需要有人观察敌军动向,时刻汇报情况……有人能做这件事吗?”

“就让SVD去吧。”PK说。

“可以。”陆久说着转向站在一旁的SVD,“你去楼顶观察,如非紧急不要开火。敌人如果有异常动向立刻告诉我,明白吗?”

一直沉默的SVD点了点头。

“很好,马上到你的位置上去吧。”

SVD依旧没有做声,只是敬了个礼就转身离去了。

“请陆司令别介意,这孩子不太喜欢说话。”PK说道。

“没什么。话太多有时也很麻烦。”想起聒噪的静默小组,陆久耸了耸肩,“那就这样执行吧,请你去把我刚才的部署传达下去。你带领机枪班做好火力掩护,如果出现情况……让那个74u来传令,多跑着点吧。我看她很适合干这个。”

“您真是慧眼识珠。”PK心领神会地一笑,“她的优点也就剩下腿脚麻利了。”

“好,开始行动。”陆久说着取下肩头的自动步枪,一拉枪栓将武器上膛。

“您……也要参加战斗吗。”PK轻声问道。

“啊,毕竟多一个人就多一份力。”

“如果我劝您不要去,您也不会听的吧。您真是与众不同,和少校说的一样。”PK点了点头,“正因为有您这样的指挥官,17区的队伍才会百战百胜……您真是名不虚传。”

“那都是以讹传讹。我也不过是个,扛枪打仗的罢了。”

“不,我看您比传言之中更加勇武。”PK认真地说着并抬手敬礼,“那就各就各位了——向您致敬,‘战地指挥官’同志。”

“也向你致敬。”陆久抬手回礼,PK转身快速走了出去。

“打起精神来,我们还没有失败!少校已经派出全部兵力来帮我们解围,支援已经在路上了!”陆久听到PK在对着战术人形们高声呼喊,“而且,现在我们中间已经来了一位人类指挥官,他就是大名鼎鼎的17区的总指挥官陆久!听到了吗?17区的‘战地指挥官’陆司令,正在我们中间现场指挥、并和我们协同作战!只要我们守住阵地,下午就能回家了!勇敢战斗吧士兵们,乌拉!”

“乌拉!”备受鼓舞的战术少女们异口同声地地回应道,高亢的呼声一时间掩盖了杂乱的枪声。

\section*{}

事情不太对。

奔行于战场各处的陆久忙里偷闲在一个房间的角落里坐了下来,看着面前正在交替射击的战术人形,心里想道。

防御组织得很成功,效果几乎可以说是完美。铁血的部队人数虽众,但是只知一味强攻的战术让他们在陆久设下的火力点上碰了钉子,进攻一次又一次地被击退。陆久在楼顶已经看到了地平线上逐渐集结的N21战区主力军,想必派瑞特少校马上就会下达总攻命令。届时如果铁血的部队若不撤退,那么用不了一小时就会被全歼在这个气象站外围。而在那个时刻到来之前,铁血是没法走进陆久固守的阵地一步的。

这看起来是上好的局势,他们只需要维持现在的状态就能坐等战斗结束。但在这平稳的局势之中,陆久却感到了一丝危机:事情似乎太过顺利了。

这些人形的作战效率很低,他们完全是在依照既定的预设模式进行作战。当然,这就是铁血的常规战术、也是他们的弱点,这种情况实属常见。但是这通常会被GK公司利用的弱点,此时却仿佛在昭示着什么……不同寻常的情况。

——几百名战术人形,全部都在自律作战?

这引起了陆久的注意。铁血的战术虽然一向简单粗暴,但是抛开战损不谈,他们的作战成效还是有的。他们经常出动大军对目标发动突袭,把目标打个措手不及然后趁机达到他们的战略目的。但这并不代表铁血就没有指战人员。

铁血也有类似指挥官的人形军官,而且这些军官的指挥能力并不输人类。

事实上,铁血的指挥人形比人类指挥官更难对付。因为他们不仅有着媲美人类的心智,还有着人类远远无法企及的计算速度、信息获取效率、以及通讯功率。

一个铁血的指挥人形,能够同一时间精确地指挥几百个只有简单战斗模块的作战人形,这和人类指挥官分散小组、各自为战的指挥模式大不相同。

这次行动怎么也不像是没有人指挥的样子。如果说围攻N21指挥部的行动只是佯攻,那么这次围困会不会也是佯攻呢?

陆久已经很长时间没有听到过有铁血的指挥人形出现了,这些“铁血指挥官”,到底在哪里、又在做着什么呢?陆久忽然心里一凛,他想起了一件事情。

……现在的时间已经接近中午,通讯依然没有恢复。95的队伍到底在干什么?

不会吧,莫非铁血在那边布了什么局?陆久心里一动,下意识地站了起来。身边的V也注意到陆久神情的变化,关切地看了过来。

难道95那边出问题了?不,不可能,陆久在心里细细思索着。95只带了有限的几个人过去,而且他对95下令不许冒险,就算行动不成功也总不至于陷入危险。况且就那么几个人,退一万步说就算是全军覆没,铁血又能从中得到什么好处?难道只有六个人的突击队,能比困在这里的七十多个人形更有价值?

这不可能,陆久心想。95的战斗力他是了解的,而且突击队十分灵活,就算是无法完成任务目标,但是想要全身而退还是没有问题的。她们一定是抵达了目标位置却发现行动无法进行,于是索性放弃了。

但陆久还是在心头感到挥之不去的不安。他甚至想要派出V去查看情况,但是考虑到从这里去往h6区域至少需要两个小时,而这段时间已经足够派瑞特少校的部队清缴面前的铁血部队,方才作罢。

别胡思乱想了,陆久对自己说。控制住这里的局面,至于95那边的事情,等一会儿和派瑞特少校会师后自然就知晓了。

正当陆久这样想着的时候,通讯器忽然发出了一阵响动。一大片通讯信号同时传来,汇成了集市里一般混乱的吵闹声。

什么情况?陆久一惊。他按下呼叫/接听按钮,通讯器开始搜索着波段,并自动连接上了派瑞特少校的电台。

“……喂,派瑞特呼叫。陆司令,喂?陆司令……能听到吗?请问……能听到吗?……”电台里传来了派瑞特少校的声音。

“陆久收到,请讲。”陆久回应道。

“陆司令,您听到了?太好了!”派瑞特少校高兴地说着,“您那边情况如何?”

……原来是通讯恢复了啊,陆久终于松了一口气。

这说明95那边的行动成功了。自己果然是在无谓地担心。

“情况良好,敌人全部在门外,一个都没进来。你那边呢?”

“我这里人员已经集结就绪,马上就要开始进攻。”派瑞特少校说,“您那边没事真是太好了,请再坚持一小会儿,我们马上就能会师了。”

“没问题,目前这里伤亡很小,坚持到晚上都可以。”陆久说,“你的副官也很安全,话说,她作战很英勇呢。”

“啊,是吗。您已经知道了啊……当然,我也想到这事瞒不住您。”派瑞特少校的声音稍微有点不好意思,“PK劳您搭救了。我欠您这个大人情,以后一定报答。”

“谈不上搭救、更不用报答。”陆久说道,“她协助我指挥战斗非常得力,而且在激励士气方面很有一套呢。很了不起呀。”

“呵呵,是呀。PK很擅长鼓舞士兵们。”派瑞特少校发出一声由衷的笑声,“您那边安全就好。我会在10分钟后发起进攻,届时首先会用炮火覆盖铁血所在的区域,请您让战斗人员找好掩护。”

“明白。炮击结束后我们从内部配合突围。”陆久说完结束了通话,长出了一口气。

总算该收工了……他心里想着,把电台调到了自己队伍的波段。

“突击队,陆久呼叫。收到请回答。”陆久对着话筒说道。

“……”

传来一阵轻微的电磁干扰声,但是没有任何回应。

“突击队,是否收到?”陆久再次呼叫。

“……”

“九五?”

“……”

依然没有答复。陆久皱起了眉头。

这是怎么了,为什么没有回应呢。莫非那边的信号依然被屏蔽着?

没有理由,这边信号恢复了,说明发生器已经摧毁了。既然发生器都被摧毁了,那么这个地区的信号都该恢复了才对。

不管那些了,陆久心想。总而言之95那边肯定是成功了,不然派瑞特少校的信号也过不来。既然如此,那么就没什么好担心的了。

不许冒险——自己下过这样的命令的没错。就算行动失败,也不会对战局造成太大影响,她该不会干什么危险的事情。95向来严格遵守命令,这一点陆久还是很放心的。

“95那边,依然没有回音吗。”V轻声问道。

“是啊,不知道什么情况。”陆久说,“不过既然通讯恢复,那说明她们已经成功了。我想是回战区营地了吧。”

听到陆久这么说,V点了点头。肯定是回去了吧。

“5分钟。”陆久看了一眼手腕上的计时器,“派瑞特少校会在5分钟后发起进攻——据说首先是炮击开路。我们分头通知一下各单位,让她们注意呆在掩护物后边不要被波及。炮击结束后准备突围。”

“是。”V说道。

“走吧。记得让士兵们相互传达一下。”陆久边说边朝着一个火力点走去,“这段行动大概要结束了,我们差不多也该撤了……”

5分钟后,总攻如约打响了。陆久首先在天空中看到一片闪烁的光点,那是$\SI{107}{mm}$火箭弹飞翔的景象。接着一阵呼啸声之后,猛烈的爆炸声在气象站外围响起。

这是一次三段式的步进轰炸,炮火的落点分三个位置,由外围逐渐向主建筑推进。陆久将大量兵力部署到了建筑面向轰炸区的一方,在炮击开始后对楼下的铁血部队发起了反击。铁血的人形们后边遭到了渐进的轰炸,面前又是格里芬部队的猛烈火力,在空旷的地带上一时间就连可靠的掩体都找不到。两轮炮击后,铁血的有生力量几乎损失了三分之一。

第三轮炮击的落点离建筑非常近,陆久命令士兵们全部进入建筑内部躲避炮火。剧烈的爆炸声在气象站的院子里响起,灼热的气流摇晃着大楼的墙壁、击碎了窗户上所有的玻璃。那些火箭弹几乎全部落在了大楼的门口,甚至有一颗落在了大楼的楼顶,炸毁了大楼的一角。

陆久坐在屋里点起了一根烟默默地抽着,他的肩膀上落了一层尘土。V和SVD静静地呆在陆久身边,面无表情、一言不发;而PK虽然也保持着沉默,但是神色明显有点紧张。

“不用担心,顶层的人员已经疏散了,屋里很安全。”陆久对PK说道。她大概是没有经历过这种大规模的炮火支援,特别是在如此近的距离上。

虽然这场炮击没有覆盖整座气象站,但是也够这些铁血的人形吃些苦头了,陆久心想。等到主力部队压上时,恐怕只剩下清理战场了。

炮火开路、装甲部队掩护、步兵协行……经典的西式作战套路。看来派瑞特少校,很喜欢大手笔啊。

几分钟之后,爆炸停止了。陆久露头往窗外一瞥,看到气象站的外围已经被炸得一片坑坑洼洼,铁血的伤亡相当可观。派瑞特少校的兵力正在从两侧围攻四散后退的铁血部队,气象大楼的周围已经没有了铁血的人形。

“PK,让士兵们整队。检查装备和弹药,我们该出门了。”

“是。”PK站起身说道,“要开始突围吗?”

“突围?”陆久说道,“啊。算是吧。不过如果出去的话,大概也么什么围好突了。”

人形们整队完毕后,陆久带头走了出去,V紧随其后。PK想要提醒陆久注意危险,但陆久只是淡然一笑。

“派瑞特少校,我们要离开建筑了。请让部队注意友军火力。”走出门口之前,陆久对着对讲机说道。

“放心吧,全都安排好了。”派瑞特少校回答道,“马上去接应您。”

片刻后,全部人形都安全撤离了气象大楼,两辆装甲车带着烟尘停在了大楼外面轰炸过后,依然有些发烫的地面上。

装甲车里走下来几名士兵,还有一个军官。陆久仔细一看,来人竟然是派瑞特少校本人。

“陆司令!”少校快步走了过来,“铁血的部队已经全部驱逐,这里安全了。您没有受伤吧?”

“没有,我很好……给你。”陆久说着从兜里掏出那把马卡洛夫手枪,在手上转了一圈递给派瑞特,“防御作战很成功,我方伤亡非常轻微。倒是你,怎么亲自来了?”

“我也想向您学习,亲身感受一下战场的温度啊。”少校接过手枪说道。

“你想感受一下倒没什么,但是千万别和别人说是向我学的。”陆久耸耸肩,“不然上峰该更看我不顺眼了。”

“呵,保证只字不提——不管他们信不信。”少校笑了笑,然后朝着陆久身后看去,目光落在了自己的副官PK身上。

PK也正看着少校,他们的眼神相遇后,副官侧身走出了队列,来到了少校面前。

“报告少校,战术人形PK参上。三营撤离完毕,可以归队了。”PK整理了一下银色的长发,对着少校说道。

“……你没事。”少校打量了一会儿PK,轻轻说道。

“我没事。”PK点了点头,“多亏陆司令指挥运筹。”

“没事就好。上车吧。”少校也点了点头,然后对着陆久说道,“陆司令和V副官,二位也请。”

PK登上了装甲车,陆久和V也跟了上去。当陆久走到车门前的时候,派瑞特少校亲自为他扶住了车门。

“谢谢您,陆司令。”少校用几不可闻的声音对陆久轻轻说道。

陆久没有回答,只是默默笑了笑。

“对了,突击队回去了吗。”登上战车之前,陆久忽然开口问道。这个问题让派瑞特少校也一愣。

“突击队?……我问一下。”少校说着拿起了对讲机,“指挥部,派瑞特呼叫,收到请回答。”

“是,少校。指挥部收到,请指示。”

“N17战区的突击队是否归营、当前是否在我军营地?”

“N17战区的军力?请稍等……报告少校,现在我部的友军,只有三个作战梯队和一个狙击小组驻守在外围。突击队出击之后,一直没有归营,也没有和我部进行联络。”

“派瑞特收到。完毕。”少校放下了对讲机然后转向陆久,“陆司令,突击队好像……还在勤务中。”

陆久把已经登上装甲车的脚拿了下来,他的神情变得严峻了。

“派瑞特少校,你在21区,听到过铁血指挥人形的消息吗。”

“不瞒您说,的确没有。”派瑞特少校的神色也严肃了起来,“怎么……难道说,还有我们不知道的情况?”

“秘密武器、大批人形,竟然没有指挥者。不觉得奇怪吗。”

“您是说……”

“报告……报告长官!”没等少校说完,忽然一辆全地形摩托车呼啦一声停在了装甲车旁,一个人形从上面跳了下来。

是陆久侦查连里的一个队员。

那个侦察连的人形气喘吁吁,被汗水打湿的头发紧贴在她俊俏的脸上,她显然匆忙地赶了很长的路。

“别慌。”陆久说道,“怎么了?”

“05……05队长,在h6区等您。”那个人形努力平复着呼吸说道,“让您过去……她希望您,马上就去……有紧急情况!”

陆久的心里涌上了一股不详的感觉。05是他事先派往N21战区调查情况的,不仅是h6区的情报,铁血的动态、气象站情况和地下管道的地图,全都来自于05。但是按照预定计划,此刻05在结束侦查后应该返回N17战区了,她为什么又去了h6区?更重要的是……为什么她不用电台联系?

95和突击队也一直联系不上,陆久隐隐感觉到这些事之间有着直接的关联。这紧急情况,难道和她们有关?

陆久感到心里一阵焦躁。

“我马上去。”陆久说着跨上了摩托车,“上车。”

后边那句话显然是对V说的。听到陆久的话,V毫不犹豫地坐上了摩托车。

“需要派人支援吗?”看到马上要投入行动的陆久,派瑞特少校说道。

“不,你把部队驻扎在气象站,部署下临时工事。”陆久发动了摩托车猛地一拧油门,裸露的引擎发出巨大的轰鸣声,“h6区域那边依然无法联系,我必须去查明情况。请等我的消息再决定下一步行动。”

\section*{}

在北部战区的平原上,全地形摩托车是一种非常常见的轻型车辆。它有着坚固耐用的车架、减震优良的悬挂,还有一部800毫升排量的四缸发动机。这是一种四驱的载具,五级变速、越野性能极佳,可以搭乘两名成员或者装载两百公斤的货物。全地形摩托车的功率很大,几乎可以媲美一辆小型汽车,但是它的速度却并不算快——因为要保证良好的越野性还要在战斗中有一定的防护能力,它的变速箱被设计成了拥有更强动力的模式。

‘战地指挥官’陆久带着他的副官,飞驰在21区崎岖的路上,他的短发已经被吹风得背了过去。

V把双肩紧贴在陆久背上,双手用力抓着车身上的扶手,以便把自己固定在正在上下跳动的摩托车上;而陆久也在夹紧双腿,好让自己不被这匹狂野的铁马甩出去。时速八十公里,恐怕还远称不上是“飞驰”,不过这已经是这部摩托车的极限。而从另一个角度说,在这种路上如果还要再快一点的话,这辆车恐怕真的就要飞起来了。

陆久已经把油门拧到了头,但也只能达到这种速度。他身后的少女虽然相当难受,但是感受到指挥官的焦急心情,她选择了沉默着什么都没说——“稳一点”、“慢一点”,这样的事情现在已经不是陆久会考虑的东西了,他心里唯一迫切想要了解的事情,就是h6区域到底发生了什么。

一个多小时之后,他们来到了h6区域的预定坐标。

这是一片临时搭建的基地,有着典型的“铁血工造”风格:钢铁支架的篱笆搭上帆布构成了简易的外墙,支架顶端拉着铁丝网。帆布外墙之内是一堆涂着迷彩图案的集装箱,相互拼接构成了两层楼房高的指挥所,再盖上伪装网。指挥所后面还竖立着一根高高的天线。

虽然不适合长时间使用,但这样的建筑方式效率极高,而且也具有一定程度的防御和伪装能力。

陆久把摩托车在铁血的基地门前,跳下了摩托车。他看到有一个人正站在那里——是05。

05的脚下还躺着两个被摧毁的铁血人形,看来这里发生过战斗。

看到陆久到来,05默不作声地看了一眼手腕上的计时器。

“汇报情况。”陆久走到05面前,开口说道。

05没有说话,也没有向陆久敬礼。她面色凝重地看了陆久一眼,然后低下了头看向地面。片刻后,她抬起头,伸手指了指铁血的指挥部。

“里边。”05说道。

05的样子很怪异,一点都没有平时在陆久面前恭敬的样子,简直一反常态——这一点陆也注意到了。但是05什么都不说,陆久也弄不清到底发生了什么。

陆久知道05的性格,她是个不会开玩笑的人。如果她什么都不说的话,那可能是因为她不想说、或者没什么可说的,那么就算问她,也是没有用的。

于是陆久点了点头,朝着铁血的指挥部里走去。

“站住。”陆久听到05在身后说道,于是停了下来。但他马上意识到,那句话不是对着自己,而是对着试图跟上来的V说的。

“只有司令能进去。”05低声说。

“在情况不明的时候,我有责任保护司令的安全。”V淡然说道。

“里边很安全。你没必要跟着。”

“我受公司委派,负有自己的使命。除了司令之外我不接受其他人的指示。”

“……”05没有再说话。她看似漫不经心地摆弄着手里的手枪,发出咔哒一声轻响。

陆久知道,那是解除保险的声音。他侧过身,看到V也拉开了战斗的姿态,身体微微前倾、手按在了腰后的匕首握柄上。

“你们在干什么?”陆久看着剑拔弩张的两个人,低声喝道。

听到陆久满含怒意的质问,两个人都收起了架势。

陆久瞪了05一眼,然后对着V说道:“在这里等着。”

他看到05用冰冷的眼神看了看V,然后转过了目光。V则听从了命令,沉默地退到了旁边。

陆久确定她们之间不会再发生冲突后,继续朝着铁血的指挥部走去。

铁血指挥部内部的装修风格如外边一样冷硬,通道相当狭窄、而且灯光也不怎么明亮,这让陆久想起了他曾经被囚禁的那间囚牢。

说起来,一切都是从那里开始的……陆久心想。但是眼前的建筑不仅是冷硬,而且还非常阴森:墙壁上沾着点点血迹,地上也是一滩一滩的血,但却看不到一具尸体。这瘆人的景象让陆久都感到汗毛倒竖,如果不是05说里边是安全的,他现在已经拔出手枪来了。

陆久走到了建筑内部的尽头,来到一间还算宽阔的房间里。这个房间显然是指挥部的中枢,布置着各种各样的通讯器材和设施,房间的中央还有一个巨大的屏幕。房间里十分凌乱,文件和纸片扔得到处都是,不知道是因为匆忙离开而没来得及收拾,还是因为发生了战斗。陆久在房间中环顾了一圈,没有发现什么值得注意的东西,这里没有人也没有重要设施——什么都没有。

陆久有些困惑,05为何要让自己来这里,又为何不让V跟随呢。他想起05看着自己时那凝重的神色,知道肯定是发生了什么严重的事情。但到底是什么呢?

“太好了,司令。您来了……赶上了呢。”

忽然,陆久的背后响起了一个说话的声音,把陆久吓了一跳。他迅速转身,看到身后的屏幕不知何时亮了起来,声音正是从那里传来的——而屏幕里的,是95的脸。

“……九五?”陆久疑惑地说道,他注意到95的脸上沾着血迹,“你……在哪?”

“您的下方……位于地下的安全屋里。”

“你被困在里边了吗?”

“不是,是我自己把自己锁在里边的。”

“这里没有敌军,你可以出来了。”

“不……我不能出去。”

“为什么?”

“……”

95没有说话,她咬了咬嘴唇,慢慢往后退了几步。

让开了被遮挡的区域后,陆久终于看清了95身处房间的全貌:那是一座没有什么摆设的房屋,四壁都是灰色的,但却沾染着一片一片的血迹。

95的身上也全都是血,她显然受了重伤。而在她的身后,是六具扭曲的尸体——那是95的突击队员。

“你……这是怎么了?还有这些突击队员……这是谁干的!?”面对面前的惨烈景象,就连陆久都震惊得瞪大了眼睛。虽然地狱般的场景他见得已经厌倦了,但是他无法想象,是什么人能够连续杀死六个突击队员、重伤95后还能全身而退。

“……您马上就会知道。”95轻声说着,“但我首先要向您汇报……最重要的信息……”

“你别说话!我马上就带你回去,05和V都在门口,你不会有事的。把安全屋打开!”

“不……已经太迟了。”95微微摇了摇头,“我……已经,被感染了。”

“感染?”听到这个词,陆久呆住了。战术人形不会像人类一样感染疾病,那个词是什么意思?

“是的。铁血的秘密武器……就是它。这是一种专门针对……战术人形的病毒武器。它通过两种方式传播:一是通过植入病毒的通讯系统,远距离感染接听电台的人形……二是通过被感染人形的核心,不断发出电波对附近的人形进行转录。这种病毒会入侵人形的自律核心,一旦感染这种病毒的人形在小范围内聚集到一定数量,就会群体性地爆发。它能扰乱人形的敌我识别系统,让人形认为附近的所有目标都是敌人,进而无差别地攻击……”

“你是说,这些突击队员……”

“是的……她们是,自相残杀……而死的。但是人类不会受这种病毒影响,所以05将您找来,是希望您能亲自了解情况。这种病毒的二期效果还会阻断人形大脑对身体的控制,再过一会儿我可能就无法控制自己的身体了。虽然我已经受伤无法自由活动,但我的核心还在不断地向周围散播病毒,靠近我的人形都有可能被感染。”

“没关系,我这就让战术人形全部撤离……我来将你带回公司进行诊断维修。”

“没用的,铁血为了防止我们破解这种病毒,早就留下了后手。病毒激活后,如果感染人形已经无法运作、或者附近没有其他的感染人形,一段时间后就会对核心发出自毁程序,熔毁的核心会连人形的神经系统一起破坏掉。无论如何,感染病毒的人形只有死路一条。”

“怎么会这样?……不,冷静下来仔细想想,一定还有其他的办法!”

“是的,的确还有其他的办法……” 95凄然一笑,“那就是在人形的核心自毁前将其摘除。这样就能采集完好的病毒,回去再作研究。” 

陆久沉默了。他知道这意味着什么:在没有专业的神经桥接设备的情况下摘除人形的自律核心,无异于直接摘除人类的心脏——就算能够保证核心的完好,但人形将必死无疑。

“……不行,不能这样。”陆久低声说道。

“我们必须这样做。只有这样,才能破解铁血的秘密武器,避免更大的伤亡。”

“我们他$\times$的不能这么做!”陆久大吼了起来。

“是啊,您做不到……您是那样温柔的一个人,就连女孩子的请求都不忍心拒绝,怎么可能对自己的部下痛下杀手呢。我当然知道。所以……就由我自己来吧。”95说着笑了笑,从腰间取出了手枪指向了自己的太阳穴,“人形彻底停止运作大约15分钟后核心开始自毁。所以在那之前您一定要狠下心来,不然,我可就白白牺牲了哦。”

“不、不不不不,别这样!”陆久慌乱地大声喊道,“放下那把该死的枪!我命令你放下!”

“对不起,这次不能服从您的命令了呢。”

“我请求你,九五。请别这样!放下枪!”

95的眼睛里流出了泪水。

“对不起,陆司令。我愿意答应您的任何请求,哪怕是刀山火海我也在所不辞……但是,这次真的不行……”

“别这样,九五。”陆久竭力控制着自己的情绪,努力让声音平静下来,“别让我再一次眼看着自己的战友死去。我不想失去你。”

“我也不想离开您,如果还有机会、如果还有选择,我想永远都和您在一起。可是如果我不这样做,就会有更多的战友死去。您曾经说过,我们都是士兵,承担这样的痛苦就是我们的义务。所以请不要悲伤、更不要为此而自责。您已经做得够多了。”95擦去脸上的泪水,微笑着说道,“知道吗,此刻我已经无法识别您的身份了。我的核心一直在对我发出攻击指令,我大概很快就要无法控制自己了……可是听到您的声音,依然是那么的让人安心。身为人形的短暂一生里,能够遇到您,我真是太幸运了。在最后时刻能够这样告别,我没有任何的遗憾……永别了,陆久。我……我爱你……”

“九五。九五!”

陆久感到自己的耳朵里响起一阵嗡鸣,他就连自己喊出的话都没有听到。眼前的一切瞬间都变慢了,本该连续的情景变成了一帧一帧的幻灯片。他看到95手里的枪口微弱的火光一闪,然后,她的头部一侧绽开了一朵红色的花朵。

95的身体缓缓倒了下去,犹如一片悄然飘落的花瓣。

陆久用手按住前胸跪在了地上,他感到胸前仿佛有万吨的重物压着,让他无法呼吸。他的五官拧成了一团,紧紧地咬住牙关,生怕一松口就会流出泪来。

陆久想要站起来,但是他仿佛全身都失去了力量,双腿无论如何都不听使唤。于是他索性额头抵地伏在了那里。干脆就躺下吧,他想。不必再起来了。

但是片刻之后,他还是摇晃着站起了身,走向前边不知道什么时候打开的安全屋的入口。

95的身躯静静躺在那里。虽然战斗服上浸透了血迹,但是她的表情十分平静,仿佛睡着了一样。如果不是六具突击队员的尸体在旁边陪伴,陆久根本无法相信刚才发生的事情。

15分钟,陆久对自己说。这用95的命换来的15分钟,自己已经浪费了一半——必须做该做的事情,快一点。

可是……不行。

她的生命已经溘然消陨,自己还要去破坏她的遗体吗?她的容颜已是如此苍白,自己还要去再刺上一刀吗?

不。为什么要这么做呢,陆久问自己。人类、公司、这个世界,真的有那么重要吗?

比自己最忠诚的伙伴的生命,更加重要吗??

他想要保护的只有区区那么一点,可到头来还是保护不了。他只想把自己的士兵带回家,可到头来还是只能眼睁睁看着她们死在自己面前。这一切到底是为了什么?

他想起他们初次相拥而眠的那个冬夜、想起他们一起经历的万万千千。

早知道是这样,那时就抱紧她一点。

早知道是这样,那时就接受她的好意。

早知道是这样,那时就不派兵去支援。

早知道是这样,那时就不让她带队出击。

早知道是这样……

陆久的拳头用力锤在了冰冷的金属地板上,发出砰的一声闷响。

天气预报说,今天午后会有雪,这该是今年冬天的第一场雪吧。95最喜欢下雪了。她就像一片天空中飞舞的雪花,纯洁无暇、不知温暖为何物……待到春天来临、万物复苏,却已经悄然消融。

要伤害她,陆久做不到,哪怕只是一具失去了生命躯体。

做不到,他在心里对自己说着。做不到、做不到、做不到……做不到。

可是,为了不辜负她的牺牲,他必须做到。陆久咬紧牙关,决然站起了身——

绝不能,让她白白死去。

强忍住想要把她紧抱在怀里的愿望,陆久用正在不住颤抖着的手,抽出了战术匕首。

\section*{}

V永远都无法忘记那一刻的情景。

当陆久走出铁血的指挥部的时候,天空正开始慢慢飘起雪花。陆久的表情狰狞、脸上全都是血,仿佛一头来自地狱的恶魔。V想要迎上去询问发生了什么,双脚却如同被钉在地上一般一步也挪不动,因为她看清了陆久怀里抱着的躯体。

而05则直直地盯着前方,眼光片刻都不曾看向陆久。

“死在这里的,本应该是你。”

当陆久走过V身边的时候,V听到05这样低声对自己说道。

陆久站住了,他显然也听到了那句话。但他随即继续向前走去,一直走到了摩托车旁,把95的遗体轻轻放在地上。接着,仿佛害怕她受凉一样,陆久又脱下自己的外套披在95那已经失去生命的身体上。

然后,陆久走了过来,停在了05的面前。

碰!!

V听到一声闷响,05瘦弱的躯体被陆久一拳打得飞出几米远。陆久的右手滴着血,他的手掌大概也因为那全力的一击而骨折了。但他依然伸手把05从地面上提了起来。

V想要冲过去拉住陆久,但又停了下来。因为她感到一丝恐惧——她看到陆久脸上的表情,仿佛想要把05撕成碎片。

但陆久终于松开了手,任由05跌落在地上。

“没有人死得理所当然,”V听到陆久用嘶哑的声音说道,“也绝不会有人死得一文不值。”

05挣扎着爬了起来,她的鼻梁被打歪,下巴也脱臼了。她双手捧住自己的下颌,V听到咔吧一声轻响。

呸,05吐出一口混着血的唾沫,脸上露出一个冷笑。

“一通废话,”她满是嘲讽地说道,“不说也罢。”

V本以为陆久会被这话惹得更加愤怒,但他什么都没说。

“去找派瑞特,让他来,带两辆车过来。”陆久转身对着V说道。

“……是。”V回应道。她看到陆久脸上的狰狞的表情忽然之间消失了,没有了刚才的愤怒和狂躁,他的面色平静得如同什么都没有发生过一样。只是那双眼睛里,凝结着看不到底的悲伤。

当V跨上摩托车的时候,她看到05已经解下了身上的武器,放在了脚边。

“感谢您长期以来的收留,陆司令。”V听到05对陆久说,“现在,我向您申请,退出在N17战区驻军的服役。”

“批准。”陆久说道,“你走吧。”

05转身离开了。但是她走了几步,又返了回来,从兜里掏出一个小小的物件递到了陆久面前。

“什么。”陆久说。

05没有立即回答,而是默默地看了陆久一阵。

“通向往昔之门的钥匙。”她说。

陆久看了看05,然后一言不发地接过她手里的东西,放进了口袋。

05再次转身,朝着N21战区边缘的方向慢慢走去。没有说原因、也没有说去向,她就这样无言地离去了,一直到消失在渐紧的风雪之中,她都没有再回头。

当V和派瑞特少校赶到h6区的铁血基地的时候,雪花已经大如鹅毛一般。陆久坐在地上,怀里抱着95的躯体,身上已经被雪盖了厚厚一层。在他的身边,整齐排列着六个突击队员的遗体。

“陆司令!你怎么……”派瑞特少校惊异地走了过来。当他看到陆久身边的遗体和他怀里抱着的人时,派瑞特少校明白了一切。

他站在陆久的面前,缓缓摘下了头上的军帽。

“……我很遗憾。”少校垂下了头,低声说道。

“没什么。这就是军人的宿命。”陆久淡然答道,“我们都会有这么一天的。”

“来人,把这些烈士的遗体放到车上。”派瑞特少校对着随行的人形说道。

几个人形走了过来,轻轻地抬起地上的突击队员遗体装进了车辆里。

“我们……先回营地吧。”派瑞特少校轻声对陆久说道。

“不。”陆久终于抬起了头,看着少校说道,“派瑞特,虽然你是此地的总指挥官,但是我必须要向你下达一个命令。”

“听您吩咐,陆司令。”少校答道。

“到这座指挥部最里边的屋子里,桌子上放着一枚人形的自律核心。”陆久低声说着,“那个核心被一种专门针对人形的危险病毒感染了,但是对人类却没有影响——那就是铁血的秘密武器。你设法把这枚核心送到公司,并对他们说明情况,看看能否找到破解的办法。我不确定这枚核心是否还有感染性,所以不要让任何人形接近。事态紧急,请立刻行动……这里剩下的事情我来处理。”

“什么?这……那枚核心是……”派瑞特少校惊呼了一声,但当他看到陆久抱着的躯体时,他明白了过来。

“知道了。”少校说道,“我马上就去。”

说着,派瑞特少校快步朝着铁血指挥部走去。

“薇。”待派瑞特少校离开后,陆久对着V低声说道。

“是,司令。”V立即回应道。

“去向驻扎在N21战区营地的三个梯队传达我的命令:间隔200米距离,包围N21战区指挥部的主建筑,切断建筑的电源、禁止任何人形出入。”

“……什么?”听到这样的命令,V感到不知所措——包围友军的总部?

但陆久并没有理会她。

“建筑内的所有人形都是敌对目标,而且极度危险。”陆久站起身,抱着95的躯体朝那辆装有突击队员遗体的装甲车走去,“如有试图离开建筑的人形,不需做任何警告——格杀勿论。”

\section*{}

80分钟前。

陆久伸手擦了一把脸上的汗水,手上的鲜血抹了他一脸。他为95整理好军服,把她打扮成平静离去的样子,努力不去想刚才所做的事情。做完这一切之后,陆久转过了身。

“出来吧,鼠辈。”他对着铁血指挥部里的通讯器低声喝道。

“……”

“别躲躲藏藏的,我知道你在看。”

“哎呀,来了来了。”铁血指挥部的通讯器里传来一个轻佻的声音,“我不是诚心默不作声,因为爱人之间的生死离别实在是太感人了,我还一直沉浸其中呢。”

陆久看到屏幕里有个娇艳的女孩,脸上带着俏皮的笑容。她的皮肤雪白,让那一头黑色长发更加显眼,相貌十分精致。

陆久大概不会把这个女孩当做有害的事情,如果现在她不是正戴着N17战区的臂章的话——陆久可不记得自己手下有这么一号人物。

“你以为装模作样地躲在21区指挥部,就能神不知鬼不觉了吗。”

“啊?已经被发现了吗?不愧是陆司令,果然名不虚传啊。那就正式自我介绍一下吧——我就是铁血工造派来此地的指挥者,人称‘播音员’是也!”女孩高兴地说道,“我本来还以为能给您一个惊喜呢。不知道我到底是哪里露出马脚了呢?”

“一开始我就注意到了,只不过我当时没有在意……在N21营地外边你用那辆装甲车吸引了所有人的注意力,然后扮作我的人形混进了N21区营地,对吧。”

“是啊,那时候的进攻就是为了让我渗透进战区指挥部。你的支援可真及时啊,不过支援的不是你的友军,而是支援了我。那一片混乱的战斗里,没有任何人留意到我。其实我一开始还发愁怎么伪装自己呢,结果还是你给了我灵感哟。当我自称是N17战区的人形溜进指挥部时,他们竟然丝毫没有怀疑,嘻嘻。”

“你屏蔽了d3区域的信号,然后在指挥部的电台里上传了病毒,想要在少校下达命令的时候感染d3区的人形,是吗。”

“完全正确,这样我就可以里应外合地感染N21战区的所有人形,把他们一网打尽!可是我没想到那个平日里缩头缩脑的少校,竟然被你愚蠢的英雄主义打动,亲自带兵去往了d3区,真是让我大伤脑筋呢。如果他不用被感染的电台对他的人形下达命令,我可就毫无办法了。看到我的计划落空后,我只能改变自己的目标咯。”

原来如此,陆久心想。少校并没有和被困的部队联系,在指挥战斗的是自己——这种病毒只能对人形感染,如果通讯链之间出现了人类,那么病毒的传播就被阻断了。

“所以你设法让突击队感染了病毒,是想要把N17战区作为目标。”

“没错!我模仿了你的声音在控制台说话,你的突击队长果然中计,拿起通讯器接听时我趁机把病毒注入了她的核心,哈哈!我本想让她把病毒带回N17区的,可惜感染病毒的人形太多,病毒自动爆发了……哎呀呀,真是设计失误,这种试制的病毒要是加一个手动抑制程序就好了。不过你的突击队长可真了不起啊,在我告诉她病毒的事情后,她没有向任何人求助,而是切断了对外的通讯,并且把自己和自己的队员锁在了安全屋防止病毒扩散。”

“你为什么要告诉她那些?”

“我想看看她们的反应嘛。反正她都已经死到临头了,让我找点乐子还不行吗?”播音员欢快地说道,“对了,我把她们最后的时光录下来了,非常精彩哟!你一定很想看吧?别客气,来吧看看吧。”

陆久还没来得及说什么,面前的屏幕一闪,切换成了播放模式。

陆久看到几个被感染的突击队员正在互相攻击,95试图制服她们,却引起了她们的围攻。95一开始只是勉强招架着,尽力不使用致命的武器。但在确认那几个队员脸上已经没有一丝的理智之后,她彻底绝望了。

陆久看到了95杀死那几个突击队员的一幕:她向后跃开一小段距离,然后弓身将身体前倾、握住了背后的匕首,这动作和陆久的攻击动作完全如出一辙。接着,95拔出了匕首——然后发生的仅仅是一瞬间的事情。

她像一道闪电一般穿梭在几个突击队员中间,只用了不到一分钟的时间,就刺穿了她们几个的后脑。

虽然沉浸在巨大的悲痛之中,陆久还是感到一丝惊讶。他知道95不喜欢近身格斗,却没想到她的格斗技巧竟然如此娴熟。

“知道吗,一直到你来到基地的时候她还在抱着膝盖哭呢。但是看到你出现,她急忙擦干了泪水、整理好姿容。嘿嘿,看来她真的很在意在你面前的形象啊。”播音员不失时机地解说道,“哦,还有你的侦查员也不简单,竟然顺藤摸瓜地理清了我的计策,话说你手下可真有点能人啊?可惜她还是来迟一步,等她赶到这里的时候,那几个突击队员已经完全感染了。”

“……你的计划至此,已经彻底破产了。”

“唔,这次行动的确不能说是成功,不过我还是有点作为的。虽然没能消灭N21战区的主力部队,但至少现在指挥部还在我手里。”

“指挥部里还有活人吗?”

“那我就不知道了。当你让少校询问突击队的事情的时候,我感觉事情大概是快要瞒不住了,就感染了指挥部里所有的人形,然后把自己锁在了控制室——毕竟这些不分敌我无差别攻击的人形是很危险的,我也怕自己被她们杀掉啊。不过外边的惨叫声已经很久没有响起了,估计是都被杀光了吧。至少人类都该死光了。”

“你别想离开了,我马上就会去找你的。”

“哎呀哎呀,您要来找我啊!”播音员的声音忽然兴奋了起来,“那么您会把我怎么样呢,杀掉我吗?折磨我吗?还是用您那强健粗壮的男性躯体蹂躏我呢?我还真有些期待呢。告诉您吧,我在向您的突击队长的核心里注入病毒的时候,顺便浏览了她的记忆。我在她记忆里看过了您的形象,这么说有点不好意思,但是您的身体,正是我喜欢的类型呢……哎呀,真是让人害羞,我竟然说出来了。我现在就感到自己的心在砰砰乱跳……和您的突击队长记忆里的感受完全一样呢!难道这就是恋爱的感觉吗?”

“……住口,不许窥探她的内心!”陆久强压着心头的怒火说道。

“干嘛啊,怎么生气了?我说的是实话啊。您知道吗……啊,您大概还不知道吧?您的突击队长对您可是一往情深。想起来她对您的幻想,甚至让我这样没有廉耻观念的存在都感到害羞,虽然人形没有生殖能力,但她也想过为您繁育后代呢。可惜啊可惜,她最后也没能得到您的一点宠幸,虽然一生对您痴心一片,却一直到完全损毁您也没肯一亲芳泽、略施雨露……”

“……住口,混蛋!!”陆久暴怒地吼了起来。

“呵呵,哈哈哈哈……看看,这就是你们人类的丑态、也是你们人类就是一个低级物种的证明。你们总是脱离不了情感的束缚,轻而易举地就被挑起情绪,不过后果地就被感性支配……如此情绪波动的你,现在还能做出客观地决断吗?被愤怒充满的你,现在还能做出正确的选择吗?你想杀了我吧,你一定很想杀了我吧?别说你没有。可是杀了我又有什么用呢,你以为我会害怕?我的心智备份在铁血的云端,存在于现世的只是一个躯壳,死是威胁不了我的,你根本拿我无可奈何!哈哈,真让人好奇,你到底能把我怎么样呢?”

“如果人类是一种低级物种的话,你又算是什么呢。据我所知,除了主脑本身,其他的智能都是来自人类的意识。别告诉我你是在电脑里被运算出来的。”

“就算我曾经是人类,但此刻的我也是更高级的人类、是摒弃了那些根植于本性的缺点的人类、是将来替代你们的完美人类。我拥有人类的意识,但是没有人类的弱点,特别是情感这种毫无意义的东西!”

“你错了。情感并不是人类的弱点,而是组成一个完整人类的重要部分,你这样的东西只是一段代码,根本不能称之为人。虽然情感有时会左右人类的判断,但情感在某些时候,也能够让人发挥出超越自身的潜力。人与人之间的羁绊,才是真正的力量之源。”

“可笑至极!情感能干什么,你告诉我情感能干什么?能让你不怕子弹吗、能让你思考速度超过计算机吗!能让你……永生不死吗?!啊——对了,说起来,您的突击队长似乎也有着饱满的情感呢。请您告诉我,她是不是也算是人类了呢?”

“如果你真的看过了她的所有记忆,你应该知道我会怎样回答。”

“哎呦哎呦,我的确想到了你会这么说——又是那套‘生而为人’的理论吗?简直让人感到恶心!一具仿制人类的躯体、几个简单的自律程序,再加上满脑子无聊又可笑的感情,就能够称之为‘人’了吗?如果我不算是人类,那她就更不是了——就算在人形之中也只能是一件次品!”

“呵,你的认识实在是太浅薄了,难以相信你竟然也曾经是人类。你把灵魂卖给铁血的时候一定还很年幼吧?”陆久冷笑了一声,“你可能不知道,人类有些劣性确实是难以去除的,只要你作为人类而生,不管你现在是电子意识还是计算机运算,那些东西永远会埋在你的心底深处无法消除。”

“哈哈,啊哈哈哈,这番高论倒是让我耳目一新!”通讯员过于激动的声音里已经不知是兴奋还是愤怒,“我连自己的记忆都可以修改,敢问到底是什么什么东西是我改不了又去不掉的呢?嗯?告诉我啊,告诉我吧,快告诉我啊!”

“想知道吗。稍安勿躁,很快我就会当面传授给你。” 

\section*{}

“司令,我需要提醒您,这可能会被当做反叛行为。”V淡淡地说道。

“我知道。所有责任我一个人承担,你可以离开了。”陆久点了点头回答道。

陆久回到N21基地的第一件事,就是解除了装甲车里几个N21战区人形的武装,并把她们锁在了车里。然后,他拿起自己的40式突击步枪,穿过三个梯队的包围圈走到了N21战区指挥部的门前——现在他要独自攻入N21战区的指挥中心。

N21战区的主力军依然驻扎在d3区域,而派瑞特少校匆忙赶往公司后,陆久下令封锁了他的指挥部。如果派瑞特知道陆久所谓的“剩下的事情由他处理”是这样一种情况,他肯定会惊掉眼镜。

不过没关系,等他回来的时候,一切都结束了。

“我说了,你可以离开了。”陆久对着身边的少女说道。虽然没有再说话,但V依然紧跟在陆久的身后。

“我拒绝。”V说道。

“你这是在抗命。”陆久说。

“根据我从公司得到的命令,我必须确保您的人身安全。和这一原则相悖的命令我有权不执行。”

……这段对白好像似曾相识,陆久心想。

“听着,这座建筑之中有着数量不明的战术人形,她们很可能都已经感染了铁血散布的病毒。这种病毒会侵入人形的自律核心,并让她们无法识别敌我、对一切目标进行无差别攻击,十分危险。走进这扇门之后,我不知道会发生什么。”

“因此您才更需要我的保护。”

“……我所说的 ‘危险’,不是指那些人形,而是指病毒。”陆久终于转过了身,看着V说道,“这种病毒不会对人类产生影响,只会感染人形。你进去之后也有被感染的可能。”

“没关系,据我所知这种病毒只有在特定情况下才会爆发,05都告诉过我了。”

“……你也想‘立功’,是吗。”陆久冷冷地说道。

“我……”V低下了头,她的语气犹豫了。她知道陆久所谓的“立功”指的是什么。

“我也想自己立一次战功呢”。这句话依然回荡在耳边,可笑着说出这句的人却已经不在了。

“不,我不想。”V抬起头说道,“但是我依然坚持自己的使命。”

“如果前面只有死路一条,也没关系吗?”

“没关系。”V轻声说,“如果这个我不能用了,再重新制造一个就是了。反正……我也不是第一个了。”

陆久侧过身看向V,V默默转过了头。

陆久脸上的表情,她曾经见过。她说不清自己是第几次让陆久露出那样的神色,那种仿佛被什么刺痛了一样的表情。

她也不知道为什么自己一定要说出那样的话,她明知道那么说会让陆久感到痛苦——陆久已经够难受了,他刚刚失去了自己最优秀的战士,和最心爱的女人。

但仿佛为了证明自己的存在一般,她还是忍不住说出了刺伤陆久的话。他们之间的关系仿佛又回到了她初来乍到N17战区的时候。

“想找死的话,就尽管跟来吧。”陆久低声说,“不过别说什么第一个第二个的。在我印象中,‘薇’只有一个。”

说着,陆久抬腿踢开了指挥部的大门。

枪声一旦响起就不会停下,从陆久射杀第一个N21战区的人形开始,事情就已经无法逆转。

病毒的存在,目前还只是陆久的一面之词,毕竟所有的知情人都已经死了。但他却在未竟许可的情况下擅自展开了行动。

就算最终他汇报的情况得到证实,但他在未遭到主动攻击的情况下,应该等待公司的指示。所以,屠杀友军的罪名已经难以避免。

但陆久毫不在乎。

他本来就是背负着“屠杀”罪名的人,这一点他从来不曾忘记。那么他现在做的一切,不过是在自己屠杀的名单上再多加几个名字罢了。

派瑞特回来之后会怎么想呢,这是陆久唯一稍有在意的事情。毕竟那个年轻人是那么地尊敬和信任自己,而自己也对他有些赏识。但那一切和自己要做的事情相比,只是一些微不足道的细节。

到底什么才是“人”?

人形是人吗、铁血是人吗。自己……是人吗?

存在于他心里的结论,他要亲自去验证。

“你留在这里。”

解决了最后一个挡在指挥室门前的人形后,陆久对着V说道。

“不行,里面……”V还想继续对抗陆久的命令,但却迎来了一个凶狠的眼神。

那个眼神是如此地可怕,甚至让V下意识地后退了一步。

愤怒、哀戚还有悲伤。这些东西混杂在一起,形成了一种让人寒冷的东西——

一种就连空气都足以冻结的决意。

“留在这里。”陆久轻声说道。忽然之间,他眼神里让人恐惧的东西消失了,声音也变得仿佛是恳求。

“……好。”终于,V顺从地点了点头。

陆久也点了点头,表情犹如感激。

“来了、来了!我等了好久,您终于来了!哈哈,哈哈哈哈……”

当陆久走进指挥室并轻轻锁上门之后,V听到里边传来一个近乎歇斯底里的笑声。那个声音亢奋地尖叫着,因为过度地激动而语无伦次,不停嘶喊着语义不明的话,但陆久一句话都没有回应。

过了很长时间,当屋里彻底安静下来之后,V终于听到陆久开口说话了:

“在我的那个时代,人们通过一种测试来判断一个人工智能是否具有了人格,能够通过测试的智能将被当做是人类的意识。那是一种相当复杂的测试,因为它的发明者叫做图灵,所以这个测试又叫做‘图灵测试’。但是今天我要进行另一种测试,测试一下一个自称是更高等形式的生命体的意识,是否归根结底还是她所鄙夷的生命形式——人类。这个测试依然会相当复杂、而且也许还有点天马行空。但我个人认为这会是行之有效的测试方案。”

“很可惜,现在没有任何相关的科学家能够见证和评定这个测试的可靠程度,而且我也不能把这项测试的具体细节流传出去。不过等到这个测试结束后,人们一定会感到信服的,因为他们会发现这个测试的结果无懈可击。所以如果后人哪天在学术论证中提起这个不太可能会再次重演的测试的话,我希望它能有一个合适的名字……嗯,按照经典的命名法则——就把它叫做‘陆久测试’吧。”

\section*{}

“我以为你会杀了她。”

一个小时之后,当陆久面色漠然地走出指挥室的时候,一直等在门口的副官在他背后轻声说道。

屋里的敌人已经被击垮,至少在精神上被彻底击垮了。她抱着膝盖紧紧瑟缩在角落里,不用说反抗,就连声音也不敢出。

“她已经死了。恐惧已经在她心里生根,她被我们抓在手心了。从那一刻开始,她就死了。”

“我是说,从肉体上消灭她。”

“不行。她还有用。”

“……你心里不恨吗。”

“恨。唉,是啊。”陆久停下脚步,抬起头望着天花板叹了口气,“但我不知道,我该恨的是谁。我是该恨敌人、恨自己,还是恨这战争?”

陆久说着,回头对着V笑了笑。那个笑容很熟悉,一如既往地混杂了无奈、落寞、悲伤——

并不是如此。V不知道陆久是把那些感情都隐藏了起来,还是索性根本就没有。他看到陆久的表情平静得如同什么都没发生过一般,他的眼里除了一丝疲惫之外,只有早已看透一切的超脱。

V忽然感到自己的胸口涌起一阵异样的感觉,仿佛自己的胃被什么人如同拧毛巾一样地绞拧着,让她感到难受得几乎窒息。

这是什么感觉?她不明白。但这种感觉她以前曾经也有过,特别是在陆久露出那样的笑容的时候。

特别是在他明明应该无奈、落寞或悲伤,脸上却要若无其事地微笑的时候。

“你没有必要这样。”V低声说道。

“……什么?”

“没什么。”

陆久耸了耸肩。

“走吧。结束了吧。”V说着快步走过了陆久的身边,朝着指挥部的出口走去。

“应该是结束了。不过……”陆久站在原地没有动,“如果你感染了病毒——”

听到陆久的话,V稍稍停下了脚步。

“……我知道。”她轻声说道,然后继续朝前走去。

陆久没有说话,只是默默地继续向前走去。她说自己“知道”。她知道什么呢。

如果她也感染了病毒,到底该怎么办,就连陆久自己都不知道。

他该好好想想这个问题的——或者说,一开始就不该让V跟过来。陆久第一次因为自己的冲动而感到后悔,但那时V的话确实让他心中大为光火。

不过关于那个问题陆久只来得及思考了两三秒钟,因为他看到走过拐角的V,脸上露出了一个惊讶的表情。

陆久立刻咬牙拔腿冲了过去,一手将V推到了一旁,一手抽出了胸前的手枪——他下意识地猜到发生了什么。

确切来说,那是V第一次看到陆久战斗的情景。她搜索着自己之前和记忆、和那些自己并没有经历过的记忆,也找不出类似的场景。

走过走廊的拐角的时候,她看到面前不远处有一个21战区的战术人形——虽然身体几乎已经被子弹撕裂了,但依然在顽强地直起身来,举起了手中的枪。

看来这位前友军是不会接受谈判的,那么是该迎战还是该躲避?

但V还没有来得及做出决定,就感到身体被一阵强大的力量推到了一边。

在跌倒之前,她看向那股力量传来的方向,只看到陆久神色焦急而凝重、咬牙切齿的脸。

然后是一阵连续的枪声。

V看到陆久的身上腾起一阵烟雾,然后他向后退了一步,接着俯下了身。

陆久的身体颤抖着、变换了好几个姿势,唯一没有动摇的就是那只握着枪的手。

……这就是过去的士兵的战斗的模样吗,V心想。陆久的身体强度远不如战术人形,但他的意志却顽强如钢铁。虽然他被子弹击中,但一直到因为痛苦而跪在地上、直到垂下了头,他手里的枪依然笔直地指着目标,手指还在有节奏地扣动着扳机。

V愣愣地盯着他,甚至忘记了他中弹的现实。直到面前的男人因为体力不支而倒在地上,V才终于喊出了那个她从来未曾叫过的名字:

“陆久……!”

……薇。

陆久瓮动着嘴唇,吐出了一个字。一个人的名字。

他眼前一片昏暗,看不清楚任何景象。他中弹了——这是他唯一所确知的事情。

子弹击中了什么位置、他又是在哪?他无法确定。他的意识正因为子弹的冲击而渐渐变得模糊。

但他感到自己正在移动。

“……医官!”

他听到一个焦急的喊声,好像是自己的副官发出的。然后,移动停止了。

“请离开这里。”陆久听到一个男人的声音说道,“人类的医疗设施里,不允许人形进入……”

但那个男人没能把话说完,因为陆久听到了一阵枪声作为回应。

……是谁在开火。

“……薇?”陆久用虚弱的声音再次重复了那个名字。

“我在这里。”

他听到了自己期待的声音,然后感觉自己的手被握住了。那是一只柔软而纤细的手,触感有些冰冷。

陆久稍稍感到安心了一点。

“你……怎么样……”陆久感觉自己几乎已经无法呼吸了,但还是用肺里残存的空气挤出了一句话。

“我……没事,请不要说话。”

“哦。我也……没事……”陆久没有听从那个声音的劝告,再次努力地说了一句。

“是的,您没事。”那个声音仿佛楞了一下,“您会没事的……一定会的。”

那是陆久听到的最后一句话。

之后,陆久感到胸口传来一阵剧痛,仿佛被什么刺了一下。他想要大喊,但是已经没有力气了。

他感到耳边一阵嗡鸣,接着所有声音都消失了,眼前只剩下一片黑暗。



\section{尾声}

陆久在离开N21指挥部的时候,遭到了残存的感染人形的袭击。因为事发突然,陆久没能及时化解袭击,身中数弹。但是多亏他的副官保护及时,子弹均未击中陆久的要害,只是被防弹衣阻挡而造成了几处骨折。他的副官把他抬到了指挥部旁的医务室做了紧急处理,并通知了本部派出运输机接应。两个小时后,陆久恢复了意识,在副官的协助下返回了本部。

派瑞特少校第二天才从公司返回。他带来了人类士兵,收押了意志崩溃的 “播音员”并对她进行了审问。“播音员”落网后,铁血的力量全部撤出了N21战区。

“播音员”供述了一切——关于病毒、关于铁血、关于她自己,关于她所知道的一切。

据N21战区幸存的人类文职军官讲述,那天当陆久从房间里走出来的时候,他的脸色平静如冬天的湖面。而那个铁血的人形则抱着头瑟缩在角落里,眼神涣散、甚至不敢抬头看一眼身边的东西。

她的意志被彻底摧垮了,对于审讯者的提问(只是没有任何审讯技巧的常规地提问,完全不能称之为“审问”),她全部照单回答了,没有哪怕一个字的隐瞒。

谁都不知道陆久到底做了什么。那个铁血的人形身上有很多伤痕,但是没有严重和致命的伤口。那些伤痕似乎是她自己造成的。

而且最离奇的是,她的左手不见了。人们后来在她的呕吐物中,发现了她手骨的残渣,但没有人知道她为什么吃下了自己的左手。

后来陆久称他只是和她进行了一场“和平的谈话”,显然事情不止如此。但关于这场 “陆久测试”的细节,无论是陆久还是播音员,都没有吐露过一个字。

V副官同样毫不知情,她的印象中只是听到了一些低声的交谈——以及播音员轻轻呜咽和哀求的声音。她没有听到任何疑似暴力的动静。

了解到事情经过的派瑞特少校选择了沉默,他没有控告陆久、也没有对陆久表示谢意,他只是无言地接受了全部事实。事情似乎就这么过去了。

受伤的陆久在N17战区的医务室躺了一个星期才能下地活动,他的副官也被“隔离”了一周以作观察。一周后,陆久和V被认定康复,准许 “出院”。

陆久出院的次日,对N17战区全军表示了对95和突击队员们的哀悼,并将她们安葬在军营的陵园之中。陵园里的“无名战士墓”被拆除了,取而代之的是一座白色大理石垒砌的平台,平台上置有一金属制作的花冠。那座平台和花冠有一个美丽而略显伤感的名字:“繁花之眠”。

侦察连的成员被编成了新的队伍,由静默小组率领,命名为“哨兵”小队,以取代过去的快速反应突击队。战区的一切又渐渐回到了正轨。

四个月后,寒冷的冬天过去了,北部战区迎来了春天,四野里散发出复苏的气息。

在一个冰雪消融的晴天,陆久悄然离开了指挥部,未留下任何信息。随之不见的还有他的副官。没有人知道他们是否在一起、也没有人知道他们去了哪里。


\section{后记}

95牺牲了。她曾经对V说,“士为知己者死、女为悦己者容”,她却不能再为她喜欢的男人装扮妆容,因为她已经为赏识她的男人而死。她曾经担心自己在陆久心中的地位无法和V副官相提并论,但她今后永远都不必再担心这一点,因为陆久永远都不会把她忘记。

陆久没有要求根据模板再去制造一个95,因为他知道95的核心已经被病毒彻底侵蚀,就算再有一个一模一样的人形出现,她也不会是那个95。她和陆久一起度过的八百多天的记忆,已经无法复制,真正的95不会再有了。

在安排好战区的日常事项后,陆久离开了战区,没有通报任何人。他随身没有带多少个人物品,除了抽屉里一摞已经发霉的现金之外,他只带了他的前侦察队长给他留下的信物。

他也许是累了、也许是烦了,总之他决定不再理会这场和他毫无关系的战争。他决定在自己被抓住然后重新投入牢狱之前,过一阵普通人那样漫无目的的生活,不再为任何人操心、也不再为任何事负责。

但是,虽然是没有任何计划的逃亡,却终于没能瞒过陆久那细心的副官的眼睛,Vector察觉了陆久的异样。和平时冷漠淡然的陆久不同,他那天表现得太过自然了——眼睛里的不仅是平静,似乎还有着一丝对什么东西的向往。这引起了V的警觉。

于是,在陆久离开营地五公里之后,他被同样徒步出行的V追上了。陆久要求V原地解散,但是他的命令遭到了拒绝——不仅是出于公司的指派,更是出于已故密友的嘱托,V坚持跟随陆久出逃。无可奈何的陆久只能任由她跟在身后。

四个月后,在某海滨城市一个盛夏的午后,公司的特派员找上了门。陆久知道这一天迟早会到来,所以他平静地跟随公司的特使回到了原单位听候发落。但是公司并没有把他投入大牢、甚至对他的无故离去问都没有问一句,仿佛早就决定给他放了四个月的假。这次找到他,是因为他又被委派了新的任务。

陆久再次回到了将功赎罪的打工生涯,只不过不同的是那个曾经时刻紧随其后的副官不见了,并且没人知道她的去向——或者说唯一知道她去向的陆久,对自己的副官闭口不提一字。有传言说是陆久在手头拮据的时候,把V卖给了需要劳动力的地方;也有传言说陆久为她安排了更好的归宿,让她从此并不劳动就丰衣足食。但这些传言很快就被逐渐遗忘了,毕竟区区一个人形的失踪,不会得到人们长久的关注。

这次,陆久终于离开了北方的穷乡僻壤,去了某个繁华的城市。他坐在办公室里看着纸质和电子表格,为手上的技术工作而苦恼,感到文职人员的活儿远比带兵打仗麻烦。虽然他保留了指挥官的军衔,但是这次他可差不多没人可指挥了,就算人们依然把他叫做“陆司令”……

而那无人知晓的四个月里发生的事情,有空再细说好了。