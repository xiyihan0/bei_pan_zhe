\chapter{背叛者(十五)}

\section*{前言}


人偶少女经历了跌宕起伏的人生,每每回顾的时候,她都感到这是她之前十几年里都没有过的体验。她决定要爱上一个人,她要知道什么是爱、如何去爱,那就必须要成长,而不是一直站在原地不动。她已经得到了目标和勇气,并且决定成为一个“人”,所以,她不畏惧为此而进行的一切战斗。她坚信为了所爱的男人,自己也能够成为一个很好的姑娘……

如果,她能活过这场战斗的话。

\lineseparator

新年好,各位读者。

故事全篇的最后一节,是SOG小组为主角的一场以战斗内容为主的剧情。

Vector的设定并不只是一个寻爱少女,她也是一件杀人武器,但对于V的作战描写却很少,在第二章快要结束的时候有过一段关于Vector的战斗描写,不过只是简单的几句话。但可以看出,Vector扮演少女杀手的角色,其实已经轻车熟路。但现在的Vector又今非昔比,她已懂得了爱和人心,不再是个懵懂的少女。这个战场上的Vector身为队长带领着SOG小队,应该有符合她特有身份的表现。

这一节的内容也很长,本想分段发的,但是想了想,还是一次性发出来吧。一口气把故事讲完,然后没有心理负担地过个好年,不是更好吗。

那就让故事开始吧。

\section*{背叛者:最终章}


“保护好自己。”

这是少女从那个老迈的男人口中,得到的最后一个真正意义上的命令。从那之后,她听到的只有“作战计划”、“任务目标”之类的东西,再也没有过直接给她的命令。那个男人似乎希望她能自己做出决定,但少女并不喜欢这样,她不想自己去做决定。

因为“做出选择”是一件过于主观的事情,而她,只不过是一件服从命令的商品。

所以即使在很久之后,少女也时常会怀念他们一起并肩作战的时光。她清晰地记得那个男人的每一个手势、每一个眼神,战斗中每一个微小的细节所发出的信号。他们曾经是相当默契的战友,那时候少女需要做的事情很简单,就只有服从命令而已。少女曾经认为所谓的“满足”不过如此——不需要进行思考、只要按部就班地执行,没有快乐但也没有痛苦。这样的满足持续了一些年头,直到有一天,那个男人不再把突击步枪带在身旁。

少女当时还不知道发生了什么,那个男人很少参加战斗了。即使有需要战斗的情况,他也只是命令其他人,或者少女去执行,很少再亲赴战场。一段时间以后少女才明白,那个男人已经不需要亲自去战斗。他成了一群战斗人员的领导者,他手下有着许多经验丰富的战士,从此他不必再亲自出战,因为对他来说指挥和调度这些人是更加重要的工作。他的价值,现在已经远远超过了一个在战场上随时可能毁灭的卑微士兵。

从那时起,少女才意识到了自己和那个男人的不同。虽然那个男人非常袒护她,但她明白了像她这样的东西,只不过是人类作战用器械的一种:因为人类的生命过于宝贵不能肆意消耗,才有了她们这些代替人类去作战、去受伤和死亡的替代品。

那是少女的内心深处第一次感到孤单。不是因为她不得不独自出击,而是因为她明白了自己存在于世上的意义——

不过是一件服从命令的商品而已。

“这个任务结束后,你就不用回来了。”

当少女听到这句话的时候,她没有任何感觉。从很久以前她就知道这一天会来临,而她也曾为这一天的到来而感到过惶恐,但当这一天真的来到时,她却毫无感受。

因为她已经麻木了。

男人的意思很明白,她已经没有用了。结束这最后的任务,然后自我毁灭,这就是少女漫长而不断重复、枯燥不堪的生命的终点。如果少女的内心还有一丝情感的话,她那时应该心怀感激,因为那个男人终于赐予了她解脱。

但少女没有就此解脱,因为她没有选择毁灭、甚至没能够完成任务。她把任务搞得一团糟,既没有消灭预定的目标、也没有自觉地自我毁灭,而是在任务失败后不知廉耻地返回了基地。但那个男人还是接受了她。

那个男人不知道少女到底出了什么问题,她这次所做的事情和他期待的大相径庭,但那个男人还是很大度地选择不问原因、也不做处罚。因为少女已经跟随他多年,在他以戎马为伴的一生中,为他立下了许多功劳。

而对于少女来说,她却已经不是那个男人所知道的士兵了。她发现一直以来都被当做武器塑造的自己,能做的不仅仅是战斗,也可以给人以温柔的安慰、也可以回应别人的关怀和期待。她不畏惧毁灭,但却不想毁灭,因为她有了自己所挂念的东西,所以她选择了活下去。

因为她意识到了生命真正的意义。

少女和那个男人最后一次见面,是在他的办公室。那时候他已经是个略显老态的老人了,而少女依然如他们并肩作战的时光里那样年轻。那个男人长久地注视着少女,不知心中是在感慨人类的生命短暂、韶华易逝,还是这个评估制造了一系列失败的改装民用人形,是否还值得信任。但他最后什么都没有说,只是对少女交代了任务。

“保护好自己。”在战术少女领命前去战场前,那个男人在她背后说道,“如果你出了什么事,现在将不仅仅是我一个人的损失了。”

听到男人的话,少女停下了脚步。

她和这个男人之间是有感情的,那个男人是第一个让她确认自己拥有“感情”这种东西的人。但他们之间的感情很奇怪,不是战友之间的同袍之情、也不是男女之间的爱恋之情,而像是一种从诞生前就存在的羁绊。少女无法形容那种感觉,因为她并不知道感情该如何分类。不过如果是让一般人去看,大概都会认为那是有血缘关系的人才会产生的感情。

但即便是生而有之的羁绊,因为两个人从来都不懂得维护,也终于在分离之后,渐渐稀薄成了秋日清晨漂浮在地表的氤氲。

“我对于其他人来说,也是重要的吗。”战术少女说。

“当然。”男人沙哑地说道,“你是公司的财产。你如果损失了,那将会是全公司的损失。”

少女觉得那不是那个男人真正想要表达的意思,他只是想把这个问题敷衍过去。但少女没再说什么,只是微微点了点头,然后走了出去。她隐约地意识到,这大概就是他们之间最后的对话了。

 \section*{}

“清点弹药、检查武器。”队长甩了甩头驱走了脑海中已然远去的回忆,低声说道,“我们得快点,再过一会儿天就要亮了。”

“哈……还要转移吗?我想休息一会儿啊……”突击手一边整理自己的武器弹药,一边抱怨地说道。

旁边的支援射手虽然背着许多装备,却早已准备就绪了。

“别抱怨了。队长一直在警戒,一分钟都没休息。”支援射手说道,“早点动身、早点搞定,这烂地方已经让我无法忍受了。”

“开什么玩笑,你以为我们是在郊游?”突击手嗤之以鼻,“还是说你以为派我们出来的那些人,还会接我们回去?”

“他们当然得接我们回去。”支援射手皱起眉头说道,“别忘了,我们可都是各战区选拔出来的精英。我们要是交代在这里,对公司的损失是难以估算的,我们的战区也不会同意的。”

“精英又如何,不过是质量高一些的炮灰。你不知道我们受总部的直接指挥吗,谁敢反对总部的命令?别忘了,我们就是为了替人类送命才被制造出来的,如果有必要,他们当然会毫不犹豫地丢下我们。”

“你简直是满口胡言、动摇军心!”支援射手呵斥道。

“你才是掩耳盗铃,别把你那自欺欺人的幻想强加到我身上!”突击手毫不犹豫地反唇相讥。

“别吵了,武器装备没问题的话就准备动身。”队长终于忍不住开口打断了两个人的争吵。

“至少你该对现在的情况心知肚明才对。”突击手对着队长说道,“难道说你也和这个天真的家伙一样,相信还有人会带我们撤离这种童话?”

“往好处想想。要是能联系上总部,我们的希望就大得多,至少可以请求附近友军的帮助。”队长说,“我们必须到通讯站去碰碰运气。”

“我们根本不知道该怎么去通讯站吧?”

“通讯站的位置明明白白地标记在地图上了。”

“可这地方早就乱套了,废墟之间到处都是潜伏的敌人。也许为了抵达一公里外的地方,我们不得不绕十公里的路!”

“马上行动的话,预计三、四个小时后能抵达。”

“真的吗?”

“当然。”

“你不会是为了安慰我们,故意这么说的把?”突击手把脸凑到了队长跟前,盯着她的眼睛问道。

“我就是为了安慰你们才这么说的,现在闭嘴,马上动身。”队长的语调平静如故,脸上没有任何表情地说道。

脏弹爆炸后,爆心附近的电磁干扰极其强烈,单兵使用的通讯器无法直接和总部进行通信,于是格里芬派人部署了一些中继信号的通讯站。当格里芬在这一区域的兵力全部撤离后,遗留下来的通讯站,成了被困在这个地方的人们的最后救命稻草,只有那里才有功率足够突破干扰的大型通讯器。

几个人从废墟里钻了出去,小心地移动着。黎明马上就要到来,但这不意味着生机,反而意味着这座城市将会更加危险——军方的势力已经离开,但还有数量可观的铁血部队在游荡,天亮之后它们的侦察范围将极大地扩张,而那些不知藏在何处、完全无法侦测的感染者也是巨大的威胁。

“发现感染者,十点钟方向、120米距离。”支援射手轻声说道。

“消灭它们。”队长说。

噗噗、噗噗噗。

突击手扣下了扳机,但脚一步都没停。三个感染者应声而倒。

“安全。”队长过去检查了一下说道,“只有三个,没有武装。”

这几个感染者应该只是这里的居民。受到辐射而变异早已失去了意识,游荡着走出了他们躲藏的地方,来到了街道上。

“我们这是在杀人。”支援射手说。

“哪又如何?可别告诉我,你为此受到了良心的谴责。”突击手说。

“它们已经不是人类了。”队长说。

“不管它们现在是什么东西,它们的识别特征依然是人类。”支援射手说。

“无论识别特征是什么,它们挡了我们的路,我们就要干掉它们。”突击手边说边往弹匣里塞了几颗子弹。

“这证明我们都是非法的人形。”支援射手说。

“谁他妈的在乎这个?”突击手叫了起来。

“我在想,如果我们是违反人形管理法规的存在——”

“集中精神,集结点还有400米距离,前方道路没有阻碍了。”队长打断了支援射手的话,“提高移动速度!”

几个人奔跑了起来。

所谓“集结点”是队长在地图上标记的几个建筑物,它们的共同特征是处于撤离路线上,而且相对隐蔽——因此被队长“推定为”安全。

也就是说,到底安全不安全,谁也不知道。不过就目前为止的情况而言,它们暂时是安全的。

一阵狂奔之集结点后,突击手呼吸稍微有点急,而支援射手则已经气喘吁吁。但队长的呼吸却没有一丝紊乱的样子。

“就算我们到了通讯站,就能唤来救援吗?事实上,我觉得通讯站这玩意是否真的存在,都很可疑。”突击手问道,“你这家伙为什么如此自信?”

“我不是怀着什么信心,只是没有其他的选择。不这么做,就只剩下等死了。”队长说。

“不可能,你肯定知道点什么内幕吧。你这家伙到底是什么来头?”

“一个普通的战术人形。”

“一个普通的非法战术人形,”突击手纠正道,“但肯定不止如此。我们都是各单位的精英,这一点一开始大家就知道。不用谦虚了,稍微自我介绍一下呗?”

“那是违反纪律的。”

“别扯淡了。”

这个小组队员之间的身份是相互保密的,任何人都不得谈论、更不得泄露。但这样的规定对这几个人来说,显然没什么约束力。在组队不久后,小组的成员们就多多少少地向自己的队友透露了一些个人信息,而对此绝口不提的只有队长一个人。

因此,队长的身份也成了队员们关注的焦点。

“我也觉得这无伤大雅。”缓过气的支援射手开口说道,“反正这里只有我们三个人,没人会知道我们说什么。如果你们愿意公开自己的身份的话,我可以公开对等的信息。”

“我对这毫无意义的事情没兴趣。”队长说。

“去你的吧,逃命路上的聊天话题,还管什么有没有意义?”突击手说,“我们搞不好路上就会死掉,难道把这种秘密带进棺材里有意义?虽然我们大概死了也不会享受到棺材的待遇。”

“我同意这个小鬼的话。队长你太过古板了,完全不懂调节队员之间的关系和气氛。”支援射手说。

“你要说我是不合格的队长,我也没什么可辩护的。但总部给我这样的任命,肯定不是为了哄你们开心。”队长耸耸肩说道。

“那我们自己哄自己开心怎么样?”突击手说。

“好吧,你们要是想公开自己的信息,我权当没有听到好了。但请你们别再问我的事情了。”队长说。

“那还有什么意思?你不参与的话,不就像是一场只有两个人的联谊会了吗!”

“嘘,别出声!我侦测到了电子信号……是铁血。”

负责电子侦查的支援射手忽然说道,几个人立即警觉了起来。

铁血的部队虽已是残兵,但依然同平时一样组织严密,然而它们的行动不像是有人在指挥的样子。它们有时候会搜索一片区域,搜得很仔细,一旦发现不是铁血的单位就彻底消灭,因此清理了不少感染者。而又有些时候,它们就完全一动不动地在原地待机,特别是在清理过一片区域之后,总是会待机一段时间,仿佛是在“休息”一样。

SOG小组则一直在试图摸清它们行动的规则来躲避它们,因为就算是小股的铁血部队,也不是她们能对付的。

“这些家伙在干嘛?”突击手低声说道。

“它们在巡逻。它们会清理它们遇到的一切。”支援射手说。

“它们为什么要这么做?有人对它们下了这样的指令吗?”

“不知道。它们的反应迟缓,不像是得到了明确的指令,但行动效率却相当高。”

“应该是低级自律程序。”队长说,“在没有得到指令并且和指挥者失去联系后,为了确保自身所在区域的安全,它们会间歇性地巡逻。铁血人形都有这种程序。”

“也就是说它们只是在无差别清场?”

“是的。它们清理得很仔细,并且和一定区域之间的铁血还有信息共享。所以我们必须藏好,绝对不能被发现。”

三个人一直处于电子静默状态,不仅关闭了对讲机,还屏蔽了生命体征信号,铁血应该是侦测不到她们的。但在铁血经过她们跟前时,紧张的压迫感还是让她们屏住了呼吸。

好在片刻之后,铁血部队走了过去,没有察觉她们的存在。

“这种提心吊胆的感觉真糟透了。”突击手小声抱怨道。

“是吗?我还以为你不怕死呢。”支援射手说。

“凭什么我就不怕死?谁敢说自己不怕死?”

“你不是都看开了吗。”

“可我也不想死!”

“别吵!铁血还没走远。”队长制止了两个人的争吵。

“队长倒是泰然自若呢。”支援射手说。

“与其说是泰然自若,不如说是与世无争。”突击手说。

“唉。在死之前,还有什么愿望吗?”

“我想饱餐一顿鱼子酱。”

“我想听肖邦的降E大调夜曲……”

“都给我安静。”

队长站起了身,走到残破的窗前望向外面。

“我们依然处在铁血的侦测范围内,站在那里,有中等概率会被发现。”支援射手提醒她说。

“你这个傻瓜,没看出来吗。队长才不在乎。”突击手戏谑地说道。

“怎么了?”

“队长显然不怕死啊。”突击手说,“因为她在这世间大概没有什么残念。对她来说,把我们带到绝望的最后就是尽职尽责,对吗,我亲爱的队长?”

“是啊,我无所谓。”队长说,“我带着你们东奔西跑,就是为了让你们在死的时候,能够没有遗憾地闭上眼。”

队长的声音稍稍提高了一些,突击手和支援射手互相望了一眼。

“……她生气了?”支援射手问。突击手没有说话,只是撇了撇嘴。

“铁血已经离开了,暂时安全。”队长的声音恢复了之前的漠然,“该出发了,行动。”

 \section*{}

几个人再次跑了起来。她们穿过一条又一条的残破街道,有些街道里游荡着感染者、有些建筑里似乎还有哭喊声,但她们都无暇顾及。她们能做的只有不停地、以最快的速度奔跑。

二十分钟后,她们终于到达了又一个集结点。

“呼、呼……”

支援射手喘着粗气,就连队长也微微有些喘息了。

“这里视野不错,在这里休息一会儿。”队长说,“这附近好像没有铁血的影子。”

“不,有铁血……我能侦测到,它们的信号。”支援射手说。

“什么情况!“突击手说。

“在西南方向、距离超过一公里,信号微弱但是交流频繁,大概在和什么人交火。”

“我去看一看。你们呆在这里。”队长说。

“我也去。”突击手说。

“那就一起吧,我们三个还是别分开的好。”支援射手说。

三个人一起走出了集结点,小心翼翼地行进了一阵,已经能隐隐听到射击的声音。于是她们找了个隐蔽的地方藏了起来。

“在那边,没多远了。它们是在打谁?军方的人?”突击手说。

她们已经能够隐约看到远处,铁血的十来个人形在朝什么人开火,但它们攻击的目标看不清。

“不是。如果是军方的人,它们的信号会很清楚,可我侦测不到任何军方的电子信号。”支援射手说。

“军方早已经撤离了,那是我们的人。”队长说,“那些人和我们一样,通讯受到了辐射的干扰。”

“不管是谁,我们还是走为上。”突击手说,“我们自己都顾不上,更帮不了别人。趁着铁血被吸引,我们赶紧去往下一个集结点吧。”

“你说呢?”队长问支援射手。

“我也认为和铁血交火不是明智之举。不过要是能救回友军,或许对我们会有所帮助,至少能得到点情报。”支援射手回答道。

“嗯。你从这个距离,能狙杀几个吗。”

“最好能再近一点。”

“我从来没有同意过要和铁血正面开干!”突击手抗议地叫道,“蠢货才在这种时候逞英雄!”

“无论什么时候,我都不会逞英雄,但那边那些人很值得注意。”队长说,“最后一批被派到这里的战术人形,任务是部署通讯装置。往好处想想,要是她们的通讯器能联系上外界,我们说不定就不用去通讯站了。”

“这简直是找死!我绝对不干!”

“那你就呆在这里吧,我去侦察一下。”队长说,“支援射手,你从最后面开始,按照距离依次消灭目标,削弱它们的火力。我去搞清楚另一边到底什么情况。”

不由分说,队长自己跳出了隐蔽的建筑。看着队长的身影远去,支援射手架起了枪。

“你帮我警戒四周。”支援射手对着身边的突击手说。

“我知道!可恶……”

噗!噗、噗!噗……

因为加装了消音器,微弱的枪声在废墟之间并不明显。支援射手十几秒的时间就击毙了三名铁血人形,敌人的火力弱了一些。但当支援射手仔细观察战场的时候,她发现事情不妙。

铁血不是人类,它们的信息交换速度很快,发现后背受到攻击后,迅速做出了反应。五个人形开始分散地向着支援射手这边移动,很快就发现了正在试图渗透的队长,并开始交火。敌众我寡,队长陷入了相当不利的境地。

“队长被压制了。我掩护,你快去支援她!”支援射手说。

“我操,我就知道会这样!”突击手骂了一句,跳出了掩体向队长冲了过去。

突击手是个中近距离攻击的好手,火力相当不俗。她冲了上去,一边奔跑一边开火,很快就消灭了压制队长的铁血人形,帮她解了围。支援射手换了个有利的位置,三个人凭借熟练的配合,歼灭了其余的铁血人形。

“我救了你的小命!这次你欠我一个了吧?”突击手得意地对队长说,但是被心不在焉地敷衍了过去。

“啊。”队长应了一声,然后转向了支援射手,“你怎么也过来了?”

“我的位置不太顺手……而且,你们跑得太远了,我一个人感觉有点危险。”支援射手说。

这次几个人可以清楚地看到铁血正在攻击的目标,那的确是格里芬的人形。那个小队有6个成员,其中4个已经彻底被摧毁、一个也严重受损,还有一个只受了些轻伤。

“我们是格里芬的SOG小队。”队长走过去说道,“你们是干什么的?”

“通……通讯小组。”伤势较轻的姑娘带着哭腔说道,“我们是通讯兵,奉命来建立通讯站,但到了这里却发现到处都是铁血和感染者。我们火力微弱,又在战斗中被冲散了……要不是你们,我一定死在这里了。”

“你有电台吧?还能用吗?”突击手急忙问道。

“应该能……”

“带上电台,马上跟我们离开这里。”队长说。

“等等,我……我要带上她!”

队长看了一眼另外一个战术人形。

“她伤得太重,我们救不了她了。”

“别管我。你快走吧……!”那个受伤严重的人形气息微弱地说道。

“不,我不会抛下你!”通讯兵哭了,“我们约好了,回去还要一起钓鱼的……”

“这里很危险,铁血之间的情报是互通的,随时都可能有更多的敌人出现。”支援射手说。

“没错,我们不能呆在这里。必须马上去安全的地方,然后发信号请求救援。”队长说。

“至少……至少让我为她的云图留下备份,不然我哪也不去!”

“我们可没空等着你下载小电影——”

突击手还没说完就停了下来,这次不用支援射手提醒,她也感到了情况不对。脚下的大地在微微震颤,似乎有什么巨大的东西正在地面下活动。

突然,轰的一生地动山摇,冰冻的泥土和石块被高高掀起,一台足有两三层楼高的四足机器突然出现在她们的头顶上——那是一台休眠的铁血机甲,本来是埋藏在地下,大概是附近发生的战斗唤醒了它。

“……蝎尾狮!”支援射手低吼了一声。

“嘘、嘘……别出声。”队长压低声音说,“它太高了,扫描不到正下方的目标。我们都关闭了所有电子仪器,它应该还没发现我们。”

“可我们怎么办?”

“它应该也有巡逻路线,我们保持在它脚下面移动,然后找个合适的时机溜掉。”

“去他妈的,开什么玩笑……”

队长的话听起来很轻松,但她们都知道“蝎尾狮”的可怕,它不仅拥有厚度惊人的装甲,更有着恐怖的火力装备:四挺重机枪、两门速射炮、两具榴弹发射器还有一门电磁主炮,是一部不折不扣的移动堡垒。这部战争机器要消灭她们,恐怕连一分钟都用不了。

几个人都屏住了呼吸,那个通讯兵明显是第一次和这恐怖的武器如此近距离接触,已经吓得全身发抖、丝毫不敢出声了。

蝎尾狮伸展了自己的四足,用位于顶端的主探头扫视了一番战场,果然,它没有发现脚下的人形。它微微停了一下,似乎是在计算行进路线,然后迈开四条机械腿开始移动起来。

“走!”队长轻声说道,突击手和支援射手立即跟了上去,那个通讯兵却没有动。

“过来啊,该死的!”突击手低声喝道,但通讯兵毫无反应。蝎尾狮已经走出了一段距离,通讯兵所在的地方已经不是它的扫描盲区了。

“呆在那里你会被发现的!快过来!!”突击手终于忍不住喊了一句。

“不要,救命!”通讯兵跳了起来,然后一边哭一边向着反方向跑去,“我不想死……救命啊!”

队长见状试图跑过去把她拉回来,但已经来不及了。

蝎尾狮发现了身后的动静,停了下来。它的主探头锁定目标,将上身的辅助火力平台转向通讯兵,然后开始射击。

四挺12.7毫米机枪同时开火,子弹像雨点一样飞去,只用了几秒钟通讯兵就中弹倒地了。但蝎尾狮并没有停火,而是换为榴弹发射器攻击,向目标射出一连串40毫米榴弹。烟尘散去,SOG小队的三名成员清楚地看到了一百多米外的杀伤效果——地面上只有几个爆炸产生的弹坑,通讯兵被彻底抹消了,就连一块碎片都没剩下。

蝎尾狮将武器转向前方,开始继续移动。几个人谁也没有说话,只是迅速钻到那部杀戮机器的身下,跟着它后腿的步伐跑了起来。

几个人跟着那台蝎尾狮跑了差不多半个小时,终于在它经过一片楼宇的时候钻进了废墟之间。队长观察了一阵四周,发现她们所在的位置已经偏离了去往通讯站的路线。

“绕了些不必要的远,我们偏离路线了。”队长说,“这里不安全,我们得赶紧走。”

“这里不安全、那里不安全,哪有安全的地方?”支援射手低声说道。

“集结点,相对安全。”队长说。

“你也看见刚才那个孩子了。那说不定就是我们的下场。”突击手说。

“也许吧。但在那之前我不会停下的。”队长说。

“下一个集结点在哪?”

“东南方向,距离大约4公里,路上有一段开阔区域,具体情况未知。”

“我不去了。”突击手说。

“你想怎样?”

“我就呆在这里。你们爱去哪去哪吧。”

“想想我们是怎么来到这里的。”支援射手说,“这里处于铁血的巡逻线路上,它们绝对还会再来的,呆在这里必然是死路一条。”

“说的好像出去就不会遇到铁血一样!”突击手说,“我要找个地方隐蔽起来、以最低功率待机,说不定能坚持到被什么人发现。我受够了这样提心吊胆地钻来钻去了!”

“如果躲在一个地方等着就行,我绝对不会这样冒险四处奔走。”队长说,“我们得发出信号,让总部知道我们还活着,这样才有可能得到救援。不然的话谁会向这种地方派出搜救小队?”

“去哪都是一个样!”突击手叫道,“到处都是高强度的辐射区、到处都是铁血和感染者,总部怎么可能为了我们三个人形派出救援!夹着尾巴东躲西藏到达了通讯站,却被告知我们早就已经被抛弃了,到时候岂不是更惨!”

“不试一试怎么知道?反正我们也没什么可以失去的东西了。”支援射手说。

“我不去。”突击手说,“跟着那台该死的蝎尾狮的时候,我觉得自己的脑子都快烧掉了。我一直在想要是被它发现了怎么办、要是遇到其他铁血怎么办,我们这次还能不能保住小命?我觉得我已经到极限了。说实话,刚才那个孩子逃跑的时候,我差点就要和她一起跑了。下次遇到铁血,说不定第一个崩溃的就是我。”

“我明白你的感受。虽然你总是满口大话,但其实是个虚张声势的胆小鬼。但我还是希望你能一起来。”队长说,“如果你不一起走的话,路上出现情况我们就更难对付了。至少你的战斗力还有点用。”

支援射手惊讶地望着队长,不知她在说些什么。队长的话不像劝慰和鼓励,她不知道队长到底是哪根筋不对了。

“谁他妈的管你们?你们的生死关我逼事!”果然,突击手被语出惊人的队长激怒了。

“我是在请求你。”队长说,“你说我不在乎,你说错了。我在乎,我不想死,这世间依然有我挂念的事情。所以我请求你们两个,继续去往通讯站,就当是在帮我好吗。”

突击手的脸上也露出了吃惊的表情。她盯着队长看了好一阵,然后站起了身。

“好吧。”她说。 

三个人又跑了一阵,终于来到了下一个集结点附近,却发现这个集结点已经被炸成了一片废墟,外面还有几十个感染者在游荡。这些感染者身上都穿着军方的制服,有些手里还拿着武器。

于是三个人找了个稍微隐蔽一点的角落躲了起来。

“那些家伙难道还会用武器?”突击手小声说。

“不好说。根据感染情况不同,有些感染者还保留着一点点意识,会使用武器也有可能。”支援射手一边观察一边说,“无论如何,这群家伙看起来要比以前遇到的那些更危险。”

“感染者都是没有意识的,不足为患。我们首先要躲避铁血的巡逻兵。”队长说,“长时间暴露在外面很危险,必须找个地方藏起来。”

“可集结点已经被感染者包围了。”支援射手说。

“那就干掉它们?”突击手说。

“大规模的战斗说不定会引来什么东西,我们也没有那么多弹药可浪费。”队长摇了摇头。

“那怎么办?”

“趁着没人发现,我们马上去下一个集结点。”

集结点只是行动路线上的一些节点,作用是用来确认路线和稍事休整。若不需要躲避铁血,没有必要一定呆在里面,以最快的速度抵达目的地才是最重要的。但这只是对于队长而言。她制定这个计划的时候,忽略了一件事,就是在这种极端条件下队员的体力问题。

队长本人是轻巧型的战术人形,即便不进食和休息,也能够长时间地行军,而突击手是个各方面能力均衡的人形,勉强还能跟得上队长的步伐。但支援射手则不同。

支援射手是为了侦测和支援而制造出来的战术人形,因为经常需要携带专用的设备,她舍弃了机动性能换取更高的负载能力,消耗的能量也高于普通人形。在一般的行动中,如果保证食物的补给充足,她尚且能和队长共同行进;但此刻她已经整整一天没有进食,体力几乎已经达到了极限,无法再维持长时间高强度行军。

“不行,我得休息一下。”支援射手说道。

“在这里休息?”突击手奇怪地说。

“你是不是早就累了?”队长问。

“是的。我的机动性是弱项,不适合这样长时间的行军。”

“原来是个大小姐啊?”难得抓到支援射手的把柄,突击手讥讽地说道。

“你们的功率有多大?”没有理会突击手的嘲笑,支援射手淡淡地问道。

“1.2匹。”队长说。

“我有1.7匹。”突击手得意地说。

“我是3.5匹。”支援射手说。

另外两个人沉默了。3.5马力的功率,意味着能够背负最多260公斤的辎重,这比她们两个人的功率加起来还要高。

“呃……对不起,看来我不该叫你大小姐,该叫你大排量才对。”突击手挠了挠头说道。

“这么说,你的口粮也早经吃完了吧。”队长问。

“是的。”支援射手点了点头。

队长闻言,拿出了自己的口粮递给支援射手。

“把这些吃了。”

“全给我,你就没有了。”

“我还能坚持,你的能量消耗大,给你吧。”

“不必了。我吃了也坚持不了多久,还是你自己留着有用。”

“到这种时候就,别再相互谦让了吧。”突击手也拿出了自己的口粮,让人意外的是她的存货竟然还不少,“不用我说,你们也该知道,要是这次搞不定,恐怕以后也不需要口粮这种东西了。”

“说的没错,成败在此一举。”队长把所有人的口粮都收集起来分成了大中小三份,“往好处想想,如果抵达下个集结点,通讯站就近在咫尺了。现在我们吃下所有的口粮,休息10分钟,然后继续前进。”

几个人互相看了看,然后点了点头,一起吃下了最后的口粮。

“我的指挥官最讨厌军用口粮,每次吃这东西他的脸色都很难看。”队长说,“没想到,我们也有不得不依赖它的时候。”

“我觉得味道还不错啊?”突击手说。

“你的指挥官……是个什么样的人?”支援射手说。

队长没有说话。支援射手意识到自己问了一个不受欢迎的问题,队长是肯定不会回答的。但沉默了一阵后,队长竟然开口了。

“我不知道该怎么形容。”队长说,“他应该算是个普通的人,稍微有点沉闷、不太合群,但很有军事才能,而且非常善于战斗。”

“你们相处多长时间了?”

“三年多,也许四年。”

“感觉如何?”

“大多数是平常的军营生活,按部就班,没什么特别的事情。”

“我不是说你们共事的感觉。我是说你和他之间,相处得怎么样?”

队长看了支援射手一眼,然后转过了头。

“一般吧。”队长说。



 \section*{}

稍稍休息了片刻,她们开始继续前进。队长说下个集结点是最后一个,在一条横穿这座城市的河流的岸边。河的对岸是一处高地,高地上有一处雷达站,通讯站就设置在雷达站里。

三个人去往集结的路上点没有遇到任何敌人,这让她们感到非常幸运,眺望着已在视野之中的雷达站,士气也开始高涨起来。但随后发现的情况却让人沮丧——当她们赶到距离最近的大桥时,发现桥已经被炸塌了。无奈之下,几个人不得不沿着河岸前寻找其他过河的地方。

“我们得向着下游走。”队长说。

“地图上最近的桥在上游吧?”支援射手问。

“的确,但上游的桥都是吊桥。下游有一座混凝土结构的桥,被摧毁的可能性比较小。”

几个人向着下游走去,走了一阵发现了一座桥,但是是吊桥,而且正如队长所说,已经被炸毁了。她们没有气馁,继续沿着河岸向下,终于来到了一座大桥前面。

那座桥是一座平直的混凝土桥,只有一座桥墩位于河中心的沙洲之上,没有桥栏。这座桥显然也经历了不少炮火,但是由于结构坚固,依然完好而牢靠,这让突击手稍稍赞叹了一下队长的远见。

“不错嘛。虽然你不是个好队长,但是还挺有预见性。”突击手说,“在战区的时候没少带兵打仗吧?”

“不,我从来没有带过兵,这是第一次。”队长说,“另外你高兴得有点早了。”

“怎么了?”

“铁血。”支援射手简单地说道,“我侦测到大量铁血的信号。”

“在哪?!”

“桥头的建筑物里。”

“……它们也学会埋伏了?”

“它们数量庞大,但相互之间的信号传输并不活跃,应该是处于待机状态。它们得到的最后一个命令,可能是守卫这座大桥。”

“那我们还能过去吗?”

“我不知道。不过我想我们一旦接近大桥,就会激活它们。敌我实力悬殊,正面作战想都不要想。”

“那我们是该绕过去,继续寻找下一座桥吗?”

“不,没有必要了。”队长再次开口说,“既然铁血守在这里,那就说明一件事情:是铁血炸毁了桥梁,然后把守着交通要道。下一座桥,如果没有铁血看守,一定也被炸毁了。”

几个人沉默了。她们走到这里突破了各种艰难险阻,但这些艰难似乎永无止境。

“要不,我们泅渡过去?”突击手说。她看了一眼远处的河流,河面上弥散着白色的氤氲,还没有结冰。

“不行。”队长说,“在零下十几度的严寒中,这条河面竟然没有冻结,只能说明一件事情:有蕴含巨大能量的物质流入了河里,加热了水体。水中的放射性不知道到底有多高,浸在里面也许比进攻铁血把守的桥梁更加危险。”

“那我们这次是没辙咯?”

“不一定。”支援射手说,“我这里有些爆炸物,合理使用的话,应该能拖住它们一阵。”

“问题是,我们有时间部署这些爆炸物吗?”队长说。

“有的。我来处理这件事,你们只要一直向前跑就是了。”

“嗯,这个主意确实很可靠。”队长明白了支援射手的意思,“但是我反对。”

“我想到了会这样,所以我们投票决定吧。”支援射手说着转向了突击手,“你呢,小鬼?你该不会也反对吧?先别急着做决定,仔细想一想。这是目前最切合实际的办法了。”

“你在说什么蠢话?我当然反对了!还有别叫我小鬼!”突击手叫到。

“我还以为你们都是现实主义者呢。”支援射手不屑地说道。

“我要的不是仅仅自己活下去,而是所有人一起活下去。所以这个建议我不会接受。”队长说。

“我也……虽然我不想死,但我……”突击手说,“我绝对不会靠别人的牺牲,来让自己活下去。”

“看来我们没法达成共识了。”支援射手说。

“那你呢。”队长说,“如果由我去实施自杀性的断后,你会借机逃跑吗。”

“我——”

“所以这就是共识了:我们要共同进退。”

“好吧……好吧。”支援射手叹了口气,“这样的话,我还有另一个想法。”

“嗯?”

“据我观察发现,铁血从待机状态激活,需要一小段时间完成准备。丢掉重型装备,全速冲向河对岸,如果我们够快,说不定能甩掉它们。”

“你为什么不先说这个主意?”突击手说。

“这样做风险太大了。河对面到雷达站之间有什么东西,我们完全不知道,丢弃装备后,我们遇到情况,就无法再以战斗来应对了。”

“那些东西没你想的那么重要。”队长说,“装备只是用来以防万一,事实上,我们一路用到武器的时候并不多。丢掉重型装备,只带轻量的武器和弹药,也足够应付多数情况。”

“那就这么决定了?”支援射手说。

“我看你们也没有更好的办法了。”突击手说。

“那就着手准备吧。”

为了尽量减轻服装,几个人脱下了自己的插板防弹衣,并丢掉了大量不必要的装备。队长还好,她的装备本来就不多,因此丢掉的东西也不多;突击手丢掉的则主要是弹匣。支援射手卸下了许多的东西,包括手榴弹、塑胶炸药还有数量可观的弹药。然后,她小心地拆解了自己的步枪装进箱子里,并把箱子藏在了河岸的隐蔽处。

“还想着过几天再回来拿呢?”突击手开玩笑地问道。

“是啊,这可是烙印过的武器。如果有可能的话,我会设法回来取的。”支援射手认真地说道。

“往好处想想,至少不必担心有人捡走它。”队长说道,“以后的时间会有很多。我们走吧。”

三个人轻装上阵,谨慎地朝着桥头前进,在距离桥头大约200米的地方停了下来。

“从这里开始,我认为我们匍匐前进会好一些。”支援射手说,“前面没有任何遮蔽物,就这样大摇大摆地过去,恐怕到不了桥头就会被发现。”

“铁血的那些家伙,是靠什么侦测敌人的?”突击手问。

“和我们一样,电子信号识别为主,动态图像监测为辅。为了避免无意义的警报,一般会过滤掉一些图像,例如移动速度和占据屏幕的比例低于一定数值的。”

“铁血的单元都是些战争机器,它们和人类不同,不会松懈也不会看漏的。”突击手说。

“的确,这就像是赤手过雷区。”队长说,“不过,姑且试一试吧。要是运气好,只靠匕首也能在雷区里犁出一条路来。”

几个人俯下身朝着桥头爬过去,慢慢爬到了桥头。

“这里真的有铁血吗?”队长低声问。

“有。就在那边的房子里。”支援射手说,“总数大概有二三十个,信号清晰。”

“靠。”突击手说,“那为什么它们一点动静都没有?”

“不知道。也许这次真的该我们走运,它们没有发现我们?”支援射手说。

“无论如何,快点离开这里。”队长说。

几个人奋力匍匐着前进,但这种姿态相当消耗体力,而且无论怎样都很难提高速度。她们爬了一阵,速度渐渐慢了下来。

“我看到前面的桥边上有爬梯,那是不是桥墩的位置?”突击手说。

“没错。我们快走到一半了。”队长说。

“铁血那边一点动静都没有,你的侦测是不是有误啊?”

“不可能。”支援射手说,“我明明……等等,不好!”

“怎么了?”

“铁血的信号突然变得活跃了,它们被唤醒了!我们一定是被发现了!”

队长扭过身,看到后面的建筑里的确正在涌出大批铁血的人形。但奇怪的是,那些人形没有立即向她们冲过来,而是先组成了有序的队形,然后一列列地开始向桥上前进。

“我们的确触发了它们的警报,但它们还没有发现我们,不然现在已经该开火了。”队长说,“它们的监视区域不在桥头,而是在桥中间。”

“我们怎么办?”突击手焦急地问道。

“继续向前,以最快的速度!”

几个人又往前爬了一段,来到了大桥的正中,桥面拱起最高的部分。前面的桥面是下坡,对面的情况一览无余,但几个人向前方看去的时候,见到的景象却让她们几乎惊呆了——大桥的对面,河岸的另一边,聚集着大量的感染者,密密麻麻数不清有多少。

“这下完了。”支援射手喃喃地说道。

“我们沿着桥墩上的爬梯爬下去!”突击手说。

“铁血就连巡逻路径上的一只老鼠都不会放过,我们不可能不被发现。”队长说,“到时候上不去也下不来,我们就真成活靶子了。”

“早知道这样,还不如带上装备。至少能抵抗一阵……”支援射手咬着牙,懊恼地说。

“你没看见它们有多少人吗?”突击手说,“就算把我们的全部武器都用上也对付不了!”

“别慌。”队长说,“对于铁血来说,我们和感染者是没有区别的。我们也许可以利用这一点。”

“你在说梦话吗!”

“只能赌一把了。我们继续向前爬,然后钻到感染者群里,让它们帮我们挡住铁血的火力。当铁血被它们吸引的时候,我们再趁机冲出去。”

“我才不要!我……那些感染者我在远处看着,就浑身起鸡皮疙瘩……”突击手说。

“它们会撕碎我们的吧。一定会的。”支援射手也低声说道,“它们的爪子和牙齿那么尖。我宁可让铁血用离子枪把我烧成灰烬……”

“冷静点!”队长喝道,“我们身上的作战服有二级防穿刺效果,能够抵挡一定的伤害。你们都做过肉搏训练吧,不要开枪、保护好自己的要害,尽量挡开它们的攻击。用不了几分钟铁血就要走过来了,枪声会吸引这些污染者,我们就向反方向冲出去。我们能做到的!”

“可是,我感觉……很害怕……”突击手的声音越来越小,似乎就要哭出来了。

“我呢,想要再见我的指挥官一面。”队长说。

“嗯?”

“比起鱼子酱和肖邦,我更想我的指挥官。我们之间还有没能实现的约定,所以无论如何,我都想再见他一面。”队长说,“我要活下去、必须要活下去,这样我们才有再见的可能。这就是我在这里像苍蝇一样钻来钻去的理由。你们呢?”

“我也想……”支援射手说,“我出发之前的那天晚上,我的指挥官说要我一定平安回来,因为他不想失去我……但是因为他当时喝醉了,我不喜欢他醉醺醺的样子,所以没有搭理他,离开的时候也没有告别。我现在感觉好后悔,有太多本该说的话,却没有说……”

突击手惊讶地看着两个人,终于流出了泪水。

“我……我也是啊……”她说,“那个家伙,还欠我……欠我好几顿饭呢……”

“别哭,我们还没完。活下去就有机会。”队长说,“这些感染者的反应速度并不快,集中全部精力躲避的话,它们没那么容易伤到我们。这一点我可以肯定。”

“好吧,”突击手点了点头说,“我知道了。”

“还是害怕吗?”队长说。

“……有一点。”

“害怕的话,就想想自己思念的人。我们要为了他们活下去。”队长说着拔出了战术匕首,“现在,白刃出鞘。”

突击手和支援射手也抽出了匕首,三个人互相看了一眼。

“去吧!……”

感染者的数量虽然庞大,但训练有素的铁血战斗力更强,那些失去了自我、只知道撕咬的怪物,最终还是被铁血消灭掉了。这一点让队长有些失望,她本希望污染者能够占据上风的,因为毕竟这些没有思想的家伙更好糊弄一点。

但无论如何,幸运的是三个人都趁乱逃了出来。

“你的伤势,不要紧吧。”队长说。

战斗中队长和突击手避开了多数感染者的撕咬,只受了轻伤,但支援射手就没那么幸运了。由于格斗经验不足,支援射手全身多处负伤、腿上被撕了一条大口子,这让本就机动性不佳的她处境更加艰难。

“还能坚持。”支援射手说。

“你的腿不能再乱动了,强行走路会让伤口撕裂得更厉害。”队长说,“我来背你。”

“还是我来吧。”突击手说道,“要论负重能力,我怎么说也比你强一些。”

“没必要。”支援射手说,“你们走吧,不要管我了。”

“怎么到了这时候,反倒开始说这种话了?”突击手说,“经历了刚才那场,我可是更加明白活着的可贵了。你不这么想吗?”

“正因为如此,我才让你们走。你们都有活下去的理由,而现在的我不仅不能战斗,还会拖你们的后腿……我不想因为我,而毁掉大家逃生的希望。”

“我知道你现在的感受。失去了烙印武器的确是件痛苦的事情,我也体验过。”队长说,“对我们来说,那东西就像是生命的一部分。一旦失去,心中那种空落的感觉,让人开始怀疑自己存在的意义。只要过一阵子就好了。如果你感到不安,拿着这个或许会好些。”

说着,队长从腿上取下了防身的手枪,递到了支援射手的手里。

“……你有过这样的经历?”支援射手接过手枪,惊讶地问。

“有过一次。”

“我想象不出什么情况下,我会丢掉自己的武器。就算是我死了它也该紧握在我的手里。”突击手说,“你这家伙的过去,绝对不只是什么‘按部就班的军营生活’吧?”

“多数时间都是,但也有点其他的。”

“除了战区,你还在其他地方生活过?”支援射手问道。

“是的。我做过私人护卫和特勤人员,也曾在民间……和分公司呆过。”

“民间和分公司?在那些地方做什么?”支援射手吃惊地说。

“餐厅的服务员,和一些文职工作。”

“快给我讲讲!”突击手叫到。

队长难得地微微笑了笑。

“好吧。”她说,“如果你们能打起精神来继续前进的话,我可以给你们说点故事,打发路上的时间。”



 \section*{}

由于支援射手执意不肯接受让队友来背负,于是她们不得不换了一种折衷方案——找了一根粗壮的树枝做拐杖,然后由两个人轮流搀扶着支援射手行走,将她因伤而对行军速度的影响降到了最低。这是经过仔细权衡的做法。

雷达站坐落于距离河岸大约不到3公里的山林中,由于位于高处,因此需要盘山而上,实际上走的距离远不止3公里。一个人背着另一个人的话,速度会慢许多,而第三个人为了统一步调又不得不跟着放慢速度,这样整体的速度至少会慢一倍。而三个人共同前进的话,虽然慢了一些,但不会慢太多。

几个人逃出那场混战的时候,铁血还在清理污染者,虽然它们不太可能追上来,但还是不能掉以轻心,因为三个人手里的弹药就连保护自己都捉襟见肘。

“就年龄而言他早已经是成年人了,而且经历过许多战斗,军事素养极高。但有时候,也会做出让人难以理解的行为。在一次战斗中,由于铁血使用了病毒武器,他手下的精锐小队遭遇了严重伤亡。他为此意志消沉,于是就一声不吭地,独自离开了战区……”

队长讲的,自然是她那位沉闷而不合群的指挥官的事迹。那些让人匪夷所思的事情,牢牢抓住了她两个队友的心。

“那不是擅离职守吗?”突击手问。

“所以才说难以理解。”队长点了点头说。

“总部一定大为震怒吧。”支援射手说。

“倒是说不上大为震怒,但也下了命令,让我去把他追回来。“

“然后呢?”

“然后,我追上了他……跟着他去到了一个陌生的镇子,在那里呆了大约三、四个月。”

“你没有把他带回战区?”

“没有。”

“你这……不也是抗命吗!为什么?”

队长沉默了片刻。她的脸上露出了复杂的表情,嘴角微微翘了翘似乎是想要微笑,但皱起的眉宇却流露出一丝寂寞的无奈。

“他作战的风格一直都很骁勇,在那次之前,我一直认为他是个铁腕的战士。”队长说,“但那时候,他的样子失落极了。他一个人走在不知通往何处的路上,背影充满孤独和迷茫,让人忍不住想要同情。所以我就……没有执行总部的命令。”

“就因为这个?”支援射手摇了摇头说,“就这样,你就违抗了总部的命令?那可是差不多最高等级的指令啊,我想不明白。”

“想不到,队长其实是个温柔的人呢。”突击手小声咕哝道,“而且很浪漫。为了一个人的背影,可以就连最高指示都弃之不顾……真是太浪漫了。好羡慕啊……”

“我说,你虽然一直张牙舞爪的样子,该不会其实是个……充满少女情怀的小鬼吧?”支援射手满脸的不可思议,她感到她对自己队友的印象正在发生颠覆。

“少女……少女情怀怎么了,我也是有个人喜好的!这叫做性情,性情你懂吗!”突击手涨红了脸说道,“队长不也一样吗,看似高冷,其实是个纯情种!反差也是一种萌点!”

“好,好。我感觉今天可学到了不少东西。”支援射手似乎陷入了思考。

“别打岔,听队长继续讲。然后怎样了?”

“然后我们就在镇上生活了一段时间。因为不能暴露自己的身份,我们谎称是……伙伴,他在码头上找了份工作,算是隐居了一阵。”队长说着脸色有些发红,因为他们那时候的身份并不是她说的“伙伴”,而是更为亲密的关系。

“听起来就像是,私奔一样啊!”突击手叫了起来。

“也许……差不多吧。我们在那里呆了大概两三个月。直到有一天,总部派人来了,向他指派了新的工作,我却没有得到下一步命令,就留在镇上做服务生。之后又过了一阵,我也被召回了总部。”

“然后呢?”突击手问。

“然后,就各自去做自己的事情了。”

“你们就分开了?” 

“是的。因为被安排了不同的工作,我们都没有再回原来的战区。”

“再就没有再见吗。”

“有再见。大约半年后,我们都被派往同一个分公司,在那里从事了一段文职人员的工作,之后我就被召入了SOG小队。”

“我听得有点糊涂了。”支援射手说,“你是说,你那位指挥官现在还留在分公司?”

“不,他也来这边的战场了。你们和他见过面。”

“哦!你是说,就是那天晚上的那个——”突击手一拍巴掌。

“和某个身份不明的人形在一起的男人。”支援射手补充说道。

“是的。”队长点了点头。

“叫什么来着?”突击手皱起了眉,“我对他的印象已经不深了,那是个不起眼的人。他说自己是格里芬的军官吧,来这里是在干什么?”

“我也不知道。”队长说,“他的任务和我们一样,涉及一些保密事情,所以我没有问。”

“……啧,你可真是个死板的家伙。”突击手不满地说道。

“我们已经不是同一个战斗序列里的成员了。”

“这么说来,那天晚上我们是被故意支开了吧。那你们有没有做……就是,那个……?”支援射手小声问道。

“这算是什么问题啊,太露骨了!”突击手叫道,“还说我少女心,你这不是爱八卦的大婶才会打听的事情吗?”

“没有。”队长淡淡地说道,“只是交换了一些关于战场的信息。”

“那以前呢。以前有过没有?”

“队长这种家伙,怎么可能会……!不过既然她懂,那么说不定也有可能?到底有没有?”

队长没有回答过度关注自己的队友。她没有说话,而是微微昂首看向远处,一路上渐渐放松的表情忽然再次凝重了起来。

她们为之不断辗转的地方已经进入视野,但同时进入视野的还有那些建筑物上痕迹。远处雷达站的外墙上满是疮痍,那些残破的墙壁和窗户,显然是武器造成的破坏。

“我们马上就要到了,做好准备。”队长沉声说,“做好战斗的准备……或许,还有白来一趟的准备。”

毕竟是训练有素的士兵,听到队长的话,两人立即离开了主路、进入了警戒状态。几个人端起各自的武器,挑选了合适的掩护物,开始小心地朝着雷达站前进。

“你的腿怎么样,走路能行吗?”队长走在最前,一边走一边说道。

“感觉好多了,不碍大事。”支援射手说道。

“侦测到什么信号了吗?”

“没有。也许是太远了,至少附近没有任何电子信号。”

“小心点,密切留意四周。”

“明白。”

十几分钟后,她们来到了雷达站的外围。围墙残破不堪,许多地方已经倒塌了,在外面可以看到里面的建筑也满是破损,这里明显是发生了激烈的战斗。

“侦测到什么了吗?”几个人躲在围墙下,队长压低声音说。

“没有,什么都没有。这个地方简直一片死寂。”支援射手小声说。

“它们会不会是藏起来了?我是说铁血。”突击手说。

“可能性不大,我们还没有见过刻意隐匿自己信号的铁血。”队长说,“但不能掉以轻心。我们得进去看看。”

“总觉得这里阴森森的,我不想进去啊。”突击手说。

“我也不想,但我们就是为了这个地方才费劲周折,不可能不进去。”支援射手说。

“别说话了。我先上,你们听我命令行动。”

队长说完,纵身越过围墙进入了庭院。眼前的场景让她惊呆了:通往雷达站主建筑的路不过三四十米,但却布满了弹坑和烧焦的灰烬,还散落着许多残缺的报废人形——

全都是格里芬的战术人形。

“怎么没动静了,队长?什么情况?”

直到听到突击手的呼唤,队长才回过神。

“嗯……安全。你们过来吧。”

“好,我来了……我操,这是什么?!”突击手从围墙的破洞里钻了进来,然后惊恐地说道。

“战术人形。或者说战术人形的碎片。”支援射手倒是很冷静,“不过这些烧焦的灰烬都已经冷了,战斗至少过去好几个个小时了吧。”

“至少过去一两天了。”队长说,“这些弹坑里有积雪,但这两天没有下雪。如果是风吹进去的,那么至少得一两天的时间,雪才能积这么厚。”

“她们为什么会在这里?”突击手说。

“不知道。我只希望不要成为她们的一员。”支援射手说。

“别想那些了,我们来这里是为了我们的事情。去屋里看看。”队长说。

其实她们几个都很明白,这些战术人形很很可能是在战斗中被冲散的各部人形,她们集结在一起却遭到了铁血的猛烈攻击,退守到了这里,但最终还是没有逃过被消灭的命运。庭院里被击毁的全都是格里芬的人形,已经说明了一切——只有失败者,才会把尸体留在战场上。

几个人小心翼翼地从破碎的窗户钻进了雷达站,正如所料,屋子里也有许多被摧毁的格里芬人形,多数聚集在墙壁的破口和窗户前,手里还拿着各各色武器。她们显然是边战边退,然后被全部歼灭在房间里。

队长走到房间的主控电脑前,看到电脑上还趴着一个人形。这个人形的后背被激光烧穿了一个大洞,临死之前一定是在试图用雷达发出信号。可惜,在她发出的信号得到回应前,她就在这里和主控电脑一起被摧毁了。

“好消息是:这里的战斗已经结束很久了,敌人早已离开,我们暂时安全。”队长说。

“坏消息呢?”突击手问。

“主控电脑被摧毁了,雷达也没有能源供应。格里芬显然没能成功在这里建立通讯站。”

“也就是说我们果然白跑了一趟?”

“也不算白跑。往好处想想,这里还有不少能用的弹药和装备,找一找的话,说不定还能找到一些食物。我们可以在这里休息一下,顺便想想离开这里的其他办法。”

“其他办法,呵呵!”突击手气得冷笑了一声,然后狠狠地踢了一脚脚下的报废人形,发出一声闷响,“要是有其他办法,我们何必来这种地方!!”

“这些人形也算是我们的同伴,你不要她们撒气!”支援射手生气地说。

“至少我们还活着。冷静下来,总能找到办法的。”队长说。

“我去他妈的冷静!我现在只想打人!”

“支援射手在这里休息,顺便警戒。我去搜搜这些‘伙伴’有没有留下有用的物资。”队长对突击手说,“你要是不开心,可以在附近散散心,不过别走太远。”

“我去他妈的散心!!”

队长没有理会突击手,独自向着雷达站二楼走去。

雷达站的二楼有一个尖顶,三角形房檐正中央是一颗水泥抹成的五星,上面曾刷着红色的油漆,但是早已经褪色。这种建筑风格现在已经十分少见,很可能是上个世纪建造的。这里曾经是军事重地,但是在和平的岁月里渐渐变成了民防设施,直到战争爆发才又一次被人们想起。

被遗忘也未必不是一种幸运,队长心想,她想起了自己和指挥官在海边小镇度过的那段被“遗忘”的时光。虽然明知没有未来,但那须臾的宁静和安详,依然让她感到留恋。

她从残破的窗户朝外面看去,雷达站的另一边是起伏的山脊和树林。背向燃烧的城市废墟,硝烟的气息也随之远去了,仿佛这场地狱般的战争只是一场让人不愉快的梦。

可惜,只要是梦,那就一定有醒来的时候。

“队长!”

队长听到一声惊呼,是支援射手发出的。

“怎么了?”

“侦测到铁血的信号!就在我们四周……到处都是!!”

“队长看了看外面,她什么都没看到。但当她把目光集中到树林里时,她看到森林里有动静——

地面之下,有一些小型机械单位在破土而出。它们有着箱形的身体和四条机械足,身体前方是侦测探头和激光武器,身体上装饰了树枝草叶做伪装,静止的时候很难被发现。这些小东西的数量相当惊人,正在快速向着门前方向集结。

“那是什么,兵蚁吗?!”突击手惊恐地说道。

“是行军蚁。”队长说,“兵蚁的加强版,装甲更厚、火力更强,却同样跑得飞快以及射程很近。它们的作战方式是集结到一定规模,然后集体冲锋。把重型武器收集起来,面向正门方向建立防御!”

“我操他妈的!”

突击手四下张望,在墙壁的破洞旁看到一挺机枪。她跑过去提起机枪和弹药箱跑到门口,却发现支援射手已经在那里了。

“让开!”突击手喊到。

“该让开的是你。”支援射手拿过突击手手里的机枪,泰然说道,“操作重型火力是的我工作。”

“这里连掩护物都没有,你腿又受伤,呆在这里找死吗!”

“你帮我搭个掩体不就好了?”

“哪来材料搭掩体?!”

“这不满地都是吗。”

支援射手用下巴一指,突击手楞了一下,因为支援射手所指的方向,她并没有看到木板和沙袋。但旋即,她明白了支援射手的意思,支援射手是让她用战术人形的残骸做掩体。

“她们不是我们的伙伴了吗?”突击手说。

“当然是。我们这不是马上就要加入她们了吗。”支援射手笑了笑。

“为什么我总是要遇到这种倒霉事啊?”

“队长不是经常说吗,往好处想想。至少你想打人的愿望实现了,现在你可以大打出手了。”

“哈,那家伙。去他妈的往好处想……”

队长找到了一部榴弹发射器,当她把那东西以及剩余的弹药搬过来的时候,看到了让她惊奇的一幕:雷达站的房门前,堆起了一列由人形残骸和墙体碎块组成的掩体。支援射手驾着机枪稳坐其中,突击手则将备用枪管和一条条弹带整齐地罗列在一旁,两个人脸上都带着笑意气氛融洽,不像是两个大敌临阵的士兵,而像是两个结伴出游、捧着便当谈笑正欢的伙伴。

“怎么,终于找到共同的话题了吗。”队长把榴弹发射器装在支架上说道。

“是啊,我们在讨论你。”支援射手望着敌人聚集的方向说道,“我一直觉得你这个人不太讨人喜欢,生硬冷漠、没有生气,甚至好像没有感情一样。现在想想其实你挺负责任的,你许下的承诺都实现了。虽然我们无法活着离开这里,但至少我们已经竭尽全力了。就像你说的,现在我死也瞑目了。”

“我……确实是个虚张声势的胆小鬼,怕死、怕疼,一想到要上战场我就紧张得不得了。我的‘精英’的头衔全是在演习和模拟战斗中获得的,说实话,这是我第一次实地作战。”突击手说,“但这些天的经历,让我感觉这些事也不是那么可怕了。我想如果生命不是如此脆弱和短暂,也就不会这么的可贵吧。我忽然感觉有很多遗憾的事情……比如说,我们没法再更多地相互了解了。”

“看来你们心里已经对我们的结局有了构想了。但我和你们想的不一样。”队长说,“我呢,确实,是个不讨人喜欢的家伙。因为这个世界让我感到厌烦透顶。不过,人活着并不是为了这个世界,而是为了自己心中在意的事情和人。或许在他人看来,那是很普通的事情或者很平凡的人,但有了他们,这个世界的存在才有了意义。这就是我这些年学到的东西。所以,我不会放弃的,因为这个世界有我愿意为之奋力活下去的人。子弹打完就用匕首、匕首折断就用双手,哪怕有一万分之一活下去的可能,我也会战斗到最后一刻。”

“……好一番豪言壮语。”支援射手笑了笑说。不远处的敌人已经开始躁动,于是她拉了一下枪栓,把手指放在了扳机上。

“那个让你为之奋力的人,如果能听到这些,一定会非常感动。”突击手托起了弹带。

“它们来了。”队长调整好榴弹发射器的射角,将榴弹链塞进了发射管,“自由开火吧,祝我们好运。”



 \section*{}

在敌人冲锋之前,支援射手抢先扣下了扳机。遭到攻击,敌人立即炸了窝,成群地扑了过来。

机枪喷吐着火舌,冲在最前的敌人立即被放倒了一片。但它们的冲锋没有减慢一拍,因为它们没有知觉和恐惧,只有预设的消灭目标的程序。敌人迫近,队长发射了一串榴弹。火焰在敌群中爆开,第一波冲上来的敌人被消灭了大半,但还是有几只漏网。突击手端起自动步枪,在它们冲到面前之前消灭了它们。

一波进攻之后,又是一波。第三波敌人被消灭之后,它们的进攻停了下来。它们在庭院之外再次聚集,但比上次远了不少,这次机枪不能直接命中它们了。

“它们为什么停下了?”突击手说。

“它们似乎是在试探我们的火力。”支援射手说。

“是的,根据对方的反应,再采用针对性的进攻战术。”队长说,“虽然是自律作战,但这种策略显然是经过了巧妙的设计的。”

“这么说我们想要击败它们,必须要技高一筹了?”

“再高明的计策,也需要一定规模的部队才能实施。我们三个人组织不起来复杂的战术,它们恐怕也不会给我们这种时间。”

“注意,它们又来了!”

敌人再次向正门发起了冲锋,但这次它们派出了两股部队,一股冲击正门,另外一股从墙壁破损的地方分散地钻了进来。三个人奋力作战,虽然再次击退了敌人的进攻,但这次有好几只行军蚁已经冲到了她们的面前,队长和突击手都受了轻伤。

“这次敌人的数量增加了一倍,榴弹发射器散热不及,无法维持这么快速发射了。”队长说。

“它们马上就会发现,只要派兵足够多,我们的火力就会应对不暇。”突击手说。

“我们不能让它们试探到我们的火力极限。”队长说,“我们需要一些更有力的武器。”

“那边有一些装了遥控引信的炸药。”突击手说,“做个陷阱怎么样?”

“姑且一试吧。我们把它放在院子大门口外面一点的地方,快!”

队长抓起炸药,把遥控器塞到支援射手手里,然后和突击手一起跳了出去。当她们跑到院子门口的时候,敌人也开始行动了,因此她们只能慌忙把炸药仍在了门口两侧,然后立即转身往回跑。

“快引爆!”队长说道。但支援射手默默地看着前方,没有动手。

“操,还等什么!!快他妈的按那个按钮啊!!!”突击手转过身,顾不上瞄准胡乱地扫射了一通,成群的行军蚁几乎已经冲到了她的面前。

轰!!

突击手被灼热的气浪包围,感觉就像掉进了锅炉,天旋地转、并且全身都在燃烧。然后,她重重地摔在了地上,眼前一黑失去了知觉。

“……你还好吗。……能站起来吗?”

“呜……”

突击手呻吟了一声。她听到有人在和自己说话,但因为意识涣散,一时想不起这个声音是谁。然后,她感觉自己被拎了起来。

“我能走……放开我……”突击手咕哝着说。

“你不像是能走的样子。”

“我没事……!”突击手叫道。她因为受到震荡而眩晕的脑袋,慢慢恢复了意识。

她睁开眼,看到对她说话的是队长。她想要扭头,却感到一阵头疼欲裂。

“别动。你最好休息一下。”队长说。

“我昏过去了吗。昏了多久?”

“也就两三分钟。气浪把你击晕了。”

“怎么样了?”突击手大口喘着气说。

“还好,没缺胳膊少腿。”

“我是说敌人。”

“被爆炸消灭了一大部分,暂时停止进攻了。”

“……你怎么没事?”

“托你的福,在那家伙引爆炸药前,我躲进了一个弹坑。”

“哈哈……咳、咳……”

突击手挣扎着爬了起来,但刚一抬腿,就摔倒在地。血从她的嘴角渗了出来,虽然没缺胳膊少腿,但她还是被气浪震伤了,而且伤势不轻。

“你得休息一会儿。”队长说。

“敌人……会休息吗。”突击手说。

“要不你去和它们谈谈,一起休息一会儿?”

“我去他妈的……谈谈……”

“队长。”一直沉默的支援射手忽然说话了。

“嗯?”

“我想你得看看这个。”

队长向着远处看去,看到行军蚁没有再次集结,而是分散地守在庭院边缘。它们的数量,似乎没有之前那么多了。

“它们没有再次集群。它们的数量,好像比之前少了一些。”支援射手说。

“但就算是现在的数量,也不是我们能够对付的。”

“我推断,它们的数量是有限的,现在已经无法达到集群冲锋的规模了。”支援射手说,“我们只要继续削减它们的数量,就有突围的可能。”

“它们对我们的情况没有主观上的认识,但是通过这几次进攻,至少掌握了一些数据。”队长说,“你看,它们现在的位置,正处于我们的射程之外。这说明它们已经试探出了我们的火力范围。能够设计出这种策略的人,不会想不到单一兵种的弊端。也许它们的数量并非减少,而是分散了。”

“你是说它们想要围困我们?”

“也许不止如此。面对仅凭手中的火力无法攻克的堡垒,你会怎么做?”

“难道说……它们在请求支援?!”

队长仰头望向树林深处,虽然看不到那边有什么,但是可以隐约看到高处的树枝正在颤动,显然是有大型物体正在朝这边移动。

“大概就是这样。”队长点了点头。

“听你们的话,我们是不是有点不妙了……”突击手嘟囔着说道。

“是的,和一直以来一样的不妙。”支援射手说。

“那就是说不是什么大事。”突击手说。

“嗯,最多也就是马上就要完蛋了。”

噗嗤,突击手笑了起来。

“唉呀,挣扎到最后还是这样。可真惨啊。”她说。

“是啊,哈哈。”支援射手也笑了。

“还没到最后。”队长说,“枪还能用、弹药也还有一些,我们还能战斗。”

“别闹了。如果你的冷静不是假装的,那你就该知道现在怎么做才对。”突击手说。

“该怎么做?”

突击手努力地坐了起来,看了身边的支援射手一眼。

“我已经没法再跑了,而这家伙想跑也跑不动。我们的逃亡就到这里了。”突击手说,“你自己突围吧,队长,趁着它们的重火力单位还没来。你是我们三个中速度最快的,你还有希望。我们去吸引一阵火力,你趁乱逃出去,应该不成问题。”

队长看了她们两个人一眼,没有说话。她默默拿出自己的通讯器,打开了开关。为了保持电子静默,她们都把通讯器关掉了,但现在已经没有必要了。

队长凝视着手里的通讯器,总部的波段已经失效,但里面还保存着另一个波段。她当然知道不可能接通,没有中继站,通讯器发出的信号最多只能传到几百米的范围,况且还有坍缩液炸弹带来的辐射的干扰。通讯器这东西,其实早已经没用了。

但她还是选择了那个波段,然后按下了呼叫按钮。

传来的是一阵杂乱无章的沙沙声,那是电离辐射产生的噪音。

队长无奈地笑了笑,扔掉了通讯器。

“赌注有多少了?”队长开口说道。

“什么赌注?”支援射手说。

“你们不是一直在拿我打赌吗?”队长说,“赌我到底是何方神圣。突击小鬼猜我是个偏僻的小地方来的村姑、而你则认为我是从总部来的钦差,不是吗?”

“……你这家伙,已经知道了啊。”突击手说。

“从一开始就知道了。赌注有多少了?”

“六十块,或者八十块吧。来到这地方以后就没再计算了。”

“都要归我了。”

“哈?!”

队长站起身,拍了拍身上的土,她看向外面,敌人的支援已经隐约可见——虽然茂密的树林遮挡了它上方的轮廓,但从那爬行的四条机械足来看,显然是一部“蝎尾狮”重型机甲。

“我的身份,对我自己来说也是一个谜,出身也好、原型也好,我从来没有去探寻过。我本是一件只懂服从命令的商品,多年里更换过许多素体、变更过许多所有者,每一次改变,都意味着不同的经历和记忆,我现在甚至不能确定那些记忆哪些是真、哪些是假。虽然那不是我的名字,但大多数时间里的大多数人,都叫我‘Vector’。”队长说,“我来自北部战区的17号区域,的确是个偏僻的小地方,但我是奉克鲁格元帅本人的命令去那里的。所以你们的猜测,只能说各自猜对了一部分。我的指挥官的名字,是‘陆久’——N17战区的前任指挥官、北部军团的现任总指挥官。而我,是陆久的战友、副官、以及用生命许下过承诺的爱人。你们听明白了吗?”

突击手和支援射手呆呆地看了队长一阵,然后互相对视了一眼。

“还有,你们想象过的那些事情,我们全都做过,而且做过许多次。”队长又补充说道。

“哈,真是想象不到啊。”支援射手说。

“绝对想象不到。特别是你和指挥官欢爱的情景。”突击手也说道。

“我不会独自逃离的。”队长说,“绝不抛弃一同踏上战场的战友,这是陆久身为指挥官时的信条,我也同样会这样做。现在,我命令你们准备战斗。”

“妙啊。我竟然遇到了个不怕死的家伙,还满口大义凛然。你是在鼓舞我们的士气?你真把自己当成指挥官了?”突击手讥讽地说。

“没用的。”支援射手摇了摇头,“无论人数还是火力,我们都无法和它们抗衡,这一点显而易见。就算战斗也是同样的结果。”

“至少战斗能够把生存下去的可能性提高一点。”队长说。

“从万分之一提高到千分之一吗。没有意义。就算你决定搭上自己,还是没有意义。”突击手说,“如果你决定一起死在这里,我不反对。但不要再号召我战斗了,我累了。等到确认了敌人的支援火力,我就在这里了结自己。就是这样。”

“我也是,不想再做徒劳的抵抗了。”支援射手说,“如果必死无疑,我希望能安静地死去。谢谢你的手枪。”

“你们已经彻底绝望了吗。”队长说,“回到重要的人的身边,已经不值得期待了吗。”

“别再自我欺骗了!”突击手叫到,“我们早已被抛弃了,不然的话,就算是失去联系,也该有人来找找我们。我们心里一直一来所抱的希望,就是向什么人求救,但事实上是,就算我们成功联系到了总部,他们也不可能会救我们!我们不是人类,我们是出于经济性考虑才被制造出来的代用品,不可能得到超出成本预算的救援,这一点我们心里都明白,这就是战术人形注定的命运!我们只是自己心有不甘,才一直挣扎到现在。不信你问旁边这个家伙,她真的相信会有人来救我们吗?!”

队长看了支援射手一眼,没有说话。不必去问,从那黯淡的眼神里就已经看出,支援射手也放弃希望了。

“我明白你们的想法,我曾经也是这么想的。我曾经对这个世界厌倦至极,我漠视生命、漠视自己,漠视一切。但和指挥官相遇后,我经历了很多、也改变了很多。”队长说,“我不喜欢‘命中注定’这句话。人形是人类的造物,就注定是人类的附庸吗。就注定是任人摆布、可以随意丢弃玩偶吗。就注定是必须默默无言地代替人类去战斗和受伤、失去使用价值后就安静地自我毁灭的工具吗?不。不管你们怎么想、不管他人怎么想,我不这么想。你们说得没错,救援大概是不会来了。但就算是这样,我们的生命,就不值得自己为之战斗了吗。”

突击手和支援射手没有出声,她们甚至没有去看队长。队长的话,她们仿佛没有听到一样。

“这个世界正并不是我们被告知的那样。在人类之中,也有许多会把我们当成伙伴和同类对待,前提是我们得自己珍惜自己。一个人并不是因为受到别人的认可才成了一个‘人’,而是因为他认为自己是人、他希望得到像人一样的尊重和对待,因为他,生 而 为 人 。但如果你们自己还把自己当做物品的话,那就永远无法改变了。”

队长说着朝外看了一眼,她看到一台庞大的战斗机器已经从树林里钻了出来。

“蝎尾狮来了——我们的攻击对它是没有作用的,但在这复杂的地形当中,它的威胁没有在开阔地那么大。我的策略是最大限度地消灭那些速度极快的行军蚁,然后利用建筑物牵制住蝎尾狮,然后逃命——如果那时候我们还有命的话。这大概就是最后的战斗了,要是你们不愿意干,我一个人干。就这样吧。”

“你叫什么名字?”支援射手问,“你说Vector不是你的名字,那么,你也是有名字的吧。”

“陆薇,‘陆’是陆久的陆,‘薇’是一种花的名字。你呢。”

“我叫卡尼娅。”支援射手说,“Kal'nya,是星星的女儿的意思。”

“我……我叫泰洛丝,名字来自一种葡萄酒。”突击手说。

“很高兴和你们一起战斗,卡尼娅、泰洛丝。”队长伸出了手。

“很高兴和你一起赴死,陆薇队长。”支援射手也伸出手,和队长握了握。

“你们这些讨厌的家伙,我不高兴、也不想死。我要拼了命的活着!”突击手在两个人的手上拍了一下。

最后的战斗持续的时间并不长,因为三个人根本无法抵抗蝎尾狮的强大火力。仅仅过了不到10分钟,她们的防线就崩溃了。突破防线的行军蚁接连不断地钻进了房间,支援射手端起机枪边打边退、队长和突击手在她的周围只能勉强掩护。

“撤!上二楼!”队长喊道。

“敌人如果占领一楼,我们的退路就断了!”支援射手说。

“顾不上那么多了,一楼已经没有多少掩护,外面还有很多行军蚁!无论是出去还是留在一楼,都会被烧成灰的!”队长说。

“那台蝎尾狮火力太它妈的猛了!得想办法干掉它!”突击手说。

“不可能的!上楼梯,我来断后,快!!”

几个人匆忙撤向二楼,顺便在楼梯上消灭了几只行军蚁,蝎尾狮的轰击也暂时停止了。

“它似乎不会在没有引导的情况下进行攻击。”支援射手说,“可以暂时喘口气了。”

“那就抓紧整理一下武器装备。行军蚁迟早会再次找到我们的。”队长说。

“你们有没有听到,像是飞机的声音?”突击手说。

“没有。”队长说。

“我也没有,铁血没有能在高空飞行的单位吧?”支援射手说。

“奇怪了,我明明听到一阵轰鸣从上面过去了。难道不是铁血?”突击说。

“专注战斗。别想那些没用——”

轰!!

一声巨响,几个人隐蔽的地方墙壁被炸出了一个大洞,是蝎尾狮在用电磁炮轰击。

“咳、咳……你们没事吧?”队长说,“有人中弹了吗?”

“我没事……这混蛋怎么突然就开火了?”突击手说。

“楼下的角落里有一只行军蚁在引导攻击!”支援射手说道,“我们必须消灭它!”

“小心,散开、散开!”队长喊道,“它又要射击了!”

轰!!

又是一炮,把二楼朝外墙壁炸掉了一大半,三个人完全暴露在蝎尾狮的视野中。紧接着,几枚榴弹飞了进来,三人急忙躲在雷达的电动机后面才没有被炸成碎片。

“这下完了!”支援射手绝望地说道,“我们完全暴露在它的火力范围之下了!”

“继续开火!”队长说,“行军蚁已经不多了!只要消灭这些家伙,我们就有机会……”

“你们有没有听到那个声音?飞机,这次更近了!”突击手说。

“谁还顾得上那些?!”支援射手叫道。

“不,我好像也听到了。”队长说着向蝎尾狮的方向看了一眼,脸上露出了惊恐的表情,“那是什么……?不好!离开这里,快走,快!!”

轰!!!

随着一阵巨大的爆炸声,热浪和烟尘笼罩了整个庭院。雷达站的二楼被炸塌了,雷达倒了下来,几个人也落到了外面。

要不是队长在最后时刻奋力将两个人拉了起来,她们现在也许已经被埋在雷达站的废墟之下了。

“那是……什么玩意?哎哟……”突击手躺在地上呻吟着说道,“那个四爪怪,不该有这么大威力的武器……”

“我看见有什么从天而降,击中了那只蝎尾狮。”支援射手说,“像是巡航导弹之类的东西?”

“是飞机。”队长说,“我看到了。击中它的是飞机……的确是飞机。”



 \section*{}

几个人的目光,齐刷刷地向着蝎尾狮所在的位置看去,只见原本矗立在那里的机械怪兽已经无力地瘫倒在地上,机械爪上面的主体部分被全部削掉了。再往后面,是雷达站的废墟,废墟上有一部喷气引擎正在燃烧、还有一部引擎落到了远处的地方。

“这是哪来的飞机?”突击手吃惊地说道。

“不知道。不过,我们好像得救了。”支援射手说。

队长则没有说话,只是默默地环顾着四周,像是在找什么东西。片刻后,一个穿着黑色防护服的人影走出浓烟,出现在雷达站的废墟上面。

“看来我来得还不算晚。”那个人说道。

虽然看不清那个人的容貌,但支援射手和突击手惊讶地发现,说话的竟然是个男人的声音。

“不晚。”队长回答道,似乎是认识那个人,“应该说正是时候。”

“那就好。我一直担心能不能及时找到你们。”男人跳下废墟,走到了队长面前。

“是你的飞机撞毁了那部机甲吧。”队长问。

“确切地说,是准将先生的飞机。”

“这下你的打火机再也要不回来了。”

“要是能把你捞出来,打火机就留给他好了。反正飞机我也赔不起。”男人耸了耸肩,“我收到一个无线电信号才定位了这里,你是不是呼叫我了?”

“……是的。”

“呵,是吗。”男人轻轻地笑了一声。

“你……请问,您是谁?”支援射手终于忍不住问道。

“我叫陆久。”男人回答道,“我们之前见过面。”

“你就是那个,北部军团的总指挥官!”突击手说。

“前任指挥官。现在已经是通缉犯了。”

“呃……这个。”

两个人面面相觑,只有队长面色平静,因为她早就知道了这件事。

“是不是发生了什么事情?”队长说。

“是的,一言难尽。”男人回答,“简单地说,在得知你们被困辐射区后,克鲁格拒绝派出救援,我和他谈崩了。加上之前的种种原因,我和他走向了决裂。”

“原来是这样。难怪我接到命令,让我一见到你就立即杀死你。”

“那你好像没有认真执行命令。”

“我拒绝了。那种事情对我来说是不可能的。”

男人无奈地摇了摇头。

“外面发生了很多事情,也许整个世界的秩序都将瓦解,甚至陷入混乱之中。”男人说。“从今以后,我大概得亡命天涯了。”

“不论发生什么、不管去哪,我会都和你在一起。”队长轻声说。

“我不想事情变成这样,让你受这种流亡之苦。”男人说着,用手指爱怜地摸了摸队长的头发,“可是,种种机缘……唉。”

“没关系。我们不是约好了吗,只要能和你在一起,那又算什么呢。”队长微微颔首,小声说道。

听着两个人互诉衷肠的话,支援射手和突击手惊讶地发现,一路上姿态凛然的队长,此时竟然像个小姑娘一样羞涩忸怩。

“……我们去看一看附近,还有没有没肃清的敌人。”感到自己在这里有些碍事,突击手拽了拽支援射手说。

“感谢你们的体贴,不过,别去了。”男人说,“我从这里飞过的时候,看到附近铁血的部队正在聚集。我不确定它们是否是往这里来,但我们绝对对付不了它们。我们得赶紧走。”

“往哪走?”

“我拜托了一些认识的人,她们会派一架直升飞机来接我们。出于安全考虑,她们将在我发出信号后行动,并且只在这里停留5分钟。我已经发出信号了,她们会在35分钟后抵达预定地点。汇合点在河边,我们马上行动的话,时间应该还算宽松。”

“我们在河边遇到了大量的感染者和铁血,沿河两岸都不安全。那里不是那么容易去的。”支援射手提醒道。

“我知道。”男人说,“我们可以利用地下的排水管道去那里,没有辐射也没有敌人,很安全。我已经看过了这一带的排水网络图,雷达站后面就有一个下水道入口。”

“不愧是总指挥官,真可靠啊。”突击手赞叹地说。

“哪里,都是别人提供的情报。事不宜迟,我们马上动身。”

几个人打开下水道的井盖,依次钻了进去。上下的通道虽然只能容下一个人,但进去之后发现,里面的管道其实相当宽敞,三四个人并排走也不拥挤。男人带头在前,队长在最后,突击手搀扶着支援射手走在中间,用了二十分钟就接近了排水口的出口。

“我已经能感到明显的空气流动了,出口应该就在附近。”男人说着停了下来,“我们在这里休息几分钟吧,提前出去在外面等着,反而不安全。”

“我们真的就这样逃出来了吗,感觉像是做梦一样啊。”突击手说。

“是啊,我也是。”支援射手说,“就在半个小时前,我们还被铁血的机甲堵在一块巴掌大的掩体后面,那时候我觉得这次绝对要死了。可转瞬之间,我们就到了安全的地方,还有直升机接应……感觉真的就像是做梦。”

“哈哈,说实话,这一路上我都觉得背后有追兵呢。”突击手说,“就好像还有行军蚁跟在背后一样,总是感觉不安。”

“等等,我也一直隐约感觉背后有动静。”队长忽然说道,“卡尼娅,你侦测到什么了吗?”

“我确实侦测到附近有铁血信号,但我一直以为是在地面上,所以没怎么在意……”

支援射手惊恐地说道,几个人猛地朝后面一转身,同时打开了战术手电。

明亮的白光把下水道照得一览无余,身后空荡荡的,什么都没有。

“什么啊,根本没人。疑神疑鬼的,紧张得发神经了吧。”突击手笑着说道。

“如果是行军蚁的话,最好不要掉以轻心。”一直沉默的男人忽然说道,“这东西经常成群结队地出现,给人一种靠数量取胜的印象,但其实拥有相当庞大的策略库,是一种非常狡猾的敌人。数量越少,它们采取的策略会越谨慎,我甚至见过它们钻到地下伪装起来,等我们的人形经过时自毁,就像地雷一样。它们可以让自己的电池在几秒钟里短路过热,引起强烈的爆炸,能量相当于几公斤的黄色炸药……”

男人小心翼翼地往回走去,他仔细地检查了来时路上阴暗的角落,但那些角落里确实什么都没有。

“是我多心了吗。”男人自言自语地说着,转身准备往回走。就当他转身的一瞬,他的余光在高处看到了一个暗红色的光点,那是红外射灯发出的微光。

他猛然一抬头。

“在上面!”男人高呼一声,“开火,击毁它……不好,它已经开始自毁了!散开、散开!!”

那是队长一生中感到最为后悔的事情,她本该跑向自己的指挥官的,但因为条件反射,她选择了和队友一样跑向相反的方向。接着,是一阵剧烈的爆炸。

队长感到一阵天旋地转,恶心得想要呕吐,只是胃里没有东西能吐出来。

在密闭空间里冲击波的威力会成倍提升,造成更大的伤害。所幸一只行军蚁的爆炸能量有限,如果是两只甚至更多,她们的内脏现在一定都被震碎了。

“陆久……”队长挣扎着爬了起来,踉踉跄跄地朝着爆炸的方向奔去,但爆炸造成的塌方堵死了他们的来路。

“陆久……!!”队长用力喊道,但却觉得嗓子里仿佛塞了棉花,怎么也喊不出。于是她用力拍打着地面,希望能够传达出信号。

“我听到了。”队长听到一个声音从塌方的另一边传来。

“你怎么样?你受伤了吗?”队长意识到,自己因为声带过于紧张而失音了。但她无论如何也放松不下来,只能发出了一阵呜咽般的声音。

“我没事,别急。”对面的声音说,“我被震得有点晕,但没什么大事,休息一会儿就好。”

“你别动,我这就想办法打通通道!”队长感觉自己的嗓子好了一些,终于能够顺畅地说话了。

“没时间了,来接人的飞机只会等5分钟。你先走,我自己想办法离开!”

“绝不!!”队长叫了出来,这次她是真的哭了,“如果你不走,我也会不走!如果你死在这里的话,我也要和你一起死在这里!”

“给我住口!说什么任性的胡话!!”

对面传来一阵愤怒的吼声,然后是片刻的沉默。

“薇,我历尽千辛万苦来到这里,不是为了死在一起这种蠢事。我们要活下去,都要都活下去。”男人用嘶哑的声音说道,“我们不是说好了要在一起的吗。远处还有我们未曾看过的风景、还有我们未曾触碰到的幸福,我绝不甘心死在这种地方。这不是我们最后一次见面,绝对不是。”

“可是,我要如何把你丢下自己离开?我做不到。我不能一个人……”

“你做得到,必须做到,无论如何也要做到!”男人说道,“我知道你在想什么,索性一起留在这里吧——这种想法的诱惑对我来说也难以法抗拒的,你不知道我有多思念你、不知道我为了见你这一面做了些什么……但我们不能这么做。我们经历了许多的磨难,这不是第一次,也许也不是最后一次,而我们必须要超越它。只要想到你还活着,我也会设法活下去,无论有多少艰难险阻。但如果你放弃的话,我也就没有活下去的力量了。你必须离开,必须活下去,我也会想办法离开这里,这样我们才能有机会再次相聚,你明白吗?”

“我……”

“明白了吗!!”

“我……我明白了。”

“你向我保证,无论如何都要安全地离开。”

“我保证。”

“我也向你保证,绝对不会死在这里,好吗?”

“好的。”

“那就走吧。快!”

“陆久。”

“嗯?”

“你一定……一定要活着。”

“我以男人的名义保证。快走!”

“……陆久。”

“还有什么事?!”

“我爱你。”

“……该死的,再说下去我就会忍不住不让你走了!我也爱你!走吧,我也要去寻找其他的出口了!马上走!”

队长咬了咬牙,站了起来。她的两个队友已经在她身后等着她了。

“我们走。”队长低声说道。支援射手张了张嘴想要说些什么,却又不知道该说什么,只好说道:

“……好。”

于是她们一起,用自己最快的速度,朝着风吹来的方向奔去。

五分钟后。

当队长走出排水口的时候,她几乎睁不开眼睛——一半是因为未能马上适应外面的光线、另一半是因为直升机的旋翼吹起的风太大了。在稍稍适应了光线后,队长吃惊地发现,天空竟然开始下起了大雪。

三个人登上等待她们的直升机,看到机舱里只有两个人——一个驾驶飞机的飞行员,似乎是个战术人形;另一个则是坐在机舱里的女人,深色长发随意地在脑后编了个结、脸上有几道疤痕、目光凌厉,还装着一只机械手臂。

“陆久呢?”那女人开口问道。

“下水道发生了塌方,陆久被困在远离出口的一端,去寻找其他出路了。”队长淡淡地说道。

“我们是冒着极大的风险来这里的,我们可没空等他从其他水管里冒出来。”女人说。

“我知道。不必等他了,我们走吧。”

女人扬了扬眉毛,似乎对这个回答有些吃惊。

“你就是Vector吧?”女人说。

“是的。”

“我听说你和陆久……交情不浅吧。你就这样把他丢在这里?”

“如果把你们全都杀了、然后夺取飞机能让我救出陆久,我会毫不犹豫地这么做。但是,不行。”队长冷冷地说道,“我必须安全地离开,并活下去。所以你们最好在我改变主意之前赶紧升空。”

女人更吃惊了。她仔细端详了队长片刻,然后大笑了起来。

“哈哈哈哈,难道说,是陆久让你这么做的?”

“没错。”

“难怪陆久会看上你,就连这种话都言听计从,真是个好姑娘。”女人说着对飞行员打了个响指,“9,起飞了。”

“指挥官怎么办?”飞行员说。

“嘿,管他呢?你没听到这姑娘的话吗,飞得慢了,我们就要小命不保啦。快爬升!”

“嘻嘻,遵命~”

飞行员轻声一笑,拉下升降杆,直升机迅速离开了地面。机械臂女人坐在副驾驶位置上,拿出了一部电话。

“喂,是我。”她对着电话说道,“战术少女们已收回,但指挥官下落不明。……他自己倒霉,被困在下水道里了,关我什么事?……不行,一分钟都不能等,我来这里就已经引起了军方的注意,别忘了,他们的驻地离这里可不远。快给我准备去东亚的飞机。……不行?那就先降落在西部边境。……我知道。……好吧,我的人还在这里,我可以派她们去。但是下面的事情我可确定不了,就算找到人,可能也没那么容易撤离。……嗯,从长计议吧,那只能如此了。这世道,能活着就不错了,耐心不是什么值钱的玩意。……明白。那就这样。”

挂断了电话,女人扭头看了看身后的三个人形,然后轻声叹了口气。

“你们的运气真好,有人肯为你们出生入死地搭桥营救,但多数人形就没这么幸运了。”女人说道,“这次战斗,格里芬损失了70%的战术人形,并且绝大多数云图都一同损毁了。这就是命。正是因为人类的生命太过脆弱,所以才制造出了人形,来替代他们去做那些危险的事情,但这种历史,也即将走向终结了。未来可能是个人类不得不和人形,在一定意义上平等共存的时代,而你们则是掀开这一时代序章的人。是的,我想从此我们可以用‘人’来称呼你们了,虽然你们大概还不知道,‘人’这个身份,意味着许多烦恼……不过,那些事到时候再说吧。反正无论是哪个时代,对咱们这些家伙来说,都不会太平。”

“我已经派人去搜救陆久了。他目前还活着。”过了一阵,她再次转向了Vector说道,“别想太多,至少你们都还活着……活着,比什么都好。”



 \section*{}

吹下了史无前例的牛皮呢,躺在阴暗潮湿的下水道里,陆久心想。

听着Vector远去的脚步,他的心里放下了一些,但还没有完全放松。“设法离开这里”,可没他说的那么简单。

陆久活动了一下自己的左臂,感到一阵剧痛。他的左臂被压在了一块混凝土板下面而且抽不出来,掌心传来的痛感告诉他,左手的手掌大概被钢筋刺穿了。

此时如果有人帮忙,能把他的手臂刨出来的几率有多大?陆久虽然估算不出来,但他知道肯定比现在要大。可他没有留下Vector,因为他更知道救援的飞机不仅会引来铁血,还会引起军方的注意。错过这趟航班,在这地方幸存下来的可能性微乎其微。

如果被困在这一边的是Vector,陆久绝对不会自己离去,原因说来可笑,陆久的想法和Vector一模一样——要是不能一起离开,那还不如一起死在这里。但心思简单的Vector没有想那么多,陆久三言两语就把她骗走了。

“我爱你”吗,陆久心想。

多么多么温柔的诅咒、多么甜蜜的劝诱,无论听多少次都听不够。但温柔和甜蜜是解决不了眼下的绝境的。未知的敌人正在迫近,他必须要做出一个重要的决定——

吃子弹,还是断骨头。

来吧,陆久心想,毒蛇啮指,壮士断腕。骨头断了还能长上,让心爱的女人等不到的男人,可不是好男人。

于是陆久解下捆扎带勒紧左臂的臂弯,然后抽出了腰间的战术匕首。

……

“陆久……。”

有什么人在呼唤自己的名字。说起来,自己到底是不是叫陆久?

陆久感觉迷迷糊糊的。

“陆久……?”

是谁?是谁在叫那个名字?

“陆久……!”

“薇?”陆久张了张嘴,答应了一句,但很快发现不对。

说话的不是Vector,而是自己的通讯器。陆久按了一下对讲键。

“喂……”

“你终于回话了,急死我了!你在哪,还好吗?”

陆久感觉清醒了一些,他能听出和他说话的人是谁了。

他看了一下自己的左臂,手臂上的肌肉被切开、骨头也锯切断,但缝了一半的伤口发干,血迹已经凝固。他给自己做了截肢手术,但在手术临近尾声的时候,他昏了过去。

幸亏扎好了臂弯,不然,现在已经失血过多而死了,陆久悻悻地想着。

自己又活下来了。

又一次,从死神手里捡回了小命。

“我还好。”陆久虚弱地说道,“我受了点伤,但是还活着。”

“我已经定位到你,安洁也派人去找你了。你最好是呆在原地等着,这样找起来能省点事。”

说话的人是帕斯卡。帕斯卡莉娅,人形研究领域首席的科学家,一个智商情商都超高、上得了厅堂下得了厨房、风情万种,并且和陆久有过一段故事的女人。

也是给陆久带来了他用一辈子也解决不完的麻烦的女人。

“我不能呆在这里。”陆久说,“这地方离撤离的位置很近,一定会招来铁血,说不定还有军方。我必须转移,我已经没有力量和它们周旋了。”

“那你还能动吗?”

“还行,唔……”

陆久挣扎着想站起来,却摔了一跤,通讯器里发出一阵杂音。他感觉自己血压很低,光是直起身就两眼发黑。

“陆久?!”帕斯卡急切地说道。

“没事,有点头晕。”

“你是不是失血过多了?”

“也不太多。”

“你必须保持清醒!千万不能失去意识!”

帕斯卡说得很对,他绝对不能失去意识。如此寒冷的地方,如果失去意识,大概就永远不会醒来了。事实上,刚才如果不是帕斯卡一直呼唤他,他很可能就要从此一睡不醒了。

但在极度疲劳困顿又失血过多的情况下,保持清醒没那么容易。

“帕斯卡……我这次,有可能真的搞不定了。”陆久说。

“你振作一点!”

“抱歉。我知道你为了救我费了不少心,但我大概坚持不到了。”

“哼,你这家伙。”帕斯卡忽然冷笑了一声,“就为了救一个民用人形,不仅把全世界都搅了个天翻地覆,就连自己的命都要搭上了,值得吗?”

“啊,很难说。至少我自己还挺满意的。”

“我和Vector比起来,到底哪一点不如她?”

“你哪都比她好。”

“你简直就是个无药可救的蠢货。”

“是的,我是蠢货。我也发现这一点了。”

“不过我还是得称赞你眼光不错。虽然你恐怕不知道,Vector和克鲁格的关系绝不止上下级那么简单。”

“……什么?”

“当然,这件事你肯定不知道。从血缘关系来说, Vector和克鲁格是有血缘关系的,至少基因绝对一脉相承。克鲁格给你看过他家人的照片吧。你难道不觉得,她和克鲁格的某位家人长得很像吗。”

“……什么?”

陆久仔细回忆着克鲁格曾经给他看过的家庭照,感觉自己的血压快速上升,头脑也清醒了许多。关于克鲁格和战术人形,UMP45也和他说过,但他从来没有想过那些事和Vector也有关系。现在听帕斯卡这么一说,他猛然意识到,照片里克鲁格女儿的面容,和Vector十分相似……几乎可以说是一模一样。

“难道说,Vector就是,克鲁格女儿的……?”陆久说。

“你终于明白了。那个项目由我一手操刀,Vector就是那次实验的最终产物。她不仅是第一个二代民用人形,而且是第一个真正意义上的战术人形。Vector是克鲁格女儿的复制品,实验结束后我把她留给了克鲁格,算是纪念品。”帕斯卡说,“克鲁格的女儿叫维多利亚,在15岁的时候得了血癌,夭折了。克鲁格就把她的基因和意识贡献出来用于研发制造‘民用人形’——替代人类从事危险工作的拟人生物体的实验。克鲁格是抱着背负世人之罪的决心献出女儿的基因的,据说这也是他那位虔诚的还俗修女夫人的意愿,是不是很像《圣经》里弥赛亚的故事啊?顺便一提,虽然从未谋面,但维多利亚是你的小粉丝呢,直到临终都想和你见一面。因为克鲁格经常给她讲老虎叔叔的故事,‘老虎叔叔’的原型,毫无疑问就是你吧?”

陆久听完呆住了。这件事情的冲击性,犹如一颗炸弹落在他的头顶上。

克鲁格。修女。维多利亚。帕斯卡。老虎叔叔。这些名字在陆久的脑海里不停翻腾,化成一个个咧着嘴的脸谱,放肆地嘲笑着他的迂腐和无知。

……还有Vector。这到底,是什么样的孽缘。

“你为什么要和我说这些?”陆久说。

“为了拯救你的血压。另外我认为你也该知道这些了。”帕斯卡说,“你现在回话的速度快多了,思维已经能跟上现实了吧?”

确实能跟上了,陆久甚至开始揣测帕斯卡的意图。但无论怎么想,帕斯卡说的都是些许多年前的陈年旧事。维多利亚要是还活着,现在该比自己的年龄还大几岁了,这些事情不可能是编造的。不知道这些,也只能证明自己的愚蠢。

“好吧,我确实感觉体力恢复了不少。我要向着其他出口移动了。”陆久站前身来说。

“我觉得你的体力走不了多远,不过要是换个地方能让你感觉安全一点,那就朝北走吧。为了防止你再次昏过去,我希望你能和我说说话。”

“你想说什么?”陆久采纳了帕斯卡的意见,向通往北边的下水道走去。

“说什么呢?要说关于你的事情,我知道得比你本人详细多了。你对我有没有什么想问的呢?”

“问什么都可以吗。”

“可以啊。不过要说私密的事情,你其实对我已经了解得不少了吧,嘻嘻。”帕斯卡笑了一声。

“我不会问你想的那些事情。”

“不问吗?说实话我还挺期待的。越是让人心跳加速的话题,越是对你的血压有好处啊。”

“不必了。那么我想问问……你为什么会想要做,民用人形这一方面的研究?”

“想知道这种事情啊。哼。”帕斯卡的热情似乎一瞬间消退了,“你这个人,倒很会戳别人的痛处呢。”

“要是不能说的事情就算了。”

“倒没什么不能说的,反正也是陈年往事了。只不过,这有损我多年来苦心经营的科学婊子的形象——”

“不要用那种词形容自己。”陆久不快地打断了帕斯卡。

“怎么了,前女友自轻自贱,让你感到难受了吗?还真是个好懂的人,呵呵。”帕斯卡笑了,“你呀,就是这样割扯不清、对谁都温柔,才总是命犯桃花。”

“……说你的事。”

“我呢,从小就患有一种病,非常难治的病。我妈妈带我去了许多医院,都治不好,也诊断不出原因,一直到我们去了一家全国有名的 医院。在那家医院,我们遇到了一个非常冷漠的专家。他说,不用确诊了,是基因里的病,治不好的,最多能活到十三四岁。他说按照政策,这种病是可以再要一个孩子的,劝我妈妈再生一个。我老妈吓坏了,那真感觉是天都塌了啊。但她还是不死心,又去了另一家医院。这家医院坐诊的是个老医生,我妈连号都没拿,去了直接就给他跪下了。老医生吓了一跳,问明原因后,让我妈别着急,他说只要治就有治好的希望。从那天起,那位老医生成了我的专属医师,因为我每天早上七点半要上学,老医生就七点就去医院专门给我看病,一直看了四年,直到我小学毕业,我不仅没死,病还渐渐好起来了。后来我考上了全国第一流的初中,高兴坏了。我拿着入学通知书去找爷爷……就是那位老医生,那时候我和他关系已经融洽,都是喊爷爷了。可是你猜怎么着?我被医院告知,爷爷心梗去世了。他走得很急,家里人也没能见到他最后一面。我这个难过啊。明明实现了和爷爷的约定,成了一个好孩子,但却没能让爷爷看到。我该怎么做呢?于是我就想,我也做个医生吧。像爷爷一样救人,也算宽慰他的在天之灵了。所以后来就学了医学了。”

“这样吗……可是这和民用人形有什么关系?”这个故事让陆久有些触动,但他很快意识到,帕斯卡没有回答自己的问题。

“呵,你说呢?”帕斯卡笑了一声,反问道。

陆久思考了一下,隐约猜出了原因。为救人而学医的人,很少有半途而废的,能让他们放弃这一行的原因通常只有一个,那就是发现学医救不了人。

“后来觉得学医救不了人?”

“没错,但只是一部分原因。”帕斯卡说,“医生确实能治愈一些人身体的痛苦,但人的痛苦往往不是来自肉体而是来自精神。如果能有什么人来给他们慰藉、代替他们去承受那些不能承受的事情,那不是能救更多人吗?一开始构思‘人形’这一概念的时,候我是这么想的,这种想法也和克鲁格一拍即合。但是后来就不一样了,我在研究过程中,又有了新的发现。”

“什么新的发现?”

“你应该早就意识到了。人形少女们无私又纯洁,比人类可爱多了不是吗?呵呵。”

“你是想……?”

陆久发现,帕斯卡永远能让自己感到吃惊。

“人类这种肮脏的物种,就是世界的毒瘤——虽然这想法不能公开说,但我不介意让你知道。你说要是人形取代了人类,这个世界会不会干净很多?”

“……我不知道。人类虽然不是什么善于建设的生物,但至少目前还是站在食物链顶端的物种。一旦消失会怎样,我不认为这个世界会更好。”

“不管更好还是更坏,都一点也关不到你的事,哈哈。”帕斯卡狎昵地笑了一声,然后用耳语般的声音说道,“趁着没有别人,我们来讨论一个只有我们两个知道小秘密吧。你说,我们两个寻爱求欢那么多次,一点防护措施都没,为什么我肚子里一点动静都没有呢?”

陆久沉默了。

这个问题他不是没有想过,但他一直都以为是帕斯卡做了防护。但他今天才知道并非如此。

“没有防护吗。”陆久说,“没想到,你竟然是一个在追求刺激方面也如此极端的人。”

“你不要以为我和谁都是这样,其他男人可是绝对不行的哦。只不过你呢……我也是个女人,心爱男人的优秀基因,谁不想得到呢。可惜,我到最后得到的,都只有失望。”

“这么说,是我已经失去生育能力了吗。”

“是的。我化验了你的精液,里面几乎已经不存在活跃的因子了。四十年的时间囹圄,并非没有对你造成一点伤害,你虽然身体没有什么大碍,但已经没有延续自己基因的能力了。”

“也就是说,我本来就是行将灭绝之人了?”

“你自己知道就好。不过对你来说,这根本无所谓吧?”

“呵,是啊。朝生暮死之人,何必在乎自己是否有后?”

“我说啊,陆久?”

“嗯?”

“别去管那个人形了,和我在一起吧。”

“谢谢,血压过高也是对身体有害的。”

“我是认真的。给她找个平静安宁的地方度过余生,这样对她也是好事,不必再奔波受苦。适当地运用生物技术,只需要一段基因,我也能为你繁育子嗣。”

“你说得越来越离谱了。”

“你如果讨厌我在做的这些事情,我也可以不做了。我会尽快结束,然后就此收手。处心积虑地谋划这么多年,现在想想,那些东西也没那么重要。”

“那些和我完全没有关系。”

“你要是实在割舍不下,让Vector跟着你也可以。反正她只是个人形……我也不是不能接受。”

“别胡说了好吗?”

“我是在向你求偶啊,你这该死的蠢货!”帕斯卡声音颤抖地说道,“我这么优秀的女人,话都说到这份上了,我什么都不要了、什么都能接受,还不行吗?我虽然上过不少男人的床,但都是在遇到你之前的事情,认识你以后我已经痛改前非了啊!就算退一万步,至少我的子宫里还是干净的,主动为你这要绝种的混蛋生传宗接代你还不肯,你是不是瞎了眼了?!”

“优秀”这个词用在自己身上,对于他人来说叫自夸,对于帕斯卡来说只能说是自谦。帕斯卡这样的女人,应该是用“杰出”、“卓越”这一类的词来形容才恰当。但就是这样一个超越了优秀的女人,陆久依然无法接受。

“帕斯卡,不要再菲薄自己了。我不接受,不是因为你不够好,我想你不会不明白。”陆久低声说。

“呵呵,我明白。我怎能不明白?”帕斯卡轻笑了一声,“男人这种狼心狗肺的东西,喜欢的人上房揭瓦也行、不喜欢的人掏心掏肺也没用,我当然明白。我还以为自己和那个人形之间的区别,只在于她肯倒贴、而我不肯;现在我终于明白了,就算是倒贴,我也贴不上。”

陆久停下了脚步。有那么一瞬间,他仿佛忽然忘记了自己正去向何方,也忘记了自己正身在何处。他感觉脸上冰凉,伸手一摸,发现脸颊竟已被泪水浸湿,而泪水是何时流下的,他却毫无知觉。

是谁爱上谁、是谁离开谁,又是谁在想着谁,他们早已说不清。但陆久知道,自己永远不会忘记这个女人的好、永远不会忘记她曾给过自己的温柔和安慰。

“我一直欠你一句坦白,帕斯卡,我不会说什么‘就算不说你也懂’,只是因为‘我爱你’这句话用过去式说出来时,总会有人受伤。”陆久说,“你如果想用说伤害自己的话来让我感到刺痛,是很容易的。但我请你不要那么做,因为过去的事情终究无法改变,我们能够维持一种普通的关系,已经很奢侈了。”

“是啊。但说出去之后,我心里也不那么难受了,呵。”帕斯卡再次笑了一声,“我只是不挣个鱼死网破,就心有不甘。罢了,我们之间的事情就到此为止吧,安洁的人差不多该到了。别担心,我挺好……你该知道,我只是为了维持住你的血压,才故意这么说的吧?”

“我知道你说这些是为了让我保持清醒。”陆久说,“你总能看透我的心思,真让我羞愧。”

“没什么,毕竟比我更优秀的人太少了。你很有自知之明,这点也很好,要继续保持哦。”

“谨遵嘱咐。不过安洁的人在哪?”

“继续往前走然后左转,再走30米有个向上的井盖,接你的人就在上面。去吧。”

“知道了。”

帕斯卡没再说话,陆久明白,他们之间已经没有什么可再说的——能够继续保持联系,就是最好的结局了。

几分钟后,陆久来到了那个下水道的出口。他用一只手艰难地向上爬去,推开井盖,有一只手抓住了他的手腕,把他拉了上去。

陆久闭上眼睛稍微缓了缓,好让自己适应外面的光线。然后,他睁开眼,看到了一个熟悉的身影。

“你好,又见面了。比想象中更快呢。”陆久面前的少女说道。银色长发、蓝花发卡,以及背后的HK416突击步枪。是她,不会错。

“我听说你受伤了,不过看起来你的状态还不错,不知道是哪里受伤——”

少女的话没说完,忽然微微睁大了杏眼,因为她终于看到了陆久袖子用捆扎带绑着起的左臂,那手臂子明显比右边短了一截。

“你的左手……?!”

“卡在塌方的地方拿不出来,只好丢在下水道里了。”陆久强装镇定地微微一笑,却感到眼前的光线越来越昏暗。体力透支的效应,在意志放松之后,正在开始全面显现出来。

终于,陆久膝盖一软,倒了下去。

“安洁,我找到陆久了,他受了很重的伤!他的左臂断了……不是骨折,是截肢!在肘下两寸的位置截断……伤口进行了缝合处理,但肯定流了很多血,而且有冻伤痕迹。准备最近的医疗站,要保暖被、血浆和高热量食物……还有葡萄糖,他可能无法进食。还有强心剂……什么,没有飞机?!我驾驶的是摩托车!该死的,知道了……赶快,用最快的速度!他已经开始休克,很可能挺不到那里……”

416在对着通讯器大声呼喊,但陆久已经听不到。他印象中最后的画面,是银发少女急切的面孔,和铺天盖地的大雪。

\section*{尾声}

空军基地里秩序井然,所有飞机都停在机库,但皮尔斯准将却失踪了——和他的副官一起。没人知道他是怎么离开的,也没有人知道他去了哪里,他在离开前很仔细地销毁了所有关于行动的资料,以及关于他的任何一丝一缕的痕迹。这让他的母公司和格里芬公司之间的劳务费结算变成了难题,因为虽然账单还在,但提供的劳务明细已经无处可寻。当然,鉴于格里芬公司目前的财物状况,即便是有完整的“供销明细”,恐怕也难以在短时间内支付费用。而且和格里芬的实际控制人克鲁格的处境相比,这笔劳务费也显得微不足道。

对于儿子的出走,老皮尔斯将军并没有感到特别意外,反而感到有些宽慰,因为永远逆来顺受的孩子,是变不成大人的——人总是会经过青春期的叛逆才会慢慢长大,虽然儿子的叛逆期来得稍微晚了一些,但这终归是个好的开始。老皮尔斯将军眼下最关注的还是克鲁格,劳务费的索要并非要事,因为他和克鲁格之间的合作远不止于此。他只是暂时转移了自己的焦点,并不意味着他会任由自己的儿子为所欲为,因为这世界上还不存在敢于忤逆他的下级。

郝丽安带着佩瑞特等北部军团的剩余指挥官和人形离开了,在损失了整个南部军团和大量指战人员之后,这残余的部队已经是格里芬最后的火种。她必须度过不知道将有多长的韬光养晦的时光,因为在可以预见的未来,以战术人形为主力的私人军事服务供应公司,将会面临产业的寒冬。但她相信自己一定会让这个公司东山再起,然后交还到它原本的主人手里,即便届时已经不再使用“格里芬”这个名字。

在格里芬全部撤离、404小队也离开后,NT77依然在格里芬的临时撤离点独自等待着陆久,但她最终等来的却是帕斯卡。如果不是是UMP9对安洁随口提到了她,她几乎已经被人遗忘了。安洁把NT77的情况告诉了帕斯卡,她本打算将NT77作为战斗力收留,而帕斯卡却提议将NT77带到16LAB。最终,她们决定让NT77自己来做出选择——加入安洁的队伍,还是充当帕斯卡技术方面的助手,或者是遵从陆久的意愿从此“走向自由”。经过考虑,NT77决定跟随帕斯卡,因为所谓的“自由”对她来说没有什么价值,她认为在帕斯卡这里能够更早地得到陆久的消息。对于这一决定安洁没有什么异议,因为她们三个人都明白,这个决定不会是长期的。

帕斯卡将研究工作交给了研究所的其他人,向她的赞助人请假去了中亚,理由是这一阶段的工作过于劳累、她需要休息和放松一段时间。她也许的确需要稍微放松一下,但每个人都知道,中亚可不是什么度假胜地,特别是在动荡暗暗涌动、治安形势急转直下的现在。在“助手”NT77的陪伴下,帕斯卡努力调整着自己的心情,她决定要走出上一次失败恋情的阴霾。她在中亚落地后做的第一件事,是去拜访一位久违的朋友……顺便重新找回一点内心曾经的美好,虽然她知道“美好”这个词对自己来说,可能太过遥远。

Vector被安置在了中亚北部的某个小镇,这里有一个收容所,是一个专门收留和保护遭到迫害、遗弃,或者无家可归的人形的权益组织。Vector知道这只是个幌子,这个组织和将她安置在这里的人,有着千丝万缕不能明言的联系。在见识过一些人类难以理解的复杂行为后,她已经不是当初那个轻易就会听信别人的姑娘了。不过她对此并不关切,因为那对她来说无关紧要,她只是在这个地方短暂地停留。

在撤离战场一周后,Vector收到了一段语音留言,来自一个称“HK416”的人形。她告诉Vector陆久在营救中受了伤,但性命无恙,伤情已经稳定。他们此刻正在为了安顿而四处奔走,所处的地区经常没有能够保持稳定通讯的条件,让Vector不必为陆久担心,如果事情顺利,或许一个月后他们就能见面。但一个月过去后,Vector没有见到陆久,也再没有得到收到进一步的通讯——从那之后,他们就完全失去了联系。

冬季过去的一个晴天,冰雪融化、树枝上冒出新绿,四野里正散发着春天的气息。Vector悄无声息地离开了收容所,正如陆久离开N17战区一般,没人知道她去了哪。但她的离去没有引起过多的关注,毕竟这个世界上一个人形的存在和消失,不是什么值得在意的事情。

世界的规则已悄然变换,但有些事总是大同小异,例如为了名利的勾斗和为了苟活的奔波。在日复一日不断重复的生活中,人们总能找到让他们适应的生存方式,人类就是这样坚韧的生物。一旦形成了习惯,痛苦也不会再觉得痛苦,而痛苦中一瞬间的不痛苦,则转变成为了快乐;为了生存,他们可以舍弃那些相对不重要的东西,接受本以为无法承受的现实,说服自己平静,相信这是理所当然。于是,曾经的誓约相守的人们终于各自天涯,而那些从情感冲动中说出的话,也在不经意间被遗忘在了身后的远方。这世上万古不变的,唯有太阳每日升起和落下……

以及,时光的流逝。

\section*{后记}

这故事,到这里就结束了……

虽然结束了,但又没完全结束。我们还有点事情没说完。

这个故事让我耿耿于怀的,并不是它有一百万字、并且是免费的,而是很多路过的朋友连个赞都舍不得点。

所以我决定把故事的结局发给真正关注故事、并尊重作者的人。对结局感兴趣的读者请留下邮箱,在这里留言也可以,私信发我也可以。

留言多的话,再考虑公开发布,我觉得这个要求并不过分。那就这样吧。

\lineseparator

结局的内容已经发在最后一章,希望看过的朋友能够留言,就当是对这个故事最后的纪念吧。

感谢大家的支持,向你们致意。