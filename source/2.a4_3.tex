\chapterul{外传:丛林之虎(三)}


\section*{}

这真是个令人悲伤的消息,悲伤到就连克鲁格听到之后,都凝重地皱起了眉头。

当然,科宁斯很清楚,克鲁格皱眉头的原因是这下他不能再说什么“你的儿子可能已经到了安全的地方”这样自欺欺人的话了,而且去叛军的军营里找人,可没有去难民营找人这么容易了。

“我们没有时间了。”克鲁格对着一脸恳求神色的向导说,“政府军已经集结完毕,我们必须在24小时内展开行动——今晚返回营地,明天太阳落山就行动,一分钟多余的时间都没有。我会将此事汇报给政府军那边,让他们代为留意的。”

这大概是克鲁格能够表现出的最大诚意了。但这样的保证到底能带来多大希望,在场的所有人都很明白——向导的儿子很可能已经被训练成了童军,政府军见到他们的时候做的第一件事,肯定不是先问他是不是一个叫尼库鲁的男人的孩子。

只有他活着被政府军俘虏的时候,才有可能再次见到他的家人。而政府军对这种事情有多大的兴趣,是不言而喻的。

“求您了,克鲁格先生……”几乎绝望的向导一边哭一边说道。

“我们已经尽力了,但这不是我们能够左右的事情。”克鲁格努力耐下心来说道,“我会向政府军提出请求的,但是我们的行动必须得到保证——已经没有时间可浪费了。”

“……那好吧。”过了一会儿,认清形势的向导终于说道。

在太阳沉入地平线的时候,几个人踏上了返程的路。他们依然是和来的时候同样的队形,但这次行进的路上气氛十分压抑,显然是受了向导的影响。为了加快移动速度,克鲁格下令先走两个小时再休息,因此他们第一次休息的时候已经是漆黑的夜里了。

几个人在靠近路边的树林里隐蔽了起来,围坐在一棵树下沉默地休息着。科宁斯很想抽根烟,但是迫于眼前的形势这显然是不可能的。他看着阿虎和克鲁格,这两个人也是抽烟抽得很凶的人,但此刻的他们没有一点耐不住烟瘾的样子,不由得让科宁斯由衷地感到佩服。

“咳……”科宁斯清了清嗓子想要说点什么缓解一下气氛。但就在时候,向导忽然说话了。

“嘘!”他说,“我听见声音了……又是汽车!”

“比昨天要早多了,啊?”克鲁格沉声说,“藏起来、保持安静,让他们过去。他们不会发现我们的。”

几个人沉默地躲藏在丛林之中,汽车的声音渐渐接近。片刻后,一辆尼桑牌的皮卡从距离他们只有十几米的小路上,缓慢地驶了过去。借着车灯光,科宁斯能够清楚地看到车里的人:驾车的是个青年叛军,而后面的车斗里坐着的几乎是清一色的少年。那些士兵最大的也就十几岁、身上穿着破烂的衣服,手里却每个人都拿着一把AK自动步枪。

这是真正的巡逻队,那些人里边没有俘虏,都是武装分子。虽然这些叛军士兵科宁斯已经不算陌生了,但如此近距离地观察还是第一次。

他们只是些孩子啊,看着那一张张稚气未脱的脸,科宁斯想道。缺乏组织和训练、只有简单的武器,而且完全是被胁迫着而战斗……这些童军的战斗力十分低下。如果不是人数占绝对优势,在克鲁格那些经验丰富的老兵面前,他们是没有对抗的能力的。因此就科宁斯所见,在这些叛军和克鲁格的战斗中,他们从来没有占到过便宜。克鲁格总能把他们轻松击退,并且最多只有些人员受轻伤;而那些叛军则通常会留下几扭曲的具尸体仓皇撤离。

人生的公平之处到底何在呢,科宁斯心想。这些孩子不仅被剥夺了童年,就连自己的生命都难以保全。在和平的国家里,他们应该还在上小学吧。可是在这里他们却不得不去杀人,而且甚至根本不知道这一切是为了什么。

有的人生来就在富庶平和的环境之中、有些人却不得不从小就在战火中成长。当这些孩子在动乱中夭折的时候,没有人会同情他们、甚至没有人会知道。他们的生命卑贱如蝼蚁草芥,只因为他们生于贫困和蒙昧的土地上。

科宁斯心中渐渐充满了愤懑。人们应该看一看真正的战争是何其残酷,他心想,特别是那些鼓动战争、从中得利的人。

“墨菲……?”

正当科宁斯沉浸在情绪之中的时候,他忽然听到身边有个人轻声说道。

科宁斯转过头,看见向导正直直地看着擦身而过的叛军,喃喃地说着一个名字。他也许是在那群孩子之间看到了熟悉的身影。

克鲁格见势不妙,连忙伸手去掩向导的嘴,但已经晚了。

“墨菲!?”看到那群正在走远的孩子,向导失声叫了出来。

“什么人!”已经开过去的汽车上,有个孩子喊了一声,“我听到那边有人说话!”

汽车立即停了下来,车上的人纷纷跳下了汽车,科宁斯听到汽车里的人在高声呼喊着什么。虽然那是些他听不懂的当地语言,但不用说他也知道那个人在下令搜索。

“快跑!往树林深处去!”克鲁格对着向导和医官说道,“莉莉安,你和向导一起,我去引开敌人!”

几个人同时从地上跳了起来。向导朝着树林里跑去,医官紧随其后,克鲁格和阿虎则朝着另外的方向跑去。

而科宁斯则跟着克鲁格一起跑了起来。

“你他妈的过来干嘛!?”看见科宁斯跟了过来,克鲁格喘着粗气边跑边骂了一句。

“我……”科宁斯不知该说什么。他只是下意识地跟着克鲁格跑起来的,因为他离克鲁格比较近。

“算了,把身子低下!”克鲁格吼道,接着拉了一下手里的枪栓。

砰、砰砰!

克鲁格朝天上打了几枪,枪声吸引了那几个进入树林搜索的叛军,科宁斯听到一阵杂乱的脚步声在快速接近。

“隐蔽!阿虎,阻击敌人!”克鲁格低吼一声,接着拉住科宁斯滚向了一边。阿虎闻声立即扑倒在地,然后转身将枪口朝向了后方。

啪啪、啪啪啪啪!

随着阿虎枪口微弱的火星闪烁,科宁斯听到一阵奇异的声响。那声音不像是普通枪支开火发出的类似鞭炮爆裂的声音,而是和皮鞭撕裂空气的声音一样——他不知道那是装配了消音器的突击步枪发射常规弹药时发出的声音。

不远处立即传来一阵哀嚎声,接着枪声大作,雨点一般的子弹飞了过来。

“趴在地上,别抬头!”克鲁格对着科宁斯说道,然后也开始开枪还击。科宁斯一手抱着头趴在地上一动也不敢动,一手紧紧握着背后藏着的手枪,不知该如何是好。

阿虎和克鲁格的反击很有效,因为敌人的武器没有消音设备,发出的声音和火焰将他们的位置暴露无遗,而阿虎和克鲁格的枪口则几乎看不到明显的火焰。几轮点射之后,敌人的枪声立即变得稀疏了,而且听起来只是在毫无目标地胡乱开火。

“停火。”交火三四分钟后,克鲁格小声说道,“嘘……听,他们应该是撤退了。”

“他们不会是埋伏起来了吧?”科宁斯低声说。

“他们没有这样的胆子。他们只知道我们有武器,却不知道我们有多少人,不敢冒险追击的。”

三个人在地上又趴了一阵,听到脚步声确实是远去了,黑暗的树林里再次恢复了宁静。

“走吧,此地不宜久留。我们得想办法和向导他们汇合。”克鲁格说着慢慢趴了起来,阿虎和科宁斯也跟着站起了身。

“我们去哪找他们?”科宁斯小声说。

克鲁格拿出对讲机,将耳麦塞进了耳朵里。然后,他轻轻拧开了通话旋钮。

“莉莉安,我是克鲁格。你们在哪?”克鲁格压低了声音说,但电台里传来的只有微弱的沙沙声。

“莉莉安?”克鲁格再次呼叫,但依然没有回迅。

“讯道里没有她的信号。大概是忙于躲藏而忘记开机了。”一直在旁沉默的阿虎低声说。

“也有可能是在躲避搜索。”克鲁格说,“不过算了,她和向导在一起应该不会迷路。我们直接回去吧。”

哒哒哒……

克鲁格话音刚落,忽然传来一阵突兀的枪声,接着又是几声零星的枪响。听声音,开枪的位置很远,至少在几百米之外。

“妈的!”克鲁格低声骂了一句,“一定是莉莉安他们!快走!”

几个人也顾不上隐蔽,奋力向枪声传来的方向奔去。几分钟后,三个人停了下来。

“差不多该是这附近……”

克鲁格话还没说完,忽然身边响起一阵噼里啪啦的声音,那是树枝被子弹打断的声音。

克鲁格立即反射性的举枪想要还击,但却被阿虎按住了枪。

“等等,是消音武器。”阿虎说。

克鲁格明白了过来,放下了手里的枪——对面的火力没有任何声音,开枪的一定是医官的MP5。

“Jungles?”克鲁格低声呼喊着接头暗号,但没有听到回答。

“Jungles!”克鲁格又喊了一句。

“Tiger...”半晌,才传来一个微弱的声音,“是你吗……上士?”

说话的声音不大,但毫无疑问是莉莉安医官的,她就在不远处。听到友军的回答,科宁斯松了一口气。

“是我。我要过去了,那边怎样?”

“安全。”

克鲁格几个人小心地走了过去,看到医官和向导正蹲踞在一棵树下。

“我们听到了枪声,是在这里发生战斗了吗。”

“是的。有三四个叛军追了过来,我开枪打倒了两个,剩下的逃走了。”

“还好。我还以为他们的大队人马朝你那边去呢,果然还是撤了啊。我们赶紧走吧。”克鲁格说。

“好……不过稍等。”医官用有些无力的声音说着,听起来状态好像不是很好,“我刚才跑动的时候绊了一下,把脚扭伤了,我要……休息一会儿再走。”

“不行,”克鲁格说,“留在这里的话敌人随时可能回来。不能行动的话,我来背你好了。”

说着,克鲁格将自己的枪交给了阿虎,然后伸手去扶医官。但当他把医官的胳膊架在身上的时候,他的动作忽然停了下来。

“……你受伤了?”克鲁格低声说道。

“没有,只是……”

“别骗我了,我闻到了血的气味!让我看看!”

克鲁格说着不顾暴露的危险打开了战术手电。借着灯光科宁斯看了一眼医官,只见她从腹部到腰部以下的迷彩服已经变成了黑紫色,显然是被血浸透了,而伤口大概是在腹部。

“他妈的!”克鲁格骂了一声,“坚持一下,我们马上回营地!”

“不,你们走吧。我……已经不行了。”莉莉安摇了摇头,虚弱地说道。

“闭嘴!”克鲁格低声喝道,“现在马上急行军。我们四个人轮流背着你,你一定会没事的!”

“没用的,伯鲁。我是医官,我知道怎么回事。”莉莉安说,“我的脾脏中弹了,出血止不住的,而且营地里根本没有血袋……你们走吧,不用管我……”

“胡说!”克鲁格咬牙切齿地说着,竭力忍住才没有大喊出来,“你想让我把你丢在这里?你想让我抛弃自己的战友吗!?”

“克鲁格,去难民营。”科宁斯伸手按在克鲁格的肩膀上说,“那里有完备的医疗设施,而且离我们比较近。我和你一起去。”

克鲁格沉默了一阵,没有说话。不知为何,科宁斯感到手下面那个粗壮的肩膀,似乎正在微微发抖。

“克鲁格?”见克鲁格不出声,科宁斯疑惑地说道。

“阿虎,你带着科宁斯和向导回营地,后边的行动按照原计划进行。我带莉莉安去难民营。”

过了好一阵,克鲁格才低声说道。

阿虎无言地看了克鲁格一阵,然后把克鲁格的枪放在地上说:“知道了。走吧。”

“你一个人带着她怎么行……”听到克鲁格的话,科宁斯有些惊讶。

“快走。”阿虎不等科宁斯说完就打断了他,并用力推搡了一把沮丧地蹲在地上的向导。然后,他头也不回地大步朝着北方走去。

“喂、喂?”见阿虎走远,科宁斯赶紧跟了上来,向导也站起来跟了过来。

“该有个人跟克鲁格一起才对!”科宁斯一边努力地跟上正大步前进的阿虎一边说,“这里离难民营至少也有好几公里,他一个人背着医官……”

“别说了,”阿虎说道,“我们还有任务要执行。已经没有时间可浪费了。”

“那他也不可能……”科宁斯说着脚步慢了下来,“不,他该不会是?!”

“住口,我没时间接受采访!”阿虎带着怒意喝道。

“克鲁格根本不会去难民营的是吧?因为这样一来你们的任务就要失败了是吧?!”科宁斯愤怒地低声吼道,“你们这些人都是他妈了个逼的铁石心肠吗,就连自己的战友都见死不救?!”

听到科宁斯的话,阿虎猛然停了下来,科宁斯险些撞到他的身上。

阿虎转过身,看了科宁斯一阵、又看了向导一阵。树林里很黑,科宁斯看不清阿虎的表情,但他知道如果阿虎的眼睛能喷出怒火,那么他现在一定已经被烧成灰烬了。

“这是克鲁格的决定,他一定已经考虑过了。”科宁斯听到阿虎说道,“我们都有自己必须去做的事情,不能问理由、也不能说意愿……我们所要做的,只能是服从。”

让科宁斯倍感意外的是,阿虎的声音很轻,里面听不到一丝愤怒的情感。他的语气里,只有背负了太多无奈的疲惫。

科宁斯忽然想起刚才把手搭在克鲁格肩膀上的时候,克鲁格的身体在颤抖——他现在才明白,克鲁格那时是在哭泣。克鲁格知道如果立刻赶去难民营的话,医官是还有希望获救的,但那么做他们的任务将功亏一篑。所以他为了任务能够完成,而选择了舍弃自己的战友……并独自去背上做出这一选择的罪。

克鲁格的选择,阿虎大概从那时起就明白了吧。所以他才会催促快走,为的是让莉莉安能不受打扰地和克鲁格度过她最后的时间。

科宁斯从阿虎那里听说过,莉莉安医官和克鲁格是一起从地狱里归来的,因此莉莉安在某些方面对克鲁格有着特殊的依赖。他们之间的感情也许不止是战友、甚至不止是性命相托那么简单。如果以命相换能救莉莉安,克鲁格一定会毫不犹豫地去做,可是他却不能……当他决定放弃的时候,心里究竟是怎样的滋味呢,科宁斯甚至不敢去想象那种痛苦。

这是一群什么人啊,科宁斯心里五味杂陈。为了所谓的命令,真的就值得做出这样的牺牲吗。

“克鲁格……无论如何,他都会陪着莉莉安的。”察觉到了科宁斯的沉默,阿虎说道,“而我们,也还有自己的事情要做。走吧,不能再耽误了。”



\section*{}

阿虎说完就继续走了起来,科宁斯和向导只好默然跟在身后。

几个人在黑暗中继续走了一阵,然后停下来休息了片刻,接着继续上路。又过了一阵,他们在一条小路分岔的路口停了下来。

“走这边,”向导指着其中一条岔路说,“这条路比较近。”

说完向导开始朝那条路走去,科宁斯也跟了上去。但阿虎站在路口没有动。

“我们的汇合点,不在那边。”阿虎低头看了一眼手腕上的计时器说,那个计时器上有一个夜光指南针。

“从这边我们能够更快到达……”

“那条路,是通往叛军营地的吧。”

阿虎打断了向导的话。

“……”

向导没有说话。

“你还想去找你的儿子?”

“……是的。我刚才看见他了,就在那群人的车上!那时他没有下车,一定是跟着他们……”

“我们出发前就约好了,只去一趟难民营的。”

“但是我的儿子……”

“我们来这里不是为了帮你找你儿子,而是另外的事情。我们所做的一切都必须以我们的任务为优先,这些克鲁格早就和你说过了吧。”

“……”

向导再次沉默了。他知道自己该做什么,但是所谓的“约定”并没有什么说服力。对他来说,自己的家人显然更加重要。

“好了,赶紧回去吧。”阿虎说道,“关于你儿子的事情,克鲁格之前也承诺过了,会委托给政府军的。当时你也同意了。”

“我不能就这样看着他从我眼前擦身而过!”向导喘着粗气激动地说道,“找不到我儿子,我什么都不会做的!你们的任务关我什么事?!”

“那个人也许不是你的儿子。我也观察过了,车上的人的确有一个没有下车,但那个人一直在黑暗中甚至连轮廓也看不清。你呼喊他的时候他也没有动,你凭什么确定那就是自己的儿子呢?”

“凭我是他爸爸!凭一个父亲的直觉!!”

阿虎说话的语气一直十分平静,而向导显然已经开始情绪失控了。

“那我要是拒绝呢。”阿虎依然波澜不惊地说道,但语气却是不容置疑的。

“求你了,先生。求求你……”向导哭着跪在阿虎的脚下,不断地向他磕头,“我们就去看看、就看一眼。您要是不愿意去,让我自己去也行……”

“……那就走吧,抓紧时间。”沉默了一阵后,阿虎终于同意了。但科宁斯感觉他并非出于同情才答应向导,而是不想在见到克鲁格的时候,无法交代向导去了哪里。

几个人沿着小路走了几十分钟,向导就带着他们再次钻进了树林,他说知道叛军营地有个可以绕进去的缺口。果然,没走多远,他们就看到了光——那是叛军营地里传来的火光。

“糟糕,他们也留意到这个缺口了,已经派了人看守。”向导懊恼地说道。科宁斯随着他的目光看去,在被铁丝网围着的营地边缘有一块被崩塌的山体冲开的口子,但那个口子旁边正坐着一个昏昏欲睡的守卫。

“你们在这里等着。”阿虎一边说着,一边从后腰抽出了战术匕首。他匍匐着爬到了里那个守卫十几米远的地方,然后从地上捡起一块石头。

阿虎将石头扔向守卫的身后,石头落地发出啪嗒一声。昏昏欲睡的守卫被声音惊醒了,急忙站起来去查看声音的来源。就在他转身的一瞬间,阿虎从地上爬了起来,迅速抛出了手中的匕首。

那把匕首直直地刺入了守卫的后脑勺,守卫一声都没吭就倒了下去。阿虎一个箭步冲过去,将死去的守卫拖进了草丛。

“你的运气不错,有新衣服穿了。”阿虎一边剥那个死人衣服一边对着向导说,“假装是个叛军士兵,混进去走一圈应该很容易。”

阿虎的语气仿如没事发生过一样,而向导看着那个虽死却尚未瞑目的敌人,吓得腿都软了。他显然没有在如此近距离上看过一个人是如何被杀死的。

“你怎么想,难道你准备就这么像逛超市一样大摇大摆地进去找人?”看见向导呆在原地一动不动,阿虎讥讽地说道,“还是说,你有更好的潜入手段?”

虽然不知道向导到底有没有逛过超市,但他的心情科宁斯是理解的。眼前的一幕让科宁斯也感到心惊胆战,虽然他知道阿虎久经沙场手下早已亡魂无数,但能够如此云淡风轻地处理血仍未冷的尸体,是他做梦都没有想到的。

阿虎剥下来那个敌人的衣服,然后扔给了向导。向导咬了咬牙,颤抖着穿在了身上。

“拿上这把枪,假装是在巡逻。不要往屋里走,也别往光线明亮的地方去——就像闲逛一样,放松点,他们不会注意到的。”阿虎一边说着一边把那个敌人的枪扔给向导。然后他拔出了插在尸体后脑上的匕首,在裤子上蹭了蹭血迹,把匕首塞回了后腰上的刀鞘中。

血从那具尸体的伤口中汩汩流了出来,科宁斯嗅到了一阵浓重的血腥味,他紧咬着嘴唇才没有吐出来。阿虎看了科宁斯一眼,然后把那具尸体推到了一边。

向导拿起枪,不知所措地看了阿虎一眼,显然心里十分害怕。

“不错,和真的一样。去吧,不然我都快忍不住想请你吃子弹了。”阿虎漫不经心地说,用意不知是鼓励还是威胁,“别浪费时间,快去快回,我们在这里等你。”

听到阿虎的话,向导像是下定了决心一样深吸了一口气,然后迈开步子朝叛军的营地走去。

“还行。”阿虎一边用突击步枪上用的光学瞄准镜观察着叛军营地一边说,“多数人都还没睡。要是他想找的人在里边,他应该能找到。不过我不觉得他儿子真的在这里。”

“……阿虎。”科宁斯低声说着,“现在说这些可能有点不合时宜,但我感觉你对这种事情,好像已经……驾轻就熟了?”

阿虎微微转头看了科宁斯一眼,随即又把眼睛放在了瞄准镜前边。

“你指的是哪种事情呢。”他淡淡地说道,“杀人、脱衣,还是参加这样的化妆舞会?”

“呃……各种吧。”

“我知道你在想什么,记者。你一定是在想‘这家伙为什么能如此面不改色地杀人扒衣服,就好像是在自己家里吃早餐一样’,对吧?”

“嗯,差不多。”

“对你们这些没有亲历过战争的人,战争可能只是一些伤亡数字。但一个人从接受普世价值的教育,到变成一个杀人不眨眼的冷血动物,绝对不是几句话就能概括的事情。”阿虎仿佛在闲聊一般说着,但眼睛一刻也没有离开正在叛军营地乱逛的向导,“你知道这些小孩子是如何‘参军’的吗?”

“我不知道。”科宁斯老实地说道。

“当你在家里吃饭的时候,你家的门被踢开了,一群手拿AK的人闯了进来。然后你全家都被叫了出去。你们被命令站成一排,然后他们就通知你被征募为士兵了。”

“……就抓壮丁而言,这手法也挺老套的。”

“当然。不过参军的考核可不是审查你全家的政治面貌。有个人会给你只有一颗子弹的手枪,来考验你的枪法……当然,射击距离随你喜欢,真正的试题是靶子——你会被要求打死你的父亲……”

“什么?”

“或者,他们会在你的面前杀死你的全家。如果是你,你会怎么做?”

“我……我——”

“不用回答,这个题目留在你的心里,以后慢慢求解吧。总之你成功‘入伍’已经证明了一切。在没有了象征着律法的父亲之后,你就已经是半个合格的士兵了,因为接下来无论让你杀谁你都能下得去手了。然后你会被囚禁三天、然后痛打一顿,接着会有个慈眉善目的头领告诉你已经经过了洗礼、摆脱了家庭与社会的桎梏、并获得了全世界最大的自由。你会得到一顿饱餐、如果喝酒的话也可以,然后你的手里会多出一把枪。一群和你年龄差不多但是资历比你老的‘老兵’会带着你闯进一个你从没去过的村子,下车之后你所要做的一切就是打光自己枪里的子弹。然后,你就可以点燃村子里的茅屋、强奸被俘虏的女人——但是掠夺到的财物是要上缴的。如果你做得够酷够抢眼,你也会得到一些具有纪念意义的战利品;如果你表现得不是那么主动,就会被当场狠揍一顿。在经历过几次这种事情之后,你渐渐也习惯了、甚至抓到了开枪的诀窍。于是就会有一群年龄和你差不多的菜鸟被送过来、由你带领着他们去打家劫舍、你教他们如何打光自己枪里的子弹、然后如何点燃茅屋强奸女人、并且可以把他们带来的战利品挑两件留给自己……当然他们要是做得不够积极,你可以随便揍他们,而且平时也可以使唤他们。怎么样,做一个自由战士也不过如此,很简单吧?”

“……”

科宁斯沉默了。在这个地方他已经目睹了许多可怕的事情,但在他所知之外,依然有他闻所未闻的疯狂。

“想要让这样的地方恢复秩序,是不能依赖软弱的手腕的。仁慈的说教、精神意义上的治疗,将会收效甚微,至少不会立竿见影。所以我们这些人才会出现在这里。”过了一阵子,阿虎再次说道,“就像是治疗患病的躯体一样——必须清理掉腐烂的组织,伤口才能渐渐愈合。而这个清理过程中,疼痛是在所难免的。”

阿虎的话冷酷而现实,让人无法辩驳。如果这些叛军没有武装,那么政府自然可以考虑将他们进行收容和劳教;但人的手里一旦拿着枪,那么说教的力量就显得微不足道了。没有人甘于束手就擒,他们一定会拼命抵抗的。

“嗯,我们的非常重要人员先生回来了。不过好像收获不多。”正当科宁斯默默地思考着阿虎所说的话时,他忽然听到阿虎在身边说道。

科宁斯定睛一看,果然有个身影在光线之外的地方慢慢接近。那个人的步伐蹒跚甚至有点踉跄,科宁斯能够看出那是有些失魂落魄的导游。

“情况如何?”阿虎问道。其实不必问,几个人心里都已经知道了答案。

“没有。”导游垂着头说道,“他不在这里,我悄悄观察了营地和所有屋子,没有发现墨菲。”

“没有也好,至少以后还有希望。”阿虎说,“总好过你找到了自己的儿子,却被他一枪崩掉,或者交给他们的头领来请赏强。”

“墨菲不是那样的孩子!”听到阿虎的话,向导愤怒地反驳道。

“随便你怎样说吧,反正多数童军的父亲都已经没有机会为他们的儿子辩解了。”阿虎耸耸肩说,“既然如此,我们就赶紧去我们该去的地方吧。每耽误一分钟,都会为接应撤离的人增加一分危险。”

说着,阿虎头也不回地走向了他们来时的方向。

阿虎在前边走的很快,显然是为了赶时间,科宁斯不得不走一阵就跑几步才能跟上。而向导则因为再次受到了失望的打击而变得有些脚力缓慢,渐渐地被阿虎和科宁斯落在了后边。

几个人走了一阵,渐渐接近了之前的岔路口。但在距离岔路口还有一段距离的时候,阿虎的脚步忽然停下了。

科宁斯走到阿虎身后,向他的视线方向一望:那个岔路口上亮起了隐约的亮光——那里有人。

“保持安静,”阿虎低声说道,“有情况了。”

说完,阿虎压低了身子,慢慢朝着亮光处走了过去,科宁斯也跟在了后边。他们在距离亮光还有四五十米的距离处再次停了下来。

因为距离已经相当近了,所以透过树丛科宁斯也能看到,那里生起了一小堆篝火。而围坐在篝火前的,是两个叛军士兵——他们的身形矮小,正怀里紧紧地抱着枪围着火堆打瞌睡,一看就是童军。

“再有两个小时天就要亮了,我们没时间绕路了。”阿虎对着科宁斯悄声说,“你呆在这里,我去搞定他们。”

说完,阿虎从胸前抽出了手枪,然后拧上了消音管。但正当他要举枪射击的时候,忽然被身后不知什么时候跟过来的向导死死拉住了。

“等等!”向导说,“别开枪,我觉得那两个孩子……”

“都很像是你的儿子?”阿虎淡然地接着说道。

向导没有说话,只是用恳求的眼神看着阿虎,然后点了点头。

“他们有两个人。想要在不发出声响的情况下同时制服他们,是不可能的。”阿虎说。

“我知道。不过,我想,如果我去吸引他们的注意力的话……”

“就这么堂而皇之地从树林里走出来,不可能不引起怀疑。”

“那……”

向导低下了头。显然他知道自己提出的事情是无法实现的,但是他无论如何都想要试一试——寻子的执念,已经让他不顾一切了。

“……我来当俘虏。”沉默了片刻后,阿虎说道。

“什么?”

“你继续假扮叛军,而我假装是你抓到的俘虏。接近他们之后我们一人对付一个,同时动手打昏他们——知道怎么做吧,用你最大的力气,猛击颈部。”

“……好,好的。”

“我先说明一下,到时候无论你看到的是谁,都要把他打昏,然后再说后边的事。这里离他们的营地很近,一旦惊动了敌人我们就连逃都逃不掉,明白吗。”

“明白了。”

“还有,这是你最后一次浪费我们的时间了。做完这件事,在回到我们的营地前,不许再提你儿子的事情了。”

“好。”

阿虎看了向导一眼,没有再说话。虽然他完全无法信任向导,但他还是取下了自己背上的突击步枪递给了科宁斯。

“科宁斯,如果我们暴露了,你什么都不要做,直接逃跑。我们的营地在北方,你要设法回去——很远,而且独自行动会有点困难,但你必须把这个情况汇报给克鲁格。”

“……我知道了。”科宁斯接过阿虎的枪,微微点了点头说。

“开始行动吧。”

阿虎说完,把手枪藏在了腰后,然后背过手假装自己被捆住了手。而向导则在他背后推搡了一把,装作自己抓到了俘虏的样子。

科宁斯看到两个人慢慢接近了那堆篝火前的童军。向导对着那两个昏昏欲睡的少年守卫说了句什么,那两个孩子站了起来——正当他们好奇而不知所措的时候,阿虎突然挥手一拳打在其中一个孩子脸上,那孩子一声不吭地就倒在了地上。而向导显然没有干过这种事情,只是用胳膊勒住了另一个孩子,然后用手捂住了他的嘴。

那个孩子还在不断挣扎,而向导却无法制服他,不过好在他没让那个孩子发出声音。阿虎果断再次出击,用手肘朝那个孩子的太阳穴猛击了一记,第二个孩子也被打昏了。

见两个人的行动得手,科宁斯赶快跑了过去。他听见阿虎和向导正在说些什么。

“如何?”阿虎问道。

“不,不是他……”向导说。

“真遗憾。”阿虎叹了口气说,“还不如一开始就按原计划执行。”

说着,他将两个孩子拖进了丛林,然后掏出了消音手枪。

“你要干什么?”向导跑到阿虎的身后,惊讶地说道。

“你说呢?”阿虎转过身反问道。

“你疯了吗?他们……只是些孩子呀!”

“又不是你的孩子。再说,他们手里的枪可不管扣扳机的是孩子还是大人。”

“你不能这么做!他们也不过和我的儿子年纪相当,我不会让你……”

嗵。

向导没能把他抗议的话说完,因为肚子上挨了阿虎凶猛的一拳,让他痛苦地弯下了腰。接着,阿虎抓住向导的头,抬腿给了他的面门一记狠狠的膝撞。

向导仰面朝天躺在了地上。

“闭嘴,你这头没脑子的猩猩。我说过不许再提你的儿子了吧。”阿虎冷冷地说道,“给我听好:如果以后我在什么地方遇到了你的儿子,我保证会把他的尸体送到你家里去的,再顺便把你一家都埋在一起,这样你们就永远不会分开了。现在不要再跟我说这些废话,不然我就先把你的脑袋割下来。”

“你敢……我会杀了你,你这个王……”

嗵、嗵、嗵。

向导奋力想要爬起来攻击阿虎,却被阿虎在脸上连踹了三脚,终于趴在地上一动也动不了了。科宁斯惊恐地发现,虽然向导是个相当粗壮的黑人,但在阿虎面前却没有丝毫抵抗的能力。

“你说的没错,我们的任务不关你事。那你的儿子又关我们什么事?”阿虎把向导的脸狠狠地踩在杂草丛生的地上,用阴冷的声音说道,“医官是怎么受伤的你该心里有数吧,你觉得你儿子的命比她的命更有价值?告诉你吧,你和你儿子的狗命加起来,在我眼里也不值一颗子弹钱。我们在为了你们的人民而战斗,但你的眼里却只有你自己。你们这些愚昧而卑贱的种族根本不值得我们去流血,因为你们只是一群自私自利的猪猡!我不是克鲁格,没有那么多的宽容和耐心去安抚劝导你。只要再有一次让我觉得你是在碍手碍脚,我就像宰掉一只丛林狒狒一样宰了你,听明白了吗?我可能没说过吧,前一阵子我和克鲁格在丛林里潜伏的时候,都是猎捕那些动物当做食物,我一直都觉得你们这些土著和那些肮脏的猴子一样烦人!”

说完,阿虎朝着地上吐了一口唾沫,然后放开了脸已经陷入泥土里的向导。正当向导想要站起身的时候,阿虎又朝他的肚子狠狠踹了一脚,向导一声不吭地再次倒在了地上,过了半天才爬起来。

眼前发生的一切让科宁斯猝不及防,他噤若寒蝉地一声也不敢出。虽然知道阿虎绝非什么温和礼貌的人,但这些天阿虎给科宁斯的印象还是相对有风度的,沉默寡言但是从不像克鲁格那样满口脏话。一直到刚才,阿虎都没有任何情绪化的表现,甚至就连看到医官负伤他都一句话也没说。但现在的阿虎,让科宁斯感觉要远比独断专行的克鲁格可怕的多,他就像是……

正如“阿虎”这个名字一样,像一头发怒的老虎。

“好好干你的事,别再惹我发火。”稍稍调整了一下情绪后,阿虎的声音平静了一些,但言语间的寒意却有增无减,“说真的,我一开始就不支持克鲁格的冒险行为,因为我们有很多办法能让你带路,没有必要绕这么大圈子——不过那家伙不同意。你大概也知道吧,带路其实不需要用腿走、用手指,只要你开口告诉我们该怎么走就行。让人开口的方法我知道得很多,不过在让你开口之前,我会很乐意先把你没用的手脚摘掉。现在用你最快的速度往汇合点行进,我会跟在你后边的。如果我觉得你在消极怠慢、或者走的方向不对,那你就见不到明天的太阳了。走。”



\section*{}

向导用自己的鼻青脸肿为代价,保住了两个童军的小命。得益于他的努力争取,那两个孩子没有被阿虎灭口,而是就那么被扔在了草丛里——考虑到也许很快会有其他叛军巡逻过来,以及那番争执浪费的时间,阿虎选择了走为上策。

在阿虎的“激励”之下,几个人的行动速度快了不少,到天微微发亮的时候他们及时抵达了汇合点。那个开车的年轻士兵依照克鲁格的指示,已经在那里等着他们了。

“太好了阿虎,你终于来了!”那个年轻人说道,“不过,呃……克鲁格和医官呢?”

“他们出了点情况,暂时过不来。”阿虎说,“我们先回营地,晚上再来接他们。”

“是吗……好吧,我知道了。”

几个人快速登上了汽车。阿虎让科宁斯坐在司机后面、导游坐在副驾位上,自己则坐在导游的后边,手里紧紧握着手枪。

趁着天还没有完全亮起来,几个人快速赶回了克鲁格的营地。行路所用的时间和来的时候差不多,但一路上没有了克鲁格的故事和医官的玩笑,几个人都非常沉默,气氛让科宁斯感到难言的压抑。

回到营地后,向导被关在了一个铁丝网围绕的帐篷里,阿虎命令三个士兵轮流看守。

因为这两天的行军实在是太疲劳了,科宁斯没有心情再去关心向导。他钻进自己的帐篷就睡着了,就连东西都没有吃。

当科宁斯醒来的时候已经是下午,太阳已经微微西斜,但还没到黄昏时分。他走出帐篷,正好碰见迎面而来的阿虎,阿虎的身后还跟着三个整装待发的士兵,以及脸尚未消肿的向导。

“来的正好,科宁斯。”看到科宁斯走出帐篷,阿虎说道,“太阳落山前我们就要出发了。”

“呃,是吗。”科宁斯伸手挠了挠头,他不确定阿虎所谓的“来的正好”是什么意思,“你是在找我?”

“是的。能说几句话吗。”阿虎说着,看了身后的士兵们一眼。

“好的。”科宁斯点了点头。

“这边来。”

阿虎说完朝着营地的一角走去,把几个士兵和向导留在了身后。科宁斯于是跟了上去。

“我们的这次行动就要到尾声了,我想我们马上该说再见了。”阿虎在营地边缘一个没有人的地方停了下来,从兜里掏出烟盒摸出两根烟,把其中一根递给科宁斯。

科宁斯接过烟,发现正是自己给阿虎的“骆驼”——不过看起来这是最后两根了。

“这么说,你们马上就要离开了啊。”科宁斯点上烟,抽了一口说。

“是的,所以这次算是和你道别吧。” 阿虎也点上了烟,“你是个不错的记者——‘记者是没有国界的’,嗯,这句话让我印象很深刻。”

“呵呵。”

“克鲁格和你说过我们的任务目标了。一旦确定目标区域的位置,我们就会将位置发给政府军,之后就会直接撤离战场。如果不出意外,时间就会在今晚——如果我们在夜里转移的话,这里将不再安全。所以,我希望你能在那之前离开我们的营地,去往村子里下榻。”

科宁斯恍然大悟——原来阿虎是在下逐客令啊。不过这话说得还真是委婉,如果换做克鲁格,大概会直接告诉科宁斯“这里没你的事了赶紧滚蛋”吧。

不对啊,科宁斯在心里嘀咕道,克鲁格还欠我一次访谈呢。虽然说是不做记录,但也不能就这样一笔勾销了吧。

科宁斯忽然想起来,阿虎大概不知道他和克鲁格之间的约定。不过那也是无可奈何,这两天发生了这么多事情,让所有人都始料未及。这样看来,对克鲁格的访谈是要泡汤了。

真是计划赶不上变化。

“话说,克鲁格呢?”

“他留在了丛林。我们约定了碰头的时间和地点,不过为了保持隐蔽他已经关闭了通讯。”

“他该不会是被敌人……”

“不会。他是用暗码传送的信息,内容十分清楚。况且就凭叛军那些废物,还抓不住克鲁格。”

那么医官呢。

这句话到了科宁斯的嘴边,但他还是没有说出来。他知道自己不该问那些,因为行动已经迫在眉睫,说这些事情难免会让士兵们分心。

克鲁格去了那么久,说不定他已经把医官送到难民营了,科宁斯自我安慰地想着。虽然阿虎那时的表情非常严肃,但克鲁格总不会真的那么无情吧。

也许经过治疗,医官已经稳定住了伤情,说不定过几天就能归队。

“是吗。这么说,这可能是我们最后一次见面了?”科宁斯笑着说。

“那也未必。山不转水转,只要活着,总有再会的可能。”

“也就是说,首先要活着才行啊。听起来真让人不安。”

“呵。”阿虎难得地也笑了一声,“不管怎么说,你也不该为这种事情担忧吧。”

“那可难说。如果是以前,我肯定不会担心这些事,不过我已经立志要做一个战地记者了,所以担心未来也是情理之中吧。”

“怎么,对职业生涯做出新的规划了吗。”

“是啊。我总是说自己追求真实,可是经历过真正的战场之后,才发现自己所发掘的那些‘真实’太肤浅了。”科宁斯认真地点点头,“所以这不会是我最后一次在战场上做报导。”

听到科宁斯的话,阿虎没说什么。他用力抽了一口手里的烟,吐出一大口烟气。

“别看克鲁格那个样子,其实他是个老好人。”阿虎说。

“嗯?”

“我们第一次一起作战就是在非洲,那次我出了意外……已经无法灵活地行动了。我让他独自撤离,可他不肯,执意要和我一起。我们迷路而且失去了补给,在丛林里艰难地潜伏了二十天,一直到救援部队抵达。那时如果没有他,我想我是不可能坚持下来的。他不像我,他在心里尚留存着一丝仁慈……因此他从不抛下同志。”

“……是吗。嗯,我能看得出来。”

科宁斯不知该说如何应答——阿虎究竟是在说克鲁格是个有情的人、还是说他自己是个无情的人呢。他觉得阿虎还不是那么冷酷,但既然阿虎这样说,科宁斯也无法去反驳。两个人陷入了沉默,气氛一时间有些沉重。

“克鲁格和医官,是怎样的关系呢?”片刻后,科宁斯问了一个他一直都想问的问题。虽然他知道这种时候不该问这些,但既然已经到了诀别的时刻了,索性就畅所欲言吧。

“他们一起经历了很多战斗,我不知道他们是什么时候相识的。”听到这个问题,阿虎稍微有点迟疑,“我想感情一定很深吧,毕竟是老战友了。”

“没有……其他方面的吗?”

“其他方面?”

“男女之间的那种。”

“你说那个啊。”阿虎恍然大悟,“莉莉安一直对克鲁格……嗯,这其实也不是什么秘密了。但克鲁格对军营里的人都是一视同仁的,也许是因为他作为长官的身份吧。如果离开军营,他们也许会是不错的一对,但是谁知道呢。我们这些人,没人敢说自己最后能不能全身而退。”

“那你呢,有喜欢的姑娘吗。”

“没有。革命尚未成功……”

“别顾左右而言他,我是认真的在问。”

“科宁斯,这是娱乐记者才会问的问题。”

“那又怎样,这又不是采访。”

“……”

阿虎没有说话。他沉默地看着远方,午后的阳光洒在他的侧脸,让他另一侧的脸庞陷入了阴影之中。他似乎在认真地思考科宁斯的提问。

“你不觉得,和我们这种人产生太深的感情,会是一种负担吗。”

过了一阵,阿虎开口轻声说。

“嗯?”这个反问让科宁斯感到有些意外。阿虎好像提出了一个相当独特的观点。

“终日穿梭在弹雨之中,能否活着看到明天都是未知的……和这样的人相识,恐怕只会让人寝食难安吧。所以我觉得还是不要和别人扯上太密切的关系为妙,以免给别人徒增烦恼。”

科宁斯大感意外,他无论如何都没想到阿虎竟然会说出这种话来。

这种想法虽然不能说毫无道理,但也太过患得患失了。而且因为担心给别人带来烦恼,而刻意地埋藏起自己的真心,听起来完全就是懵懂少年才会有的忧愁啊。

“人总不可能事先考虑好后果,再和另一个人产生感情啊。”科宁斯说,“再说人都是有同类的,因为自己和别人不同而感到不自信,那是小孩子的社交障碍吧。”

“……同类吗。”听了科宁斯的话,阿虎无谓地咧了咧嘴,“战争机器的同类,是什么东西呢。无非还是战争机器。而机器,又何故需要感情呢。”

科宁斯看着阿虎的脸,没有说话。他是在笑吗,科宁斯心想。

他的确是在笑。但他不是因为感到高兴而笑,而是因为讽刺了自己而自嘲地笑。

科宁斯忽然感到,那个笑容中似乎有着几分落寞。

“别那么说。”科宁斯拍了拍阿虎的肩膀说道,“你可不是什么机器,你是个人。你有快乐和悲伤、有着在意的东西和憎恶的东西,那都是真实存在的感情。心存渴望就不要回避,寻找自己情投意合的伙伴或者伴侣,是天生的本分,人人都有这样的权力。”

“嗯,如果有了动心的姑娘,我一定会不会错过的。”

说出这句话的时候,阿虎的脸上依然是那样的笑容,所以科宁斯不知道他是不是认真的。

不过想想一起经历过的一切,作为临行的饯别,这个笑容也足够了。

“那就这样吧。”科宁斯说,“我马上就去村子里。祝你们行动顺利。

“放心,我们总是行动顺利。”阿虎点点头说,“后会有期,记者先生。”