\chapterul{外传:丛林之虎(四)}



\sectionul{前言}

重新整理这个故事之后,心中许多感慨。最大的感慨就是回头一看才发现这个故事里也有浓烈的宿命论。

是什么让克鲁格动了救世之心,是医官还是修女?克鲁格是混蛋的外表下面藏了一颗圣人的心。而阿虎则是个斯文杀手,只对朋友忠诚、对敌人没有一丝怜悯。但多年过去,他们都变了。

这个故事包含了许多线索,其实适合在全篇结束之后再去看,这样大概能对克鲁格这个角色多一分了解。

我也有点痛恨自己,把克鲁格捏造成了一个遭遇了诸多不幸、终生都在痛苦中挣扎的男人。但我又非常喜欢这样的男人,他是个彻底的硬汉,就算再多不幸也没有把他打倒,他一生都在贯彻自己的救赎——用战术人形来代替人类去战斗、去受伤、去死。但是他自己也不知道这种做法是不是正确,因为战术人形和人类之间的界限,正在不断模糊。

在故事的最后时刻,克鲁格才解开了自己一生的郁结,在遭到最信任的朋友的背叛之后,他才明白,人们的认识是不断改变的,他倾尽一生想要成就的事情在其他人,甚至在他的亲信眼中,也许是罪恶的。他无法统一所有人的意见,因此也不需要在乎别人的看法。他不必为自己所做的事情定义对错,对错自有后人评说,他只要做了他认为正确的事情就好,这个世界自然会融洽合理。

因此,最后的背叛恰恰也是最后的救赎。因为克鲁格终于可以放下“对与错”的包袱,坦然地面对自己的余生。

看了上面这段话的朋友,也许会困惑,不明白我在说什么,这很正常,因为我不能剧透过多。克鲁格做了什么,克鲁格的妻子和家人后来怎样了,陆久又和这些人有着怎样的关系?等到读完《背叛者》故事全篇,自然会有答案。

正文在下一页。

\lineseparator


\section*{}

科宁斯走进村子里的时候,太阳尚未完全落山。火红的夕阳坠在天边,远处的丛林看起来若隐若现,仿如地平线上的氤氲。

如果没有亲临那里,对科宁斯来说,那里就只是一片在人们的谈话中才会出现的海市蜃楼。

但科宁斯现在知道那个地方是真实存在的,而且正是这片地区的灾祸之源。曾经作为他的“朋友”的一些人,此刻正在去往那个方向,想要悄然潜入那片虎穴龙潭。

科宁斯从兜里掏出一根烟放在嘴里点上,那是他从难民营里的同事手中要来的半盒“情义烟”。德国的烟草滋味醇香而厚实,让人想起家乡的味道,但科宁斯却并不喜欢这种口味。

他喜欢的是辛辣干烈的美国香烟,一大口吸进去直呛嗓子,一股狂野西部的冲劲。

就像自己所追求的人生一样,科宁斯心想。如果老实呆在家里、做个办公室里的职员之类的,现在估计也早就过上了安稳的家庭生活了吧。但不知为何他却偏偏不甘平凡,总喜欢做一些让别人睡不安稳的事情。

因此他才从家乡汉堡跑到了法兰克福,成了个让主编又爱又恨的新闻记者。

也是因此,他才到现在还是单身一人。

想到这些,科宁斯笑了笑。幸亏刚才阿虎没人问他的个人问题,不然阿虎会发现这个满口说教、谆谆善诱的记者,其实也是个坏榜样。

不过阿虎有些话说得也对啊,科宁斯有点发愁地心想。自己这种总爱东奔西跑、四处惹麻烦的家伙,到底谁会看上呢?

一边心怀这微微的惆怅,科宁斯一边慢慢走着,不知不觉他的双脚已经把他带到了修道院的门前。透过修道院那不怎么高大的篱笆院门,科宁斯能够看到教堂的门前正站在一个人——身形高挑又有几分纤细、穿着深色的修道服,正默然面对着夕阳的方向,仿佛在企望着什么。

科宁斯不用细看也知道那是修女黛雅。

“你好,姐妹。在等什么人吗?”科宁斯对着黛雅说道。

黛雅转过了头,看向了科宁斯。有那么一瞬间,科宁斯感觉她的眼神似乎有些恍然。不过旋即,她的脸上就浮现了一丝浅浅的笑容。

“没有,只是闲来无事而已。”黛雅点了点头说道,“你回来了,记者先生。”

“是的,又要叨扰你们了。”

“哪里。主是仁慈而博爱的,自当庇护远行的旅人。”

“那个……。”

科宁斯挠了挠头。

“怎么了吗。”他欲言又止的样子,让修女感到有些奇怪。

“我们顺利抵达了难民营,找到了向导的家人。”科宁斯深吸了一口气说,“但是他的儿子不在那里。他们……走散了,至于下落目前还没有消息。克鲁格说会将此事知会政府军方面,请他们代为留意。”

“这么说,向导已经同意为克鲁格带路了吗。”

“嗯,是的。我想他们很快就会开始行动。”

“愿他的家人能够平安。”

“……姐妹,我想问您几个问题,不知道可以吗。”

“请问吧。”

“请问你认识向导……就是那个叫墨菲的孩子的父亲吗?”

“我想还不能说认识,但是我们见过几次面,有那么几回我在教堂讲课讲得晚了,墨菲的父亲曾经来接他回家。听说那个人是个非常优秀的猎人,经常带着他的族人外出捕猎,每次都有丰富的收获。不过在我看来他只是个朴实的男人,和本地的其他人也没什么不同。你为什么忽然问起这个,向导他出了什么事吗?”

“呃……没什么,只是随便问问。因为找不到自己的孩子,向导十分沮丧,情绪让人不免感到有些担心。”

科宁斯掩饰地说着。他知道黛雅不太喜欢克鲁格一行人马,如果把向导和阿虎的冲突告诉修女,只怕会更让她不快。

“那是当然了,虽然只是见过几次,但我也能感到墨菲的父亲非常爱他。那是个非常伶俐的孩子,学习也十分努力,他说他长大希望成为一个医生为村民们治病。他在课余时间也会给神父帮忙,神父教给他的医疗知识,他学得非常认真。在这个迷信的蒙昧之地,肯相信医学的人甚至都很少,而墨菲却很有这方面的天分……真希望他能早日回家。”

黛雅说完,闭上眼睛轻声祈祷了几句,然后在胸前划了个十字,看来她是真的很担心那个孩子。但就科宁斯所知,墨菲的处境绝不乐观。

“他一定会平安无事的。”科宁斯违心地说着,走上了教堂二楼。他回到他的小房间里,把随身的背包扔在床上。除了他出来的时候带的水壶、笔和记事本,背包里还被塞了两份军用口粮。那是阿虎的临行赠礼,他还顺便教会了科宁斯如何使用里边的生石灰来加热口粮。不过想起那些口粮糟糕的味道,科宁斯没有去吃那些东西。

太阳落山后,科宁斯在教堂里吃了点简单的晚餐。之后神父在逐个检查伤员的病情,而黛雅则一直跪在神坛下面默默地祈祷着。

“她真是个虔诚的信徒。”当神父走过科宁斯身边的时候,科宁斯小声地对神父称赞道。

“你这样说不就让我情无以堪了吗。”神父笑着说道,“我一个星期里祈祷的时间,都没有黛雅一天多。”

“抱歉,神父先生。我并非在指责您。”科宁斯有点窘迫地说。神父一直在为伤员治疗和检查,的确忙到停不下来,这一切科宁斯都是看在眼里的。

“我知道,我只是开个玩笑。”神父宽容地说,“也许我在这里做的事情,比起神父更像是个医生。幸亏还有黛雅,时刻提醒我们保持坚定的信仰不要迷失……她不仅帮了我很多忙,而且也给了我很多鼓舞。”

“我一定会把你们的事迹传播出去的。你们都是些很了不起的人。”

“圣子所到之处,必有奇迹降临——他要救那些信祂的人;而那些不信的人,见了这奇迹,也便信了。我们身为使徒,只是依照祂的旨意行事,不堪谬赞。”

“请您不必自谦。无论信与不信,你们的善行都彰显着仁爱,即便是不信的人也会知道那是美德的光辉。”

“能得到记者先生的称赞我深到宽慰。”听到科宁斯的话,神父高兴地说,“你说得对,无论信与不信,只要我们行了真的教义,就是传了祂的福音。”

“我要说您做的一切都实至名归。”

“病人现在情况都很稳定,我要先回去房间了。”神父在水盆里洗净了双手说,“一日未受神之感召,我也该做些祷告了。如果有什么事,请和黛雅说就好。”

说完,神父轻声念了一句以马内利,朝着楼上走去。科宁斯默默地坐在教堂的角落里,看着依然在祈祷的黛雅。

过了好一阵,黛雅祈祷完了,站起了身。她看了一旁默不作声的科宁斯一眼,没有说什么,只是向着自己的房间走去。

片刻后,黛雅再次回到了教堂的大厅,只不过身上的修道服已经换掉了。

“你很会说话呢。”

黛雅走到科宁斯跟前说。

“此话怎讲?”

“神父很少会引经据典,只有和真正的信徒交谈时他才会借用典籍里的话……可以看出他对你的称赞非常欣赏,你的话让他很高兴。”

“可我并没有刻意地阿谀。我说的都是事实。”

“也许那就是他高兴的原因。你让他觉得自己所做的事情,并非怠于虔敬,而是真切地践行了主的教诲。”

“那他理应感到高兴。你也是。”科宁斯笑了笑。

黛雅也笑了笑,然后转身向着病号们走去。她依次巡查了那些病人,确认了他们的情况全部正常,于是默默地站在了教堂的门前。

“我回来的时候,克鲁格留在了丛林。”科宁斯轻声说道, “医官……受了伤,克鲁格要带她去难民营治疗,所以没有跟我们一起回来。不过阿虎说他一切正常。”

“……我没有问他的事情。”黛雅沉默了片刻,说道。

“那就当是我说了些多余的话吧。”科宁斯笑了笑。

“莉莉安……受伤了?伤势严重吗?”

“……那个。当时是黑夜所以我看得不是很清楚,但是既然克鲁格和她在一起,所以我想……应该,没事吧。”

科宁斯并非刻意说谎,因为他迄今为止没有听到关于医官的消息,因此确实不能确定医官的情况。但根据当时克鲁格的反应来看,恐怕形势要比“应该没事”要严重得多。

所以他的话,不知道是在安慰修女,还是安慰他自己。

听到科宁斯的话,黛雅没有说说明,只是在胸前划了个十字。

“我能,问问关于克鲁格他们的事情吗。”

“我对他们的事情不太了解,也没什么可说的。”对于这个问题,黛雅显得有些回避。

“他们的行动就在今晚。阿虎说他们行动结束后可能会连夜转移,让我离开了他们的营地。所以明天早上之前,他们可能就会撤离……然后去别处作战。也就是说,也许不会再回来了。”

“……是吗。”

“所以……有些事情,我只能向你了解了。”

听了科宁斯的话,黛雅沉默了。过了一阵,她轻轻打开了教堂的门:

“请到外边来,不要打扰了病人休息。”

科宁斯随着黛雅走了出去,看见黛雅已经坐在教堂门外的台阶上了。今晚的天气也十分晴朗,月色皎洁而明亮,把教堂院子里的一切都照得很清晰可见。月光洒在黛雅的脸庞上,反射着淡淡的光辉,仿佛在她的脸庞镀上了一道银边,为她长长的睫毛拉下两道细细的影子。

科宁斯不由得再次在心中赞叹,这真是一位难得一见的美人。

“克鲁格他们在这里开展行动,有多久了?”科宁斯坐在黛雅身边,然后掏出一根烟点上,开口问道。

“一年吧……嗯,差不多。”黛雅轻声应答。

“一直都在这个村子附近?”

“最近一年都在这里,就算不在村外安营也在附近。他们在这里获得了很多补给,也为村子提供了一些…安全保障。但来这里之前的事情我也不知道。”

“冒昧一问,你好像……不太喜欢克鲁格?”

“我……并不是不喜欢克鲁格,而是不喜欢他那一类人。这些事情,德鲁巴应该和你说过吧。”

“嗯。不过有时候我倒觉得……”科宁斯想了想说,“啊,算了。克鲁格这家伙,确实不怎么会讨好人。不过他这个人还不坏。”

“我说了,我并不是不喜欢克鲁格……”黛雅开口纠正科宁斯说,但忽然意识到自己说错了什么,“不,没什么。”

这无意的失言,让科宁斯微微笑了笑。

“阿虎说,你总是对克鲁格有点过度关注。”

“我哪有?”因为急躁,黛雅的声音稍微有些提高,“我只不过是……”

“没有吗。”

“……他们总是在做些危险的事情。”

“没事吧。那些家伙打仗挺厉害的。”

“可是枪弹无眼。这里的伤员有很多都是在战乱中受伤的,并不是所有人都那么幸运,能够在接受治疗后渐渐好起来。克鲁格他们已经……”

“他们……难道也有人……?”

黛雅再次沉默了。过了好一阵她才说道:

“克鲁格的人里,已经有四名士兵牺牲了,他们都埋在教堂的后面。每次有新的伤员被抬过来,我都担心会是克鲁格的人……或者是他本人。”

……已经四个人了吗,科宁斯心想。本以为克鲁格那群人能征善战,就算是出入沙场也能保得了性命,但不想战争果然比他所知的更残酷。

是啊,那么多子弹飞过来的时候,谁知道哪一颗会打中自己呢。科宁斯心想。在自动武器面前,再坚毅的人也不过是血肉之躯。

“这次他们只是进行侦查,不是正面作战,应该会安全一些。”科宁斯安慰黛雅说道,“所以你也不必太担心。”

这样的安慰显然是毫无说服力的,科宁斯自己也清楚。就算这次过去了,但后边还有其他的任务,战斗对克鲁格那一群人来说是无穷无尽的。就连阿虎都说,没人敢说自己最后能全身而退。

“我知道。”黛雅说,“他们那些人就是做的那样的事情,担心也是没有用的……我明白。”

“我听阿虎说,医官对你的事情……非常敏感。所以他告诉我千万别在医官面前提起你。”

“莉莉安她……因为一些原因,在某些方面非常神经质。”黛雅喃喃地说着,“我们也曾是非常亲近的朋友,但自从她觉得我对克鲁格……我们的关系就开始渐渐变得难以把握了。也许等她回来,我该和她好好谈谈,我并不是她想象的那样……”

“所以你并没有对克鲁格产生过分的……?”

“……”

黛雅没有说话,只是扭头看了科宁斯一眼。然后她的目光渐渐变得淡漠,甚至于说是有些冷峻了。

“我是服侍神明的人,记者先生。我已经发誓要把自己的全部都献给天主。您这样的想法,是对我信仰的亵渎。”

“很抱歉,姐妹。”科宁斯也随着变得恭敬了起来,“我无意冒犯,只是……随口说些无关大局的事情。”

“我真心希望他们能够平安,就像希望其他人平安一样。”黛雅说着站起了身,“但事情没有你想象的那么复杂。世俗的事情,早已不会让我为之在意了。”

说完,黛雅快步走进了教堂。科宁斯叹了口气,他知道今天的“采访”结束了。



\section*{}

当科宁斯被一阵嘈杂的声音吵醒的时候,天还没有亮。他从那扇小小的窗户朝外看了看,发现天空中晨星依稀可见——他最多不过睡了四五个小时。

科宁斯穿上衣服,朝着楼下走去,想要弄清楚到底发生了什么。走到楼下,他看到神父和修女已经在大厅里了,两个人穿着整齐的服装正站在教堂门口望向外边。而教堂的门外,科宁斯依稀可以看到有许多村民正在匆忙地穿行。

“怎么了?”还有点惺忪的科宁斯一边走向神父和修女,一边说道。

“不知道。不过我想很快就会知道了。”黛雅说道。

“人们的神色很慌张,通常只有在有叛军入侵的时候才会有这种情况。”神父说,“可如果是那样,人们通常会来教堂避难……可这次一个人都没有。而且由于村子里组织起了民兵,已经很久没有叛军的侵扰了。难道这次出了什么前所未有的状况吗。”

科宁斯没有说话,虽然不知道到底发生了什么,但他知道肯定不是什么好事——而且他能够预感到,一定和克鲁格那些人有关。

果然,没过一会儿,一辆吉普车停在了教堂院门外。车上跳下一个人,朝着教堂的方向走来,那个身影很熟悉,但却不是克鲁格——那个人是阿虎。

科宁斯的心里咯噔一声。如果来的是克鲁格,那么他也许只是需要一些援助……而来的是阿虎的话,那么很可能真的有什么严重的事情发生了。

毕竟,阿虎是个讷于言而敏于行的人。

“威利斯神父、黛雅,你们好。”阿虎走到教堂的门前说道。

“你好,阿虎。”神父笑着对阿虎说。而黛雅只是微微点了点头示意,没有说话。

“我是来通知你们紧急撤离的。据可靠消息,这里很快将发生大规模战斗,留在这里将会非常危险。”

“大规模战斗?!发生了什么事情?”科宁斯惊讶地说道。

“政府军今天凌晨袭击了位于丛林里的某个叛军营地,但是有部分叛军武装逃离了围剿。据政府军提供的消息,那些残余的叛军正在向这个方向前进,预计清晨就会到达。”阿虎看了科宁斯一眼说道,“无论目的为何,我想他们绝对不是出于友善的意图。我已经通知了德鲁巴让村民们马上撤往我们指定的安置点,你们有十五分钟的时间来收拾一下,然后和我一起离开。”

“安置点在什么地方?”神父问道。

“此地以东,距离约三十公里,有一处我们临时建立的防御性营地。你们可以暂时安置在那里,等待政府军的救援。”

“这里都是伤员,多数无法自由行动。”神父摇了摇头,“那么远的地方,如果没有车辆,他们恐怕去不了。”

“……很遗憾,我们也无能为力。”阿虎说,“我们只有三辆车,而且车队已经转移到别处了。”

“……”

听到这样的消息,所有人都沉默了。眼下的情况十分严峻,但他们却一筹莫展。

“黛雅,你和记者先生跟着阿虎去往安置点,我留在这里照顾伤员。”神父说,“这里没有战斗人员,就算叛军来了也不会伤害我们的。”

“不,神父。我不会离开的。”黛雅淡淡地说道。

“我强烈建议您立即跟我离开,还有神父先生也是。”阿虎说。

但神父并没有答应阿虎的建议,只是微微笑了笑。

“谢谢你的好意,但我们不能离开。我倒是希望黛雅能跟你们走,但是她的脾气你知道的……也许你可以劝劝她。”

阿虎看了黛雅一眼,没有说什么。他稍稍思考了一阵,把目光放在了科宁斯身上。

“你呢,记者?”阿虎说。

“如果你是在要求我离开的话,我会跟你走。”科宁斯说,“不过我也想留下来,毕竟这种近距离接触叛军的机会难得,而且就这样把神父和修女扔在这里也不好。”

听了科宁斯的话,阿虎拿出了对讲机。

“克鲁格,我是阿虎。”他对着对讲机说道,“记者同意离开,但神父和修女拒绝撤离。请求指示。”

“……明白。我来处理。”科宁斯听到对讲机里说道。

只过了一小会儿,克鲁格就出现在教堂的门口。虽然阿虎说部队已经转移了,但看来克鲁格并没有走远。

“我们昨晚揭发了叛军的一个营地,今天凌晨的时候,政府军把那个营地掀了个底朝天。”克鲁格一句寒暄的话都没有说,只是直白地开始说明情况,“但政府军的目标不是武装分子,而是营地里面的东西,所以很多叛军士兵都逃散了。政府军完全没有追击……因为那些散兵游勇在政府军眼里已经不成气候了。他们只是告诉我们,那群混蛋重新集结后朝着这个村子来了——具体是朝我们来的还是朝你们来的,没人知道。我猜大概两方面都有吧。不过他们肯定不是来干好事的。所以我觉得明智的人不会呆在这里等着麻烦找上门。”

“我们不能走,克鲁格军士。这里的伤员无法行动,我得留下来照顾他们……”

“神父,我想你还没有听明白我在说什么。那伙混蛋已经是穷途末路了,他们还要来这里要干什么呢?除了报复,我想不出其他理由。所以本该撤离的我们,才特意过来通知你们赶紧去避难。因为一旦等到叛军包围了这个村子,他们才不管你是医生、平民还是神的使者。那些穷凶极恶的混蛋不会留下任何活口的,这里唯一会发生的事情,就是血流成河。无论如何你都救不了这些人,留下来只能徒增不必要的牺牲。”

“……”

几个人再次沉默了,因为克鲁格所说的事情,比阿虎说得要直白得多、也可怕得多。

“黛雅,克鲁格军士所说的很可能是真的。不论如何,照顾伤员的话这里有我一个人就够了。你跟着他们离开吧。”

“主见世人蒙昧,便派祂的圣子降世,宣讲祂的福音。”黛雅轻声说道,“圣子被无知的世人陷害、被钉在十字架上时,背负的乃是全世人的罪——因此被他所救的我们,才一直为此而忏悔、直到今天依然忏悔着。圣子知道自己被人出卖,即将遭受苦难,但他可曾退缩吗。没有。我们如果要宣讲主的福音,就也该要像圣子那般。就算是戴他戴过的荆冠、受他受过的鞭挞,又有何畏惧呢。”

听了黛雅的话,神父没有出声。他闭上眼睛,泪水从他的眼眶无声地滑了下来。

“对不起,黛雅,我很惭愧。我虽为主的仆人,但我对主的信奉,远不如你虔敬。”神父颤声说道,“就算知道了一切都是祂的旨意,但我终究觉得生命才是主所珍爱的,不该轻易舍弃。因此,我才不惜怠慢了对主的悔罪,而将时间用于挽救人的性命……直至此刻我依然觉得如此。所以,就算你当我是违逆了主的教诲,我还是希望你能跟着克鲁格军士撤离此地。”

“笃信主的,皆为主的臣民,将其置之不顾,是为不义。我不畏地狱的硫磺之火,旦畏让我所信的,救我、爱我的主蒙羞。”黛雅决然说道。

“说够了吧。”克鲁格说道,“说够了,就赶紧动身。”

“我不能走,克鲁格军士。虽然叛军来势汹汹,但我也不能这样抛下这里的病人。”神父说,“教堂自古就是避难之地,这是我的所辖的圣所,我怎能自顾逃离呢。并非不信任你,但身为神职人员,我早有献身的觉悟。”

说完,神父转身走进了教堂。

“好吧。在船只行将沉没之时,虽然于事无补,但有些船长会选择和它的船共存亡。我明白。”克鲁格点了点头说,“那么黛雅,你也不走吗?”

“不。”黛雅摇了摇头,轻声说,“感谢你冒着巨大的危险前来通知我们,克鲁格先生。但我是不会改变我的决定的,请你离开吧。愿上帝保佑你的平安。”

克鲁格看了黛雅一阵,没有说话。然后,他转过脸,深深地叹了一口气。

“我的平安从来都不是靠上帝的保佑,而是靠我身边的战友和手里的枪。可你根本不知道自己将要面对的是什么。”克鲁格将脸转向黛雅,直直地看着她的眼睛说道,“向导死了。他在叛军的营地找到了他的儿子,却被他的儿子当做敌人,失手开枪打死了。你自诩不畏地狱之火,那么请问,你见过地狱吗?”

“什么……”克鲁格带来的消息让卓雅无比震惊,但克鲁格并没有给她时间去回味。

“你没有,但我见过。地狱不在末日审判后的往生,它就在人们的心里,人的残暴远超出你的想象。杀光一整个村子的男人、将女人玩弄之后大卸八块、在孩子身上浇上汽油点燃取乐……这些事情我已经看得多到厌倦了,但你大概一次也没见到过吧。嗯,我并不意外。因为只要见过一次,你就会明白,上帝不会保佑这里的任何人——他老人家,早就已经遗弃非洲了!那么就让我来告诉你,那些叛军来之后,会对你这样年轻漂亮的女孩做些什么——首先,将会有几十个男人在大庭广众之下轮奸你,但你不会因此而休克,因为他们会不断地用水冲洗你的头而让你保持知觉。等你身上的每一个孔洞都被他们进出一遍之后,他们会割下你身上某些珍贵的零件作为纪念品,因为毕竟可以肆无忌惮地玩弄一个白人姑娘的机会是可遇不可求的。最后,他们会把你的身体犹如穿刺牲口一样用木桩穿起来,从直肠一直到口腔——然后竖在村子的最中央,制成他们这次狩猎和复仇的图腾。你如果幸运的话,首先会被砍头;如果不幸,你可能会在阳光下暴晒一整天才会死去。当然在你死后或者半死不活的时候,有秃鹫或者鬣狗来撕咬啄食你的身体,自然也是无人问津的。对于这样的前景,你又作何感想呢?我猜你大概连男人的手都没牵过,不知你那具娇贵的躯体,将如何承受这样残暴的屠戮。恕我冒犯,恐怕这要比被钉上十字架、或者绑在柱子上施以火刑,还要糟糕得多吧?”

克鲁格用冰冷的声音低声说着,但他所说的每一个字都清晰可闻。

这番话让科宁斯都起了一身鸡皮疙瘩,黛雅的表情却出乎意料地十分淡然。但不论是科宁斯还是黛雅,两个人都知道克鲁格绝非骇人听闻——他所说的这些事情,那些残暴的叛军是做得出来的,而且他们会很乐意这样做。

“就算是恐惧也不能动摇我的内心,因此你不必恐吓我。”黛雅平静地说道,“肉体的痛苦,也是主对他的仆人的试炼。如果献出身躯能够驱逐人们心中的恶,那我会欣然接受。”

“真的是这样吗。”克鲁格伸出手,一把抓住了黛雅的胳膊,“就是说连这具身躯,也是为了神明而献上的祭品吗。”

“你……要做什么?!”黛雅被克鲁格吓了一跳,声音因为紧张而微微发颤。显然以前从来没有人这样对待过她。

“既然你想要驱逐人性之恶,何不先净化一下内心充满污浊的我呢。”克鲁格一边说着一边扯下了黛雅的修道服,“反正你也已经时日无多了,不妨就先服侍我一下吧。一会儿等我享用够了,会在你脑门上开一枪,让你死得干净利落、也省得受那些苦。放心,我至少也算个文明人,不会太粗暴的。”

“我……不会害怕的……”黛雅的声音颤抖得更加厉害了,她可能就要哭出来了。可即便如此,她也没有向克鲁格示弱。

“很好,就让我来看看你为了自己所谓的信仰,能够献出多少。”

说着,克鲁格的手伸进了黛雅的衣襟,攀上了她的胸前。黛雅闭上眼睛,泪水大滴大滴地从她的脸上滑落,但她没有做出任何反抗,只是任由克鲁格在她的身体上摸索。

“住手,克鲁格!你疯了吗?”再也忍不住的科宁斯大吼了一声,“你在做些什么?!你他妈的难道是魔鬼吗!”

克鲁格看了一眼科宁斯,冷笑了一声。然后,他再次撕扯下了黛雅的衬衣。

黛雅的上身几乎完全赤裸,她只能不知所措地用胳膊遮住胸部,纤细而白皙的双肩不住地颤抖着。

“老克!你这个混蛋……”科宁斯向克鲁格冲了过去,但却被阿虎先一步拦下,并推到了一边。

“这么做是没用的,克鲁格。你该明白。”阿虎走到克鲁格身边说道。

“……”

克鲁格看了阿虎一眼,又看了看虽然正在发抖,却毫无退缩之意的黛雅。

然后,他深深地叹了口气。

“去召集士兵。”克鲁格扭头对阿虎说,“让作战人员在教堂的门前集合。”

阿虎点了点头,转身离开了。克鲁格从地上拾起了黛雅的修道服,抖去了上边的尘土,披在了黛雅的肩膀上。

“能请你来一下吗。”

克鲁格对依然在瑟瑟发抖都黛雅轻声说了一句,然后迈步朝着教堂后面的墓地走去。黛雅迟疑了一下,裹紧了被撕裂的修道服,跟着克鲁格走了过去。科宁斯见状连忙也跟了上去。

“莉莉安牺牲了。她在去往难民营的路上,为了保护向导而中弹负伤,我却没能救下她、甚至没能带回她的遗体。”克鲁格站在荒草丛生的墓地前,背对着黛雅说道,“虽然她也不信奉神明,但我想……请你为她祈祷。”

说着,克鲁格从兜里掏出一串身份铭牌,轻轻放在黛雅的手里。

“就算是不信神明的人,若有你这样虔诚信徒的祈祷,或许也能把她送到神明的身前吧?嗯,说不定能。”

“什么,莉莉安她……?”

黛雅的眼睛瞪大了。她难以置信地看着手里的铭牌,泪水再次汩汩地流了下来,却不再是因为屈辱。

“莉莉安和我,一起经历了许多战场,其中不乏十分残酷而危险的地方。但我们总是设法完成任务并活了下来。”克鲁格看着面前的简陋的墓碑,轻声说道,“有一次在柬埔寨,我们遭到了叛徒的出卖,整支队伍都沦为了敌人的俘虏。敌人无法从我的士兵嘴里挖出有用的消息,就开始用残暴的手段折磨我们——他们把我和莉莉安扔进了集装箱,并命令我奸污莉莉安来为他们取乐。当然,我没有那么做,于是他们就开始屠杀我的士兵来威胁我。我的士兵被用各种各样的手法杀死了,枪杀、棍杀、割喉、用汽车碾死、浇上柴油烧死……但那些士兵都是有种的硬汉,他们没有一个人向敌人求饶。失望的敌人杀光了我的士兵,然后把那些破碎而扭曲的尸体扔到我的面前,并锁上了集装箱任由我和莉莉安等死。可是我们相互安慰着坚持了下来。在被幽闭了五天之后,阿虎带人清剿了敌人的营地、把我和莉莉安救了出来。就是在那时,莉莉安受到了严重的精神创伤,一离开我就会出现不良状况。”

“主啊……仁慈的,主啊……”听到克鲁格描述的恐怖回忆,黛雅震惊得一句话都说不出来,只是在嘴里喃喃地叨念着。

“也许你不相信。但虽然我们不信奉神明,但我也有我的信仰,那就是共产主义……一个遥不可及、但却坚定不移的信仰,我们这些人就是为了那个伟大的信仰而不断战斗着。所以你的心情我是理解的。如果在这里的是我的战友,也许我会选择留下来和他们共同进退。我只是希望,自己能够改变些什么,改变这……不得不眼看着优秀的人,在自己面前失去生命的命运。”

“……”

“当你决定要做一件事的时候,你就必须为你要做的事情付出代价;当你要做一件影响深远的事的时候,你可能就要为之付出巨大的牺牲。我并不畏惧牺牲,只是不想让别人因为我做出的决定而牺牲。我们来到这里的时候一共有十七个人,已经有四个人长眠在这异乡的土地上……现在加上莉莉安,是五个人。就算这样的牺牲立即停止,这代价也早已远远超出了我所能接受的范围。可我却无法停止。救一个人就得牺牲另一个,我总是在不断面对这样的抉择。有时候我会想,我们所做的一切,到底是否值得这样的牺牲?可是没有人能给我答案。就像是现在,我不会知道自己所做的是不是正确的,我能决定的,只有做,或者不做……”

“……克鲁格?”

克鲁格轻声低语着不知所云的东西,仿佛是在对自己说话,让黛雅感到有些困惑。

“黛雅。我也想做一个你喜欢的那种人,可是我却不能。所以非常抱歉。”克鲁格转过身,对着黛雅笑了笑说,“我知道你厌恶我、厌恶我所做的事情。就像记者说的那样,在你的眼里,我一定就是个恶魔吧。”

“不,我……”

黛雅半裸的身上只披着一件残破的修道服,克鲁格的注视让她感到一阵慌乱,不知该如何作答。

“没关系,不必勉强否认,我知道你的想法。”克鲁格的表情恢复了严肃和冷峻,“你觉得我像个恶魔,你没有看错。但之所以会如此……是因为,我总是站在地狱的中心。现在,请祈祷吧,为了莉莉安。如果可以的话……也为了我们其他还活着的人。”

说完,克鲁格朝着教堂的院门走去,把愕然的黛雅和科宁斯留在了身后。

“报告,上士。特别行动小队集结完毕,请下达指示。”

教堂门前,十个整装的士兵有序地列队在前,见克鲁格走来,阿虎上前说道。

“稍息。”克鲁格说,“听着,诸位。‘丛林之虎’行动已经圆满结束,现在我宣布撤离,请各位即刻登上载具赶往14号撤离点。现在就动身。”

但没有人行动。所有人都一动不动地站在那里。

“行动!”克鲁格提高了声调说道,“你们没有听到命令吗?”

“别下达言行不一的命令,克鲁格。” 阿虎说,“很显然你不是那个意思。” 

“我就是那个意思。”

“可是你却不会离开,对吧。”

“是的,我要留下来对抗即将来袭的敌人。所以我希望你们能把武器留给我。”

“以一己之力,去对抗一个排的敌人?”

“这不需要你来操心。”

“克鲁格,你知道这些人为什么站在这里。”

“为了抗命吗?”

“‘岂曰无衣,与子同袍’。”

“……你在说什么鬼话。”

“那是一句来自我家乡的战歌。一起走上战场的士兵会一起离开战场,抛下战友的事情,你做不出来,所以今天没有人会独自撤离。”

“我说过任务已经结束了!”

“‘只要还有一个人留在战场,任务就不算结束’,你说的。”

克鲁格沉默了。他抬起眼睛,用决然的目光依次扫过面前的士兵们,他看到这些人也在用同样的目光注视着他。

“我明白你们的想法,”良久,克鲁格终于开口说道,“你们是我的同志、战友……还有兄弟。可这是我独自做出的决定,是和我们被赋予的任务毫无关系的决定。你们付出的已经够多了,我不能让你们再为了我一个人而流血。”

“不是为了你一个人。你的意愿,也是这些人的意愿。”阿虎笑了笑,“我们能够在此持续长达一年的作战,全都依赖这个村子里村民们提供的补给和教堂的医疗,我们还有同志的忠骨埋在此地。并非只有教徒才懂回报的美德——不要让别人认为,无产者就没有信仰。我们已经打了那么多次仗,谁也不介意再多一次。哪怕这就是最后的斗争,但我们会团结起来、直到明天。我想你明白。”

“‘这是最后的斗争,团结起来到明天’吗,呵。”克鲁格也笑了起来,但他却伸出手擦拭了一下眼角,“那么说,英特耐雄纳尔……”

“没错。就一定要实现。”



\section{尾声}

……“英特耐雄纳尔,就一定要实现”。

克鲁格和“阿虎”,或者这个被模糊了名字的战士之间,至少也曾经有过共同的信仰。但在岁月的变迁中,他们共同的信仰不可避免地渐渐变作了个人的信仰、又变作了不知是否存在的信仰。但他们曾经是生死与共的战友这一点,不会、也不能改变。

那是一场激烈的战斗,以克鲁格为首的十二名士兵围绕教堂展开了防御作战,和疯狂的叛军持续交战了四个小时,击退了他们一波又一波的进攻。克鲁格预计叛军的人数在四十到六十人之间,但直到赶来增援的政府军剿灭了所有叛军,才发现这群武装分子超过了一百人。战斗中克鲁格及其士兵均有负伤,其中有四人伤势较重,但得益于修女和神父的救治,他们全都活了下来。当天夜里,克鲁格的部队离开了非洲,此后的动向无人知晓。

记者科宁斯在战斗打响前对克鲁格进行了简单的口头采访,并经全程见证那场战斗。然而不幸的是,他在战斗中被流弹击中面部、生命垂危,因故未能将访谈内容记录下来。在简单地包扎后,科宁斯被政府军紧急送往最近的医院,虽然保住了性命,但容貌受到了严重损伤。之后,科宁斯回国接受了容貌重塑手术并修养了半年,又再次回到了岗位,并一直活跃在全球各处的战场上,为了让更多人了解战争的真相而不断“战斗”着。

村民也们回到了自己的家中,这个边陲的村落又恢复了以前的安详——也许比以前更好。虽然战争的伤痕需要很长时间才能愈合,但至少再也不会有持枪的叛军骚扰他们了。在村民们的协助下,神父重建了在战斗中受损严重的教堂,修女的讲学班也得以继续开课。隔年的春天,修女黛雅被班吉市的教会召去,在那里会晤了一位“昔日的朋友”,之后她便豁免还俗、并离开了教会,此后去向不明。

值得一提的是,向导的儿子也参加了那次战斗并幸存了下来。作为被胁迫作战的童军,政府免除了对他罪行的追究,并将他交予威利斯神父看管。一年之后,向导的儿子作为战争遗孤和已经移居欧洲的家人相会,此后定居德国,一直从事着控诉战争暴行的代言人的事业。有消息称,这个孩子的赦免以及移民皆源于某些国外势力的运作,但这个消息从未得到证实。



外传:丛林之虎 完



\section{后记}

字数最多的第二部分至此结束。终于搞完了,心力交瘁,不过多评论了。

第三部分虽然依然难逃沉重的气氛,但多数情节是轻松的,主要在陆久和Vector的工作和生活之间展开。喜欢V的朋友有福了,因为第三部分的故事主要是在办公室里卿卿我我,下班后也住在一起,时不时出去玩一圈,尽享美食美景美人……基本都是些秀恩爱的剧情。

嗯,我怎么又开始自言自语了……
