\chapterul{第四部分:最后的外传,以及后记}

\section*{前言}

山川不会相逢,因为它们永远只懂相互遥望。但人不同。只要他们决心向着对方走去,那么,总有一天——

\lineseparator

这就是“背叛者”故事的全部,已经没有什么再多的话可说。2017年2月3日开贴,至今已五年。回顾过往,一刹那恍惚,唯余若有所失的感觉。

外传已发给我留邮箱的朋友,让我先自己给自己完结撒花吧。

最后的文字,在下一页。

\clearpage

这是《背叛者》故事最后的章节。

献给所有喜欢、并且用热情来支持这个故事的人。

\vdots\footnote{由于作者的标注,本部分未编入此文档。若有需要,请自行前往\url{https://www.pixiv.net/novel/show.php?id=17017058}查看。}

\clearpage

\section*{后记}



后记啊……

其实我从一开篇就想着要写点什么后记,例如创作时的灵感、设定,以及对各个角色的访谈录之类的东西,里面也包含了一些陆久和V的私人生活……但真正到完结的时候,却一点都写不出来,因为中间经历的一系列事情,例如可悲的关注度和网络空间的封杀,已经将作者的热情消磨殆尽了。我甚至不太能记清自己到底为什么要给Vector写这样一个篇幅巨大的同人文,她的热度已经可怜到P站上连色图都没有几张。唯一值得欣慰的是终于把这个故事完整地地交代完了,并且和一开始说的那样,用一个温暖的结局给它画上了句点。不论结局好坏,这个故事里的人们都是在命运大潮里不断浮沉着,他们的人生在波澜之中可谓是辛苦而艰难的,作者想要通过这个故事来表达的,就是希望身在平静生活中的我们能够珍惜当下拥有的一切。

关于这个故事,有两处值得一提的事情,其一是关于续作的事情、其二是关于其结局。

“背叛者”的故事原本还有一个写了一半的大篇幅的外传的,外传描述了陆久和V分别的几个月里,两个人各自的经历。其中陆久是和416一起进行了充满浪漫色彩的长途跋涉旅行;而V则和NT77以及05(也就是46号实验体“山茶”,不知道大家还记不记得这号人)相处了一阵,相互交流适应人类社会的心得体会。而续作也曾考虑过,内容想象为归隐田园的陆久,因为受到极端人形权力组织的威胁,而不得不再次走向战争的泥潭……不过从这个故事的受欢迎程度来看,确定是没必要再写下去了,所以那些没收回或者收得不太好的条线,就让它们永远成为这个故事的悬念吧。

这个故事的剧情基本上是按照我一开始的设计进行加工的,唯独在处理帕斯卡部分剧情的时候,花费了不少的精力。我个人是非常喜欢帕斯卡这个角色的,我认为她是这个故事里塑造得最好的人物。如果说V的形象代表了理想的“纯洁之爱”的话,那么帕斯卡就代表了现实的“世俗之爱”。她的人生很复杂,不像V那样无得无失、喜欢一个人可以不顾一切地去爱他。对于帕斯卡来说,人生有许多值得或者是需要去追求的事情,爱情只是其中之一,在不得不权衡取舍的时候,她经常要放弃爱情去选择其他;而她又和每一个女人一样也是爱情动物,在最后时刻幡然醒悟,认识到什么野心阴谋都不如一个她爱的并且也爱过她的男人重要,但当她醒悟的时候却又为时已晚。她是个优秀的人,情商智商都极高,但她的追求更高。许多人都曾经爱过她,比如罗本、46号实验人形……还有我们的陆久先生,可帕斯卡辜负了所有人。利用自己的感情和身体做筹码,去实现那些常人难以实现的目的,对帕斯卡来说是惯用的手段,但正是因为她总是在利用感情,让她对感情抱有不信任感,所以总是无法收获真正的感情。这个故事的第二章的时候,我其实考虑过让这个故事的结局走向帕斯卡的:在南宁和V的对峙中,陆久选择了杀掉V来保护帕斯卡。在这个结局中,陆久和帕斯卡秘密结婚、入职16LAB,并最终成了帕斯卡的私人武装力量的头目,为她进行暗中铲除异己、实现她那庞大野心而活动着。这是个相当有吸引力的结局,但经过不断地思想斗争之后,我还是认为陆久应该选择V,因为人生中多数时候我们都不得不选择世俗之爱,而既然是故事,就应该超越现实。所以,这个故事最后还是走向了既定的方向。

故事的尾声之后,曾经被命运联系在一起的人们,走向了各自的人生:

帕斯卡带着NT-77来到西亚后,和某个人在一间叫“无名芳草”的花店……其实是一个庞大的情报网的终端会了面。花店的老板名字叫“山茶”,她有着许多的身份,包括16LAB的46号实验人形、以及N17战区的临时侦查队长05。但她现在已经是个自由的人了。帕斯卡这次前来的目的是情报,但那只是其中之一。经过这一段时间的辗转之后,她希望能够好好地处理自己的感情事务,因此她坦白且详细地相互介绍了几个人的情况,这让山茶陷入了极端为难的境地。帕斯卡这个历史复杂的姐姐,追求男人失败之后转投自己,之前的恩怨情仇倒不在话下;NT-77这个人,一方面可以说是杀害95的罪魁祸首,但从另外的意义上,她也可以说是95本人,就让山茶不知道该如何去对待她了。在经过一整天的思考之后,她们都决定放下过去的恩怨用新的身份开始相处。因为世界已经是新的气象,如果执着于过去,她们也许是仇人;如果放下过去,那么她们也可以是亲人——她们的仇人各自都有很多,但亲人,却已所剩无几。

离开了战场之后,陆久保留了帕斯卡的联系方式,不仅是出于对过去的怀念,更是因为陆久知道帕斯卡能为他提供一些极具价值的东西。陆久知道,不彻底斩断他和帕斯卡之间的孽缘本身就是一种极大的风险,但他还是希望帕斯卡能够有个好的人生……当然,这次他会仔细考虑需要付出的代价,不会再去做她的棋子。陆久同和安洁也偶有接触,进行一些合约性的合作,因为安洁是个爽快的女人,开出的劳务费总是相当可观。他参与了404小队的数次行动,但是只负责战略和战术方面的策划,不再参与直接作战。

关于皮尔斯,陆久唯一确知的是,他带着自己的人形伴侣逃离了父亲的控制。陆久完全没有皮尔斯的任何情报和联系方式,陆久推测以后也不会再有了。陆久知道皮尔斯是一个心思缜密的人,如果他决意要人间蒸发,那绝对不会留下任何多余的线索。虽然就这样和一个朋友诀别未免会让人有些感伤,但陆久知道,这样的代价对皮尔斯来说是值得的。他们对彼此来说并非必须的,相忘江湖、各自安好,是最好的结局。

而名为陆薇的少女,陆久给她的第一个任务,是了解和融入人类社会。虽然V只想要在陆久的身边,但她也有许多要学习的事情,因为陆久要她做她自己——不是别人给她安排的角色,而是她自己真正想要成为的人。喜爱的东西、喜欢的事情,还有存在的意义,这一切需要自己用心去寻找和发现,无法靠其他人传授。

在新的时代里,陆久已经成了一个符号,象征着反抗对人形的奴役、为了人形平权而斗争的先驱——为了救出自己所爱的人形,不惮和曾经的同袍反目、只身飞蛾扑火般地向着绝境而去,最终消失在辐射区的中心……人们相信陆久死了,因为他生还的可能性实在微乎其微,而且殉情的结局也更符合人们对浪漫而悲壮的英雄主义的想象。陆久的故事经过媒体精心地包装和渲染后,成了人们津津乐道的话题,也成了无数热血少年的榜样,对世界的冲击远大过一颗摧毁城市的脏弹。这个故事虽然是陆久斗争的结束,但却是无数其他人斗争的开始,从此,人形出现在公共场合不需要再佩带身份标识,和人形并肩乃至挽手而行,也成了思想前卫的表现。而这一切给陆久带来的便利就是:在埋藏身份的归隐生活中,当他和V一起走在市井的时候,再也不会有人对他们投以额外的目光,减少了许多不必要的麻烦。陆久对此感到很满意,在经历了无数各怀鬼胎的阴谋、流尽了众多无辜者的鲜血之后,他的愿望最终得以实现。这世界的诸多重大变革都是这样——一个宏大的伟业的实现,最初可能只是源自一个简单而渺小的愿望。但混沌之中,众生芸芸劳碌奔波,又有几人会在意这种事呢?毕竟,每个人都有自己的生活。

至此,这个故事总算是尘埃落定。至于未完成的外传之类的,大约是要永久搁置,因此请不必期待了。

就这样吧。感谢一直支持我的朋友们,江湖再见,如果有缘的话。

\rightline{某年某日\quad 某夜上}
