\documentclass[openany,a4paper]{ctexbook}
% Comment the following line to NOT allow the usage of umlauts
\usepackage[utf8]{inputenc}
\usepackage{geometry,tocloft}
\usepackage{titlesec,titletoc}
\usepackage{enumerate}
\usepackage{hyperref}
\usepackage{ulem}
\usepackage{siunitx}
\usepackage{graphicx}
% Uncomment the following line to allow the usage of graphics (.png, .jpg)
%\usepackage{graphicx}

%\titleformat{\chapter}[block]
\geometry{
left=27mm,
right=27mm,
top=25.4mm, 
bottom=25.4mm,
}

\title{背叛者}
\author{阑夜珊}


\newcommand\specialsectioning{\setcounter{secnumdepth}{-2}}

\newcommand{\lineseparator}{{\noindent} \rule[-1pt]{\linewidth}{0.05em}}

\newcommand{\sectionul}[1]{
\clearpage
\phantomsection
\addcontentsline{toc}{section}{#1}
\section*{#1}
}

\newcommand{\chapterul}[1]{
%\clearpage
%\phantomsection
%\addcontentsline{toc}{chapter}{#1}
%\chapter*{#1}
\setcounter{secnumdepth}{-2}
\chapter{#1}
\setcounter{secnumdepth}{0}
}

\newenvironment{QuoteEnv}[2][]
{\newcommand\Qauthor{#1}\newcommand\Qref{#2}}
{\medskip\begin{flushright}\small ——~\Qauthor\\
\emph{\Qref}\end{flushright}}

% Start the document
\begin{document}
\maketitle
\tableofcontents

% Create a new 1st level heading
\mainmatter
%\addcontentsline{toc}{part}{序章}


\specialsectioning{}
\part{序章}

\setcounter{secnumdepth}{0}
“知道吗,我可不是在江河日下的时候才找到你的。”

宽大的办公桌后面,年迈的老人身陷在豪华的办座椅里,嘶哑的声音里带着一丝疲惫。

“我将你捞出来的时候,整个公司的事业正如日中天。我本来希望有了你的辅佐,我更能如虎添翼……这本该是顺水推舟的事情,其实就算没有你,我一样能够实现自己的理想。比起从你那里得到什么,我更在意的是能够有你来见证,见证我们曾经一同为之奋斗的一切——真的,我甚至曾经幻想过那一刻的情景……可你,却让我如此失望。”

站在他对面的年轻男人一言不发,对他的话似乎完全无动于衷。

男人的背后是四个荷枪实弹的士兵,他们的枪口全部对准着男人的后心。他们只要扣下扳机,手中的火力足以把男人打成两截,但他们没有一个人敢这样做。

原因有很多:一是因为没有得到命令、二是因为担心流弹误伤、三是因为那个男人手里也有一把枪——那是一把已经磨得掉漆的老式自动手枪,正直直地指着他面前的老人。

哪把枪会更快,士兵们并没有十足把握。

“大势已去却依然不肯接受既定结局的人,临终总是喜欢哀嚎‘为什么’。但是想不到,这竟然也成了此刻我唯一想问的问题。”见男人没有说话,老人继续说道,“是我带身负重伤的你撤出战场、是把被判处终身监禁的你救出死牢,也是我给了你这一身的功名利禄。就算你不知恩图报,至少我不该遭到你如此的背叛……不,无论是曾经还是现在,我都相信你不会是这样的人。告诉我,这到底是为了什么?”

“事已至此,无论答案为何,都不能改变任何事情。”男人简单地说道,“我们今天这样相对,并非因为个人感情,而是因为只有你的死,才能终结这错误的一切。所以我只能请你舍生取义。”

“‘舍生取义’?呵……我难道不是一直都在这样做吗。金钱、地位、荣誉,这些东西只要假以时日我都能拥有,但我追求的难道是这些吗?几十年的经营和奋斗,是为了实现那个伟大的梦想——我曾经以为,那也应该是你的梦想……而你却将我所建立的一切都毁于一旦,只是为了一堆军用设备。是我看错你了吗?”

“军用设备”。听到这个词,男人笑了笑。

“曾经我们一同出入战场,你救过我很多次;而如今我现在拥有的一切,也都是你给的。你说得没错,于情于理,我都该对你感恩戴德……但就算背负叛徒的罪名,我要做的事情依然不会改变。如果你一定想要知道为什么的话,我想要告诉你,就如同我们曾经发誓要守护的那些人一样:人们并不是因为受到我们的认可才被称之为‘人’,而是因为他们认为自己是人、是因为他们要求被像人一样对待,是因为他们……生而为人。”

“呵呵呵呵……”老人发出一阵阵嘶嘶的干笑,“好一个‘生而为人’。这么说,你是要把那些工厂里的造物都叫做‘人’,并且决意要为它们争取一席之地了?这就是你背叛我的理由吗,真让人痛心。我在做什么、想什么,你根本不曾去了解过。这些年我所付出和经历的,远非你所能想象,那绝不仅仅是人力和财力……可以说,这个公司倾注了我的毕生心血和全部生命。但现,在只是因为自己的一厢情愿,就要武断地把这一切彻底推翻?你一定觉得自己就是正义,对吗?”

“克鲁格……坦白说,没有人比我更不想这么做。可是我们已经走得太远,远到我们都已经错过了拨乱反正的机会。”仿佛被定罪的被告在做最后陈述一般,男人的声音里透出一丝疲惫。但他手里的枪却握得更紧了。

“呵,是啊。是走得太远了……”老人冷笑一声,“如果一开始就能好好考虑的话,那么也许很多事都不会发生。但就算一切都已经无法纠正,我们依然有自己的解决问题的办法……你和我这样的人,不就是为了这个目的才被训练出来的吗?”

说着,老人的脚尖轻轻踩下了一个按钮。砰然一声,办公桌的前面冒出一列防弹玻璃将老人包围了起来,他所坐着的位置地板向两侧滑开,老人随着座椅慢慢向着地板之下沉了下去。

“永别了,曾经的同志。你还不如在那时就带着光荣死在战场上,也总好过像今天这样可悲的死法——真是可耻。”随着一阵嘲讽的话语,老人彻底消失了。见势不妙的男人立即侧身向着办公桌旁边的窗户翻滚而去,却为时已晚,他身后的士兵已经扣下了扳机。

空旷的办公室里,响起一阵杂乱的枪声。
\part{新世界}

\include{1.1}
\include{1.2}
\include{1.3}
\include{1.4}
\include{1.5}
\include{1.a1}
\chapter{附录一:罪犯H的日志}
\paragraph*{文件0105:}

DM行动纪要:关于军官H屠杀、抗命、危险行为以及推定背叛罪行的定罪依据(副本)

\lineseparator

\paragraph*{说明:}

以下是军官H在执行“DM”行动时的纪要日志,由军官军官H记录,军官F提供。

注:此文档为0105文件副本,因保密需要文件中人名以及时间、地点均略去或作简写。其真实信息参见文件正本及附件。

\lineseparator

\begin{enumerate}
    \item **月**日(0)
    
今日抵达Q市。落地后由阔别一年多的军官K接待,老友久别重逢,心内十分高兴。

从K处获悉,此次行动主旨为清除Q市滩头敌军所遗留之爆炸物,称之为DM行动,第一阶段为期约3-4个月。

总部配备的排爆人员为没有或者缺乏任务经验的新人,称之为“学员”。我负责训练以及监督其行动之任务。

抵达指挥站后,认识此次行动负责人之一,军官F。在F的带领下,检阅了我手下的第一期学员,共计16人。

学员均为战败国之战俘,以少年居多,其中有军士2人。
    \item **月**日(0)
    
多数学员的排爆经验较为匮乏,正在由F组织进行简单训练。训练完毕后,我将带领学员负责Q-B13海滩排爆工作。

B13海滩为沙质海滩,海洋环境温和,曾为Q市度假地。海滩的爆炸物有较为详细的图纸和数字资料,适合作为第一阶段任务的适应性训练。

根据F所述,当第一阶段任务结束时,存活学员将被允许作为投降人员释放回国。但我的任务未作知会,似乎还将继续。所幸战争已经结束,纵然尚需在此地执行任务,但归营只是时间问题。

今日在排爆训练中发生意外事故,有学员一人死亡。
    \item **月**日(1)
    
今日风和日丽,海滩天气良好。15名学员均已到位,营地位于海滨度假村的一所别墅内,为征用之民房。这里曾有别墅群存在,但在战争中已悉数被毁,现仅存一座楼房,有六间房间、一座仓库。

别墅的主人A先生在卫国战斗中牺牲,现有A太太及其女儿B小姐(6岁)管理。

我住在楼房一楼,A太太及女儿住在二楼,学员被安置于仓库中。

略加操练后,排爆工作终于起步。如进展顺利,约4个月内可结束第一阶段,届时应是初秋。

因学员多有恐惧心理,第一日的行动进度缓慢。相信几日后速度将会有所提高。
    \item **月**日(3)
    
今日是行动展开第三日,指挥站送来补给若干、以及一条排爆犬用于协助搜索爆炸物以及看管学员。排爆犬性格活泼、惹人喜爱,我将它命名为M犬。

但指挥站送来的补给中,仅有本人的口粮,没有学员的口粮。考虑到学员已经三日未进食,我将所有口粮置于大锅中煮成汤水让学员分饮。但因为食物有限,学员饮下后体力恢复不多。

于是今日休息时间较前二日早一小时。
    \item **月**日(7)
    
今日是行动展开第七日,指挥站依然没有送来学员的口粮,询其原因,答复曰战俘之补给不是首要事项。

这几日都在向A太太借粮用以勉强维持学员行动,如这样的情况持续,恐怕三日内学员将无法行动。另外因为我用粮食饲育敌人士兵,A太太对我态度多有抗拒。

我已就此事向军官K求助,希望一、二日内得到解决。

学员之间已经出现身体不适的症状,推断是过度饥饿所致。

行动开展一周后学员的经验都已经比较熟练,但是因为身体状况,效率未能显著提高。
    \item **月**日(9)
    
学员内爆发群体疾病。

学员A向我坦白,他们因为饥饿难耐偷吃了A女士给禽类准备的饲料,因而生病。

饲料主要材料为麦糠,纵然导致腹泻,却不该引起发热症状,因此我到禽舍实地调查。发现大量鼠粪,推断学员因吃下含微生物之鼠粪而致病。

我命学员饮下海水后扣喉咙呕吐洗胃,用清洁的水冲洗全身、暴晒衣物。然后今日休息一天。

指挥站的补给仍未送达,因此我驱车前往,取回食物和药品若干。

回到营地后将食物及药品让学员服下,学员的精神明显好转。

我指派学员A为本批学员的负责人,负责组织和监督学员的日常生活。
    \item **月**日(11)
    
今日多数学员已基本正常,年轻的躯体恢复速度很快。但仍有三人十分虚弱。

其中一人学员B向我请假一小时,未予准许。另有学员A似乎为请假者的朋友,表示愿承担前者的工作,请求我准假一小时,未予准许。

下午,之前的请假者学员B因为神志不清误触地雷引信,爆炸导致双臂损毁,伤势严重。因营地设施简陋无法救治,只能简单包扎后送往指挥站的伤兵营。

我本对受伤学员存活的可能性不报希望,但诸多学员一致央求,我不得已而施救。

对我而言,这些学员本身就是用以消耗的耗材,对指挥站而言恐怕更是如此。
    \item **月**日(12)
    
今日是行动展开第十二日,指挥站终于将学员的口粮送达。这也是托K施加压力的结果。

虽然只有粗粮且分量不多,但是至少能够保证学员有体力去完成排爆工作。

目前我们面临的主要是反步兵地雷,排爆方式为在沙滩上使用铁钎匍匐探索,找到地雷后徒手挖出并拆除引信。这一方式虽不及金属探测器高效,但在资源有限的情况下,只能用此等简易排雷法。

所幸学员学习很快,此刻排爆已经接近计划中速度,且没有更多人员伤亡出现。

目前沙滩上已经清理出一片安全区域,相信第一阶段最终结束时间,不会比预计中推迟很多。
    \item **月**日(20)
    
今日是行动展开第二十日,总体工作约完成五分之一。沙滩上的安全区域面积已经颇为可观,因此拟划出一片区域供学员操练运动。

我去往指挥站汇报行动进度,会见了K及军官F,军官F对我的行动进度表示赞赏,K为我多加美言。

在指挥站我欲顺便探视之前受伤的学员B,获悉学员B已于几日前死于伤口感染。此事并不出我所料,但为了不影响学员们的士气,我谎称学员B正在康复。

学员A似乎对我带来的消息并不信任。于是我在四下无人时,对学员A通报了学员B的死讯。他对此事表示理解,但事后在我不注意的地方哭泣了很久。
    \item **月**日(29)
    
今日天气晴朗,海滩温度适宜外出放风。因为一个月来行动进展速度让人满意,我决定给学员放假一日,以休息放松身心。

学员多为少年,在沙滩上组织足球运动玩耍、以及跑步竞赛,心情十分欢愉。我认定适当的休息和娱乐对于提高行动效率是有益的。

下午指挥站又送来两人,以补充学员数量。这些学员之间相处得十分融洽,此时我感到管理这支队伍的难度并不太高。这让我积累了宝贵的经验。

现在战争业已结束,或许我们不该一直将这些少年当做敌人对待。
    \item **月**日(30)
    
今日是黑色的一日。

我带领M犬于滩头巡视时,在安全区域触发了地雷。因为躲避及时,我未受重伤,但M犬却在爆炸中丧生。

我询问学员的时候,他们说已经仔细检查过所有爆炸物,和指挥站上提供的材料中数量一致。这说明指挥站所提供的材料数字也许并不准确——又或者,这些学员存在混淆视听、甚至消极怠工的行为。无论原因是哪一个,我都知道,此刻的海滩安全区已经并不安全。

我想我对于他们的管理过于松懈了,在朝夕相处的过程中我渐渐淡忘了他们是敌人这一事实、忘记了这些地雷正是他们和他们的战友亲手埋下的。负责这一区域的学员被我亲手处决。

我要求他们并排手挽手走过他们所清理过的所有区域,以保证这片区域的彻底安全。一名军士表示抗议,我亦当场将他处决。其他人不敢再有怨言。

这些学员走过他们清理过的区域后,没有再次出现爆炸。我想他们今后会对自己的任务更加认真。
    \item **月**日(40)
    
学员A向我汇报,他发明了使用铁框分割沙滩的方法让工作更加严密、保证万无一失。我对这个方法并不感兴趣,但还是同意推广。

我更中意的是用他们的双脚去探查他们所排查过的区域,这样就算为了他们自己负责,也不敢对自己的工作有所懈怠。

这些少年只是些孩子,也许他们对于自己在战争中犯下的罪过不该全权负责。我知道他们和我一样,只是遵守命令的军人。但我无法确定自己到底该不该相信他们,以及能信任他们到何种程度。

毕竟他们和那些我在战场上杀死的、以及杀死我的战友的人们,是同一群人。我不会再心怀友善地去对待他们。
    \item **月**日(50)

这些天的行动进展平稳,距离预期目标只剩下一半了。我向指挥部汇报了工作情况,指挥部对我的工作表示赞赏。

K表示十分期待工作结束后能再和我并肩作战,但我觉得可能不会有这样的机会了。一个原因是这里的战争已经结束,我们的战功足够我们晋升到中层军官,不必再亲临战场;另一原因是因为这些天的工作让我感到十分疲惫,甚至比我所经历的一线作战更糟。我甚至考虑过能够在适当的时候解甲归田。

但是我知道K是那种向往战场的人,他的心中充满了战斗的荣耀,肯定不会想象退役之类的事情。所以我并没有说出我的想法。
    \item **月**日(56)

今天发生了意外的事情。

B小姐,也就是A女士的女儿,独自外出玩耍的时候误入雷区。当A女士发现自己的女儿坐在隔离线之外和洋娃娃玩耍的时候,她吓得魂飞魄散——虽然那个小女孩离隔离线只有二十米距离。

而此时的我正在指挥站听取关于下一阶段行动的简报。

当我回到营地的时候发现学员宿舍里只有一个人,躺在床上面色抑郁地望着仓库的顶棚,他就是学员A。

我问他其他学员哪里去了,是不是逃走了。他不说话。

过了好一阵子,他才说,在滩头。

我赶到滩头的时候,看见那些学员们都在那里——正在用他们每天都在练习的技能,试图在隔离线和那个小女孩之间清理出一条安全通道。他们已经排出了两颗地雷。

虽然他们都知道只要耐心等待,最终每个人都会是安全的,但那个孩子似乎已经失去耐心了。她不断站起来又坐下,只是因为她母亲的苦苦哀求。我认为她已经不会在原地呆太久了。

这时候学员A站了出来。他不顾危险地走到了小女孩身边,开始和她说话,安抚她的情绪。

十五分钟后,安全通道建成了。一个学员过去把那个孩子抱了回来,但学员A却没有过来。他直直地走向了海边——那边是一大片没有经过排查的雷区。

他无视我们的呼喊漠然地走着,终于在走出二十多米后,被一颗地雷炸成了碎片。
    \item **月**日(57)

所有的学员都被学员A的死震撼了,低落的情绪在学员之间蔓延。也许每个人都像学员A一样,丧失了活下去的希望,只是没有表现出来。学员A的自杀一幕让他们积累在心头的绝望开始爆发了。

我只好让他们放假一天,并对他们进行动员和鼓励。

我终于意识到,对他们来说,战争还没有结束——他们依然都在每天面对着死亡。可是在这场还在继续的战争中,我又扮演着怎样的角色呢。

我曾经以为他们只是一群用于消耗的耗材,但是我发现我错了。也许对于我来说的确如此,但对于他们来说,我是他们初入战场时带队的中士。

我想起了自己还是新兵的那些日子。如果中士已经不顾手下新兵的死活,那这些士兵该是如何的绝望呢。我对自己的行为感到可耻。

无论如何,我们的工作已经进行大半,这是唯一给我安慰的事情。军官F曾经承诺工作结束后这些人就可以回家,我认为我可以完成这项工作。

我想起那时我的带队中士,也就是如今的军官K说过的话:如果你不能带领你的士兵走向胜利,那么至少带领他们回家。不该有更多人在这里死去了。
    \item **月**日(70)

这两周里又有一名学员因为事故而死去,但我一直在鼓励他们、他们也在鼓励着自己。

这不是任何人的过错,既然战争还在继续,那么出现伤亡总是难免的。

这是一群优秀的军人,虽然他们是如此的年轻,虽然他们曾经是我的敌人。这群孩子纵然在夜里因为想家和害怕而哭泣,但是在面对白天的战场的时候,也从不退却。

第一阶段的行动已经接近尾声,回家的愿望正在驱使他们努力前进。我希望他们每一个人都能顺利踏上归途。
    \item **月**日(82)

今天事故再次发生了,有三个学员在装卸已经拆除引信的地雷时牺牲。

那堆本来应该安全的东西因为不明的原因发生了爆炸,三百多颗地雷把方圆百米范围内的一切都炸成了灰,而我们永远无法知道到底是他妈的为什么。

他们已经离回家只有一步之遥了,我昨天还在听他们谈论回家后要做些什么。

真他妈的混蛋。
    \item **月**日(85)

所有的爆炸物都已经排查干净了。今天是我这三个月以来最为轻松的一天。

我请求A女士为我的学员们做一些可口的饭菜,她欣然接受了。我用自己的军饷支付给她报酬,她不肯收,但我还是把钱塞到了她手里。

我在饭桌上为他们祝贺,那些孩子们喜极而泣,让我也不由得心里为之触动。

明天我就去向指挥站汇报,学员们已经开始准备自己的行李了。
    \item **月**日(86)

今天我到指挥站汇报工作,顺便了解了一下第二阶段的行动。

下一阶段的工作重点是N39地区,依然是滩头。不过这次可没有B13海滩那么简单了。那里部署的都是些构造复杂的地雷,而且海滩十分泥泞,也没有爆炸物的准确数字。

另外,我的学员们将不被允许回家——他们作为有经验的排爆人员,将继续他们的任务,一直到Q市海滩上的爆炸物全部排除,或者全部阵亡。

真是讽刺。我其实一开始就该想到这一点的,活下来的都是经验丰富的士兵,战斗还未结束怎么可能让他们退役。

指挥站的人们,他们从一开始就根本没有打算让这些孩子活着离开。他们不仅欺骗了那些战俘,还欺骗了自己人,因为最真挚的表演就是要自己也深信不疑——这种事我早就了然于心,却不肯相信会发生在自己身上。
    \item **月**日(87)

我虽然反对指挥部的做法,但是和F的交涉没有任何结果。他说这一切都是上边的意思,无论是我、是他还是K,都没有选择的余地。我知道他说的是真的。

K也劝我顺其自然,不要过多考虑那些于己无关的人,但我不能不考虑。纵然他们曾经是我的敌人,但现在已经是我的战友。我从不抛弃战友。

违抗上级命令显然是错误的,可是如果这命令本身也是错误的,又该如何抉择呢?
\end{enumerate}
\lineseparator
\paragraph*{附注:}

以上为军官H所做的行动纪要全文。

在最后一次纪要的一天后,H本该奉命将剩余战俘送往N39地区,但是他没有照做,而是私自释放了那几个战俘。

前去接收战俘的官员,在B13滩头误触地雷身亡。根据**检举,此事军官H嫌疑重大,军官H因此事被送往军事法庭。

\lineseparator
\paragraph*{附:}

检方起诉事项、被告定罪以及其他人员情况:
\begin{itemize}
    \item 放任战俘逃亡是既定事实,因此军官H的抗命罪名成立。
    \item 军官H曾枪杀两名战俘,屠杀罪名成立。
    \item 因对于交接友军的死亡原因不作供述,军官H的背叛罪名推定成立,但因证据不足暂不定罪。
    \item 军官H在与军官F的争论中行为过激,被认为有危险行为罪。
    \item 逃亡的战俘下落不明,推测已经离境,不再进一步追查。
    \item 军官H的好友军官K,确认对此事并不知情,和H的行为没有关系。
    \item 军官F存在失职,但事发后积极检举,作降职处理以期后效。
\end{itemize}

\lineseparator
文件记载结束。
\include{1.b2}
\include{1.b3}
\include{2.1}
\include{2.2}
\include{2.3}
\include{2.4}
\include{2.5}
\include{2.6}
\include{2.7}
\include{2.8}
\include{2.9}
\include{2.10}
\include{2.11}
\include{2.a1}
\chapter{\sout{雪线}}

\footnote{原文链接为\url{https://www.pixiv.net/novel/show.php?id=15609997},已于\date{2022-05-31}被作者删除。}



\include{2.a3}
\include{2.a4_1}
\include{2.a4_2}
\include{2.a4_3}
\include{2.a4_4}
\include{3.1}
\chapter{昨夜的星辰(二)}
\section*{前言}
让陆久惊讶的是,本以为不会再见的V竟然再次出现了,而且和陆久在同一个办公室。陆久不知这是不是好事,因为他和V在一起的时光总是充满了蹉跎。但他认为这也并非坏事,首先因为V是个漂亮姑娘;其次是因为,V是这个世界上陆久为数不多的心有好感的人之一。

然而陆久也隐约感到,V的出现绝非偶然。有人在刻意制造这种“巧合”但其中的用意,陆久依然无从揣测。

\lineseparator

\section*{}

办公室里一下子陷入了尴尬的沉默,这让陆久感到很不自在。他揉了揉额头,默默翻看着新来的女孩的调令:

“信检中心:兹介绍本部人形7709a2到你部任职,请安排接洽。格里芬劳动资源部

人形编号:AS7709a2-‘Vector’

科(部)室:安全保密警卫部-信件函件检视中心

职务:科室职员

岗位职责自报到日起生效。”

陆久又瞥了自己对面的女孩一眼。从总部派来的人形,如果和之前的某个人形看起来很像,也是有可能的。但他跟前的这个一定不是仿制品——7709a2是V的编号,陆久记得很清楚。是她,绝对错不了。

“雷蒙。老谢这是怎么了?”陆久问道。

“报告陆主任,老谢……谢振同志不太喜欢和人形打交道,但是具体原因我也不知道。不过以前他的反应从来没有这么激烈过。”

“‘主任’就免了。”陆久疲惫地摆了摆手,“那你呢,难道也不喜欢人形?”

“我……”雷蒙迟疑了一下说道,“我倒没什么。”

“不方便说吗。”

“不,我是真的无所谓,没什么特别的感受。”

陆久点了点头,没再问下去。他能理解雷蒙的难处:自己和谢振的想法明显有些分歧,但站在雷蒙的角度去看,一边是老前辈一边是顶头上司,这队可真不好站。

“我之前没注意自己的职务。不过就像你说的,这个部门有没有负责人又有什么区别呢。所以请你也不要介意。”陆久说着站了起来,“劳驾你对她交代一下工作的事情,我出去抽根烟。”

陆久拿了一根烟点上,然后走出办公室站在门口抽了起来。他倒不是因为有女士在跟前所以才出去抽烟的,以前在战区的时候他在办公室里可没少抽过。他是因为感觉心里有点乱,短短一小会儿的时间发生太多的事情了。

首先,是他竟然成了部门的负责人;其次他完全不知道谢振为什么对人形那么反感。但这都不是最重要的。

最重要的是,V竟然也来到了这个地方。这到底是怎么回事?要说没有人在里边刻意安排,陆久是绝对不信的,但是谁、又是出于什么目的这样做的呢。

陆久忽然心里一动,从口袋了掏出了手机。

“是你的主意吗”

他编辑了一条短信发给了皮尔斯,因为他忽然想起了上次和皮尔斯之间的通话。

“这么说来,人已经到了”

陆久很快就收到了回复。

“为什么”

“自己体会”

陆久有些困惑。自己体会?他自己应该知道这个问题的答案吗。

他……

对V的出现,是怎么想的呢。刚才见到V的时候,他是怎样的心情呢。

陆久回忆了一下片刻前的情景。他首先感到的是震惊、让人不知所措的惊讶。但惊讶之余,还有一点欣慰,还有一丝莫名的……高兴。

正是如此。他应该感谢皮尔斯,因为他很高兴能和V再次重逢,无论是何种原因。

陆久丢掉抽完的烟头,转身回到了办公室。他看到V依然坐在她的桌前,而雷蒙则抱着胳膊站在一旁,表情有些苦恼。

“LUWEI小姐的职务,是我们正在空缺的书记员。但她却不懂汉字的输入法,这可有些难办。”雷蒙对陆久说,“没有录入经验的人,想要把这些信件内容编辑成文本恐怕会很吃力。”

“没事。”陆久说,“我会帮她的,她可以把信读出来然后我来输入,反正我们有时间。”

陆久想起来第一次和V见面的时候,她使用的语言是英语。不过当她作为陆久的副官出现在战区的时候,她已经能够使用中文交流了,显然是特意进行了学习。但她的汉字书写还是在战区工作时闲暇时间(在陆久的指点下)自己学的。

至于如何用键盘输入中文,想必是没有人教过她……

不,等一等。陆久突然感到自己错过了什么非常重要的信息。

刚才雷蒙说她叫什么来着??

“LUWEI……?”

陆久用疑问的眼光看了V一眼,却得到了一个若无其事的目光作为回应。陆久只得无奈地皱了皱眉。

她在调令上的称谓可不是这样,这是她自己起的名字吗。肯定错不了,因为不会再有别人去为她想个这样的名字了。不过这算是什么?

算了吧,陆久心想,向别人报上怎样的称谓是她的自由。他管天管地,也管不着别人叫什么。

陆久那里堆积了一些之前审查过的信件,正好这些信成了V的工作内容。但因为V不知道如何用字母拼出汉字,所以只好像之前说的那样由她将信件读给陆久然后再由陆久来输入文本。一上午的时间很快过去了,因为V对汉字的辨识也不算熟练,所以录入的进度缓慢,总共也没完成几封信件。

中午的工作餐依然是雷蒙从食堂里带回来的,几个人用餐过后,V很自然地把陆久的餐具拿去清理了,但却没有问雷蒙需不需要帮忙。看着走出门外的V,雷蒙露出了一个明白了点什么的笑容。

“陆哥说自己已经习惯了孤独,但我看您其实并不孤独呢。”趁着V不在,雷蒙打趣地对陆久说道。

“那个人考虑问题总是比较简单,请不要介意。”陆久有些尴尬地说道。但说完他马上就后悔了。

因为这么说的话,岂不是显得他很了解V了吗。但雷蒙只是再次笑了笑没说什么。

下午的工作依然如故,但一直到下班谢振也没有回来,看来请假之后他是不打算再来打卡了。

五点三十分,雷蒙准时从座位上站了起来。

“我先走了各位。明天见。”说完,他礼貌地朝着陆久和V点了点头,然后离开了办公室。

“我会学习汉语拼音的。”雷蒙离开后,V对陆久说道,“争取早日赶上你们的进度,不会用太长时间。”

“那倒没什么,这里的工作不是那种紧急的事项。”陆久说,“下班了,回去吧。”

“好的。”V点了点头,站起来朝门口走去。

“薇。”正当V即将走出门外的时候,陆久忽然开口叫住了她。

“有事吗。”V停下脚步转身看着陆久说道。

“啊……没什么。”陆久说。

V站在那里默默注视着陆久,有一会儿没有说话。

“如果有需要我做的事请,就请告诉我……如果您依然愿意相信我的话。”片刻后,V开口轻声说道。

“没事。只是觉得该和你打个招呼而已。”陆久摸了摸下巴,“我没想到……还能再次见到你。”

“我也是。”V说,“我也以为再不会见到您了。”

听到V的话,陆久笑了笑,因为这次他们的想法难得地一致了。

他和V已经有多久没有这样轻松地谈话了呢,陆久几乎已经想不起来了。上次是在战区指挥部的某个风和日丽的上午吗。

此后,似乎他们的每一次相见,都是一次颠沛流离的开始。

“刚才雷蒙说的那个,是你的名字?”陆久说。

“是的。”

“为什么要起这样一个名字呢。”

“因为……我也想有一个人类的名字。”

陆久看着V,他看到她的眼神平静淡然。

回答这个问题的时候,V稍微迟疑了一下,也许是对陆久的问题感到稍微有些不好意思,但她的眼睛里没有一丝的掩饰。

她可以让别人用她喜欢的名字称呼她,毫无疑问人人都有这样的权利。陆久喜欢这个回答,虽然V似乎误解了他的问题。

“你当然可以决定别人怎么称呼你。”陆久说,“不过我的意思是,LUWEI是怎么写的……你是怎么想出这个名字来的?”

“这不是我想的名字。这是您给我的名字。”

V说着,在纸上写下了两个字:

陆、薇。

陆久默默地看着那两个字。

“薇”是陆久对她的称呼,是那次在南美战场上,陆久为了记录下他们的情况而写下来的称呼。

虽然后来证明陆久是误读了她的名字,但陆久一直在非官方的报告中用这个字来代表V。陆久不知道V是何时知道这个字的,但是在北部战区的时候陆久的驻军日志里多次记录过有关V的情况,作为副官的V如果看过陆久的日志也是理所当然的。

那么,这个“陆”字,显然是来自自己这个杜撰的姓氏。

陆久感觉有点好笑。这两个字都是源自陆久的手笔不假,不过组合成这样的名字显然是她自作主张吧。

“叫薇还不够吗。”陆久说,“为什么是‘陆薇’呢。”

“汉语里没有一个字的名字吧。”

“那你可以叫‘薇薇”什么的……”

“我叫陆薇。”V加重语气强调了那两个字,看起来她是真的很在意自己的名字。陆久轻轻叹了口气。

“是和我有关吗。”

“……是的。”

“别人如果不知道,至少你该知道。‘陆久’这个名字只是我随口编出来的。”

“那您本来的名字又是什么呢。”

“我……不知道。”

“那次在海边,我问到该如何称呼您的时候,您让我称呼您的名字‘陆久’。”V说道,“您还记得吗。”

陆久感到一阵恍然,他想起自己曾经做过一个梦——

那是一个清晨,他和V徒步走在被海风扫得一尘不染的滨海马路上。V一再地轻声呼唤着他的名字,一直到他感到稍微有些不耐烦。

“为何要一再地喊我呢。”陆久问。

“只是想试试,用这个名字称呼你的话,你会有怎样的反应。”V回答。

只是这样的理由啊,陆久感到哭笑不得。不过,那时的不久前,他的确说过——

“那么,我该怎么称呼您呢。”她问。

“直接称呼名字吧,”他回答说,“叫我‘陆久’就好。”

……是真实发生过的事情吗。原来那不是梦。

“嗯。我记得。”陆久说。

“所以对我来说您是陆久。无论您是谁,对我来说您就是陆久、只是陆久。”

——他是陆久?那“陆久”到底又是什么呢,这个名字该如何定义呢。

以前陆久从来没想过这个问题。不过现在他的心里稍微有了一点概念:“陆久”,就是V印象中的某个男人。

“好,我知道了。”陆久适时地结束了这个话题,因为他感觉再说下去就要和以前一样,开始向着让人压抑的方向发展了。他不想再去思考我是谁、从哪里来、要到哪里去这种哲学命题,而且他也不想每次和V谈话的时候都陷入沉闷的气氛之中。

“你叫我陆久、或者什么其他你喜欢的称呼都行,但不要再说‘您’这种恭敬的称谓了。太拘束的话让我感觉也有点不自在。”陆久说,“还有,要一起去吃点东西吗。”

“好的。”

\section*{}

陆久收拾了桌子、锁好办公室的门,和V一起走出了公司的地下室。离开办公室前,他伸手摘下了V脖子上的颈环。

“陆……主任,”V提醒陆久说道,“根据民用人形管理条例,在社会活动中人形必须佩带……”

“陆主任是你的负责人,这点责任还是能负的。”陆久打断了她说道,“忘了那时在海边,我是怎么介绍自己的吗?”

“……我知道了。”

两个人走出了公司大门,一起朝着市区走去,两个人并肩而行,一时没有再交谈。

一边走,陆久一边偷眼打量了一下自己身边的女孩:在这个严冬季节里,她的打扮依然很合身,牛仔裤勾勒出了她腿部的完美线条、灰色短羽绒服给人一种端庄的气质,而深红色的围巾更是增添了一分娇美。陆久身上那身白色衬衫和黑色西裤跟之前的作训服几乎没有区别,只不过裤子的材质从尼龙变成了毛料……这套衣服对于一个三十岁的男人来说倒是能够适合所有场合,但站在一个窈窕美丽的女孩跟前就显得有些老土了。所幸他身上的大衣遮住了大部分的细节。

明天去买一件别的颜色的衬衫吧,陆久心想。

“你的头发短了呢。”为了打破这奇怪的沉默,陆久说道。他记得在北镇的时候V是扎着马尾的,但现在她又恢复了以前经典的短碎发型。

“头发长了不方便战斗,所以就弄短了。”V说,“你喜欢长发吗。”

“不,只是随便问问。”陆久连忙说道。

“嗯。”

郝丽安说得没错,自己果然是社会性缺失的人格啊,陆久无奈地想着。就连V这样熟悉的人,自己也找不到什么可以进行下去的话题。

两个人再次来到了陆久之前一直吃饭的饭馆。因为每次都点同样的餐点,饭馆里跑堂的小姑娘已经认识陆久了。

“您好,先生。”见陆久走进餐厅,那个孩子赶紧打招呼说,“今天比以前要早呢。还是和以前一样吗。”

“是的。”陆久说,“哦,不,请给我看一下菜单。”

“今天您竟然不是一个人来的,这位是您的朋友吗。” 那个姑娘递上了菜单,看着V笑了笑说。

“是公司的同事。”陆久说,“也是朋友。”

“她真好看。是外国人吗。”

“啊……是的。”陆久这才想起V没有佩带人形的颈环,被这个单纯的姑娘当成人类了。

“我是……”

“看看有什么想吃的吗。”

V正要解释,却被陆久递上的菜单打断了。她抬头看了看陆久,看到陆久微微摇了摇头示意她不要多说。

“和你一样就好。”V领会了陆久的意思,没有看菜单简洁地说道。

“面条,两份。”

在延续的沉默中,两个人各自解决了自己的晚餐。走出饭馆,陆久站在街头心里再次感到了茫然。

这个……他心想。

然后该去哪呢。

和以前那样像个幽灵一般在街市上游荡两个小时然后回去睡觉吗。那倒不是不行,不过V就在他旁边呢,他不想让V认为他这一个多月里就是过着这样乏味不堪的生活。

就是过着这样乏味的生活又怎样啊,陆久稍微有点恼火地想到。当然,他是因为自己的乏味而恼火。

他忽然有点怀念在上海的时候,虽然现在看来他一直都是被牵着鼻子走,但有帕斯卡的话至少他现在不会如此不知所措。

……帕斯卡,陆久心想。想起了一个不该想起来的人呢,呵。

“我们回去吧。”他说。

“这家饭店的菜,你喜欢吗。”V说。

“好。嗯?”陆久好像听到了一句他没想过会听到的话。

“还可以吧……”陆久不明所以地说,“我对吃的东西,没什么特殊的偏好。”

“那边有家味道不错的店,明天我带你去吧。”V看着远方的街市说道。

“你以前,来过这里?”陆久大感意外地说。

“这个办公地点,曾经是备勤人形的训练中心。”V说,“那时候在没有勤务的时间,我们就被安置在这里,所以对这边还算熟悉。”

“是吗,那真是太好了。”陆久笑了,“我是第一次来这座城市,对这里几乎一无所知。”

“所以你这阵子都在这家餐馆吃晚饭吗。”

“……你怎么知道的?”

“要不是这样,这里的店员怎么会认识你呢。你一直都不是个引人注目的人。”

“呵,这种评价还真是让人尴尬。不过你说的确实没错。”

陆久无奈地耸了耸肩。竟然被V这样的人形明示作不起眼,实在是太丢人了。不过陆久知道自己就是这样一个无人问津的人,只有在别人眼前模式化地反复出现,才会让人有些印象。

“好了,回去吧。”V说。不知道是不是自己的错觉,陆久感觉V好像不易察觉地笑了笑。

“对了,你住在哪?”走进公司的大门,陆久问道。他忽然意识到离开了办公室,他就无法联系到V了,因为V肯定是没有手机的。

“公司的宿舍。”V说。

“能不能告诉我是在哪个房间,如果有事的话,我也好知道去哪找你。”

“好的。”V想了想说道,“我带你去。”

V带着陆久朝着公司后边的楼房走去,他们前进的方向让陆久有些奇怪。他之前看到有公司的员工推着载物车从那座楼房里进出,一直以为那里是仓库。

“那边,难道不是仓库吗。”陆久说。

“是仓库。”V回答。

“你的宿舍怎么和仓库在一起?”

“人形当然是在仓库里储藏了。有什么不对吗。”V说。

“噢……这我倒不知道。”

这样的回答让陆久感到意外,他还以为V的宿舍应该是和他一样在办公楼的客房区。

陆久跟随V走进仓库,来到了其中一间房间,V推开了门,陆久发现房间的门上甚至没有锁。

房间的大小和陆久的宿舍差不多,但里面几乎可以说是空的——

那间屋子里甚至就连衣柜都没有。房间里只有一根钉在墙上的铁丝,看起来是用作挂衣物的。然后就是几个木质弹药箱拼起来的长台子,上边铺着一层硬纸板,似乎是被当做床铺来使用。除此之外,就只有墙皮脱落的墙壁了。

房间里阴暗而冰冷,就连取暖的设备都没有。陆久简直无法想象该如何在这样的房间里过冬。

“这就是你的宿舍?”陆久说。

“是的。”

“就算最偏远的战区,人形的宿舍都要比这个强。”陆久严肃地说道,“我刚到战区的时候虽然连房屋都没有,但至少还有帐篷和睡袋。而这里为什么就连铺盖都没有?”

“战术人形是公司的重要武装力量,为了保证她们的战斗力,最低的后勤标准也会为她们配备睡袋。”V淡淡地说道,“但我现在是储备人形,不能为公司带来任何收益,所以是没有任何配给的。”

怎么会这样,陆久震惊地想着。他还以为战区的条件已经够艰苦了,但没想到在这和平的地区人形的待遇竟然更加恶劣。在格里芬公司,这些人形到底算是什么?

“那就搬到我的宿舍来吧。”陆久不假思索地说。

话音未落,陆久马上就意识到自己说了蠢话。他和V现在的关系只是同一个部门的同事,客观地说V也是一个年轻的女孩子,他作为一个男人怎么能提出让她住在自己的房间这种有悖道德的建议呢。

但话已出口,要收回也来不及了。V看着陆久,眼睛微微睁大了,显然陆久的建议让她也感到了吃惊。

一定会被拒绝吧,陆久心想。自己在胡说些什么呀。

但V却没有拒绝。她低头想了想,然后说:“可以吗。”

“咳,可以。我的房间有两张床,只有我一个人住……没有别的意思。”

陆久假装清了清嗓子,整理了一下自己的语言才没有绊住舌头。虽然他有些顾虑,但V却完全没有去想什么合适不合适的问题。她果然还是和以前一样,陆久心想。

“好。”V说着拿起“床”上被她当做枕头的行李包,“那我们走吧。”

于是两个人离开了仓库,来到了陆久的宿舍。

“这里的条件果然比仓库好多了。比战区的宿舍还好。”V环顾了一下房间说道,似乎对陆久的房间很满意。

“虽然是一般的客房标准,不过至少起居用品都有。”陆久说,“现在还少一套铺盖,明天再去买来好了。今晚你就盖我的被子,我盖这件大衣就行。”

“不用。”V说着把两张床推倒了一起,“这样就足够睡两个人了。”

陆久看了一眼,没有做声。这样的确可以睡两个人,不过被子还是只有一条,这是要两个人同盖一条被子吗。

“我……还是盖衣服吧。”陆久说道。

“怎么了。”V不解地说道,“那时在北镇的时候,不就是这么睡的吗。”

“那时是……”

真是哪壶不开提哪壶,陆久皱起眉头心想。他之所以说有两张床,就是为了避免回忆起那段不堪回首的时光。

“对不起。”看到陆久的表情,V垂下了目光说道,“我不该说那些让你想起不愉快的回忆的事情。”

不愉快的回忆吗,陆久心想。的确不假,那时候确实是自己情绪最低落的时候。但那不是V的责任。相反是V一直在他身边陪伴和鼓励他,而他却……

“不,该说对不起的是我。”陆久说,“我是对自己的所作所为感到可耻。如果有人在那段时间里受到伤害的话,那都是我的错,没有你的任何责任。”

“我也没有尽到自己的职责。所以……”

“不必再说了。”陆久坚定地说道,“请不要再说了。那些事情,如果可以的话就让它过去吧。如果不可以,只有我一个人负责就够了。”

“……好。”V轻声说,“知道了,我不会再提了。”

“呼,早点休息吧。” 陆久长出了口气说,“今天发生了这么多事情,让人感觉有点累。”

“好的。”V说着稍稍犹豫了一下,“那个,浴室……我能用吗。”

“随便用。”

陆久把大衣扔在写字台前的座椅上、脱下衬衫,然后躺在了床上盖上了被子。他听到浴室里传来了流水的声音,然后过了一会儿,水声停了。

被子被掀了起来,有人躺在了他的身旁。陆久闻到一股熟悉的气息,一丝略带温暖和潮湿的芬芳。

陆久感到后背发烫,许多回忆涌上了他的脑海,让他驱之不及。

“陆久。”

正当陆久忙着努力赶走那些几乎让他窒息的回忆的时候,他忽然听到一个轻柔的声音呼唤了他。

“唔?”

“能够再次见到你,我很高兴。”

“……我也是。”

片刻之后,陆久听到背后传来到了均匀的呼吸声,V很快睡着了。

\section*{}

“凡耶和华所亲爱的,必教他安然入梦”——虽然陆久不信牛鬼蛇神那一套,但至少他相信V的心中是问心无愧的。但他就不同了。第二天醒来的时候陆久感到非常疲倦,一是因为他一夜没有脱衣、二是因为他根本睡不着,而且因为担心惊醒V,他一夜都保持着侧卧的姿势没有翻身。

醒来之后陆久看到V已经穿戴整齐地等着自己了。还是昨天那身装扮,但没有穿羽绒服外套,所以陆久得以看到V里边穿的是白色衬衫套着灰色的V字领的羊毛衫。这男女皆宜的职业化装扮可谓相当中性,但V那窈窕的身姿让陆久知道,他面前的毫无疑问是一个漂亮的姑娘。

睁眼就看到美人在前可谓一种难得的福利,但浑身的酸痛让陆久的心里只有一个念头,那就是今天无论如何也要买一条新被子。陆久忽然注意到V的胸前挂着一条细细的金属链,上边吊着一个精致的坠饰。那是一个小小的海螺切割而成的薄片。

“你还带着那个呢。”陆久看着那个吊坠说道。那不是在北镇那时候,自己为了打发那个纠缠他们的老妇人买的……

“嗯。”V低头看了一眼自己的胸前,赶紧把那个吊坠塞进了羊毛衫里面,脸上显出了一丝羞涩。

那是什么奇怪的表情,陆久一边迷迷糊糊地想着一边向洗手间走去。

洗漱完毕后陆久走出洗手间,他看见V的脖子上又戴上了那个蓝色颈环。考虑到公司里的人形都有佩带这样的标识,为了避免搞特殊的嫌疑,陆久没有再让V取下来。

“几点了?”陆久说。

“八点。”V回答。

“……看来没时间吃早餐了。”

“我不饿。”

陆久整理好衣衫,然后看了看手腕上的计时器。他算准了时间,和V一起走出宿舍,来到办公室时是准确的八点三十分。

办公室里只有雷蒙一个人,但经过安检的包裹在谢振的桌子上堆了一大堆,看来老谢今天也是很早就来工作了。

“早上好,陆主任。”雷蒙先对陆久打了个招呼,然后又对V说道,“早上好,陆小姐。”

“别叫主任。”陆久摆了摆手。

“我叫陆薇。”V正色说道。

“啊,好、好的。”一大早就被两个人纠正称谓,让雷蒙有些不知所措。

“老谢呢,又是早早就干完一整天的工作然后出去了?”陆久说。

“是啊,这次我连人都没见到,只看见了包裹。”雷蒙耸了耸肩。

昨天不是还说天冷了不想出去了吗,陆久无奈地想着。果然是对自己有意见吧,人和人之间的关系真是难以把握。

算了,不耽误工作就好。其他的只有慢慢来了。

“我们该干什么干什么吧。”陆久说,“薇……咳。陆薇,你看看这些信件,对照汉语拼音录入试试。有不明白的可以问我……或者问雷蒙。”

“是。”V点了点头。

陆久莫名地有点烦躁,因为他感觉那个他本来叫惯了的名字,此时忽然显得无比的暧昧。他和V之间不同寻常的关系,无论如何都很难掩饰。

陆久、陆薇,怎么看都让人感觉有不一般的关系吧,陆久无奈地想着。这家伙到底是怎么想的,非要叫这么个名字?

“陆久。”正当陆久暗自苦恼的时候,忽然听到V轻声说道。

“怎么了。”

“这些信件里的内容,我不明白。”

“我看看。”

陆久伸手接过了V递来的信,飞快地扫了一遍。和之前的没什么不同,不外乎还是写些日常琐事的家书。

“写的都是些琐碎无聊的事情,哪里不明白呢。是有不认识的字吗。”

“不认识的字我可以查。但里边说的这些事情我不理解,所以阅读起来不太顺利。”

你不懂,那很正常,陆久心想。你有没有家人朋友,怎么会懂这些家长里短的日常牢骚呢。

不过陆久马上就开始自嘲起来。自己不是和V一样吗,有什么资格笑话她。

“先按这个录进去吧,有问题下班后我再给你解释。现在说这些会打扰别人工作。”

“好的。”

“没事。”雷蒙闻言接口说道,“您尽管解释,我听不见。”

滑头的小子,陆久心想。这家伙也不像第一次见面时,他感觉的那么诚恳啊。不过机灵一点倒不是什么坏事,不必像自己这样愚钝的人一样到处遭人嘲笑。

三个人低头开始工作,雷蒙熟练地封装着被拆开的信件和包裹,并为那些邮件贴上印有“已检视”的标签;而陆久则默默地审阅着一封又一封的信件。V的工作明显不怎么顺利,因为过了一阵子陆久悄悄看了她一眼,发现自己抽屉里的信已经一小摞了,而V还在翻腾着那几页信纸。

本来是作战单位,干这种工作也是难为她了,陆久有点同情地想着。对V来说,解读这些不知所云的信件,大概和让自己去揣摩那些难以度测的人心一样难办吧。

午餐依然是食堂清单的工作餐,陆久依然是飞快地扫荡了午餐然后由V收拾了餐具。雷蒙则很知趣地自己丢掉了装着剩饭剩菜的餐盒,并且什么都没说。用餐过后,陆久忽然站起了身:

“我去买点东西,你们自便。”说着他朝着外面走去。V闻言想要跟上去,却被陆久用目光制止了。

“很快就回来。”陆久轻声对她说。

她显然是还习惯于之前他们相处的那种模式,陆久心想。总是紧跟在自己身后,生怕自己有什么闪失。但是时候改变一下了。

这里是和平的地区,离战场已经非常遥远,不必再为什么潜在的危险而担心。他们也许应该有一点自己的个人空间。

陆久来到了市区,首先想到的就是置办一套铺盖。但他却发现自己不知道去哪买这一类东西。陆久在街上徘徊了好一阵也没看到相关的商店,只好停了下来。

该向谁打听呢,陆久苦恼地心想。他认识的人只有皮尔斯,但是为了这种事向他求助的话,不知道要遭到怎样的嘲笑。

绝对不能让那个口舌恶毒的人抓住这样的把柄,陆久心想。忽然他灵机一动,朝着一家饭馆走去,那正是他之前每天都来惠顾的地方。

“您好,先生。”见陆久走进来,服务员小姑娘诧异地说道,“今天比昨天更早了呢。要吃点什么吗?”

“不。”陆久有点不好意思地说,“我想问一下,被褥之类的东西……附近哪里有卖的?”

“……啊?”

陆久的问题显然让那个孩子吃了一惊,但马上她就笑了起来。

“您真有意思。”她笑着说道,“附近可没有卖那些东西的。这一块的租金可贵了,开一家卖被子的店可要赔干家底吧,谁会整天买被子呢?”

“那么我该去哪买呢。” 陆久窘迫地说道。

“我给你问问。”说着女孩朝着柜台喊道:“妈妈!这位先生想买被子,你知道哪有卖的吗?”

“买被子?”柜台后面的女人不可思议地说道,“哪有卖现成的被子的,那要去棉花店做才行。”

“那么哪里有棉花店?”

“附近没有这种店。我倒是知道一家,不过离这里可远了。”女人想了想说道,“你要多大的?我家老头子出去买菜了,我让他帮你去棉花店做一床吧。回来了你再给我钱。”

这个提议倒真是太贴心了,陆久心想,可是他会知道那种东西的尺寸吗?

“那么,一般被子的尺寸是……”

“对不起,对不起。”女人噗嗤地笑了起来,“是我不该问,男人家怎么会知道这些事情。是你一个人盖的吗?我估摸着让他做一条吧。”

“是的。啊,不……”陆久挠着头说,“如果可以,最好是能……稍微大点。”

“知道了。”女人看着陆久抿嘴一笑,“包在我身上,一个人能盖、两个人也能盖。”

说着女人拿起了电话:“哎,你到哪了?……去棉花店里做床被子去,就咱家那么大的。……没人结婚!你打听那么多干嘛,让你做你就去做。”

“还有褥子。”陆久赶忙说道。

“还有褥子!双人床的,快点啊。晚上拿回来。”

说完,女人放下了电话对陆久说:“成了。晚上过来拿吧。”

“谢谢。”陆久感觉像是抓着某根救命稻草爬上了岸,“不过我晚上可能有点事情来不了,请问能帮我送到我单位吗?我会付运费的。”

“你在哪上班呢?太远就不行了。”

“格里芬安保。”

“你是安保公司的啊。没问题,他回来的时候路过那里,顺便给你放门岗上。不过你们那边管得挺严的,你最好先跟门卫打个招呼,要不我怕门卫不收。”

“没问题。我先把钱给您,大概得多少?”

“我也不知道。不过肯定没多少,下次来了给吧!”

“这怎么好意思。”陆久忙从兜里掏出几张钞票,“这些够吗?”

“用不了这么多——”女人看见那些钞票一愣,然后说道,“绝对用不了。两张也用不了。”

陆久闻言,抽出两张钞票放在了柜台上。

“那就麻烦你们了。”

“不麻烦,顺路的事。”女人笑着说道,“多出来的你下次来了找给你。”

“不用找我,就当是略表谢意吧。”陆久说,“就靠我自己可干不了这事,真帮了我大忙了。”

走出小饭馆,陆久的心里有点感动。虽然只是没见过几次的陌生人,但这些人非常友善热情。

然后他要去干什么来着,陆久思索了一阵。好像也没别的事情了——

对了,他得去买部手机。他自己已经有手机了,但V没有,联系起来太不方便了。

但指望V去买手机显然不可能,因为据陆久所知,那家伙比自己要更加自闭。而且也不知道她有钱吗。

格里芬的人形,有劳动报酬吗?陆久忽然有点好奇。据他所知,战区的军费列支清单上是没有人形的军饷的,那么公司有没有给她们费用呢?

陆久忽然想起了V昨天说过的话:“不能为公司创造效益,所以没有任何配给”。就连被褥都没有,工资更是不可能吧。人形这一物品对公司来说是公司的资产,所以她们很可能没有任何个人财产。

这样的念头,让陆久心里忽然有些阴郁。这甚至不能算是剥削,因为她们连作为民用人形的权益都没有。

陆久走进之前的手机店,同一位人形服务员再次热情地接待了他。陆久思考了一下女孩子该用什么样的手机比较合适,但他很快就发现对于手机他只知道一款机型。于是他又买了一部和他手里那个完全一样的手机。

离开手机店,陆久顺便在街上的提款机上取了一些现金,然后返回了公司。经过公司大门的时候,陆久对门口的人形警卫交代了一下下午会有人给他送被褥的事情,那位警卫女孩同意帮他收下。

\section*{}

回到办公室,陆久看见雷蒙又在漫不经心地摆弄手机,而V则端正地坐在桌前一个字一个字地录入信件。

“还顺利吗。”陆久站在V的旁边问道。

“我觉得熟练一些了。”V说,眼睛依然没有离开屏幕。

“慢慢来。”

“嗯。”

陆久回到了自己的位置,然后坐了下来。雷蒙已经开始继续工作、V也一直没停,但他却没什么事可做了。

他感到这个部门的工作分工明显有些不合理。他和谢振两个人通常是干得飞快,特别是谢振,就连人都看不到都能完成一天的工作,而V则处于刚刚上手阶段,雷蒙面对他和谢振一同交来的函件有时候也会有点应接不暇。这种串联式的工作模式中,有一个环节跟不上就会影响整个工作链,像是V那里录不完的信件就无法让雷蒙发出,进而导致陆久渐渐变得闲置起来了。

但陆久想了想,还是决定不去打乱现有的工作模式,因为郝丽安把他派到这里可不是为了优化工作效率。事实上,她正是要他闲下来才让他去了这么一个部门。

但她希望自己把空闲下来的时间,用在什么地方呢,陆久思忖着。他把手伸进了自己的裤兜。

那里有一个U盘,无论何时他一直都带在身上,却从来没有打开过。那是一位他曾经认识的人形的临别赠礼。

不能说是战友、甚至不能说是朋友,只能说是曾经相识,陆久心想。但作为对她微不足道的恩惠的报答,她给自己留下了这个。

“通向往昔之门的钥匙”,她是这么说的。那么……

陆久在兜里摸索了一阵,最终还是没有把那个U盘拿出来。他知道自己随时都可以打开它,但他缺乏的是打开的勇气。

因为就如皮尔斯曾经所说的那样,一旦打开了那扇通往往昔的门,他就不再是现在的这个他了。这让他感到不安。

也许,他甚至将不再是“陆久”。

那段未知的过去,对自己来说到底有多重要呢,陆久自问。这个U盘里装的一定是克鲁格曾经认识的那个人,但却不是他所认识的人,更不会是V认识的人。

“你就是陆久”。回想起V所说的话,陆久悄悄地看了她一眼。他看到V依然在认真地敲打着键盘,完全没有注意到对面的人在想什么做什么。

没有必要,陆久心想。这些东西等需要的时候再去了解不迟。他是谁,不该由他的历史决定,而是该由他本人决定、由这个此时此刻的他自己决定。V说她的名字是“陆薇”,她很在意自己的名字(虽然这个名字给陆久带来了不少麻烦),因为那是她决定要成为的人。那么身为人类的自己,难道不能决定自己是谁吗?

他可以。他有这样的权力,那是身为人类的基本权力、是降生于这个世界上的人天赋的人权。他会自己决定要成为怎样的人。

陆久把那个U盘从兜里拿了出来,又从抽屉里拿出一个信封,把U盘装了进去。然后他从桌子上拿来了胶水,涂在信封口仔细沾好,把信封扔进了抽屉的最里面。

做完这些,陆久忽然感到如释重负般地轻松。他不由得从兜里掏出一根烟,然后去门口抽了起来。

一下午转眼过去,下班的时间已经临近了,但这次陆久不指望谢振会再次出现提醒大家该回家了。果然,五点三十分已过,没有任何人从办公室的门口走进来。雷蒙看了看表,时间已经过了下班的点,于是站了起来。

“那我就先撤了,各位明天见。”他说。

“好。”陆久点了点头,“我马上也……”

咔哒。

正当陆久要向雷蒙告别的时候,忽然办公室的门开了,一个人走了进来。

陆久一看,来人正是谢振。

陆久看着谢振,谢振也看着陆久,两个人对视了片刻。

“啊。这个,我……”谢振有点不好意思地说道。

“该下班了。”陆久迅速接口说道。

“是该下班了。但我不是要说这个。”谢振说。

“那是?”

“那个,陆主任。晚上有时间……?”谢振顾左右而言他地说着,顺便对雷蒙使了个眼色。

“对,晚上有时间吗?”心领神会的雷蒙马上说道,“一起吃个饭吧,难得大家都在。”

“呃,晚上我得……”

陆久犹豫着说道,偷眼看了V一眼。V正在专心地敲着键盘,仿佛身边这三个人和她完全没有关系。

陆久晚上当然有的是时间,不过问题是V怎么办。谢振说过他不喜欢人形,但是把V丢下自己跑去和雷蒙老谢一起吃饭,这种事对陆久来说是不行的。抛开他和V的私人关系不谈,就算是办公室的普通同事,他也要站在彼此都平等的位置去处理关系。

“还有这位……姑娘,”看到陆久的眼神,谢振说道,“也请……我是说如果方便,希望也能赏脸一坐。”

V听到谢振和她说话,眼睛这才离开了显示器。她看了看谢振、又看了看雷蒙,见两个人都在期待着她的答复,于是她对陆久说:“我怎么都行,你决定吧。”

于是两个人的眼睛又望向了陆久。

“好的。”陆久说,“不过得稍等会儿,我有些东西放在门岗了让我先拿回去。你们要去哪?我一会儿自己过去吧。”

“我准备好车了,一起走吧。”谢振说。

“那得劳驾你们等我一小会儿。”陆久说。

“没问题。”谢振点了点头,“我先去热热车,小雷,你跟着陆主任,要是主任的东西多你也好帮着拿一下。”

“我就——”雷蒙欲言又止地说道,看了一眼谢振又看了一眼陆久。

“不用。没多少东西,一只手就能拿走。你们去车上等我就行。”陆久说。

“行,听您指示。”雷蒙说着和谢振一起走了出去。

“我去帮你把东西拿过来吧,在门岗的警卫室吗。”雷蒙和谢振离开后,V对陆久说道。

“我自己去。”陆久摆摆手说,“你跟他们一起去车上等着。拿着这个。”

说着,陆久从兜里掏出一摞钞票和一部手机,递到了V的跟前。

“我不需要钱。”V说。

“拿着。这东西用途很广,不一定什么时候用得着,带着点以备不时之需。”

“好的。”V没再说什么,接下了陆久给她的钱和手机。

“手机带好,方便联系。记得充电……你会用手机吗。”

“我会。”V的嘴角微微撇了撇。

是看错了吗,陆久有些怀疑地心想。她刚才是不是露出了某种不屑一顾的表情?

“这是和你的同一型号的吗?”V的手指在手机上熟练地拨弄着问道。

“啊……是啊,怎么了。”

“没什么。”V小声说,“Android3700的系统,这机型快比你还老了。”

陆久的眉头拧到了一起,他好像听到了一句相当专业的测评。另外“比你还老”是什么意思?

这家伙有点不太对劲,陆久心想,从昨天开始就不太对。好像学会挖苦人了?她以前可不是这样的。

——以前不是这样的吗?

陆久忽然感觉这是一个值得深思的问题,不过他这会儿没工夫去深思。

“总之,去车上等我。”

“嗯。”

V走了出去,陆久也收拾收拾桌子朝门岗走去。当他走到门岗的时候,警卫亭的警卫显然看到他了,但那个人形女孩只是向屋里比划了一下。

很快屋里就有人迎着走了出来,陆久仔细一看是个年轻的男人。

“陆主任,您好您好。”男人对陆久热情地打着招呼,“我是警卫处的队长,叫邵敬勤。我一直等您呢。”

“哦。邵队长你好。”陆久对那个男人点了点头。

“哎,什么队长不队长的,您叫我小邵就行了,别客气。”警卫队长说,“您是来拿您的行李的吧,我已经给您送到宿舍去了。”

“啊,这怎么好意思。”陆久有点受宠若惊地说道,“也没多少东西,怎么能麻烦你呢。真是太过意不去了。不过你怎么认识我的?”

“哎呀,您这话说的。”警卫队长笑着说,“我怎么能不知道您呢。信检中心和警卫处都是安保部下面的科室,您也算是我们的领导啊。”

陆久想起来自己的介绍信上确实写的是“安全保密警卫部-信件函件检视中心”,这么说警卫处也算属于同一个部门管理的。不过要说自己是他们的领导可是太牵强了,警卫处和信检中心,显然管的不是同一样东西。

“我可领导不了你们。”陆久笑了笑说,“我那个部门算上我才四个人,整天干点没人管没人问的事情,你们的岗位可要重要多了。”

“您说得太见外了。您可是总部派来的,放在哪个部门也是钦差啊。不仅郝丽安女士特意叮嘱我们多给您行方便,就连皮尔斯准将都打听过您的消息,哪天要是总部那边想起我们来了,还得多靠您美言几句呢。”

听到这话,陆久心里笑了起来。什么领导,原来自己是跟着升天的鸡犬,这位年轻人对自己这么恭维不过是因为觉得自己和郝丽安、皮尔斯能说上话。

据说我还是克鲁格的战友呢,陆久心想。不过他老人家现在可正看我不顺眼,要是你知道了,恐怕马上就得对我敬而远之了吧。

“邵队长年轻有为,人事上要有提携,你肯定当仁不让。”陆久说。

“哎,陆主任过奖过奖。”听到陆久的称赞,警卫队长更高兴了,“不知今天陆主任有没有安排,要是没有,晚上我请您一起吃顿便饭……”

“我们部门的同事今天刚刚凑齐,我正要和他们一起坐坐熟悉一下。”陆久说,“改天有时间吧,到时候我请你。”

“好的,没问题。来日方长。”警卫队长说,“啊不不不,我怎么能让陆主任请我呢。我请,我请您。”

“一样、一样。对了,我那几件行李是给我放宿舍门口了吗。”

“哎呀,宿舍门口人来人往,的给您弄脏了多不好。我请客房部打开门,给您放屋里了。”

“那真是太贴心了。唔?”陆久忽然意识到了有点不对,“你……进屋了?”

因为陆久突然想起来,自己的屋子里的两张床被某个人推到了一起,如果有人看见两张床上只有一床被褥,说不定会怎么想。

听到陆久的问题,警卫队长一愣。然后,他马上笑了起来。

“没、没。我放门口就出去了,没往里走。”

“哦。没什么。我随便问问。”

“没事没事,我们公司的安防和保密是一流的,您放心好了。”

“是,这我倒不担心。”

这个小伙子非常精明,两句话就知道别人在想什么了,陆久心想。拜某位公关高手女士所授,陆久也有了一点看人的道行,能够看明白一些人的脾气秉性了。

不过看这意思,这小伙是认定自己在金屋藏娇了,陆久无奈地想着。算了,管他作甚。

告别了殷勤的邵队长,陆久走出公司大门,见一辆黑色轿车正停在马路对面。陆久一出去汽车的玻璃就落了下去,雷蒙探出头对他摆了摆手。于是陆久快步走过去坐在了汽车里。




\include{3.3}
\include{3.4}
\include{3.5}
\chapter{昨夜的星辰(六)}

\section*{前言}

房子车子和马子果然是幸福男人必备的三大件,虽然房子是租来的、车子是二手的,而马子则是来历不明的。

陆久和V之间也有过一些真正意义上愉快相处的时光。这样的时光是如此的美好,以至于就连陆久都产生了这泡沫般的幸福,有可能会长久存在的幻觉。

\lineseparator
\section*{}

第二天早上,陆久很早就醒了,他洗漱之后做的第一件事就是把自己的铺盖捆扎好。然后,他一改平时提前赶到办公室的习惯,坐在光秃秃的床板上无言地看着计时器,一直到八点二十五分才离开房间。

到达信检中心的时间是准确的八点三十分,陆久站在办公室的门口,听到里面有交谈的声音,显然已经有人先到办公室了。他深吸了一口气,然后轻轻推开了房间的门。

所有人都已经在自己的岗位上了——谢振正在检查面前的一大堆包裹,看来他没有像往常那样提前做完自己的工作然后离开,而雷蒙也正在忙着处理堆积起来的信件。而V,则端正地坐在自己的办公桌前,似乎正等着他的到来。

“呵,今天陆主任竟然是最后一个到的。真难得啊。”

看到陆久走进来,谢振开玩笑地说道。

“更难得的是在八点半之后依然能够在这里看到你。”陆久笑着回敬道。

“别那么说,平时我也没耽误工作啊。”谢振挠了挠头,尴尬地笑了笑。

“今天我来的时候办公室里只有陆薇小姐一个人,这才是最难得的。”雷蒙说,“平时你们不都是一起来上班的吗?”

“我早上有点事耽误了一会儿。”陆久说。他发现自己也学会轻描淡写地撒谎,来掩饰自己不想说的事情了。而V则一直坐在自己的位置上,什么都没有说。

直到中午陆久才知道,收发处一早就送来了大量的信件和包裹,是V第一个到办公室后接收了那些函件。也许是春节将近,信检中心收到的函件忽然多了起来,这让几个人难得地忙碌了许多。陆久上午审了大约上百封的信件,在桌子上堆了厚厚一摞,就连午餐时间到了都没有注意。简单地用餐之后,他又协助V把那堆信件都录制成了文本,一天的工作下来让他的眼睛感到有些酸痛。

这一天几个人一直在满负荷地工作,谢振也十分罕见地一整天都没有离开办公室。下午到了下班的时间,谢振和雷蒙一边活动着有些僵硬的肩膀一边和陆久道别然后离开了公司,陆久也把脸埋进了双手来休息一下盯着信件看了一天的眼睛。直到这时,V才走了过来。

“眼睛不舒服吗。”她说,“要不明天我把信件念给你听?”

“别了。”陆久说,“虽然是审查,但毕竟是私人信件,里面的内容还是别让太多人知道比较好。”

因为这些天V的进步很快,陆久也已经不再给V读信了,而是一边审查一边顺便帮她把信件进行录入。

“好的。”V说,“那晚上是在食堂吃饭吗。”

对了,陆久忽然想起来点事情。今天本来是准备把行李搬去刚租下的公寓的,而且V似乎也该透露一下她去买了点什么了,但一天紧张的工作让陆久把这些事暂时忘掉了。

“先回房间吧。我也有点事情要和你说。”

回到房间,V看到屋里已经整齐打包的铺盖和行李,眼神里露出了微微吃惊的神色。

“我们两个人住在这间屋子里不太方便,所以我在外面租了房子。”陆久看了一眼依然挂在窗户上的V的内衣说道,“趁现在天还没黑,早些把东西搬过去把。走过去得走一阵子呢。”

听到陆久的话,V显然更吃惊了。她沉默了片刻没说话。

“是我在这里打扰到你了吗。”过了一会儿,V说道,“要不,我还是回仓库去住吧。”

“不行,那种地方怎么能住人。”陆久皱起了眉头,“连供暖都没有,北方的冬天可是冷得滴水成冰。不要再回那里了。”

“可是……”

“怎么了?”陆久奇怪地问道。他感觉V的语气里有些为难。

“我不想……离开公司到外面住。”

“为什么?”

“因为……”V犹豫着说道,“算了,没什么。那就走吧。”

陆久不解地看着去拿行李的V,没有说话。过了一阵,他忽然明白V为什么会面露难色了。

她该不会以为自己是要把她撵出去,让她自己到公寓去住了吧,陆久好笑地想着。想法简直像个纯真的小女孩一样。不过这副担忧的表情,倒是让陆久莫名地有点喜欢。

“我也会一起搬过去的。”陆久指了指自己的铺盖,“公寓的房间要大得多,两个人住方便一些。你以为我是让你自己去住?”

“啊。我……”

听到陆久的话,V眉宇间的担忧不见了,但脸却忽然红了起来。

“当然,我只是为了有什么情况的话方便处理。”陆久说,“如果你一个人也没问题,我倒是在哪都行。你说呢,要不我就不去了?”

“不,你要不去的话,我也不去。”V坚定地说道,“你去哪我就去哪。”

陆久笑了,他表示了解地点了点头。这样的回答当然在意料之中,因为无论是在战区还是在北镇,V都一直在紧紧追随着他——他知道这个世界上有很多虚假的东西,但他面前的这个女孩,一定是真的。

“那就走吧,晚饭在那边解决好了。”陆久说,“对了,昨天你去哪买东西了,买了整整一天?”

“对不起,我忘了告诉你了。”V说,“走吧,我带你去看。放在仓库前边了。”

\section*{}

两个人提着行李来到之前V“居住”的仓库前,但陆久什么都没看到。其实他不是没看到,那件东西很显眼,但陆久没有注意到,因为他没想到V会去买这样的东西。

“这是……”

“前天晚上我在网上订购的。”V解释说,“我找到了两个合适的选择,你说不要太显眼,所以我就选择了相对低调的一个。”

陆久没有说话,因为他相当吃惊。用他惊讶之余勉强整理好的思绪来描述他所看到的东西的话,那是一辆汽车——

一辆高大的越野车,白色的陆地巡洋舰。

那辆车应该是纯白色,但因为出厂时间有点长了,所以颜色看起来是经过日晒的乳白色。但漆面很光亮,没有一丝刮蹭过的痕迹。

“4.2升V6涡轮增压发动机、八档手自一体双离合变速箱,最高功率460匹马力。因为是全时四驱,油耗可能有点高,不过越野性能应该不错。而且有针对多种路况的电子动力分配系统。”V用相当专业的词汇解说道,“车龄已经十年,但是因为车主开得很少,所以行程只有三万多公里,可以说还很新。我昨天驾驶的时候各种系统运转都很稳定,没有什么毛病。”

“你昨天,就是去买这个了?”

“是的。因为过户手续有点麻烦,所以耽误了整整一天的时间,不过还好都办完了。总是步行也不太方便,我想早点取回来的话就能早点用上,所以就连夜开回来了。”

陆久轻轻抚摸着汽车的引擎盖,没有说话。虽然汽车的外观很普通,但那高大的车身、宽阔的轮胎和方正的大灯,正无声地透出一股力量感。这辆车不仅性能完全符合陆久的要求,实用配置更是远超他的预期。

“……那么,购车款是哪来的?”

半晌,陆久终于说道。这辆车可以说他十分满意,但他没有忘记重要的事情——他虽然给了V一些现金,但那点钱绝不足以购置一辆汽车。虽然是车龄十年的二手车,但要买这样豪华的一辆越野车,所需的钱恐怕也不会是一个小数。

“合法途径。”V说。

“哦?”陆久说,“莫非格里芬的战术人形还有工资?还是说,你支付了一些我不知道的东西做报酬?”

陆久当然知道公司的战术人形是没有薪酬的,因为她们不过是公司购置的“作战设备”。但正因如此,陆久才对这辆车的来历感到了担忧,他担心V是付出了一些他不愿意去想的代价才得到的这辆汽车。

虽然没有明说,但V似乎明白陆久的意思。

“我才没有。”V稍微有些窘迫地说,“虽然格里芬公司的战术人形没有工资,但我之前在其他地方工作的时候……得到过一些劳务费。”

她果然不是一个简单的战术人形,陆久心想。关于她“之前工作的地方”,陆久认为有机会应该稍微了解一下。不过现在不是问这些事情的时候。

“这我倒不知道,不过我不能让你为我如此破费。”陆久笑着说,“多少钱?我给你吧。”

“我不要。”V轻声说,“金钱对我来说没什么用。既然你需要汽车,我想把它送给你,如果你愿意收下的话。”

V的话,让陆久思考了一下——这是收到别人的礼物了吗。但是接受女孩子如此贵重的赠礼,到底合适吗。

不过他随即哑然失笑。她给予自己的,又何止一辆汽车呢。自己从她那里得到的东西,价值只能用生命去衡量。

“说的也是。”陆久说,“金钱这东西,对我们来说本是最没用的玩意啊。那我就厚着脸皮笑纳了。”

“不用客气。”V难得地露出了一丝高兴的神色,“不过不知道,你对这个还满意吗。”

“那当然。”陆久竖起大拇指说道,“岂止是满意,简直应该说是棒极了。”

\section*{}

有了代步的车辆,行动起来顿时方便了许多。两个人把行李放在容量可观的后备箱中,驱车只用了十几分钟就到了小区里。搭乘电梯来到房间,V把行李随手放在了客厅里,然后在房间里环顾了一圈。

“这套公寓真好。”V情不自禁地说道,“不仅有阳台和客厅,还有厨房。”

当然了,陆久心想,这又不是职工宿舍。正常的居室,都会有厨房的吧。不过对于厨房的实用性,陆久倒没抱太多期待,因为他可不是什么热衷下厨的人……而对于V他就更不指望了。他清楚地记得他们相识多年里V唯一一次制作食物是在北镇的海边,那天晚上V很贴心地把压缩饼干磨碎后,用热水给陆久冲泡了一下。陆久隐约感到那就是她的“烹饪天赋”所能发挥的极限了。

卧室有一大一小两间,但陆久没有再去征求V的意见,因为他知道就算问了V也不会有什么明确的意见。他拿起自己的被褥铺在了次卧的床上,因为那间屋子里放的是一张单人床,V的那床铺盖要稍微大一些铺在上面可能会不大合适。

“周末去买一台洗衣机,以后洗过的衣服就可以搭在阳台了。”陆久一边把自己的衣服挂在衣柜里一边说,“好了,去吃饭吧。”

“嗯。”

吃饭对陆久来说是一件总会让他头疼的事情,因为他有害怕面对五花八门的菜单的弱点。所以正如皮尔斯所言,他总是会点菜单上的第一款主食。但自从皮尔斯在前一阵子毫不留情地指出陆久的这个毛病之后,陆久已经决心加以改变了——但他改变的方式不是认真地总结自己的口味偏好,而是仔细地选择一个用餐效率最高的地方,像是单位的食堂或者路边的小摊之类的。所以从另一个角度说,优雅豪华的大饭店是陆久所最为忌讳的。

“有什么想吃的吗。”

走出小区,站在华灯初上的街道上陆久对着V问道。

“什么都行。”

这样的答复陆久早已了然于心,这种对话对他来说更像是一种免除责任的策略——既然V对于晚餐没有任何意见,那么决定权就落到了陆久手里。那么无论他点的餐有多糟糕,也不该有人再提其他意见了。

不过这次陆久还是稍微有点失算,因为他刚搬进这个小区,并不知道附近有什么方便的用餐地点,而且再跑回之前吃饭的那家饭馆就显得有点太蠢了。

一阵寒风吹过,陆久缩了缩脖子。依照旧历现在已经是小寒时节,气温早已降到了冰点之下,站在黑夜的街道上的确是有点冷了。于是陆久迅速做出了决定:去最近的地点用餐。

陆久朝着小区门口一家营业开张的店面走去,V紧随其后。走进店里,陆久看到大厅里只有两桌人在吃些便饭,气氛有些冷清,不知是不是因为来得太早的原因。陆久不喜欢吵闹的环境,于是带着V在角落的隔间里坐了下来。

年轻的服务员小伙很快递上了菜单,陆久注意到这里的小店几乎都是人类在充当跑堂的伙计,只有比较大规模的“正规”店面里才有服务人形出现。这和民用人形随处可见的上海是相当不同的。

菜单上都是些家常菜,陆久随便点了几样然后把菜单还了回去,小伙子抄了单子飞快地向后厨跑去了。放下遮住一半入口的帘子,隔间里稍稍安静了一些,只剩下了相对而坐的陆久和V两个人。

“你以前在这里,呆了多长时间?”

V坐在那里一言不发,为了缓解沉闷的气氛,陆久随口问道。

“从第一次来到最后一次离开,大约有两年的时间吧。”V说,“不过期间断断续续地外出执勤,也不是一直都在这里。”

“是吗。那你对这里要比我熟悉多了。”

“不是太熟悉,因为备勤的时候都是呆在公司,只是偶尔出门去小芮的店。”

“对了,你知道小芮家在哪吗。”

“不知道,我只在店里见过她。”

“她也住在这个小区。这套公寓,还是小芮的父亲帮忙找到的,如果你再去她的店里的话记得替我向她道谢。”

“好的。”

“别忘了邀请她有时间来做客。”

“可以吗。她会来吗。”

“她来不来就由她决定了,搬家后对朋友发出邀请,是一种惯常的礼节。”

“嗯,我会对她说的。”

“话说,我的手机是不是买到过时的型号了?”

“是的。不过这样的配置也足够日常使用了。”

“我不太常接触高科技的产品,何况是这个时代的……不过你好像对手机很了解?”

“不算很了解,但这也是日常用品了。为了方便联系,公司曾经给我们配发过移动终端,几年前的型号都比你买的这个先进。”

“那还真是让你见笑了。”

“还好吧,至少旧型号的手机待机时间都很长。”

“非常感谢你照顾了我的面子。”

“不,我没有……”

两个人有一搭没一搭地闲聊着,气氛渐渐轻松了起来。没多久,陆久点的便饭被服务生端了上来,两个人照例很快地消灭了晚餐。

\section*{}

回到公寓,时间已经是晚上八点多,虽然不早了但还没到休息的时间。陆久坐在客厅的沙发上,面对着空旷的墙壁发呆——那个地方看起来本该有个电视机之类的东西,大概是房东在离开的时候搬走了,所以显得有点空荡荡的。

该买一台电视机放在那里吗,陆久心想,一来算是装饰、二来也好在晚上回来早的时候打发一下时间……

但陆久马上就否定了自己的想法,因为他知道这个屋子里住的两个人是绝对不会看什么电视节目的。他倒是因为自己这种过于世俗的想法而有些吃惊。

人类果然是社会性的动物,这一个多月的时间里他已经基本融入一般人的日常起居了,就连想法也在渐渐和“普通人”接近起来。

人的适应性,真是超出想象……陆久心想,在残酷的战场、艰苦的战区、大都会和这样的中型城市,无论是怎样从未想象过的环境,也总能很快地融入其中。他本来以为自己永远都不会习惯这种几乎是无所事事的生活呢。

“那面墙,有什么不对吗。”

陆久忽然听到耳边传来一个声音。

“啊,没什么。只是觉得有点空荡荡的。”

陆久连忙说道,他意识到自己又走神了,竟然连V坐在了他旁边都没有发觉。

“要找点什么装饰一下吗。”

“免了吧,反正也不会有客人上门拜访……对了,你以前在公司这边的时候,没有任务的时候都会做些什么呢。”为了掩饰自己的尴尬,陆久迅速转移了话题。

“……没什么。偶尔会去天台上呆一会儿,其他的时候,大多是什么都不做地待命。”

“什么都不做?就那么在仓……宿舍里呆着吗。”

“是的。”

“呵,那可真够无聊的。其他人形呢,也是这样吗?”

“差不多吧,因为我们不能随意离开公司,所以多数时间都是如此。”

这个回答让陆久相当意外,他知道人形大概也不会有什么丰富的业余生活,但就这样什么都不做地备勤他也没想到。简直就和闲置的设备一样。这样说来,战区的那些人形还要好一些,至少可以在营地自由活动和去靶场射击……甚至偶尔还能偷偷溜出去逛一逛。 

“我还以为在公司里会多少比在战区自由一点呢,没想到还不如战区啊。”陆久有些嘲笑地说道。

“公司对人形而言,只是维修的车间和储存的仓库。不执行任务的人形是无法给公司带来效益的,对公司来说实际上是一种损耗,所以备勤人形的配给很少,只有长时间静置才能最大限度地节约资源。”

“原来是这样。我以前呆在公司的时间很少,不怎么了解这些。说起来,我对公司的人形……还有你的了解,也许并不多。”陆久叹了口气。 

“就算你知道了也会失望的,因为我没有什么值得去了解的事情。”V说。

她说的大概是真的,陆久心想,因为既然在公司只是呆在仓库的话,那么显然也没什么值得细说的东西。据他以前在16LAB认识的一个服务人形所言,人形的在人类社会生态环境并不乐观。那么在公司里想想也不会好多少,至少“储存”她们的地方,陆久已经见识过了。

但陆久忽然心里一动,想起了一些事情。

“如果这样说的话,我倒是有些事情想问问,不知你能不能告诉我。”

“什么事情?”

“在格司成立之前,你和克鲁格就认识了吧。”陆久说,“能和我说说你和他的事情吗。”

听到陆久的话,V没有出声。陆久向着V看去,看见她脸上的表情仿佛凝固了。

“怎么了,突然变得那么严肃。”陆久半开玩笑地说着,“不会是需要保密吧。”

“不,只是……”

“我是随便问问。不方便说就——”

“不是那样!”V打断了陆久的话,“让我……想想……”

陆久愣住了,他从来没有见过V这样着急。她的语气急切、脸涨得发红,仿佛在努力地在自己的记忆中搜索着什么。

“没错,我的确是在公司成立之前,就在克鲁格的手下了。那时候……”V双手撑着额头,紧紧皱着眉说道,“虽然有些记忆经过了处理,但我还记得一些。那时候,我……我是……”

“是他的士兵吗。”陆久替V把她正在苦苦思索的话说了出来,“我记得你说过,你经历了很多战斗。”

“是的。”沉默了一阵后V开口说道,“你说的没错,我是……一名士兵。”

\section*{}

V一边努力回忆,一边断断续续地讲述着。克鲁格在成立格里芬公司之前,曾经经营过一只雇佣兵队伍,陆久之前也有耳闻。但那是怎样的队伍,陆久不知道,而V也不甚了解。根据V的回忆,她只知道克鲁格的队伍里曾经主要是人类士兵组成的,但她从来都没有见过那些人,克鲁格也从来没有提过那些人后来都去了哪里。自从她拥有记忆,就只记得指挥自己见过的只有克鲁格一个人……克鲁格的士兵,一直都只有她一个。

“我参与的那些战斗,战场不仅是在野外、还有在城市之间……正面和非正面的都有。”V说,“和敌人的正面作战很有限,更多的是利用自己的外表做伪装……去执行秘密任务。侦察、破坏、内应,和直接攻击……”

“好了,我知道了。不用再回想了。”陆久开口打断了V的诉说,因为他看到V的表情显出了痛苦的焦灼。陆久想起V曾经说过她的记忆也曾被反复审查删减,虽然她在努力地回忆,但显然她已经无法完整地回忆起来那些事情了——

从某种意义上说,就和被再社会化改造过的陆久一样。

但V仿佛没有听到陆久的话一样,依然在梦呓般地喃喃低语。

 “……我在战斗中杀死过人。那是……活生生的,人类。我是一个非法的人形。我还一直以为……”

“薇,不要再回想那些事了。停下吧。”

“可是……我依然依稀记得那些人,临终时候的样子。他们看着我,那样的眼神……”

“停下!”

陆久抓住V的肩膀,用力摇晃了一下。V这才睁大了眼睛,如梦初醒般慢慢抬起了头。

“我……刚才……”

V看着陆久,眼睛里满是迷惘,仿佛不知道刚才发生了什么事情一样。

“你的记忆里已经没有那些内容了。”陆久说,“所以别再去想了,以后也别想了。”

“对不起,我没能想起多少你所问的事情。”

“你已经告诉我很多了,而且那些事情对我来说也没什么要紧的。”陆久轻轻抚摸着V的肩膀,“我只是好奇而已。”

“但我至少想起了一件事,那就是我曾经是个为了杀人而生的机器。呵。”V摇了摇头,无力地笑了笑。

“你只是个执行了命令的战士。”看着V疲惫的样子,陆久说道,“这没什么,我也一样。我也曾是个士兵,我也……消灭过许多敌人,也许远比我所知道的要多。但那又如何呢。我们只是奉命行事,仅此而已。”

“嗯,也许吧。回忆起那些事情,我感觉十分陌生,就像是别人的记忆一样。但我知道那是我的过去,因为那就是战术人形诞生的目的。为了替代人类去杀或被杀,而作为工具制造出来的——”

“那些事情都已经过去了。”陆久说,“现在的你也不是以前的那个人了,你只要,做你现在想成为的那个人就好。”

“……那你呢。”休息了片刻后,V轻声说道。

“我?”

“你也,不记得以前的事情了吧。”

“的确。过去对我来说,是许多破碎的片段。有很多细节我至今也想不起来,而且努力去想的话我也会感到头疼。”陆久耸了耸肩。

V默默地注视了他一阵。

“那你心里的自己……到底是谁呢。”过了一会儿,V开口说道,“如果……当知道了自己的过去之后,你还会是……‘陆久’吗。”

“你为什么会这么问呢。”

“那个人形在离开前,给了你一些东西吧。在战区的时候。”

陆久稍微思索了一下,才想起来V说的是谁。她说的,应该是从16LAB逃出来的那个‘05’……在北部战区那个初雪之日的诀别。

的确,陆久心想,那时候V就在他的身旁,亲眼目睹了一切。

“你是在担心我有朝一日回想起了过去的一切,就会变成另外的人吗。”

“……嗯。”V低声说,“如果你不是陆久了……那我,又该是谁呢。”

陆久看了V一阵,明白了过来。如此说来,这还真是件值得担忧的事情。

“不会的。”陆久说,“你的担心是多余的。”

“嗯。”

V没有再多说什么,但她脸上略有凝重的表情没有一丝缓解,显然她还是对自己的担忧难以释怀。

“唉。”陆久叹了口气说,“在16LAB的时候……我差一点,就和帕斯卡结婚了。”

“是吗。”

“唔。正如你所说,帕斯卡非常希望我能留在她的实验室里,因为她在某些方面十分需要我这样的人。她为此或明或暗地,把许多事情都做了精心的安排。我并不是说她做的那些都是出于虚情假意,但我也知道她有她自己的目的,她的感情也许并非假装,但她另有更加庞大深远的图谋,我能感觉得到。但我无论如何也没想到她竟然会提出结婚的事情。”

“那你是怎样回应她的呢。” 

陆久好像突然说起了一个毫无关系的话题。不过这个话题,显然吸引了V的注意力,她对陆久说的事情……

很感兴趣。

“我说我会认真地考虑。呵。”陆久说,“当时她说出结婚这个词的时候已经喝醉了,我以为她只是酒后的玩笑。但她第二天清醒之后再次提起了这件事,让我感到不知所措,完全不知该如何作答。因为对于我这种朝不保夕、习惯于独来独往的人来说,那是不现实的。但那也是我第一次去认真地考虑那些事情。” 

“你是说……结婚吗。”

“不是结婚……”陆久摇了摇头,“我是说那些我一直在刻意回避的事情……许多事情。”

“比如自己的过去?”

“就是那些。真的,以前我从没认真地去想过那些。因为我觉得那些事情确实不能改变现在……不,因为我觉得那些事情,我如果详细地了解之后,恐怕自己也会一时难以接受。如果我知晓过去之后,就不能像那时候那样得过且过地随波逐流,那我该怎么办呢?这样的想法让我非常惶恐。所以我就接受了皮尔斯的那套‘不必追寻过去’的说辞。”

“是吗……我还以为,你真的不在乎呢。”

“对不起,呵,那时候我只是装腔作势。因为对于那时候的我来说,‘陆久’这个名字确实是个不错的避世之所。”陆久笑了,“但在和帕斯卡相处的时间里,我再次认真地考虑了这些事。虽然不知道细节,但那是对于自己过去的身份我已经有了一点了解了——和你一样,我也曾是克鲁格旧日的相识,但我的关系可能要比你更近一点,我是他年轻时的战友。可是我觉得这样的身份对我来说并没什么特别的意义。”

“你怎么会这么说呢。你和我完全不同。”V稍稍有些吃惊地说道,她显然是第一次知道陆久竟然有着这样的过去,“如果事情真的是你说的那样,那么克鲁格元帅显然是对你抱了极大的期望才把你从牢狱里救出来的。如果你满足了他的期待,你以后一定会是格里芬公司数一数二的人物,你的地位一定会比现在要显赫得多。”

“可是那真的是我想要的吗。”陆久说,“功名利禄、地位财富,对我来说到底意义几何?”

“那么你想要的是什么呢?”

听到V的提问,陆久沉默了片刻。

“我想自己决定要成为怎样的人。”陆久说,“我们的经历有所不同, 所以我说的这些事情也许你还不能明白。但我可以告诉你,现在的我对‘陆久’这个身份很满意,我正在用这个身份去做自己真正想做的事情。”

陆久说着,伸手轻轻拨弄了一下V的头发。

“我可不是你这样的小女孩,会整天摇摆不定地不知道该何去何从。我既然已经决定自己要做谁了,那我就不会再动摇。那个人形给我的数据磁盘就扔在我的抽屉里,我只是没有兴趣去看那些,但无论那个磁盘里有什么,我向你保证,我都会是陆久,是你所知道、所认识的这个陆久。”

“我不是小女孩。”V抗议地嘟哝了一句。但她脸上的神色,明显轻松了很多。

“是啊,你也是个战斗中成长起来的老兵。”陆久笑了,“别再去想了,早点休息吧。明天我们还有工作。”

“好的。”V说着站了起来,“我去睡觉了。”

“薇。”

“嗯?”

当V正要走进自己的房间时,陆久开口叫住了她。

“无论以前我们是谁,都已经和现在的我们现在无关了,所以别在意那些。你就是你、我就是我。”

“……嗯。”

看到V走进了她的房间并关上了门,陆久也起身朝自己的屋里走去。
\chapter{昨夜的星辰(七)}

\section*{前言}

活着、死去,这些事情对于以前的陆久来说,并没有什么特别的意义。他总是在按部就班地活着,从不曾考虑过生命的意义,因为他认为自己这样的人,就算是死去也不会有人为之哀悼。

这样的人,是从什么时候开始明白生命的价值的呢。大概就是从看到那些活着的人,为了死去的人而悲痛的时候。那时候,他才开始了思考:自己失去珍惜的人会是怎样的心情,自己死去是否也会有人承受同样的悲伤。

因此,他也就感到了死亡是沉重的、剥夺人的生命是罪恶的,感到了活着是一件值得庆幸和珍惜的事情。

\lineseparator
\section*{}

早上,陆久从房间里出来的时候V已经在客厅等候了。她的表情很平静,似乎听从了陆久的建议,没有再去想关于以前的事情。两个人如往常一样在八点三十分准时抵达了办公室。

这一天的工作和昨天一样,早早地就有一大堆信件堆在面前,几个人一直忙碌着到了中午才稍稍喘了口气。到了下午下班的时间,谢振首先起身离开了办公室,但雷蒙却没有走。当陆久正想要闭眼休息一会儿干涩的眼睛时,雷蒙走到了他的座位旁边。

“陆主任,我有点事情想向您……汇报一下。”

“唔,怎么了?”

陆久揉了揉眼睛,坐直了身子。但雷蒙却没有继续说下去,而是朝着V望了一眼。

“我能和您单独谈谈吗。”他说。

陆久疑惑地看着雷蒙,心里有点纳闷。难道有什么需要保密的事情要说吗。

“知道了。薇,请你去先把车热一下吧。”

“好的。”V看了陆久一眼,表示明白地点了点头,起身走出了办公室并关好了门。

“不好意思,我不是不信任陆薇小姐,但这件事知道的人少一点会比较好。”V离开后,雷蒙从抽屉里取出一件东西递给陆久,“陆主任,我们今天收到了……这个。”

陆久接过来一看,是一个很普通的蓝色信封,寄件人是公司总部、收件人是一个他不认识的名字,地址则是这个城市下辖的某个区县。

看起来这只是一封从总部发出,需要代为递送信件。陆久知道由总部发出的信件是不需要审查的,直接按照流程投递即可,却不知雷蒙为何要把它交给自己。

“这封信,有什么问题吗。”陆久说。

“蓝色的信封……是装阵亡通知书专用的封套。”雷蒙轻声说。

“……是这样吗。”

陆久不由得感到了惊讶。再有十几天的时间就是春节了,在这样阖家团圆的节日前收到这种东西,对一个家庭来说会是怎样的噩耗,就连曾见惯了死亡的陆久也不忍去想。

“我们该怎样处理这封信件呢。”陆久说。

“依照惯例,信检中心在收到‘蓝信封’的时候会向综合办公室汇报,然后由综合办去安排公司专门的人员将信件亲自送往收件人的家中、并进行安抚工作。这件事……通常是由分公司的一把手负责出面。”

“那么,我是要将这封信亲自交付综合办公室吗。”

“流程上本该如此,不过……”雷蒙似乎有些犹豫,“有件事我不知道该不该说。”

“说吧。”

雷蒙看了看陆久,又看了看门口。然后,他仿佛下定决心一般,深深吸了一口气。

“这位阵亡的士兵我认识,他曾经是谢振手下的人。在谢振刚刚来到信检中心的时候,这个人曾经来找过他,那时我们还一起吃过一顿饭。所以我在想的是,这件事要不要和老谢说呢。如果报到综合办的话,迟早是会被老谢知道的,而我们要是没有提前知会他……到时候他恐怕不会高兴吧。”

陆久沉默了一阵,然后伸手摸了摸下巴。战斗总会有牺牲,如果是以前的陆久,他是不会为这种事情动容的。但现在不同了。

他已经不是运筹帷幄的指挥官,甚至就连战斗人员都不算了。他必须站在别人的角度设身处地地去想一想,必须像个普通人那样去同情、去理解别人的心情。

所以他意识到,这件事远比他想象的要复杂得多。

既然是谢振的往日战友,那么于情倒是该让他知道,但陆久觉得没有理由这样做。谢振已经不在特勤中队了,这件事也就和他没有关系了……而且这样悲伤的事情,谢振知道了一定会影响他的情绪。但如果不告诉他,就像雷蒙所说的,以后老谢知道了恐怕也会不高兴。这让陆久一时犯了难。

“我觉得,老谢知道了这件事只会难过,还是不要让他知道的好。”沉思了一阵后,陆久开口说道,“如果我们去把这封信送到家属手中,符合公司的规定吗。”

“公司只是说通知书须内部人员亲自送达家属,但没有明文规定必须由谁送……由分公司的最高负责人出面,只是为了表示公司的重视而形成的惯例。您如果您愿意去的话,综合办公室应该不会反对,毕竟,如此沉重的事情……您也能理解吧,谁都不会愿意去做的。如果有人主动请缨,我想他们反而要松一口气。”

“是这样。这封信,先放在我这里吧。”陆久点了点头说,“我考虑一下怎么办最为妥当,然后我来和综合办汇报。”

“好的。”雷蒙点了点头没再说什么。他把那个蓝色的信封放在陆久桌子上,然后离开了办公室。

\section*{}

陆久看了看那个信封,上面的名字他并不认识。这是当然的,他从来没有和特勤人员打过交道。他把信放在了自己的抽屉里,然后想了想,又把信取出来塞进了自己的兜里。

陆久走出离开办公室,V正在门口等着他。他一言不发地上了车,和V一起回到了公寓。两个人依然是在昨天吃饭的地方解决了晚饭,但这次陆久相当沉默,吃饭的时候一句话都没有说。

“发生什么事了吗。你看上去精神不太好。”

回到公寓,陆久依然坐在客厅里发呆,这时候V终于忍不住问道。

“我们收到了一封……阵亡通知书。”陆久长长地出了口气,从兜里取出那封信放在桌子上,“我在思考该如何处理这件事。”

“按照标准流程,该怎样处理的呢。”

“交给综合办公室,让他们安排人员去递送。”

“你不打算这么做吗。”

“是的。这个阵亡的士兵……是谢振曾经的战友。我不想让他知道这件事。”

“为什么呢。”

陆久没有说话,抬头看了看V。

“你觉得我们应该让谢振知道这件事吗。”陆久说。

“我觉得……应该吧。”V想了想说,“毕竟是曾经和他有关的人,他有权知道。”

“正因为如此,我才和你的想法相反。谢振是非常珍重战友的人,他知道了这件事也不会让死者复生,但一定会非常悲痛。我认为这只会给他徒增烦恼。”

“可他总有一天会知道的。”

“如果他知道的时候这件事已经过去了很久,至少他就不会那么难受。”

“我觉得我们不该向谢振隐瞒这件事。那样做……不诚实。”

陆久看了看V,然后笑了笑。是啊,陆久心想,“诚实”。那也许正是V身上最美好的品质。但对于人类来说,过于诚实有时候也是一种鲁莽。

“你总是很诚实,薇。我毫不否认诚实是一种美德。但有时候我们讷口不言,才是真正的人文关怀。”

“我不明白。”

“你一定认为说出真相并没有错。当然,如实陈述,任何人都会觉得是合乎情理的。但在人类的世界里却并不总是如此。我们了解到的真相,不去透露给那些不需要知道的人,这样那些人就能活得更心安理得——既然他们知道了也不会改变任何事情、知道了反而会徒增苦恼,那我们为何非要让他们知道不可呢。所以说,有时候不把自己知道的事情说出来,也是一种慈悲。有句话叫做‘沉默是金’,就是说一些时候,保持沉默才是最恰当的。”

“ ‘知道了也不会有所改变,所以还是不知道为好’,以前你曾说过这样的话,但你后来也承认这是一种逃避。所以这样的思考方式我不太理解。我觉得无论如何那些事实就摆在那里,如果早一些知道的话,也许就……”

V似乎对陆久的逻辑早有微词,但当她无意间扫了一眼桌子上的信封的时候,她的眼睛忽然微微睁大了。

“怎么了?”

陆久有些奇怪地说道。他注意到V那总是很平静的表情,出现了一些细微的变化。

“不,没有。那个……”

V的样子忽然慌乱了起来,就连说话都有些语无伦次。

“你想起什么事了吗。”陆久更加不解了。

“没什么。”V说道。

“你的样子可不是‘没什么’的样子,我看显然有什么吧。”陆久又疑惑又好笑地说,“你知道自己根本不会撒谎的,更不会隐藏自己的想法。”

“……但是,我不想说。”

陆久闻言一愣。V拒绝回答他的问询,这还是第一次。明明昨天晚上陆久的提问让V回忆得非常痛苦,她也不曾推脱。

“那能告诉我不说的理由吗。”

“因为……‘沉默是金’。”

陆久皱起了眉头。这家伙在说什么啊。词语学得倒是很快,可这个显然用错地方了吧。

“我已经知道了你有话要说,再沉默就没用了。”陆久说,“你不想说我也不会勉强你,但要是和我们眼前的这件事有关的话,我还是希望你能告诉我。因为我正在为此而伤脑筋,也许你知道的事情能对我有所帮助。”

“……不会有帮助的。”V低声说,“只会起反作用。”

听了V的话,陆久也沉默了。他从桌子上拿起那封信,凝视了一阵,然后又放在了桌子上。

“就算你不说,我也能大概地推断出来。”陆久说,“这封信是从总部寄来的,你不可能知道里边的内容。那么只有一个可能:你认识信封上的这个名字,对吧。”

“……”

V沉默着没有出声。

“你说会起反作用,莫非我也认识他?”陆久接着说道,“不,不会的。我一个特勤人员都不认识。不过我倒是见过几个,但也非常有限。但这倒是个有价值的消息:你不是热衷交际的人,认识的特勤人员会有谁呢。这里面和我见过面的就更少了。我想我的搜索范围已经缩小了很多了。”

“……铁杉树行动。”仿佛放弃了一样,V终于叹了口气,开口说道。

“铁杉树”?陆久觉得这个词有些耳熟,但一时想不起在哪听过。

“在南宁。”V又接着说道。

南宁……?

陆久一时间还是没有把这两个词联系起来。但当他意识到V所说的事情的时候,他感到如遭雷击、脑海一片空白。

毫无疑问V说的是那一次,因为陆久只去过一次南宁。

“难道他是……”

陆久难以置信地说道,他感到嗓子有些发干。

“是的,”V低声说,“南宁那次行动的四名特勤人员之一。我曾经在那次行动的简报上看到过他的名字。”

\section*{}

陆久感觉有点头晕,于是轻轻靠在了沙发的靠背上。这么说,眼前这名牺牲的士兵、谢振的战友,是在那次行动中阵亡的……

若是这样,那么他可以说是被陆久杀死的。

“……没事吧。”看到陆久失魂落魄的样子,V屈身凑到了他跟前,关切地问道。

“没什么。”陆久摇了摇头。

“你的样子可不是‘没什么’的样子。”V学着陆久说道。

“我……只是感觉有点,出乎意料。我实在没有想到会是那些人……”

“你的推测完全正确,我还以为……你都知道了。”

“……我其实是骗你的,我一个特勤人员都不认识,又怎么会知道你认识的是哪个呢。”陆久勉强笑了笑,“我甚至觉得我们都见过的特勤人员根本不存在。”

“你……”

听了陆久的话,V才意识到自己被骗了,一时间有点生气。但看到陆久的样子,她又生不起气来了。

“好了,我知道怎么回事了。”陆久说,“我还真得感谢你,提供了一条非常重要的信息。”

“然后你打算怎么办?”

“按原计划办。我会亲自把这封信交给这位士兵的家属的,所以这件事也请你保密。”

“还是交给综合办公室吧,不要勉强自己。”

陆久看了看桌子上的信、又抬头看了看V。他看到V虽然表情平静,但眼里满是关切。于是陆久长长地出了一口气。

“这件事确实有点麻烦。不过我想还是我亲自出面比较好,我觉得我还能应付。”

“我和你一起去。”

“没有必要。”陆久说,“这种事我一个人去就够了。”

V没有说什么,只是看了陆久一阵。

“我要去。”她轻声说。

陆久抬起头看着V的眼睛,但V毫不退缩。陆久在她的眼中看到了熟悉的眼神,让他想起了在战区的时候,他决定向临近战区发起支援的那个晚上。

“我又不是去战场。”陆久说。

“可你的脸色,比去战场时还要沉重。”

“……好吧。”过了一阵,陆久终于无可奈何地说道,“那就一起去吧。”

听到陆久的这句话,V才移开了盯着他的目光。

“你的表情很难看,所以一个人去的话我不放心。”V小声说,“如果觉得我碍事,我可以在车里等你。”

“不,我只是不想让你也去承受这种……算了。”陆久说,“说实话,我一个人去心里也没把握,你如果愿意和我一起,我倒很高兴。”

“好的。”听了陆久的话,V的表情明快了许多,“那我们什么时候出发呢。”

“很快,不过在那之前我们有些东西要准备一下。我记得在战区的时候给你发过常服,那套衣服还有吗。”

“有。一直放在我的行李中,但从来没穿过。”

“很好,明天上午把它拿到洗衣店整理一下。我的军装……被没收了,我也要去找套像样的衣服来。”

“要去商店里买吗?”

“不,商店里没有那种服饰。我得找个人帮帮忙。”

\section*{}

说完,陆久拿起电话拨了一个号码。没过多久,电话接通了,里面传来一阵嘈杂的吵闹声。

“晚上好,准将先生。”陆久说。

“什么事?快点说,我现在很忙。”

“我听出来了。那就长话短说吧,我需要我的军装常服。”

“这种事可别找我。想要的话,自己去找克老爷子请示。”

“那就换个说法,我要一套和我以前那套尺码完全一样的常服。”陆久说,“这种公司人手一套的衣服,犯不着去找大老板吧,分公司难道没有后勤处吗?”

“一套衣服倒不在话下,但老板要是看到你穿着军装,那么给你这套衣服的人就不会有好果子吃了,知道吧。你以为你的军装为什么被扒了,难道你不知道克老板有多烦你?给你军装,那不是打他老人家的脸吗?”

“……我不会在公司里穿,而且我保证不会让任何公司里的人看到。”

“你要去干什么?”

“与你无关。”

“好的,我也没空听你瞎撤淡。再见。”

“别挂!我真的急需这东西。”

“少废话。要么告诉我理由,要么就别浪费我的时间了。”

“我要去……给一个士兵的家属,递送阵亡通知书。我不能穿着便装去。”

“……”

电话那边沉默了。陆久听到里边的嘈杂声渐渐小了,想必是手持电话的人去了一个相对安静的地方。

“行啊,有你的。”过了片刻准将先生说道,“我正琢磨该怎样让你死心呢,没想到你竟然找了个我无法拒绝的理由。”

“是你自己要问的,我没必要编这种故事。”陆久稍微有些恼火地说道。

“我到希望你是编的,这样我就会对你的创作才能刮目相看了。”准将先生嘲弄地说道,“我会跟那边的人打个招呼,不过你办完这件事,要马上脱下那身衣服好好藏起来才行,不然我麻烦就大了。还有别的事吗?”

“没有了。这次算我欠你一个情。”

“哼,你欠我的情只有一个吗。”

说完一句挖苦的话,对方挂断了电话。陆久也没说什么,把手机随手扔到了一边。

“明天你先去准备自己的衣服,我去公司拿了东西就出发。”

“知道了。”V说,“我们……该怎么做呢。”

“你觉得呢。”陆久反问道。他也没有这样的经验,所以想先听听V会发表怎样的看法。

“我……觉得只能把事情如实相告,并争取家属的宽恕吧。”

不可能得到宽恕的,陆久心想,这可不是鞠躬道歉就能算了的事情。而且他们是代表公司去安抚遗属的情绪,而不是去负荆请罪。

“那要是家属不肯宽恕呢。”

“他们如果要责难的话……就由我来承担吧。”

陆久摇了摇头。几乎毫无和人相处的经验的V思考方式太过简单了,她的意见陆久实在无法采纳。

“不能那样。”陆久说,“我们此去的目的是为了抚慰家属的情绪,而不是去刺激他们。我们不能让任何人知道这个士兵的死和我们有关,包括谢振和雷蒙。”

“那不是成了……”

“薇。那些遗属,失去了他们的家庭成员,那是他们最重要的人。”陆久打断了V的话,“他们要是知道自己的亲人只是毫无价值地——” 

如果是我死了,杀死我的人向你请求宽恕,你会怎么想?陆久想问V这个问题,但他没有说出口。

“别忘了我说过的话,‘沉默是金’。”

“……我知道了。”

“我们得把这通知书交给他们,但之后的事情只能随机应变了。到时候事情就让我来处理吧,你不必说话。虽然我也不是什么擅长交际的人,但至少在人情世故这方面,我比你经验稍微多一些。”

“好的。”

\section*{}

早晨陆久很早就去了公司,当他经过公司大门的时候被警卫处拦了下来。

“陆主任,昨天晚上后勤处委托我们将这个交给您。”执勤的人形少女恭敬地向陆久递上了一个包裹。陆久接过包裹摸了一下,里面的东西显然是衣服。

效率真高啊,陆久心想。大人物说话就是不一样。

“谢谢。”

陆久把衣服放在车里,来到了办公室。谢振和雷蒙都还没有来,办公室里一个人都没有,于是陆久便留了一张字条说民给自己和V要出去一天。

回到公寓,V不在屋里,应该是带着衣服去洗衣店熨烫了,于是陆久决定先试一试刚取来的衣服。他脱下身上的便装,穿好军装走进洗手间,看着半身镜里的自己。

那套常服非常合身,就像量身裁剪的一样,他的样子和在战区时没有多大区别。不过不同的是,镜子里的那张脸却有所不同——也许是想到今天要去做的事情,虽然他努力地做出平静的样子,但眉宇间还是难掩凝重的神色。

自己就是带着这幅表情四处走来走去的吗,陆久心想,怪不得会被V担心。他做了几次深呼吸、用力揉了揉脸抹去了脸上的阴郁,然后开始对着镜子里的自己思考关于那封慰问信的事情。

无论他怎样做,对于那个士兵的家属来说这都是个难以接受的噩耗,陆久很清楚这一点。但他希望自己能尽量妥善地完成自己的任务。

敲门之后,该怎样自我介绍、又该怎样打招呼呢。该站在离门口多远的距离比较合适?是该先敬礼,还是先握手?陆久一遍遍地揣摩着自己的动作,在心中模拟着悲痛的家属可能出现的反应已经自己该如何应对。当陆久听到门口传来有人进来的动静的时候,他低头看了看手上的计时器,发现时间已经过去了一个多小时。

陆久走出洗手间,正碰上刚刚进门,手里还拿着熨烫过的衣服的V。看到衣帽整齐的陆久,V楞了一下,她的眼睛微微睁大了。

“怎么了。”陆久有点不解地问道,他注意到V的表情有点奇怪。

“没什么。你的衣服……很合身。”V小声说了一句。

陆久更加不解了,他整理了一下领带,又低头看了看自己的衣服。这身衣服和他当时在战区的时候穿的应该是完全一样的,合不合身V总不该现在才发现。

“和在战区的时候有什么不同吗。”

“和那时候完全一样。”V说,“所以才觉得有点……熟悉的感觉。”

“也就是说看着还算顺眼?”

“是的。”

陆久这才恍然大悟,V大概是想称赞他的装扮,却没有合适的词汇可以形容。形象得到了V的肯定,让陆久稍稍多了一点自信。

“那就好。虽然不能光明正大地穿,呵……但我希望这套制服能符合我所要扮演的形象。”陆久说,“你也试试自己的衣服吧,如果没有问题我们就准备出发。”

“好。”V说着把手里的衣服放在沙发上,然后开始在客厅里换衣服。这种毫不避讳的做法陆久并不感到意外,不过在V脱下衣服的时候陆久还是转过身回避了一下。

“好了。”

听到V的话,陆久转过身。当他看到V的新装时候,不由得也和V刚走进门时那样楞了一下。

她那身衣服非常……合身。陆久忽然明白为什么V要这么说了,因为他一时间也找不到恰当的词语来形容。

白衬衣、黑长裤,深灰色的小西服和深蓝色的领带,女装的常服要比男士黑白两色的常服的色彩丰富一些,但依然保持了军装庄正素雅的风格。以前在战区的时候V总是穿着作为民用人形时穿着的那身洋装,虽然优雅但是谈不上什么气质,看久了就觉得不过是日常服饰而已。但眼前的这身制服穿在身上,让人感到庄严而不失大方,秘书官端庄秀丽的气质立即油然而生。陆久有些后悔,早知这身制服的观赏效果如此出众,他那时就下令所有人在营地都要穿常服了。

“好,很适合庄严肃穆的场合。”陆久点了点头赞许地说道,“希望我们能够……做好这件事。”

“听起来你好像没什么信心。”V说。

“确实。”陆久叹了口气,“虽然参加过很多战斗,但我从来没有和士兵的家属们会过面,更何况是遗属。我也不知道,该怎样……”

“我们只要把这个消息传达给他们就行了。”

“我明白。虽然他们会很悲痛,但我们是什么都改变不了的。不过这件事——”

“别去想那些了。”V打断了陆久的话,“你不是说了吗,我们是去安抚他们,而不是去请罪的。顺其自然吧。”

“你说的对,想太多也没用。”意识到自己正在从V那里寻求安慰,陆久有些不好意思地笑了笑,“时间有限,我们出发吧。”

简单地在楼下用餐后,两个人走向了他们的越野车。V本想坐在司机位上,却被陆久阻止了:

“开车是男人的工作。” 陆久说。

\section*{}

虽然不太明白这样分工的理由,但V没有多问,只是按照陆久的意愿坐在了副驾上。陆久已经在地图上查好了目的地的位置,那是这座燕山脚下的城市周边的郊县,距离大概有六十多公里,还不算太远。

这是个风和日丽的冬日午后,虽然气温很低但阳光很好,照在身上让人感觉很温暖。两个人坐在飞速行驶的汽车里随着地势的起伏在马路上时上时下,村庄和田野在身边交替掠过,他们工作和生活的城市就这样在身后渐渐远去了。

“今天的天气真好。”汽车行驶了一阵,一直沉默着的V忽然开口说道。

“是啊。”陆久说。

天气的确很好,但陆久的情绪并没有跟着变好,因为他们不是去出游。想到那天在南宁发生的事情,陆久的心里只感觉到沉重。

“还在想那天的事情?”

“啊……没有。只是在思考该怎么和那些家属开口去说。”

“……你以前从来没有这样茫然失措过。”

陆久知道V说得没错。自从他得知这个阵亡士兵的身份之后,那天晚上的战斗场景就一直在他脑海里反复播放,他不断地猜测这位士兵是那天阵亡的四个人中的哪一个。他知道这样的猜测是没有结果的,但他越是这样想就越是无法自已。

“也许吧。”陆久说,“我也不知道自己这是怎么了。我知道去想那些事情毫无意义,以前我从不会过多考虑这样的事情,但现在却感觉有点控制不住自己。”

“那是因为我们都变了。”

陆久觉得这句话有点耳熟,微微扭头看了V一眼。他看到身边的女孩只是默默地注视着前方,脸上依然是不变的平静表情。

“那么是变好了,还是变坏了呢。”陆久说。

“变好了。你渐渐开始关注身边的事情,我觉得是好的改变。”

V的语气很肯定,这让陆久也开始思考起她的话来。曾经就连生死都不放在心上的人,忽然变得多愁善感,真的是件好事吗。

“你也变了。”过了一会儿陆久笑了笑说,“变得学会安慰人了。”

地图上六十公里的距离放到平面上就不止六十公里了,而且山区的路也高高低低地起伏不定,汽车行驶了大约两个小时才到达了目的地的郊区。在进入郊县之后,陆久先在路边停下了车,然后在地图上仔细地再次确认了那位士兵的家的位置。

那是一片小小的村庄,村子里全都是一排排整齐排列的房子,高的有三层、低的有两层,一看就是居民们自己盖起来的。这样的建筑风格陆久非常熟悉,他能够感到自己大概也曾经在这样的村庄里生活过,但具体的细节却又若有似无地无法捕捉。

“第三排第四户,就在那边。”隔着田地,陆久凝望着不远处的一片房屋说道。

“这里的房子有点像北镇呢。”V说。

“嗯,北方的村庄都是这样。”

“我们要过去吗。”

“稍等一会儿。”

陆久站在车后面,点燃了一根烟。他抽了一口,然后吐出烟气、整理了一下自己的情绪。

“我感觉你有些紧张。”

V说的没错,陆久确实心里在惴惴不安。第一次上战场的士兵也许就是这样的感受吧,虽然陆久已经记不起来自己第一次去打仗时的情景了。

“是啊,说实话我现在有点后悔揽下这事情了。”

“要是感觉不好,我们就回去吧。”

陆久笑了笑,他有时候不知道V到底是真的不谙世事,还是在故意讽刺他。就这么临阵脱逃,且不说分公司或者谢振知道了会怎么想,就连皮尔斯和雷蒙那边都交代不了吧。

“还不至于。”陆久说,“至少我们要面对的人没有武装。走吧。”

说完,陆久扔掉烟头、拉紧了领带,然后坐到了副驾上。

驾驶员换成了V,只用了几分钟,他们就来到了那座房子前。房子没有直接坐落在街道上,而是和街道之间隔着一片宽阔的庭院,而院门则是敞开的,说明家里有人。陆久下车后整理了一下衣服,把蓝色的信封拿在手里,深吸了一口气,然后向着庭院内走去。V则迅速跟在了他身后。

穿过修葺整齐的院子,陆久来到了房子的大门前,然后伸手在门上轻轻扣了三下。

片刻后,门开了一条缝,陆久看到一个瘦弱的年轻女人走了出来。

\section*{}

“请问这里是檀春野先生家吗。”陆久轻声说。

“你们……”

那个女人打量了陆久和V一番,显然没有想到家里会有这样的客人来访。看着这两个身穿军装的人,她一开始甚至没有反应过来他们是干什么的。

陆久稍稍迟疑了一下,没有立即自我介绍。因为他意识到很多雇佣兵都没有对自己的家人公开过自己的“工作”,也许这个士兵的家属还不知道“格里芬公司”这个组织到底是干什么的。

不过过了片刻,陆久知道自己不需要为这些事操心,因为这个女人显然知道檀春野是做什么工作的。陆久看到面前的女人的脸色变了,在短短几秒钟的时间里,他见证了一个人的表情从迷惑到惊讶、再到悲伤的变化全过程。

“你们走错门了,”那个女人用发抖的声音说着,“这里不是檀春野家。你们找错了!”

说完,她咣当一声关上了门。

“我们找错门了吗。”V在陆久背后轻轻说道。

“我想没有。”陆久沉声说。“稍等一会儿吧。”

果然,过了一阵,门再次打开了。不过这次走出来的不是刚才的年轻女人,而是一个头发有些花白的老人。那个老人大概有六十岁上下,身体有些佝偻手里还拿着拐杖,似乎腿上有残疾。

“请问这里是檀春野先生家吗。”陆久再次轻声说道。

“是的,你们是谁?” 老人打量了陆久一番,开口问道。

“我们是格里芬公司的办事人员。”陆久说,“您是檀春野先生的家属吧。”

“我是他父亲。”

陆久闻言,点了点头,然后轻轻吸了一口气:

“您好,檀先生。非常遗憾,我们为您带来了悲痛的消息。”

听到陆久的话,老人并没有什么反应,只是直直地看着陆久的眼睛。陆久没有回避,而是用尽量肃穆的目光回望着那个人。过了一阵,那个老人终于点了点头。

“说吧。”他说。

“您的儿子在几个月前的战斗行动中牺牲了。我代表格里芬公司将这一让不幸的消息转达给您,并向您致以最沉痛的哀悼,和最深切的慰问。”

说完,陆久递上了那个蓝色的信封。

老人接过信封看了一眼,没有说什么。但陆久听到屋里传来一声尖利的哀嚎。

“对不起,是我的儿媳在哭。她实在承受不了这样的打击。”陆久面前的老人用沙哑的声音说道,“你们的消息我收到了,让你们听到这样不堪的声音实在是不好意思。”

“这样悲痛的消息任谁都会感到难以接受。”陆久说,“说实话倒是您的镇定让我感到有些……吃惊。”

“你是春野的上司吗。”老人说。

“不,我们不是同一个部门。不过我曾经也是作战人员,所以也可以说是他的战友。”

“那就容易解释了。”老人说着提起了左边的裤腿,“我也曾经是战斗人员,这是最后一次战斗时留下的纪念。”

陆久看到他的左腿上装的是金属的义肢。于是陆久立正向老人敬了个军礼:

“作为军营的晚辈,我向您致敬。”

“不必,我已经不是军人了。”老人微微摇了摇头,“我负伤退役的时候春野才4岁,不过那时候他倒是很喜欢这条铁腿。后来他长大后想去当兵,我不同意,于是他去上了大学。大学毕业后据说去了个什么保安公司,呵呵,我就知道他不是普通的保安。”

“他是个英勇的战士。”

“他是怎么死的?”

“他在一次战斗中孤身掩护几位战友撤离,结果陷入了困境。很遗憾,当救援部队抵达的时候已经太晚了。”

“他救了很多人吗。”

“是的。得益于他的大无畏牺牲,有多位士兵得以安全撤离危险区域。”

“那就好。”老人点了点头说,“那就好……我也是靠别人的牺牲,才只丢了一条腿就活了下来。那时候我如果也留下断后,我们的队伍也许就不会……算了。知道他没有白白死去我很高兴,他比我要更加勇敢。”

“他值得您为之骄傲。”

听到陆久的话,老人笑了笑。

“年轻人,我不是不懂政治部那一套。”老人说,“战斗中很多士兵都是默默无闻地牺牲的,也许是被流弹击中、也许是被炮火波及,没有几个真正死得像个英雄。很多美丽的故事都是用来慰藉那些家属的,因为人死终归不能复生。但就算这是个故事,我也很感谢你给我带来的安慰。”

“这不是故事。我以军人的荣誉保证,我所说的句句属实。”

老人没说什么,只是再次对陆久笑了笑。陆久不能确定他到底有没有相信自己的话,但他可以看到那双眼睛里的悲伤已经难以抑制。

“抚恤事宜公司会另派人员前来联系的,届时遗资遗物也会一并送达。”陆久说,“檀春野的英勇事迹将永远激励我们,也请您节哀。”

说完,陆久再次敬礼,然后转身离开了正在强忍着丧子之痛的老人。

\section*{}

短短不足半小时的谈话,让陆久感到心力交瘁,犹如经历了几天不眠不休的战斗一样疲倦。他走到汽车跟前的时候,直接拉开车门坐在了后排。

V默默地启动汽车,朝着归途而去。开了半小时之后,她把汽车在路边的田野里停了下来。

“你还好吧。”V扭头对独自坐在后面的陆久说道。

“还好。”陆久说。V看了陆久一阵,然后走下汽车,从另一侧登上了后排坐在了陆久的旁边。

“你的脸色很差。”V说。

“没什么,稍微有点恶心。哈。”陆久笑了,“真是的,我果然不擅长做这种事。让你看了这么一场拙劣的表演,实在是太不好意思了。”

“我觉得你成功地说服了那个老人,他应该是相信了你的话。”

“那又如何,事实是怎样的你我都很清楚,那个士兵根本没什么英勇事迹,因为他是被自己反叛的同僚杀死的。一个活生生的人,死得像一只蝼蚁一般。但我们不能吐露事实,因为这就是人类的交际,面不改色地说谎就是其中的主要内容。军人的荣誉?哈,价值不过一句廉价的谎言。”

“至少有人从中得到了宽慰。我有点理解你所说的‘沉默是金’了。”

“呵呵。去他妈的。”陆久冷笑了一声,然后骂了一句脏话。

听到陆久的咒骂,V沉默了片刻。然后,她朝着陆久靠了过去,抓住陆久的胳膊将他拉到了跟前,然后从背后环抱住了他的肩膀。

起初陆久对V的动作有点意外,但当他明白V在做什么之后,他没有抗拒,而是顺从地把头靠在了V的胸前。

“在我们离开的时候,我看到那个老人的眼睛里除了悲痛还有感激。”V轻声说着,“我觉得你做了正确的事情,这让我学习到了很多东西。所以不要那么自责。”

“啊。我只是觉得有点累。”

“那是因为你尽力了。稍事休息一下吧。”

“好的。”陆久听从了V的劝告,在V的怀里闭上了眼睛。

当陆久醒来的时候天已经黑了。他睁开眼睛感到的第一件事就是手脚有些冰凉,因为日落后外面气温很低,而汽车又没有开空调,所以他感觉有些冷。不过至少他的身上没有感到冷,因为他不仅披着两件大衣,而且V依然在紧紧地抱着他用体温为他取暖。

“你醒了。”察觉到陆久动了,V轻声说道。

“啊。”陆久急忙坐直身子离开了V的肩头,“抱歉,我睡了很久吗。”

陆久其实睡的时间并不长,但因为北方的冬季白天很短,所以太阳已经落山。

“只是一小会儿。”V说,“感觉好些了吗。”

“好多了。谢谢。”陆久有点不好意思地说,“我们继续上路吧。”

“好,我来开车吧。”

两个人再次坐在了汽车的前面,开始朝着回去的方向走去。大约一个多小时之后,他们回到了自己的住所。

走进房间,陆久首先按照和皮尔斯的约定,换下了制服仔细叠好,然后把衣服放在了衣柜里。而V则端坐在沙发上,不知道陆久为何回来就急匆匆地先去换一身衣服。

“陆久。”V忽然开口说道。

“嗯?”

“那时候第一次出来开门的那位女士,是那个士兵的妻子吗。”

“应该是这样吧。”

“据我所知,特勤人员也经常外出作战,很少回家。所以她平时和那个士兵见面的机会应该不多吧。”

“是的,所以我们才有那么多的信件需要处理。虽然打电话或者发电子邮件会更方便快捷,但手写的书信更有实感、更能寄托人们的思念吧。为什么突然问起这些了?”

“没什么。想到总是要等着一个不知能否归来的人,忽然觉得她这样的人类也很可怜。”

当然。胡麻好种无人种,正是归时底不归,苦等征人总是令人心焦……不过年纪轻轻就成了寡妇,不更是悲哀的事情吗。而且这种话从V的口中说出,让陆久感到很有些讽刺,人类竟然被人形所同情,不知道这算不算另一种悲哀。

陆久意识到V也开始思考关于人类社会的问题了,但她的思路恐怕还是和普通人有些差异。

“所以我才一直觉得才要和身边的人少点瓜葛。自己要是出了什么事,也少点人为之伤怀。”陆久有些自嘲地说着,朝自己的卧室走去。一直到他睡着,客厅的灯都没有熄灭,V似乎还在思考今天发生的事情。
\include{3.8}
\include{3.9}
\include{3.10}
\include{3.11}
\include{3.a1}
\include{4.1}
\include{4.2}
\include{4.3}
\include{4.4}
\include{4.5}
\include{4.6}
\include{4.7}
\include{4.8}
\include{4.9}
\include{4.10}
\include{4.11}
\include{4.12}
\include{4.13}
\include{4.14}
\include{4.15}
\include{4.a1}
\chapterul{第四部分:最后的外传,以及后记}

\section*{前言}

山川不会相逢,因为它们永远只懂相互遥望。但人不同。只要他们决心向着对方走去,那么,总有一天——

\lineseparator

这就是“背叛者”故事的全部,已经没有什么再多的话可说。2017年2月3日开贴,至今已五年。回顾过往,一刹那恍惚,唯余若有所失的感觉。

外传已发给我留邮箱的朋友,让我先自己给自己完结撒花吧。

最后的文字,在下一页。

\clearpage

这是《背叛者》故事最后的章节。

献给所有喜欢、并且用热情来支持这个故事的人。

\vdots\footnote{由于作者的标注,本部分未编入此文档。若有需要,请自行前往\url{https://www.pixiv.net/novel/show.php?id=17017058}查看。}

\clearpage

\section*{后记}



后记啊……

其实我从一开篇就想着要写点什么后记,例如创作时的灵感、设定,以及对各个角色的访谈录之类的东西,里面也包含了一些陆久和V的私人生活……但真正到完结的时候,却一点都写不出来,因为中间经历的一系列事情,例如可悲的关注度和网络空间的封杀,已经将作者的热情消磨殆尽了。我甚至不太能记清自己到底为什么要给Vector写这样一个篇幅巨大的同人文,她的热度已经可怜到P站上连色图都没有几张。唯一值得欣慰的是终于把这个故事完整地地交代完了,并且和一开始说的那样,用一个温暖的结局给它画上了句点。不论结局好坏,这个故事里的人们都是在命运大潮里不断浮沉着,他们的人生在波澜之中可谓是辛苦而艰难的,作者想要通过这个故事来表达的,就是希望身在平静生活中的我们能够珍惜当下拥有的一切。

关于这个故事,有两处值得一提的事情,其一是关于续作的事情、其二是关于其结局。

“背叛者”的故事原本还有一个写了一半的大篇幅的外传的,外传描述了陆久和V分别的几个月里,两个人各自的经历。其中陆久是和416一起进行了充满浪漫色彩的长途跋涉旅行;而V则和NT77以及05(也就是46号实验体“山茶”,不知道大家还记不记得这号人)相处了一阵,相互交流适应人类社会的心得体会。而续作也曾考虑过,内容想象为归隐田园的陆久,因为受到极端人形权力组织的威胁,而不得不再次走向战争的泥潭……不过从这个故事的受欢迎程度来看,确定是没必要再写下去了,所以那些没收回或者收得不太好的条线,就让它们永远成为这个故事的悬念吧。

这个故事的剧情基本上是按照我一开始的设计进行加工的,唯独在处理帕斯卡部分剧情的时候,花费了不少的精力。我个人是非常喜欢帕斯卡这个角色的,我认为她是这个故事里塑造得最好的人物。如果说V的形象代表了理想的“纯洁之爱”的话,那么帕斯卡就代表了现实的“世俗之爱”。她的人生很复杂,不像V那样无得无失、喜欢一个人可以不顾一切地去爱他。对于帕斯卡来说,人生有许多值得或者是需要去追求的事情,爱情只是其中之一,在不得不权衡取舍的时候,她经常要放弃爱情去选择其他;而她又和每一个女人一样也是爱情动物,在最后时刻幡然醒悟,认识到什么野心阴谋都不如一个她爱的并且也爱过她的男人重要,但当她醒悟的时候却又为时已晚。她是个优秀的人,情商智商都极高,但她的追求更高。许多人都曾经爱过她,比如罗本、46号实验人形……还有我们的陆久先生,可帕斯卡辜负了所有人。利用自己的感情和身体做筹码,去实现那些常人难以实现的目的,对帕斯卡来说是惯用的手段,但正是因为她总是在利用感情,让她对感情抱有不信任感,所以总是无法收获真正的感情。这个故事的第二章的时候,我其实考虑过让这个故事的结局走向帕斯卡的:在南宁和V的对峙中,陆久选择了杀掉V来保护帕斯卡。在这个结局中,陆久和帕斯卡秘密结婚、入职16LAB,并最终成了帕斯卡的私人武装力量的头目,为她进行暗中铲除异己、实现她那庞大野心而活动着。这是个相当有吸引力的结局,但经过不断地思想斗争之后,我还是认为陆久应该选择V,因为人生中多数时候我们都不得不选择世俗之爱,而既然是故事,就应该超越现实。所以,这个故事最后还是走向了既定的方向。

故事的尾声之后,曾经被命运联系在一起的人们,走向了各自的人生:

帕斯卡带着NT-77来到西亚后,和某个人在一间叫“无名芳草”的花店……其实是一个庞大的情报网的终端会了面。花店的老板名字叫“山茶”,她有着许多的身份,包括16LAB的46号实验人形、以及N17战区的临时侦查队长05。但她现在已经是个自由的人了。帕斯卡这次前来的目的是情报,但那只是其中之一。经过这一段时间的辗转之后,她希望能够好好地处理自己的感情事务,因此她坦白且详细地相互介绍了几个人的情况,这让山茶陷入了极端为难的境地。帕斯卡这个历史复杂的姐姐,追求男人失败之后转投自己,之前的恩怨情仇倒不在话下;NT-77这个人,一方面可以说是杀害95的罪魁祸首,但从另外的意义上,她也可以说是95本人,就让山茶不知道该如何去对待她了。在经过一整天的思考之后,她们都决定放下过去的恩怨用新的身份开始相处。因为世界已经是新的气象,如果执着于过去,她们也许是仇人;如果放下过去,那么她们也可以是亲人——她们的仇人各自都有很多,但亲人,却已所剩无几。

离开了战场之后,陆久保留了帕斯卡的联系方式,不仅是出于对过去的怀念,更是因为陆久知道帕斯卡能为他提供一些极具价值的东西。陆久知道,不彻底斩断他和帕斯卡之间的孽缘本身就是一种极大的风险,但他还是希望帕斯卡能够有个好的人生……当然,这次他会仔细考虑需要付出的代价,不会再去做她的棋子。陆久同和安洁也偶有接触,进行一些合约性的合作,因为安洁是个爽快的女人,开出的劳务费总是相当可观。他参与了404小队的数次行动,但是只负责战略和战术方面的策划,不再参与直接作战。

关于皮尔斯,陆久唯一确知的是,他带着自己的人形伴侣逃离了父亲的控制。陆久完全没有皮尔斯的任何情报和联系方式,陆久推测以后也不会再有了。陆久知道皮尔斯是一个心思缜密的人,如果他决意要人间蒸发,那绝对不会留下任何多余的线索。虽然就这样和一个朋友诀别未免会让人有些感伤,但陆久知道,这样的代价对皮尔斯来说是值得的。他们对彼此来说并非必须的,相忘江湖、各自安好,是最好的结局。

而名为陆薇的少女,陆久给她的第一个任务,是了解和融入人类社会。虽然V只想要在陆久的身边,但她也有许多要学习的事情,因为陆久要她做她自己——不是别人给她安排的角色,而是她自己真正想要成为的人。喜爱的东西、喜欢的事情,还有存在的意义,这一切需要自己用心去寻找和发现,无法靠其他人传授。

在新的时代里,陆久已经成了一个符号,象征着反抗对人形的奴役、为了人形平权而斗争的先驱——为了救出自己所爱的人形,不惮和曾经的同袍反目、只身飞蛾扑火般地向着绝境而去,最终消失在辐射区的中心……人们相信陆久死了,因为他生还的可能性实在微乎其微,而且殉情的结局也更符合人们对浪漫而悲壮的英雄主义的想象。陆久的故事经过媒体精心地包装和渲染后,成了人们津津乐道的话题,也成了无数热血少年的榜样,对世界的冲击远大过一颗摧毁城市的脏弹。这个故事虽然是陆久斗争的结束,但却是无数其他人斗争的开始,从此,人形出现在公共场合不需要再佩带身份标识,和人形并肩乃至挽手而行,也成了思想前卫的表现。而这一切给陆久带来的便利就是:在埋藏身份的归隐生活中,当他和V一起走在市井的时候,再也不会有人对他们投以额外的目光,减少了许多不必要的麻烦。陆久对此感到很满意,在经历了无数各怀鬼胎的阴谋、流尽了众多无辜者的鲜血之后,他的愿望最终得以实现。这世界的诸多重大变革都是这样——一个宏大的伟业的实现,最初可能只是源自一个简单而渺小的愿望。但混沌之中,众生芸芸劳碌奔波,又有几人会在意这种事呢?毕竟,每个人都有自己的生活。

至此,这个故事总算是尘埃落定。至于未完成的外传之类的,大约是要永久搁置,因此请不必期待了。

就这样吧。感谢一直支持我的朋友们,江湖再见,如果有缘的话。

\rightline{某年某日\quad 某夜上}

\end{document}
